\section*{Equazioni differenziali}
\rule{\textwidth}{2pt}
\subsection*{Modelli differenziali}
Si dice equazione differenziale di ordine $n$ un equazione del tipo:
\[
    F(t, y, y', y'', \dots, y^{(n)}) = 0
\]
dove $y(t)$ è la funzione incognita.\newline
Si dirà soluzione, o (curva) integrale, di un equazione differenziale, nell'intervallo $I \subset \mathbb{R}$, una funzione $\phi(t)$, definita almeno in $I$ e a valori reali per cui risulti
\[
    F(t, \phi(t), \phi'(t), \phi''(t), \dots, \phi^{(n)}(t))= 0 \;\;\;\;\;\;\;\; \;\forall\;t \in I
\]
Infine si dirà integrale generale di un equazione differenziale una formula che rappresenti la famiglia di tutte le soluzioni.\newline
\rule{\textwidth}{2pt}
\subsection*{Equazioni del primo ordine}
\rule{\textwidth}{0,4pt}
\subsubsection*{Generalità}
L'insieme delle soluzioni di un'equazione differenziale del prim'ordine prende il nome di integrale generale dell'equazione.\newline
Un equazione differenziale si dice in forma normale se è nella forma
\[
    y'(t) = f(t,y(t))
\]
e ha infinite soluzioni del tipo
\[
    y(t) = \int f(t, y(t)) dt + c .
\]
La condizione supplementare
\[
    y(t_0) = y_0
\]
permette di selezionare una soluzione particolare.\newline
Il problema di risolvere queste equazioni prende il nome di problema di Cauchy, e si intende sempre che bisogna trovare come soluzione una funzione:
\begin{itemize}
    \item definita su un intervallo $I$, contenente il punto $t_0$;
    \item derivabile in tutto $I$ e che soddisfa l'equazione in tutto $I$.
\end{itemize}
\rule{\textwidth}{0,4pt}
\subsubsection*{Equazioni a variabili separabili}
Le equazioni a variabili separabili sono equazioni del tipo
\begin{tcolorbox}
\[
    y' =a(t) b(y)
\]
\end{tcolorbox}
con $a$ continua in $I \subset \mathbb{R}$ e b continua in $J \subset \mathbb{R}$.\newline
Per risolverle notiamo che per tutte le $\bar{y}_0$ per cui $b(\bar{y}_0) = 0$, primo e secondo membro si annullano.\newline
Prendendo in considerazione, invece, tutte le $\bar{y}_1$ per cui $b(\bar{y}_1) \neq 0$, otteniamo:
\begin{tcolorbox}
\[
    \int \frac{dy}{b(y)} = \int a(t) dt +c 
\]
\end{tcolorbox}
Quindi se $B(y)$ è una primitiva di $\frac{1}{b(y)}$ e $A(t)$ è una primitiva di $a(t)$, otteniamo:
\[
    B(y) = A(t) + c
\]
e se si può ottenere la funzione inversa di $B$ possiamo scrivere:
\[
    y = B^{-1}(A(t) + c)
\]
\[
    y = F(t,c)
\]
\begin{tcolorbox}
\textbf{teor.} Problema di Cauchy per un'equazione a variabili separabili.\newline
Si consideri il problema di Cauchy:
\[
    \begin{cases}
        y' = a(t) b(y)&\\
        y(t_0) = y_0&
    \end{cases}
\]
Dove $a$ è continua in un intorno $I$ di $t_0$ e $b$ è continua in un intonro $J$ di $t_0$. Allora essite un intorno $t_0$ $I' \subset I$ e una funzione $y \in C^1(I')$ soluzione del problema.\newline
Se inoltre $b \in C^1(J)$, allora tale soluzione è unica.\newline
\end{tcolorbox}
\rule{\textwidth}{0,4pt}
\subsubsection*{Equazioni lineari del prim'ordine}
Sono equazioni la cui forma normale è
\begin{tcolorbox}
\[
    y'(t) + a(t)y(t) = f(t)
\]
\end{tcolorbox}
e viene chiama equazione completa.\newline
Se $f = 0$ l'equazione si dice omogenea.
\[
    z'(t) + a(t)z(t) = 0
\]
\textbf{teor.} L'integrale generale dell'equazione completa si ottiene aggiungengo all'integrale generale dell'omogenea una soluzione particolare della completa.\newline
\textbf{Soluzione dell'equazione omogenea}
\[
    z(t) = c e^{-\int a(t) dt}
\]
\textbf{Ricerca di una soluzione particolare dell'equazione completa}\newline
Se una soluzione particolare non si riesce a trovare facilmente, si può usare il seguente metodo, detto di variazione della costante
\[
    \bar{y}(t) = c(t)e^{-A(t)}
\]
Dove per $A(t)$ si intende una primitiva di $a(t)$ (essendo una primitiva qualunque, si può omettere la $c$).\newline
Sostituendo $\bar{y}$ nella forma completa otteniamo l'integrale generale della forma completa:
\[
    y(t) = ce^{-A(t)} + e^{-A(t)}\int f(t)e^{A(t)}dt
\]
\textbf{oss.} Nella primitiva $A(t)$ e nell'integrale $\int f(t)e^{A(t)}dt$ non c'è bisogno di aggiungere la solita costante di integrazione arbitraria.\newline
\newline
\textbf{Soluzione del problema di Cauchy}\newline
La soluzione
\begin{tcolorbox} 
\[
    y(t) = ce^{-A(t)} + e^{-A(t)}\int f(t)e^{A(t)}dt
\]
\end{tcolorbox}
sarà determinata da una condizione iniziale
\[
    y(t_0) = y_0
\]
scegliendo la primitiva $A(t)$ tale che $A(t_0) = 0$ (cioè $A(t) = \int_{t_0}^{t}a(s)ds$), sarà
\[
    y(t) = ce^{-A(t)} + e^{-A(t)} \int_{t_0}^{t} f(s)e^{A(s)}ds
\]
\begin{tcolorbox}
\textbf{teor.} Problema di Cauchy per un equazione lineare del prim'ordine.\newline
Siano $a, f$ funzioni continue in un intervallo $I$ contenente $t_0$. Allora, per ogni $y_0 \in \mathbb{R}$ il problema di Cauchy
\[
    \begin{cases}
        y'(t) + a(t)y(t)= f(t)\\
        y(t_0) = y_0
    \end{cases}
\]
ha una e una sola soluzione $y \in C^1(I)$ e tale soluzione è
\[
    y(t) = ce^{-A(t)} + e^{-A(t)} \int_{t_0}^{t} f(s)e^{A(s)}ds
\]
\end{tcolorbox}
\rule{\textwidth}{0,4pt}\newline
\begin{tcolorbox}
    \subsubsection*{Note sugli esercizi}
    Per trovare il massimo intervallo della soluzione bisogna prendere un intervallo contenente il punto $t_0$, per cui la soluzione e la funzione di partenza siano continue e definite (si può anche prolungare per continuità).\newline
\end{tcolorbox}
\rule{\textwidth}{2pt}
\subsection*{Equazioni lineari del secondo ordine}
\rule{\textwidth}{0,4pt}
\subsubsection*{Spazi di funzioni}
\textbf{def.} definiamo $C^n(I)$ lo spazio delle funzioni dotate di derivata $n$-essima continua.\newline
\subsubsection*{Generalità sulle equazioni lineari. Problema di Cauchy}