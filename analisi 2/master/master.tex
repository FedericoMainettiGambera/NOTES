\documentclass[a4paper, 9pt]{report}
  
\author{Federico Mainetti Gambera}
\usepackage{amsmath}
\usepackage{amssymb}
\usepackage{graphicx}
\usepackage[italian]{babel}
\usepackage{import}
\usepackage{xifthen}
\usepackage{pdfpages}
\usepackage{transparent}
\usepackage{xcolor}
\usepackage{cancel}
\usepackage[a4paper,left=35mm,top=26mm,right=26mm,bottom=15mm]{geometry}
\usepackage{color}
\usepackage{tcolorbox}
\usepackage{hyperref}
\usepackage{makeidx}
\usepackage{soul}
\usepackage{multicol}
\usepackage{pdflscape}
\makeindex
\definecolor{lightgray}{gray}{0.75}
\renewcommand{\familydefault}{\sfdefault}
\newenvironment{rcases}
  {\left.\begin{aligned}}
  {\end{aligned}\right\rbrace}
\newcommand{\incfig}[1]{%
    \def\svgwidth{\columnwidth}
    \import{../images/}{#1.pdf_tex}
}
\begin{document}
    \maketitle
    \newpage
    \input{../capitolo0/capitolo0c.tex} % ripasso random
    \newpage
    \section*{Serie}
\subsection*{serie geometrica}
\[
    \sum_{n=0}^{\infty}q^n = \begin{cases}
        \frac{1}{1-q} & se \;\; -1<q<1\\
        +\infty &se \;\; q \geq 1\\
        irregolare \;\;& se \;\; q\leq-1 
    \end{cases}
\]
\subsection*{serie armonica}
\[
    \sum_{n=1}^{\infty}\frac{1}{n} \geq log(n+1) \rightarrow +\infty
\]
\subsection*{serie armonica generalizzata}
\[
    \sum_{n=1}^{\infty} \frac{1}{n^{\alpha}}
\]
per $\alpha \leq 1$ 
\[
    \sum_{n=1}^{\infty} \frac{1}{n^\alpha} \geq \sum_{n=1}^{\infty}\frac{1}{n} \rightarrow +\infty \;\;\;\text{diverge}\;
\]
per $\alpha > 1$ 
\[
    \sum_{n=1}^{\infty} \frac{1}{n^\alpha} = converge
\]
per $\alpha = 2$
\[
    \sum_{n=1}^{\infty}\frac{1}{n^2} = \frac{\pi^2}{6} (\sim  \sum_{n=1}^{\infty} \frac{1}{n(n+1)} = serie \;\; di \;\; mengoli)
\]
\subsection*{serie di mengoli}
\[
    \sum_{n=1}^{\infty}\frac{1}{n(n+1)} = \sum_{n=1}^{\infty} \frac{1}{n}-\frac{1}{n+1} = 1- \frac{1}{n+1} \rightarrow 1
\]
\subsection*{numero e}
\[
    \lim_{n\rightarrow \infty} \left( 1 + \frac{1}{n} \right)^n = e
\]
\subsection*{sviluppi di Taylor delle funzioni elementari}
\[
    e^x = \lim_{n\rightarrow +\infty}\sum_{k=0}^{n} \frac{x^k}{k!} = \sum_{k=0}^{+\infty} \frac{x^k}{k!}
\]
\[
    sin (x) =\sum_{k=0}^{\infty}(-1)^k \frac{x^{2k+1}}{(2k+1)!} =  \frac{e^{ix} - e^{-ix}}{2}
\]
\[
    cos (x) =\sum_{k=0}^{\infty}(-1)^k \frac{x^{2k}}{(2k)!} =\frac{e^{ix} + e^{-ix}}{2}
\]
\subsection*{Serie di potenza}
Con $a_k$ costanti reali (o complesse) e $x$ variabile reale (o complessa)
\[
    \sum_{k=0}^{\infty}a_k x^k
\]
\[
    Sh (x) = \sum_{k=0}^{\infty} \frac{x^{2k+1}}{(2k+1)!}
\]
\[
    Ch(x) = \sum_{k=0}^{\infty} \frac{x^{2k}}{(2k)!}
\]
\[
    log(1+x)= \sum_{k=1}^{\infty} (-1)^{k+1} \frac{x^k}{k} \;\;\; per \;\; |x|<1
\]
per $\alpha \in \mathbb{R}$
\[
    (1+x)^\alpha = \sum_{k=0}^{\infty} \binom{\alpha}{k}x^k \;\;\; per \;\; |x|<1
\]
\textbf{teor.} Condizione necessaria affinché una serie $\sum_{n=0}^{\infty} a_n$ converga è che il termine generale $a_n$ tenda a zero. (Cioè perchè la serie converga, il termine $a_n$ deve tendere a zero, ma non per forza se il termine $a_n$ tende a zero allora la serie converge)\newline
\newline
\textbf{teor.}supponiamo che una serie $\sum_{n=0}^{\infty} a_n$ converga, allora per ogni $k$ anche risulta convergente anche $\sum_{n=k}^{\infty} a_n$.\newline
\newline
\textbf{Criterio serie a termini non negativi} Una serie $\sum_{n=0}^{\infty}a_n$ a termini non negativi è convergente o divergente a $+\infty$. Essa converge se e solo se la successione delle somme parziali n-esime è limitata.\newline
\newline
\textbf{Criterio del confronto} Siano $\sum an$ e $\sum b_n$ due serie a termini non negativi tali che $a_n<b_n$ definitivamente, allora:
\begin{itemize}
    \item $\sum b_n$ convergente $\Rightarrow \sum a_n$ convergente. 
    \item $\sum a_n$ divergente $\Rightarrow \sum b_n$ divergente. 
\end{itemize}
\textbf{Criterio del confronto asintotico} Se $a_n \sim b_n$, allora le corrispondenti serie $\sum a_n$ e $\sum b_n$ hanno lo stesso carattere (o entrambe divergenti o entrambe divergenti )\newline
\newline
\textbf{Criterio della radice} Sia $\sum a_n$ una serie a termini non negativi. Se esiste il limite 
\[
    \lim_{n\rightarrow +\infty}\sqrt[n]{a_n} = l
\]
\begin{itemize}
    \item $l>1$ la serie diverge $+\infty$
    \item $l<1$ la serie converge
    \item $l=1$ nulla si può concludere
\end{itemize}
Spesso utilizzato con termini che hanno come esponente $n$.\newline
\newline
\textbf{Criterio del rapporto} Sia $\sum a_n$ una serie a termini positivi. Se esiste il limite 
\[
    \lim_{n\rightarrow +\infty} \frac{a_{n+1}}{a_n} = l
\]
\begin{itemize}
    \item $l>1$ diverge $+\infty$
    \item $l<1$ converge 
    \item $l=1$ nulla si può concludere
\end{itemize}
Spesso utilizzato quando si hanno termini come $n^n$ e $n!$.\newline
\newline
\textbf{Criterio serie a termini di segno variabile} Una serie $\sum a_n$ si dice assolutamente convergente se converge la serie $\sum |a_n|$. Se la serie $\sum a_n$ converge assolutamente, allora converge.\newline
\newline
\textbf{Criterio di Leibniz} Sia data la serie 
\[
    \sum_{n=0}^{\infty}(-1)^n a_n \;\; con  \;\; a_n\geq 0 \;\forall\;n
\]
Se la successione $\{a_n\}$ è decrescente e se $a_n \rightarrow 0$ per $n \rightarrow \infty$, allora la serie è convergente.\newline
Il criterio di Leibniz può essere applicato anche se i termini sono definitivamente di segno alterno e la successione $a_n$ è definitivamente decrescente.\newline
Per verificare la decrescenza bisogna dimostrare che $a_{n+1}<a_n$ oppure mediante il limite a $+ \infty$ della derivata prima di $a_n$ o studiano quando la derivata prima di $a_n<0$ .\newline
Per determinare se una serie è decrescente non vanno usati gli asintotici !\newline
\newline
\textbf{Criterio della somma di serie convergenti} Se $\sum_{n=1}^{\infty} a_n$ converge e $\sum_{n=1}^{\infty} b_n$ converge, allora $\sum_{n=1}^{\infty} a_n +b_n$ converge.\newline
\newline
\textbf{Criterio della somma di serie convergenti e divergenti} Se $\sum_{n=1}^{\infty} a_n$ converge e $\sum_{n=1}^{\infty} b_n$ diverge, allora $\sum_{n=1}^{\infty} a_n + b_n$ diverge.\newline
\newline
\textbf{Criterio serie a termini complessi} Sia la serie $\sum_{n=0}^{\infty} a_n$ con $a_n$ complesso, se la serie  $\sum_{n=0}^{\infty} |a_n|$ converge, allora anche $\sum_{n=0}^{\infty} a_n$ converge \newline
\newline
\textbf{Criterio di Dirichlet} Siano $a_n$ e $b_n$ due succesioni tali che:
\begin{itemize}
    \item $a_n$ è a valori complessi e la sua successione delle somme parziali è limitata.
    \item $b_n$ è a valori reali positivi e tende monotonamente a zero
\end{itemize}
allora la serie $\sum a_nb_n$ è convergente. % serie di analisi 1
    \newpage
    \input{../capitolo0/capitolo0b.tex} % integrali di analisi 1
    \newpage
    \section{Calcolo infinitesimale per le curve}
\rule{\textwidth}{2pt}
\subsection{Richiami di calcolo vettoriale}
vettore
\[
    \vec{x} = (x_1, x_2, \dots, x_n)
\]
modulo di un vettore
\[
    |\vec{x}| = \sqrt{\sum_{i=1}^{n}x_i^2}
\]
versore
\[
    vers(\vec{x}) = \frac{\vec{x}}{|\vec{x}|}
\]
$\vec{x}, \vec{y}$ sono paralleli se
\[
    \lambda \vec{x} = \mu \vec{y}\;\; \text{ per qualche } \;\;\lambda, \mu \in \mathbb{R}
\]
Somma di vettori: si sommano le componenti simili.\newline
Prodotto fra un vettore e uno scalare: si moltiplica ogni componente per lo scalare.\newline
Prodotto scalare fra vettori: ha come risultato un numero reale ottenuto dalla formula
\[
    \vec{u} \cdot \vec{v} = (u_1 v_1 + u_2 v_2 + \dots + u_n v_n)
\]
Il prodotto scalare può essere espresso anche come
\[
    \vec{u} \cdot \vec{v} = |\vec{u}| |\vec{v}| cos(\theta)
\]
dove $\theta$ rappresenta l'angolo tra i due vettori. Di conseguenza $\vec{u} \cdot  \vec{v} = 0$ solo se i due vettori sono ortogonali.\newline
Dal prodotto scalare si può ricavare l'angolo fra due vettori
\[
    cos(\theta) = \frac{\vec{u} \cdot  \vec{v} }{|\vec{u}| |\vec{v}|}
\]
Inoltre $\vec{v} \cdot  \vec{v} = |\vec{v}|^2$.\newline
Prodotto vettoriale:
\[
    \vec{u}\text{x}\vec{v} = \left|\begin{matrix}
        i \;\; & j \;\;& k \;\;\\
        u_1 & u_2 & u_3\\
        v_1 & v_2 & v_3\\
    \end{matrix}\right|
\]
Il prodotto vettoriale si annulla solo se i vettori sono paralleli.\newline
Regola della mano destra: il primo fattore va sul pollice, il secondo sull indice, il risultato è nel medio.\newline
Inoltre $\vec{v} \text{x} \vec{v} = 0$.\newline
Prodotto misto:
\[
    \vec{u} \cdot (\vec{v} \text{x} \vec{w}) = \left|\begin{matrix}
        u_1 \;\;& u_2 \;\;& u_3\\
        v_1 &v_2 & v_3\\
        w_1 &w_2 & w_3
    \end{matrix}\right|
\]
Il prodotto misto si annulla solo se i tre vettori sono linearmente indipendenti.\newline
\rule{\textwidth}{2pt}
\subsection{Funzioni a valori vettoriali, limiti e continuità}
Si dice funzione a valori vettoriali una funzione $\vec{f} : \mathbb{R} \rightarrow  \mathbb{R}^n$ con $n > 1$. Le funzioni $\vec{f} : \mathbb{R} \rightarrow \mathbb{R}^2$ o $\mathbb{R}^3$.\newline
Il limite della funzione a valori vettoriali si calcola componente per componente:
\[
    \lim_{t\rightarrow t_0}(r_1(t), r_2(t), \dots, r_n(t)) = \left(\lim_{t\rightarrow t_0}r_1(t), \lim_{t\rightarrow t_0}r_2(t), \dots, \lim_{t\rightarrow t_0}r_n(t)\right)
\]
Valgono allo stesso modo delle funzioni unidimensionali il teorema di unicità del limite e la definizione di continuità (una funzione a valori vettoriali è continua in un punto se lo sono tutte le sue componenti).\newline
\rule{\textwidth}{2pt}
\subsection{Curve regolari e calcolo differenziale vettoriale}
Nel caso $n = 2$ o $3$, una funzione $\vec{f} : \mathbb{R} \rightarrow  \mathbb{R}^n$ rappresentano curve nel piano o nello spazio tridimensionale.\newline
Sia $I$ un intervallo in $\mathbb{R}$. Si dice arco di curva continua, o cammino, in $\mathbb{R}^n$ una funzione $\vec{r}: I \rightarrow \mathbb{R}^n$ continua.\newline
La curva si dice chiusa se $\vec{r}(a) = \vec{r}(b)$ con $I=[a,b]$.\newline
La curva si dice semplice se non ripassa mai nello stesso punto.\newline
Il sostegno della curva è l'immagine della funzione, cioè l'insieme dei punti di $\mathbb{R}^n$ percorsi dal punto mobile.\newline
Una curva si dice piana se esiste un piano che contiene il suo sostegno.\newline
\rule{\textwidth}{0,4pt}
\subsubsection{Derivata di una funzione vettoriale. Arco di curva regolare}
\textbf{def.} Sia $\vec{r}: I \rightarrow \mathbb{R}^n$ e $t_0 \in I$, si dice che $\vec{r}$ è derivabile in $t_0$ se esiste finito
\[
    \vec{r}'(t_0) = \lim_{h\rightarrow 0}\frac{\vec{r}(t_0 +h) - \vec{r}(t_0)}{h}
\]
\begin{tcolorbox}
Se $\vec{r}$ è derivabile in tutto $I$ e inoltre $\vec{r}'$ è continuo in $I$, si dice che $\vec{r}$ è di classe $C^1(I)$ ($\vec{r}\in C^1(I)$).
\end{tcolorbox}
\begin{tcolorbox}
Notiamo che il vettore derivato è il vettore delle derivate delle componeneti:
\[
    \vec{r'}(t_0) = (r'_1(t_0), r'_2(t_0), \dots, r'_n(t_0))
\]
\end{tcolorbox}
\begin{tcolorbox}
\textbf{def.} Sia $I \subseteq \mathbb{R}$ un intervallo. Si dice arco di curva \textbf{regolare} un arco di curva $\vec{r}: I \rightarrow  \mathbb{R}^n$ tale che $\vec{r} \in C^1(I)$ e $\vec{r}'(t) \neq 0$ per ogni $t \in I$.
\end{tcolorbox}
\begin{tcolorbox}
Di conseguenza per le curve regolari è ben definito il \textbf{versore tangente}:
\[
    \vec{T} = \frac{\vec{r}'(t)}{|\vec{r}'(t)|}
\]
\end{tcolorbox}
\textbf{def.} Si dice arco di curva regolare a tratti un arco di curva $\vec{r} : I \rightarrow \mathbb{R}^n$ tale che: $\vec{r}$ è continua e l'intervallo $I$ può essere suddiviso in un numero finito di sottointervalli, su ciascuno dei quali $\vec{r}$ è un arco di curva regolare.\newline
\newline
Alcune proprietà del calcolo differenziale vettoriale:
\[
    (\vec{u} + \vec{v})' = \vec{u}' + \vec{v}'
\]
\[
    (c \vec{u})' = c \vec{u}' \;\;\; \text{con c costante}
\]
\[
    (f \vec{u})' = f' \vec{u} + \vec{u}' f \;\;\;\text{con} \; f \; \text{funzione}
\]
\[
    [\vec{u}(f(t))]' = \vec{u}'(f(t))f'(t)
\]
\[
    (\vec{u} \cdot \vec{v})' = \vec{u}' \cdot  \vec{v} + \vec{u} \cdot  \vec{v}'
\]
\[
    (\vec{u} \text{x} \vec{v})' = \vec{u}' \text{x} \vec{v} + \vec{u} \text{x} \vec{v}' \;\;\; in \;\mathbb{R}^3
\]
Dato un vettore $\vec{r}(t)$ si dice
\begin{itemize}
    \item velocità: $\vec{r}'(t)$
    \item velocità scalare: $|\vec{r}'(t)|$
    \item accellerazione: $\vec{r}''(t)$
    \item accellerazione scalare: $|\vec{r}''(t)|$
\end{itemize}
\rule{\textwidth}{0,4pt}
\subsubsection{Integrale di una funzione vettoriale}
\[
    \int_{a}^{b} \vec{r}(t) dt =\left( \int_{a}^{b}\vec{r_1}(t) dt,\int_{a}^{b}\vec{r_2}(t) dt, \dots, \int_{a}^{b}\vec{r_n}(t) dt,\right)
\]
Diremo che $\vec{r}$ è integrabile in $[a,b]$ se lo è ognuna delle sue componenti.\newline
\[
    \int_{a}^{b} \vec{r}'(t) dt = \vec{r}(b) - \vec{r}(a)
\]
Può essere utile utilizzare il seguente lemma:
\[
    \left| \int_{a}^{b}\vec{r}(t)dt \right| \leq \int_{a}^{b}|\vec{r}(t)|dt
\]
Se $\vec{r}(t)$ è una curva regolare e chiusa in $[a,b]$, allora $\int_{a}^{b}\vec{r}'(t) dt =0$.\newline
\rule{\textwidth}{0,4pt}
\subsubsection{Classi di curve piane}
\textbf{curve piane, grafico di funzioni}
\begin{tcolorbox}
Curve ottenute da grafici di funzioni in una variabile:
\[
    y = f(x) \quad\text{per x in} \;[a,b]
\]
Forma parametrica:
\[
    \begin{cases}
        x=t \\
        y=f(t)
    \end{cases} \;\;\;\; \text{per t} \;\in[a,b]
\]
\end{tcolorbox}
Proprietà:
\begin{itemize}
    \item è continua se e solo se $f$ è continua
    \item è regolare se e solo se $f$ è derivabile con continuità (le condizioni di non annullamento della derivata prima sono automaticamente verificate perchè $x'(t) = 1$)
    \item è regolare a tratti se e solo se $f$ è continua e a tratti derivabile con continuità
    \item non è mai chiusa
    \item è sempre semplice
\end{itemize}
\textbf{curve piane, forma polare}\newline
\begin{tcolorbox}
L'equazione
\[
    \rho = f(\theta) \;\;\;\text{per } \; \theta \in [\theta_1, \theta_2]
\]
è una forma abbreviata che, tramite la sostituzione $x= \rho cos(\theta)$ e $y= \rho sin(\theta)$, può essere riscritta:
\[
    \begin{cases}
        x = f(\theta) cos(\theta)\\
        y = f(\theta) sin(\theta)
    \end{cases} \;\;\; \text{per } \; \theta \in [\theta_1, \theta_2]
\]
\end{tcolorbox}
Ricordiamo che $\rho = \sqrt{x^2 + y^2}$
Osserviamo che 
\[
    \vec{r}'(\theta) = (f'(\theta) cos(\theta)- f(\theta)sin(\theta), f'(\theta) sin(\theta) - f(\theta)cos(\theta))
\]
\begin{tcolorbox}
\[
    |\vec{r}'(\theta)| = \sqrt{\rho(\theta)^2 + \rho'(\theta)^2} = \sqrt{f'(\theta)^2 + f(\theta)^2}
\]
\end{tcolorbox}
Geometricamente la forma polare $\rho = f(\theta)$ può essere visualizzata come la curva tracciata da una penna posta su un braccio che ruota attorno all'origine a velocità costante (in modo che il tempo $t$ combaci con l'angolo $\theta$). Mentre il braccio ruota la penna si sposta lungo il braccio in modo da essere a distanza $f(\theta)$ dall'origine all'istante $\theta$.\newline
\newline
Proprietà:
\begin{itemize}
    \item è continua se e solo se $f$ è continua
    \item è regolare se e solo se $f$ è derivabile con continuità, inoltra $f$ e $f'$ non si annullano mai contemporaneamente
    \item è chiusa se e solo se $f(\theta_1) = f(\theta_2)$ e $\theta_2 - \theta_1 = 2n\pi$ per qualche intero $n$.
\end{itemize}
\textbf{coniche in forma polare}\newline
Equazione polare della conica:
\[
        \rho = \frac{\epsilon p}{1- \epsilon cos(\theta)}
\]
con $\epsilon > 0$ e $p> 0$ e $\theta$ varia nell'intervallo in cui il secondo membro è definito e positivo.\newline
Questa equazione rappresenta:
\[
    \begin{cases}
        \text{un'ellissi} \;\; &\text{se} \; \epsilon< 1\\
        \text{una parabola} \;\; &\text{se} \; \epsilon= 1\\
        \text{un'iperbole} \;\; &\text{se} \; \epsilon> 1
    \end{cases}
\]
Notiamo che il segmo $-$ davanti al coseno non è importante, infatti se cambiamo $\theta = t + \pi$ l'equazione si trasforma in $\rho = \frac{\epsilon p}{1 + \epsilon cos(t)}$.\newline
Equazione della circonferenza:
\[
        \rho = R
\]
\rule{\textwidth}{2pt}
\subsection{Lunghezza di un arco di curva}
\rule{\textwidth}{0,4pt}
\subsubsection{Curve rettificabili e lunghezza}
\textbf{def.} Si dice che $\gamma$ è rettificabile se 
\[
    sub_\mathcal{P} l(\mathcal{P}) = l(\gamma) < + \infty
\]
Dove l'estremo superiore è calcolato al variare di tutte le possibili partizioni $\mathcal{P}$ di $[a,b]$.\newline
In tal caso $l(\gamma)$ assegna per definizione, la lunghezza di $\gamma$.\newline
\newline
\textbf{teor.} Sia $\vec{r}:[a,b] \rightarrow \mathbb{R}^n$ la parametrizzazione di un arco di curva $\gamma$ regolare. Allora $\gamma$ è rettificabile e 
\begin{tcolorbox}
\[
    l(\gamma) = \int_{a}^{b}|\vec{r}'(t)| dt
\]
\end{tcolorbox}
Nello spazio tridimensionale la formula diventa:
\[
    l(\gamma)= \int_{a}^{b}\sqrt{x'(t)^2 + y'(t)^2 + z'(t)^2}dt
\]
\textbf{Lunghezza di un grafico}\newline
Sia $\gamma$ una curva piana regolare che sia grafico di una funzione, ossia:
\[
    \gamma : \begin{cases}
        x =t \\
        y= f(t)
    \end{cases} \;\; \text{per} \;t \in[a,b]
\]
allora
\begin{tcolorbox}
\[
    l(\gamma) = \int_{a}^{b}\sqrt{1+f'(t)^2}dt
\]
\end{tcolorbox}
Sia $\gamma$ l'unione di due curve rettificabili, allora $\gamma$ è rettificabile, la stessa proprietà si estende a un numero finito qualsiasi di curve rettificabili. \newline
\newline
Se una curva $\gamma$ è regolare a tratti, allora è rettificabile.\newline
\rule{\textwidth}{0,4pt}
\subsubsection{Cambiamenti di parametrizzazione, curve equivalenti}
Passando da $\vec{r} = \vec{r} (t)$ a $\vec{r} = \vec{r} (f(u))$ diremo che abbiamo riparametrizzato la curva, con $f(u)$ monotona (crescente o decrescente).\newline
Notiamo che dopo una riparametrizzazione la lunghezza dell'arco di curva rimane la stessa, anche se il verso o la velocità di percorrenza cambiano.\newline
Per invertire il senso di percorrenza si può sostituire $t$ con $-t$.\newline
\rule{\textwidth}{0,4pt}
\subsubsection{Parametro arco o ascissa curvilinea}
Se invece di calcolare la lunghezza in $[a,b]$ la calcolassimo da $[t_0, t]$ otterremmo una funzione di $t$:
\begin{tcolorbox}
\[
    s(t) = \int_{t_0}^{t}|\vec{r}'(\tau)|d\tau
\]
Se si è in grado di calcolare esplicitamente tale funzione e poi di invertirla, esprimendo $t$ come funzione di $s$, è possibile riparametrizzare la curva in funzione del parametro $s$, detto parametro arco o ascissa curvilinea (ricordarsi di ricalcolare anche gli estremi dell'intervallo secondo il nuovo parametro).\newline
\end{tcolorbox}
\begin{tcolorbox}
Notiamo che se $|\vec{r}'(t)| = 1$, $t$ coincide con il parametro arco, quindi la curva sarebbe già parametrizzata secondo  il parametro arco.
\end{tcolorbox}
Se $\vec{r} = \vec{r}(s)$ è un curva parametrizzata mediante il parametro arco, il vettore derivato $\vec{r}'(s)$ coincide col versore tangente $\vec{T}$
\newline
\begin{tcolorbox}
Se $\vec{r}(t)$ è una curva parametrizzata rispetto a un parametro $t$ qualunque (non necessariamente il parametro arco), si ha:
\[
    \frac{ds}{dt} =|\vec{r}'(t)| = v(t)
\]
che si riscrive anche nella forma
\[
    ds =|\vec{r}'(t)|dt =v(t) dt
\] 
dove il simbolo $ds$ prende il nome di lunghezza d'arco elementare.
\end{tcolorbox}
\rule{\textwidth}{2pt}
\subsection{Inegrali di linea (di prima specie)}
Sia $\vec{r}: [a,b] \rightarrow \mathbb{R}$ un arco di curva regolare di sostegno $\gamma$ e sia $f$ una funzione a valori reali, definita in un sottoinsieme $A$ di $\mathbb{R}^n$ contenente $\gamma$, cioè $f:A\subset \mathbb{R}^n \rightarrow \mathbb{R}$ con $\gamma \subset A$.
\begin{tcolorbox}
Si dice integrale di linea (di prima specie) di $f$ lungo $\gamma$ l'integrale
\[
    \int_{\gamma} f ds = \int_{a}^{b}f(\vec{r}(t))|\vec{r}'(t)|dt
\]
\end{tcolorbox}
\begin{tcolorbox}
Lintegrale di $f$ di prima specie lungo $\gamma$ è invariante per parametrizzazioni equivalenti e anche per cambiamento di orientazione.
\end{tcolorbox}
\textbf{applicazioni fisiche}
\begin{tcolorbox}
\begin{itemize}
    \item Il \textbf{baricentro} di $\gamma$ è il punto $B = (\bar{x},\bar{y},\bar{z})$ con:
    \[
        \begin{cases}
            \bar{x} = \frac{1}{m}\int_\gamma x\rho ds = \frac{1}{m}\int_{a}^{b}x(t)\rho(\vec{r}(t))|\vec{r}'(t)|dt\\
            \bar{y} = \frac{1}{m}\int_\gamma y\rho ds = \frac{1}{m}\int_{a}^{b}y(t)\rho(\vec{r}(t))|\vec{r}'(t)|dt\\
            \bar{z} = \frac{1}{m}\int_\gamma z\rho ds = \frac{1}{m}\int_{a}^{b}z(t)\rho(\vec{r}(t))|\vec{r}'(t)|dt
        \end{cases}
    \]
    Se il corpo è omogeneo ($\rho =$ costante), il baricentro si dice centroide e ha coordinate:
    \[
        \begin{cases}
            \bar{x} = \frac{\rho}{m}\int_\gamma x ds = \frac{1}{l(\gamma)}\int_{a}^{b}x(t)|\vec{r}'(t)|dt = \frac{\int_{a}^{b}x(t)|\vec{r}'(t)|dt}{\int_{a}^{b}|\vec{r}'(t)|dt}\\
            \bar{y} = \;\text{stesso di sopra}\\   
            \bar{z} = \;\text{stesso di sopra}
        \end{cases}
    \]
    \item \textbf{momento di inerzia} di $\gamma$ rispetto a un asse fissato. Se $\delta(x,y,z)$ indica la distanza del punto $(x,y,z)$ da quest'asse fissato, si ha:
    \[
        I = \int_\gamma \delta^2 \rho ds = \int_{a}^{b}\delta^2(\vec{r}(t))\rho(\vec{r}(t))|\vec{r}'(t)|dt
    \]
    Se il corpo è omogeneo:
    \[
        I = \frac{m}{l(\gamma)}\int_\gamma \delta^2ds = m \frac{\int_{a}^{b}\delta^2(\vec{r}(t))|\vec{r}'(t)|dt}{\int_{a}^{b}|\vec{r}'(t)|dt}
    \]
    Può essere molto utile cercare di ridefinire la curva spostandola in modo tale che l'asse di riferimento sia uno degli assi cartesiani, in questo modo esprimere la distanza $\delta$ di ogni punto dall'asse può essere più facile.
\end{itemize}
\end{tcolorbox}
\rule{\textwidth}{2pt}
\subsection{Elementi di geometria differenziale delle curve}
\rule{\textwidth}{0,4pt}
\subsubsection{Curvatura e normale principale per una curva in $\mathbb{R}^n$}
Il versore tangente è definito come:
\[
    \vec{T}(t) = \frac{\vec{r}'(t)}{|\vec{r}'(t)|}\text{,} \quad \text{mentre} \;\vec{T}(s) =\vec{r}'(s)
\]
in quanto $|\vec{r}'(s)| = 1$, se $s$ è il parametro arco.\newline
Sia $\vec{u}: I \rightarrow \mathbb{R}^n$, una funzione vettoriale di modulo costante, ossia $|\vec{u}(t)| = c$ per ogni $t$. Allora $\vec{u} \cdot \vec{u}' = 0$, cioè $\vec{u}$ è ortogonale a $\vec{u}'$ in ogni istante.\newline
\newline
\textbf{Normale principale e curvatura, rispetto al parametro arco}\newline
Chiamiamo versone normale principale della curva $\vec{r}(s)$ (parametrizzata rispetto al parametro arco), il versore
\[
    \vec{N}(s) =\frac{\vec{T}'(s)}{|\vec{T}'(s)|}
\]
definito nei punti in cui $\vec{T}'(s) \neq 0$.\newline
Chiamiamo curvatura della curva la funzione scalare
\[
    k(s) = |\vec{T}'(s)|
\]
da cui ricaviamo che $\vec{T}'(s) = k(s) \vec{N}(s)$. \newline
La normale principale ci dice la direzione verso cui sta curvando la nostra curva, mentre la curvatura è uno scalare e rappresenta l'intensità con cui curva.\newline
Definiamo invece come raggio di curvatura
\[
    \rho(s) = \frac{1}{k(s)}
\]
nei punti in cui $k(s) \neq 0$, nei punti in cui invece $k(s) = 0$ si dice che il raggio di curvatura è infinito.\newline
\newline
\newline
\textbf{Normale principale e curvatura, rispetto a un parametro qualsiasi}\newline
Versone normale principale:
\[
    \vec{N}(t) =\frac{\vec{T}'(t)}{|\vec{T}'(t)}
\]
e definendo $\frac{ds}{dt} = |\vec{r}'(t)| = v(t)$ da cui ricaviamo che la curvatura è:
\[
    k(t) = \frac{|\vec{T}'(t)|}{v(t)}
\]
da cui $\vec{T}'(t) = k(t)v(t)\vec{N}(t)$.\newline
Raggio di curvatura:
\[
    \rho(t)= \frac{1}{k(t)}
\]
\newline
Vale la seguente formula:
\[
    \vec{a}(t) = v'(t)\vec{T}(t) + v^2(t) k(t) \vec{N}(t)
\]
dove $\vec{a}(t) = \vec{r}''(t)$\newline
\newline
\rule{\textwidth}{0,4pt}
\subsubsection{Calcolo della curvatura per curve nello spazio $\mathbb{R}^3$ o nel piano}
\[
    k(t) = \frac{|\vec{r}'(t) \text{x} \vec{r}''(t)|}{v^3(t)}
\]
\[
    k(s) = |\vec{r}'(s) \text{x} \vec{r}''(s)|
\]
\newline
\textbf{caso di curve piane}\newline
Sia $\vec{r}(t) =(x(t), y(t))$ per $t \in[a,b]$, possiamo vederla come curva in $\mathbb{R}^3$ così: $\vec{r}(t) = (x(t), y(t), 0)$ per $t \in[a,b]$.\newline
Perciò,
\[
    \vec{r}' \text{x} \vec{r}'' =\left|\begin{matrix}
        i \;\;& j \;\;& k \\
        x'(t) & y'(t) & 0\\
        x''(t) & y''(t) & 0
    \end{matrix} \right|
\]
da cui derivano:
\[
    k(t) = \frac{|x'(t)y''(t) - x''(t) y'(t)|}{(x'(t)^2 + y'(t)^2)^{\frac{3}{3}}}
\]
\[
    k(s) = |x'(s) y''(s) - x''(s) y'(s)|
\]
\newline
\newline
Si dicono vertici di una curva i punti in cui $k'(t) = 0$.\newline
\rule{\textwidth}{0,4pt}
\subsubsection{Torsione(manca) e terna intrinseca per curve nello spazio $\mathbb{R}^3$}
Data una curva $\vec{r} : I \rightarrow \mathbb{R}^3$, regolare di classe $C^3(I)$, definiamo il versore binormale della curva come
\[
    \vec{B} = \vec{T} \text{x} \vec{N}
\]
La terna intrinseca $\vec{B}, \vec{T}, \vec{N}$ costituisce un sistema di riferimento ortonormale.\newline
\rule{\textwidth}{2pt}
\subsection{Note sugli esercizi}
\begin{tcolorbox}
Proprietà delle curve:
\begin{itemize}
    \item continua: se le componenti sono continue.
    \item chiusa: se agli estremi dell'intervallo in cui son definite la curva ha lo stesso valore (se fosse definita su tutto $\mathbb{R}$ si controllano i limiti all'infinito).
    \item asintoti: se i limiti all'infinito hanno un valore finito per una delle due componenti.
    \item semplice: se la curva non passa mai due volte per lo stesso punto. Si verifica per logica, spesso è utile notare se almeno una delle due componenti è strettamente monotona (crescente o decrescente). Spesso per funzioni trigonometriche si cerca di ragionare sulla loro periodicità.
    \item regolare: Si calcola la derivata della curva (derivata delle componenti) e in seguito il modulo della derivata (radice della somma delle componenti alla seconda). Se le derivate delle componenti sono funzioni continue e il modulo della derivata non si annulla mai per i valori di $t$ per cui la curva è definita (estremi esclusi, sempre) allora la curva è regolare. (Ci sono metodi particolari per calcolare il modulo della derivata, per esempio, per funzioni in forma polare il modulo della derivata è: $\rho = f(\theta) \; \Rightarrow \; |\vec{r}'(\theta)| = \sqrt{f'(\theta)^2 + f(\theta)^2}$; ma, inoltre, non bisogna scordarsi di calcolare le derivate delle componenti per controllare che siano continue).\newline
    Infine per individuare i punti di singolarità (cioè quelli in cui la funzione non è regolare) si calcolano i punti in cui $\vec{r}(t) = (x,y,z)$ per cui $t$ annulla la derivata.
    \item piana: \newline
    Primo metodo: si controlla se il versore binormale alla curva è costante. Il vettore binormale della caruva $\gamma(x(t), y(t), z(t))$ si calcola come prodotto vettoriale di $\gamma'(t) \text{x} \gamma''(t)$, una volta normalizzato (dividendo ogni componente per il modulo (componenti al quadrato sommate sotto radice)) si ottiene il versore e si controlla se è costante (curva piana) o non costante (curva non piana) e dunque basta assicurarsi che non sia presente la variabile $t$ (p.s. $\gamma'(t) = (x'(t), y'(t), z'(t)$, $\gamma''(t) = (x'(t), y'(t), z'(t))$ ). Concettualmente, se i calcoli sono troppo complicati, si può ragionare sul fatto che se il vettore binormale (che è sempre ortonormale alla curva) ha sempre la stessa direzione (anche se di modulo diverso) allora la curva è piana.\newline
    Secondo metodo: sostiuiamo le equazioni parametriche della curva $\gamma(x(t), y(t), z(t))$ nella formula del piano generico $ax + by +cz + d = 0$. Se risulta che l'equazione è sempre soddisfatta, allora la curva è complanare.
\end{itemize}
\end{tcolorbox}
\rule{\textwidth}{0,4pt}
Equazione di una retta passante per due punt:\newline
Dati due punti $(x_1, x_2)$ e $(y_1, y_2)$ la retta passante per questi punti è definita dall'equazione
\[
    \frac{x-x_1}{x_2-x_1} = \frac{y-y_1}{y_2-y_1}
\]
Questa formula vale se i due punti non sono allineati verticalmente o orizzontalmente.\newline
\rule{\textwidth}{0,4pt}\newline
Negli esercizi sul calcolo delle lunghezze di archi di curve e di parametrizzazione secondo il parametro arco è risultato molto utile l'intergale:
\[
    \int_{0}^{2\pi} \sqrt{1 + cos(t)}dt = \left[\text{moltiplicando per} \; \frac{\sqrt{1-cos(t)}}{\sqrt{1-cos(t)}}\right] = \int_{0}^{2\pi}\frac{|sin(t)|}{\sqrt{1-cos(t)}}dt =
\]
\[
    = [2 \sqrt{cos(t) +1}\cdot  tan(\frac{t}{2}) ]_0^{2\pi} = 4 \sqrt{2}
\]
\rule{\textwidth}{0,4pt}
Equazione della retta tangente a una curva $r(t) = (x(t), y(t))$ in un punto $t_0$:\newline
La generica retta ha forma $y = mx +q$. \newline
Il coefficiente $m$ si trova facendo il rapporto $\frac{y'(t_0)}{x'(t_0)}$.\newline
Il valore di $q$ si ricava sostituendo nell'equazione $y = mx +q$ la $m$ trovata, la $y$ con il valore di $y(t_0)$ e la $x$ con il valore di $x(t_0)$.\newline
\rule{\textwidth}{2pt} % curve
    %- finito
    \newpage
    \section*{Calcolo differenziale per funzioni reali di più variabili}
\rule{\textwidth}{2pt}
\subsection*{Continuità di una funzione in più variabili:}
Una funzione $f \;\;:\;\; \mathbb{R}^n \rightarrow  \mathbb{R}$ è continua in un punto $x_0$ se 
\[
    \lim_{x\rightarrow x_0} f(x) = f(x_0)
\]
La continuità di una funzione è anche deducibile dal fatto che sia costituita (somma/ prodotto/ quoziente/ certe volte anche composizione) da funzioni elementari continue.\newline
\rule{\textwidth}{2pt}
\subsection*{Calcolo di limiti in più variabili}
\rule{\textwidth}{0.4pt}
\subsubsection*{Non esistenza del limite}
Per mostrare che un certa funzione in più varibili non ammette limite in un determinato punto, è sufficiente determinare duee curve passasnti per il punto lungo le quali la funzione assume limiti diversi.
\textbf{es.} 
\[
    \lim_{(x,y)\rightarrow (0,0)} \frac{xy}{x^2+y^2}
\]
Analiziamo la funzione lungo due curve:
\begin{itemize}
    \item con $y=x$ ottengo $f(x,x) = \frac{1}{2}$
    \item con $y=-x$ ottengo $f(x,-x) = - \frac{1}{2}$
\end{itemize}
non ammette limite.\newline
\rule{\textwidth}{0.4pt}
\subsubsection*{Uso di maggiorazioni con funzioni radiali per provare l'esistenza del limite}
Per dimostrare l'esistenza di un limite per $(x,y) \rightarrow (0,0)$, si impone $x=\rho \cdot  cos(\theta)$ e $y= \rho \cdot sin(\theta)$, successivamente si l'intera funzione sotto modulo e procede con semplificazioni e maggiorazioni (per eliminare i seni e i coseni). E' essenziale che la funzione non dipenda da $\theta$.\newline
Più in generale se si volesse calcolare il limite per $(x,y) \rightarrow (x_0, y_0)$ si pongono $x=x_0 +\rho \cdot  cos(\theta)$ e $y= y_0 + \rho \cdot sin(\theta)$ \newline
\rule{\textwidth}{0.4pt}
\subsubsection*{Note sugli esercizi}
\begin{itemize}
    \item Se il limite non presenta una forma di indeterminazione allora il valore cercato si ricava sostituendo direttamente il punto nella funzione.
    \item Tecniche standard della maggiorazione: 
        \begin{itemize}
            \item disuguaglianza triangolare:
                \[
                    |a+b| \leq |a| + |b|
                \]
            \item maggiorazione di frazioni, con $a,b,c \geq 0$:
                \[
                    \frac{a}{b+c} \leq \frac{a}{b}
                \]
            \item maggiorazione di funzioni trigonometriche:
                \[
                    |cos(\theta)| \leq 1 \;\;,\;\;|sin(\theta)|\leq 1
                \]
        \end{itemize}
    \item Il criterio che ci permette di trovare il limite richiede di trovare una funzione maggiorante di $|f|$ che sia radiale (dipenda solo da $\rho$, non $\theta$) e infinitesima. Da notare è che è anche possibile semplificare la funzione anche senza passare subito in coordinate polari.
    \item Solitamente si suddivide la funzione in una serie di somme di funzioni e si studiano quest'ultime separatamente.
    \item Spesso la formula
        \[
            cos^2(\theta) + sin^2(\theta) = 1
        \]
        risulta molto utile.
\end{itemize}
\rule{\textwidth}{2pt}
\subsection*{Topologia in $\mathbb{R}^n$ e proprietà delle funzioni continue}
Sia $E$ un sottoinsieme di $\mathbb{R}^n$, un punto $x_0$ si dice:
\begin{itemize}
    \item interno ad $E$, se esiste un intorno centrato in $x_0$ contenuto in $E$;
    \item esterno ad $E$, se esiste un intorno centrato in $x_0$ contenuto in $E^c$;
    \item di frontiera per $E$, se ogni intorno centrato in $x_0$ contiene almeno un punto di $E$ e uno di $E^c$.
\end{itemize}
Un insieme $E \subseteq \mathbb{R}^n$ si dice:
\begin{itemize}
    \item aperto, se ogni suo punto è interno a $E$;
    \item chiuso, se il suo complementare è aperto.
\end{itemize} 
Sia $E$ un sottoinsieme di $\mathbb{R}^n$, si dice:
\begin{itemize}
    \item interno di $E$, e si indica con $E^o$, l'insieme dei punti interni di $E$;
    \item frontiera o brodo di $E$, e si indica con $\delta E$, l'insieme dei punti di frontiera di $E$;
    \item chiusura di $E$, e si indica con $\bar{E}$, l'insieme $E \cup \delta E$.
\end{itemize}
Alcune informazioni extra: 
\begin{itemize}
    \item si ha sempre $E^o \subseteq \delta E \subseteq \bar{E}$;
    \item il complementare di un aperto è chiuso e viceversa; 
    \item esistono insiemi nè aperti nè chiusi, gli unici insiemi sia aperti sia chiusi sono quello vuoto e $\mathbb{R}^n$;
    \item l'unione di una famiglia qualsiasi (anche infinita) di insiemi aperti e l'intersezione di un numero finito di insiemi aperti sono insiemi aperti 
    \item l'intersezione di una famiglia qualsiasi (anche infinita) di insiemi chiusi è l'unione di un numero finito di insiemi chiusi sono insiemi chiusi;
    \item un insieme aperto non contiene nessuno dei suoi punti di frontiera, un insieme chiuso contiene tutti i suoi punti di frontiera.
\end{itemize}
Un inieme si dice:
\begin{itemize}
    \item limitato se esiste un intorno che lo contiene tutto;
    \item connesso se per ogni coppia di punti dell'insieme, esiste un arco continuo che che li connette contenuto nell'insieme.
\end{itemize}
Proprietà topologiche delle funzioni continue:
\begin{itemize}
    \item \textbf{teor.} Teorema di Weierstrass. Sia $E \subset \mathbb{R}^n$ un insieme chiuso e limitato e $f \;\;:\;\; E \rightarrow \mathbb{R}$ sia continua, allora $f$ ammette massimo e minimo in $E$, cioè esistono $x_m$ e $x_M$ tali che $f(x_m) \leq f(x) \leq f(x_M)$ per ogni $x in E$.
    \item \textbf{teor.} Teorema degli zeri. Sia $E$ un insieme connesso di $\mathbb{R}^n$ e $f \;\;:\;\; E \rightarrow \mathbb{R}$ sia continua. Se $x$, $y$ sono due punti di $E$ tali che $ f(x) < 0$ e $f(y) > 0$, allora esiste un terzo punto $z \in E$ in cui $f$ si annulla. In particolare, lungo ogni arco di curva continua contenuto in $E$ che congiunge $x$ e $y$, c'è almeno un punto in cui $f$ si annulla.
\end{itemize}
\rule{\textwidth}{2pt}
\subsection*{Derivate parziali, piano tangente, differenziale}
\rule{\textwidth}{0.4pt}
\subsubsection*{Derivata parziale}
Calcolo di una derivata parziale tramite la definizione di rapporto incrementale in un punto $(x_0, y_0)$. \newline
Per prima fissiamo $y=y_0$ e deriviamo rispetto alla $x$:
\[
    \frac{\delta f}{\delta x} (x_0,y_0) = \lim_{h\rightarrow 0}\frac{f(x_0+h, y_0)- f(x_0,y_0)}{h}.
\]
Successivamente facciamo l'opposto, cioè fissiamo $x=x_0$ e deriviamo rispetto alla $y$:
\[
    \frac{\delta f}{\delta y} (x_0,y_0) = \lim_{k\rightarrow 0}\frac{f(x_0, y_0+k)- f(x_0,y_0)}{k}
\]
Una funzione $f: A \subseteq \mathbb{R}^n \rightarrow  \mathbb{R}$ si dice derivabile in un punto del suo dominio se in quel punto esistono tutte le sue derivate parziali; si dice derivabile in $A$ se è derivabile in ogni punto di $A$.\newline
Se $f$ è derivabile in un punto, chiameremo gradiente ( $\nabla f(x)$ ) il vettore delle sue derivate parziali.\newline
\rule{\textwidth}{0.4pt}
\subsubsection*{Piano tangente}
Costruire il piano tangente a una funzione in due variabili in un punto $(x_0, y_0)$:
\begin{enumerate}
    \item troviamo la retta tangente alla funzione nel piano $y=y_0$:
    \[
        \begin{cases}
            &z = f(x_0, y_0) + \frac{\delta f}{\delta x}(x_0, y_0) \cdot (x-x_0)\\
            &y=y_0 \\
        \end{cases}
    \]
    \item troviamo la retta tangente alla funzione nel piano $x=x_0$:
    \[
        \begin{cases}
            &z = f(x_0, y_0) + \frac{\delta f}{\delta x}(x_0, y_0) \cdot (y-y_0)\\
            &x=x_0 \\
        \end{cases}
    \]
    \item costruiamo il piano che contiene entrambe le rette:
    \[
        z = f(x_0, y_0) + \frac{\delta f}{\delta x}(x_0, y_0) \cdot (x-x_0) + \frac{\delta f}{\delta x}(x_0, y_0) \cdot (y-y_0)
    \]
\end{enumerate}
Il procedimento appena mostrato individua il piano tangente nell'ipotesi che esso esista, potrebbe però non esserci.\newline
\rule{\textwidth}{0.4pt}
\subsubsection*{Differenziabilità e approssimazione lineare}
In due o più variabili la sola derivabilità non implica nè continuità nè l'esistenza del piano tangente.\newline
Concetto di differenziabilità in più variabili: l'incremento di $f$ è uguale all'incremento calcolato lungo il piano tangente, più un infinitesimo di ordine superiore rispetto alla lunghezza dell'incremento $(h,k)$ delle variabili indipendenti. In formule:
\[
    f(x_0 + h, y_0 + k) - f(x_0, y_0) = \frac{\delta f}{\delta x} (x_0,y_0) \cdot (x-x_0) + \frac{\delta f}{\delta y}(x_0, y_0) \cdot (y-y_0) + o(\sqrt{h^2 + k^2})
\]
per $(h,k) \rightarrow  (0,0)$.\newline
Tutto ciò che è prima dell'uguale (primo membro) rappresenta l'incremento della funzione, i primi due addendi del secondo membro rappresentano l'incremento calacolato lungo il pinao tangente. Ricordiamo che l'ultimo addendo rappresenta una funzione tale che $\lim_{(h,k)\rightarrow (0,0)}\frac{o(\sqrt{h^2 + k^2})}{\sqrt{h^2 + k^2}} = 0$.\newline
Se l'equazione di prima è soddisfatta, diremo che la funzione è differenziabile in $(x_0, y_0)$.\newline

Da notare che la differenziabilità implica la derivabilità, cioè se una funzione è differenziabile in un punto, allora è anche derivabile nello stesso.\newline

Se $f$ è differenziabile in $x_0$, si dice differenziale di $f$ calcolato in $x_0$ la funzione lineare $df(x_0) \;:\; \mathbb{R}^n \rightarrow  \mathbb{R}$ definita da:
\[
    df(x_0) \;\;:\;\;h \rightarrow  \nabla f(x_0) \cdot h.
\] 
Nel caso in due varibiali, il numero $\nabla f(x_0) \cdot h$ rappresenta l'incremento della funzione nel passare da $x_0$ a $x_0+h$, calcolato lungo il piano tangente al grafico di $f$ in $x_0$.\newline

L'approssimazione dell'incremento di $f$ con il suo differenziale prende il nome di linearizzazione.\newline
\rule{\textwidth}{0.4pt}
\subsubsection*{Verifica della differenziabilità}
Per dimostrare la differenziabilità in un punto $(x_0,y_0)$ bisogna provare che:
\[
    \lim_{h,k\rightarrow 0,0}\frac{f(x_0+h, y_0+k)-\{
        f(x_0,y_0)+ \frac{\delta f}{\delta x}(x_0,y_0) h + \frac{\delta f}{\delta y}(x_0,y_0)k
        \}}{\sqrt{h^2+k^2}} =0.
\]
dove $h = x-x_0$ e $k = y-y_0$.\newline
Ma per certi casi peticolari esistono criteri molto comodi e più semplici.\newline

Teorema di condizione sufficiente di differenziabilità: se le derivate parziali di $f$ esistono in un intorno di $x_0$ e sono continue in $x_0$, allora $f$ è differenziabile in $x_0$.\newline
In particolare se le derivate parziali esistono e sono continue in tutto $A$, allora $f$ è differenziabile in tutto $A$.\newline
Una funzione le cui derivate parziali esistono e sono continue in tutto $A$ si dice di classe $C^1(A)$, dunque: $f \in C^1(A) \rightarrow$ f differenziabile in $A$. \newline

Negli esercizi spesso si usa anche l'omogeneità di una funzione per sapere se essa è differenziabile o continua, oppure le proprietà delle funzioni radiali.\newline

Negli esercizi seguire quest'ordine:
\begin{itemize}
    \item E' continua nel punto richiesto? se non lo è, può essere allungata?
    \item Funzione radiale? (vedi più avanti)
    \item Funzione omogenea? (vedi più avanti)
    \item Calcolo delle derivate parziali nel punto. Sono continue in quel punto?
    \item Verifica della differenziabilità tramite la definizione.
\end{itemize}
\rule{\textwidth}{0.4pt}
\subsubsection*{Derivate direzionali}
Si dice derivata direzionale della funzione $f$ rispetto al versore $v$, nel punto $x_0$, il limite
\[
    D_vf(x_0) = \lim_{t\rightarrow 0}\frac{f(x_0 + tv) - f(x_0)}{t}
\]
purchè esista finito.\newline
Detto in maniera diversa signifca considerare la restrizione della funzione $f$ alla direzione della retta passante per $x_0$ con direzione $v$, cioè $g(t)= f(x_0+tv)$, e calcolarne la derivata, cioè $D_vf(x_0)= g'(0)$.\newline

Calcolo di una derivata direzionale per un generico vettore $(cos(\theta), sin(\theta))$ nell'origine di una funzione $f$: Per prima cosa si ottiene la funzione $g(t) = f(t \cdot cos(\theta), t \cdot sin(\theta))$ e la si semplifica per $t \rightarrow 0$ (anche usando asintotici). In seguito si studia la derivata $g'(0) = \frac{\delta f}{\delta t}(t \cdot cos(\theta), t \cdot sin(\theta))$.\newline
Se è richiesto il calcolo in un punto generico, e non nell'origine, è sufficiente usare $t \cdot cos(\theta) + x_0$ e $t \cdot sin(\theta) + y_0$.\newline

Formula del gradiente: $D_vf(x_0) = \nabla f(x_0) \cdot v =  \sum_{i=1}^{n} \frac{\delta f}{\delta x_i}(x_0) \cdot  v_i$. Cioè la derivata direzionale è il prodotto scalare del gradiente con il versore nella direzione in cui si deriva, quindi tutte le derivate direzionali sono combinazioni lineari delle derivate parziali. Nel caso in due variabili la formula si riduce a $D_vf(x_0) = \nabla f(x_0,y_0) \cdot v = \frac{\delta f}{\delta x} (x_0,y_0)cos(\theta) + \frac{\delta f}{\delta y}(x_0, y_0) sin(\theta)$.\newline
Se la formula del gradiente non vale in un punto, allora la funzione non è differenziabile in quel punto.\newline
Inoltre la formula del gradiente non vale se la generica derivata direzionale non è combinazione lineare di $cos(\theta), sin(\theta)$.\newline

Da notare è che $\nabla f(x_0)$ indica la direzione di massima crescita di $f$, ossia la direzione di massima derivata direzionale, invece $-\nabla f(x_0)$ rappresenta la direzione di minima derivata direzionale, infine nelle direzioni ortogonali al gradiente le derivate direzionali sono nulle. \newline
\rule{\textwidth}{0.4pt}
\subsubsection*{Riepilogo}
\begin{itemize}
    \item $f \in C^1(A) \Rightarrow f$ differenziabile in $A$ (cioè $f$ ha iperpiano tangente) $\Rightarrow f$ è continua, derivabile, ha derivate direzionali, vale la formula del gradiente.
    \item $f$ continua, derivabile, dotata di tutte le derivate direzionali $\nRightarrow f$ differenziabile
    \item $f$ derivabile, dotata di tutte le derivate direzionali $\nRightarrow f$ continua 
\end{itemize}
\rule{\textwidth}{0.4pt}
\subsubsection*{Calcolo delle derivate}
\[
    \delta(\alpha \cdot f + \beta \cdot g) = \alpha \delta(f) + \beta \delta (g)
\]
\[
    \delta(f \cdot  g) = g \cdot \delta(f) + f  \cdot \delta(g)
\]
\[
    \delta(\frac{f}{g}) = \frac{ g \cdot \delta(f) - f \delta(g)}{g^2}
\]
\[
    h(x) = f(g(x)) = g \circ f \Rightarrow h'(x) = f'(g(x)) g'(x)
\]
\[
    \frac{\delta}{\delta x} [|x|] = \frac{|x|}{x} 
\]

Per calcolare il valore di una derivata parziale in punto $(x_0, y_0)$ secondo la definizione, seguire questo procedimento. Se si richiede di calcolare il valore della derivata parziale di $x$ ( cioè $\frac{\delta f}{\delta x}(x_0,y_0)$), si parte dalla funzione $f(x,y)$ e si sostituisce $y= y_0$, ottenendo quindi $f(x,y_0)$, successivamente si calcola la derivata parziale, ottenendo dunque $\frac{\delta f}{\delta x} (x,y_0)$. Come ultima cosa si sostituisce $x = x_0$ e si arriva a un risultato numerico. Per la trovare il valore della derivata parziale di $y$ in un preciso punto seguire lo stesso procedimento opposto.\newline

In alcuni esercizi è richiesto di calcolare le derivate parziali in tutti i punti in cui esistono. Il procedimento tipico consiste nel calcolare per prima cosa le derivate parziali generiche. Una volta calcolare sapremo che sicuramente esistono dove queste sono definite (dominio), ma non siamo sicuri dei punti in cui non lo sono (al di fuori del dominio). Quindi dobbiamo analizzare singolarmente tutti i punti al di fuori del dominio e per farlo sfruttiamo il procedimento visto sopra, calcolando esplicitamente le derivate nei punti richiesti. Finchè si tratta per esempio di calcolarle per un punto preciso non ci sono problemi, il calcolo è facile, ma ci sono alcuni casi difficili, per esempio:
\begin{itemize}
    \item Calcolare le derivate parziali secondo la definizione lungo una retta. Per esempio in $y=0$, per calcolare la $\frac{\delta f}{\delta x}(x_0, 0)$ non ci sono problemi, si procede come al solito. Ma per la $\frac{\delta f}{\delta y} (x_0, 0)$ ci sono difficoltà, siccome non possiamo sostituire le $y$ con $0$ e poi derivare per la $y$, dobbiamo ragionare così: la derivata non esiste a meno che non ci sia un valore che le $x$ possono assumere che annullino la funzione (per gli es che ho fatto fino ad ora sono solo al numeratore). Il concetto generale è che se non si trovano valori per $x_0$ tali che annullino la funzione e quindi ci permettano di calcolare la derivata parziale, si finisce per tornare a guardare la derivata parziale generica e quindi a non trovarla per quella retta. [spiegato davvero male, ma è un concetto strano].
\end{itemize}

Per stabilire dove la funzione sia derivabile bisogna calcolare le derivate parziali e osservarne il dominio.\newline
(n.b. tipicamente negli esercizi le funzioni sono descritte da un sistema che contiene una funzione prolungata nell'origine, in questo caso bisogna calcolare le derivate parziali al di fuori dell'origine e studiarne il dominio, in seguito bisogna calcolare il valore della derivata parziale nel punto $(0,0)$ col metodo descritto precedentemente).\newline
\rule{\textwidth}{0.4pt}
\subsubsection*{Gradiente di una funzione radiale}
Si chiama funzione radiale una funzione $h$ che dipende solo dalla distanza di dall'origine, ossia
\[
    h(x) = g(|x|).
\]
ponendo $\rho = |x| = \sqrt{\sum_{j=1}^{n}x_j^2}$ si ha:
\[
    \nabla_\rho = ( \frac{x_1}{\rho}, \frac{x_2}{\rho}, \dots, \frac{x_n}{\rho}).
\]
\[
    \nabla h(x) = g'(|x|)( \frac{x_1}{|x|}, \dots, \frac{x_n}{|x|})
\]
\[
    |\nabla h(x)| = |g'(|x|)|
\]
Le funzioni radiali sono spesso utilizzate negli esercizi in cui le incognite compaiono solo all'interno del termine $\sqrt{\sum_{j=1}^{n}x_j^2}$, in tal caso si ottiene $g(\rho)$ sostituendo ogni $\sqrt{\sum_{j=1}^{n}x_j^2}$ con $ \rho$, successivamente si può procedere sfruttando le proprietà di continuità e differenziabilità delle funzioni radiali.\newline
\rule{\textwidth}{0.4pt}
\subsubsection*{Criterio di continuità e differenziabilità per funzioni radiali}
Sia $f \;\;:\;\; \mathbb{R}^n-\{0\} \rightarrow  \mathbb{R}$ una funzione radiale, cioè $f(x) = g(|x|)$ con $g \;\;:\;\;(0, +\infty) \rightarrow \mathbb{R}$ e sia $f$ continua fuori dall'origine. Allora:
\begin{itemize}
    \item $f$ è continua in $0$, se e solo se esiste finito $\lim_{\rho\rightarrow 0^+}g(\rho)$;
    \item $f$ è differenziabile in $0$ se e solo se esiste $g'(0)=0$.
\end{itemize}
Negli esercizi spesso si controlla prima la continuità nell'origine, se non lo è si allunga la funzione e successivamente si calcola la differenziabilità nell'origine.\newline
\rule{\textwidth}{0.4pt}
\subsubsection*{Funzioni omogenee}
Una funzione $f \;\;:\;\; \mathbb{R}^n \rightarrow \mathbb{R}$ (eventualmente definita solo per $x\neq 0$), non identicamente nulla, si dice positivamente omogenea di grado $\alpha \in \mathbb{R}$ se
\[
    f(\lambda x) = \lambda^\alpha f(x) \;\;\;\;\; \;\forall\;x \in \mathbb{R}^n, x\neq 0, \lambda>0.
\]
La funzione $f$ si dice omogenea di grado $\alpha$ se la formula di prima vale anche per $\lambda<0$.\newline
Se $f$ è positivamente omogenea vale
\[
    f(x) = f( |x| \cdot \frac{x}{|x|}) = |x|^\alpha f(\frac{x}{|x|}).
\]
In particolare se $f$ è omogenea (o positivamente omogenea) di grado zero, significa che è costante su ogni retta (o semiretta) uscente dall'origine. Infatti, indicata con
\[
    r(t)=tv
\]
con $v$ versore fissato, sarà
\[
    f(r(t))=f(tv)=t^0f(v)=f(v)=costante.
\]
Più in generale, per una funzione in due variabili positivamente omogenea di grado $\alpha$ vale la seguente rappresentazione in coordinate polari:
\[
    f(\rho, \theta)=\rho^\alpha g(1,\theta)
\]
per qualche $\alpha \in \mathbb{R}$ e qualche funzione $g \;\;:\;\; [0, 2\pi) \rightarrow \mathbb{R}$.\newline

Sia $f \;\;:\;\; \mathbb{R}^n \rightarrow \mathbb{R}$ una funzione positivamente omogenea di grado $\alpha$, definita e continua per $x\neq 0$. Allora:
\begin{itemize}
    \item $f$ è continua anche nell'origine se $\alpha>0$; in questo caso $f(0) = 0$; $f$ è discontinua nell'origine se $\alpha<0$; è discontinua anche se $\alpha=0$, tranne il caso banale in cui $f$ è costante.
    \item $f$ è differenziabile nell'origine se $\alpha>1$; non è differenziabile nell'origine se $\alpha <1$, tranne il caso banale in cui $\alpha=0$ e $f$ è costante; se $\alpha=1$, $f$ è differenziabile se e solo se è una funzione lineare, (ossia $f(x) = a \cdot x$ per qualche vettore costante $a \in \mathbb{R}^n$).
\end{itemize}
\rule{\textwidth}{0.4pt}
\subsubsection*{Ortogonalità del gradiente con le curve di livello}
Il gradiente è ortogolane il ogni punto alle linee di livello. \newline
\rule{\textwidth}{0.4pt}
\subsubsection*{Equazione del trasporto}
Si definisce equazione del trasporto la seguente:
\[
    c \frac{\delta u}{\delta x} + \frac{\delta u}{\delta t} = 0 \;\;\;\;\; \;\;\;\;\; (?)
\]
\newline
Teorema del valor medio. Sia $A \subset \mathbb{R}^n$ un aperto e $f \;\;:\;\; A \rightarrow \mathbb{R}$ una funzione differenziabile in $A$. Allora per ogni coppia di punti $x_0, x_1 \in A$, esiste un punto $x^*$ tale per cui:
\[
    f(x_1)- f(x_0) = \nabla f(x^*) \cdot  (x_1 + x_0).
\]
In particolare:
\[
    |f(x_1) - f(x_0)| \leq |\nabla f(x^*)| \cdot |(x_1 + x_0)|.
\]
\rule{\textwidth}{2pt}

\subsection*{Derivate di ordine superiore e approssimazioni successive}
\rule{\textwidth}{0.4pt}
\subsubsection*{Derivate di ordine superiore}
Teorema di Schwartz. Sia $f \;\;:\;\; A \subseteq \mathbb{R}^n \rightarrow \mathbb{R}$ con $A$ aperto. Supponiamo che (per certi indici $i,j \in \{1,2,\dots,n\}$) le derivate seconde miste $f_{x_i, x_j}$ e $f_{x_j, x_i}$ esistano in un certo $x_0$ e siano continue in $x_0$; allora esse coincidono in $x_0$.\newline

Una funzione che ha tutte le derivate parziali seconde continue in un aperto $A$ si dice di classe $C^2(A)$.\newline
Se $f \in C^2(A)$, allora $f \in C^1(A)$ (in particolare $f$ è differenziabile), le derivate parziali prima sono differenziabili, le derivate parziali seconde sono continue, le derivate seconde miste sono uguali.
\rule{\textwidth}{2pt}
\subsection*{Differenziale secondo, matrice hessiana, formula di Taylor al secondo ordine}
Se $f \in C^2(A)$ e $x_0 \in A$, si dice differenziale secondo di $f$ in $x_0$ la funzione
\[
    d^2f(x_0) \;\;:\;\; h \rightarrow \sum_{i=1}^{n}\sum_{j=0}^{n}\frac{\delta^2(f)}{\delta(x_i) \delta(x_j)}(x_0)h_ih_j.
\]
I vari coefficienti $\frac{\delta^2(f)}{\delta(x_i) \delta(x_j)}(x_0)$ possono essere ordinati in una matrice detta Hessiana:
\[
    H_f(x_0) =\begin{matrix}
        f_{x_1x_1}(x_0) \;\; &f_{x_1x_2}(x_0) \;\; &\dots \;\; &f_{x_1x_n}(x_0)\\
        f_{x_2x_1}(x_0) \;\; &f_{x_2x_2}(x_0) \;\; &\dots \;\; &f_{x_2x_n}(x_0)\\
        \dots \;\; &\dots \;\; &\dots \;\; &\dots\\
        f_{x_nx_1}(x_0) \;\; &f_{x_nx_2}(x_0) \;\; &\dots \;\; &f_{x_nx_n}(x_0)
    \end{matrix}
\]  
In particolare, per due variabili:
\[
    H_f(x_0, y_0) = \begin{matrix}
        f_{xx}(x_0, y_0) \;\; f_{xy}(x_0, y_0)\\
        f_{yx}(x_0, y_0) \;\; f_{yy}(x_0, y_0)
    \end{matrix}
\]
Se $f$ è di classe $C^2$, la matrice Hessiana è simmetrica.\newline

Formula di Taylor (resto secondo Lagrange). Sia $f \in c^2(A)$; per ogni $x_0 \in A$ e $h \in \mathbb{R}^n$, tale che $x_0 + h \in A$, esiste un numero reale $\delta \in (0,1)$, dipendente da $x_0$ e $h$, tale che:
\[
    f(x_0 + h) = f(x_0) +\sum_{i=1}^{n}\frac{\delta f}{\delta x_i}(x_0)h_i + \frac{1}{2}\sum_{i,j=1}^{n}\frac{\delta^2f}{\delta x_i \delta x_j}(x_0 + \delta h) h_i h_j.
\]\newline

Formula di Taylor (resto secondo Peano). Sia $f \in C^2(a)$. Per ogni $x_0 \in A$ vale la formula:
\[
    f(x_0 + h) = f(x_0) +\sum_{i=1}^{n}\frac{\delta f}{\delta x_i}(x_0)h_i + \frac{1}{2}\sum_{i,j=1}^{n}\frac{\delta^2f}{\delta x_i \delta x_j}(x_0) h_i h_j +o(|h|^2).
\]
\rule{\textwidth}{2pt}
\subsection*{Ottimizzaione, estremi liberi}
\rule{\textwidth}{0.4pt}
\subsubsection*{Generalità sui problemi di ottimizzazione}
\begin{itemize}
    \item $x_0$ è detto punto di massimo (minimo) globale se per ogni $x$ si ha $f(x) \leq f(x_0)$ ($f(x) \geq f(x_0)$);
    \item $x_0$ è detto punto di massimo (minimo) locale se esiste un intorno di $x_0$ detto $U$ tale per cui per ogni $x \in U$ si ha $f(x) \leq f(x_0)$ ($f(x) \geq f(x_0)$).
\end{itemize}
\rule{\textwidth}{0.4pt}
\subsubsection*{Estremi liberi, condizioni necessarie del prim'ordine}
Teorema di Fermat. Sia $f \;\;:\;\; A \subseteq \mathbb{R}^n \rightarrow \mathbb{R}$, con $A$ aperto e $x_0 \in A$ un punto di massimo o minimo locale per $f$. Se $f$ è derivabile in $x_0$, allora $\nabla f(x_0) = 0$.\newline

I punti in cui il gradiente di una funzione si annulla si dicono punti critici o stazionari di $f$. Una volta individuati tutti i punti stazionari, si può iniziare un'analisi su di essi per verificare se sono o meno punti di massimo o minimo. Se non lo sono essi prendono il nome di punti di sella o colle. Da notare particolarmente è che una funzione può assumere valori di massimo o minimo anche in punti in cui non è derivabile, dunque questi punti vanno analizzati separatamente. \newline
\rule{\textwidth}{0.4pt}
\subsubsection*{Forme quadratiche, classificazione}
Un modo per determinare la natura di un punto stazionario è quello di analizzare il segno dell'incremento $\nabla f(x_0) = f(x_0 + h ) - f(x_0)$. Se infatti si riesce a stabilire che $\nabla f(x_0)$ si mantiene di segno positivo o negativo, per ogni $h$ di modulo abbastanza piccolo, possiamo dedurre che $x_0$ è punto di minimo o massimo locale. Se invece al variare di $h$, $\nabla f(x_0)$ cambia segno, siamo in presenza di un punto si sella.\newline
Lo studio del segno di $\nabla f(x_0)$ riconduce all'analisi del segno del polinomio omogeneo di secondo grado nelle componenti di $h$ (che prende il nome di forma quadratica) dato da
\[
    \sum_{i,j=1}^{n}f_{x_ix_j}(x_0)h_ih_j.
\]
Ogni forma quadratica risulta associata a una matrice simmetrica $M$. Nel caso del differenziale la matrice $M$ coincide con la matrice Hessiana.\newline
Il segno della forma quadratica è quindi studiabile analizzando la sua matrice $M = \begin{matrix}
    a \;\;\; b\\
    b \;\;\; c\\
\end{matrix}$ con $a\neq 0$ nel seguente modo:
\begin{itemize}
    \item è definitivamente positiva (negativa) se e solo se $det(M)>0$ e $a >0$ ($a<0$);
    \item indefinita se $det(M)<0$;
    \item semidefinita positiva (negativa) se e solo se $det(M)=0$ e $a>0$ ($a<0$).
\end{itemize}
Se $a = 0$ e $c\neq 0$, nelle affermazioni precedenti occore sostituire $a$ con $c$. \newline
\rule{\textwidth}{0.4pt}
\subsubsection*{Forme quadratiche, test degli autovalori}
Un importante test per determinare il segno di una funzione quadratica in $\mathbb{R}^n$ è basato sul segno degli autovalori della matrice $M$.\newline
Ricordiamo che un numero complesso $\lambda$ e un vettore non nullo $v \in \mathbb{C}^n$ si dicono, rispettivamente, autovalore e autovettore (di $\lambda$) di una matrice $M$ di ordine $n$, se soddisfano la relazione:
\[
    Mv = \lambda v
\]
oppure
\[
    (M-\lambda I_n)v = 0.
\]
Quest'ultima equazione ha soluzioni $v$ non nulle se e solo se la matrice dei coefficienti e singolare, ovvero se $\lambda$ è soluzione dell'equazione caratteristica:
\[
    det(M-\lambda I) = 0
\]
esistono esattamente $n$ autovalori di $M$ ciascuno contato secondo la propria molteplicità.\newline
Le matrici $M$ simmetriche hanno prorpietà importanti:
\begin{itemize}
    \item gli autovalori di $M$ sono reali e possiedono autovettori reali;
    \item esistono $n$ autovettori lineari che costituiscono una base ortonormale in $\mathbb{R}^n$;
    \item La matrice $S = {w_1, w_2, \dots, w_n}$ le cui colonne sono gli autovettori lineari è orotognale e diagonalizza $M$, precisamente:
    \[
        S^TMS = \Lambda = \begin{matrix}
            \lambda_1 \;\; &0 \;\; &\dots \;\; &0\\
            0 \;\; &\lambda_2 \;\; &\dots \;\; &0\\
            \dots \;\; &\dots \;\; &\dots \;\; &\dots\\
            0 \;\; &0 \;\; &\dots \;\; &\lambda_n
        \end{matrix}
    \]
\end{itemize}
Tornando allo studio del segno della forma quadratica con la sua matrice $M$:
\begin{itemize}
    \item definitivamente positiva (negativa) se e solo se tutti gli autovalori di $M$ sono positivi (negativi);
    \item semidefinita positiva (negativa) se e solo se tutti gli autovalori di $M$ sono $\geq 0$ ($\leq 0$) e almeno uno di essi è nullo;
    \item indefinita se $M$ ha almeno un autovalore positivo e uno negativo.
\end{itemize}
Da notare è che per una forma quadratica l'origine è sempre un punto stazionario. \newline
\rule{\textwidth}{0.4pt}
\subsubsection*{Studio della natura dei punti critici}
Per estrarre informazioni su un punto critico $x_0$ occorre studiare e classificare la forma quadratica 
\[
    q(h) = \sum_{i,j=1}^{n}f_{x_ix_j}(x_0)h_ih_j = h^TH_f(x_0)h
\]
dove $H_f(x_0)$ è la matrice Hessiana di $f$ in $x_0$.\newline
\begin{itemize}
    \item se la forma quadratica è definitivamente positiva (negativa), allora $x_0$ è un punto di minimo (massimo) locale forte;
    \item se la forma quadratica è indefinita, allora $x_0$ è un punto di sella;
    \item se la forma quadratica è in $x_0$ semidefinita positiva (negativa) e non nulla, allora $x_0$ è di minimo (massimo) debole oppure di colle; la situazione cambia se la forma quadratica è semidefinita positiva (negativa) non solo in $x_0$, ma anche per ogni $x$ in un intorno di $x_0$, f è convessa (concava), dunque un punto di minimo (massimo) debole;
    \item se la forma quadratica è nulla, allora non possiamo estrarne informazioni significanti.
\end{itemize}
Vediamo una strategia da seguire:
\begin{enumerate}
    \item si isolano i punti di $f$ che non sono regolari (es. non derivabili una o due volte). Questi punti dovranno essere analizzati separatamente;
    \item trovare i punti critici risolvendo:
        \[
            \begin{cases}
                &f_{x_1}(x_1,x_2,\dots,x_n) = 0\\
                &f_{x_2}(x_1,x_2,\dots,x_n) = 0\\
                &\dots\\
                &f_{x_n}(x_1,x_2,\dots,x_n) = 0\\
            \end{cases}
        \]
    \item si studia il segno della forma quadratica per ogni punto critico, se è definita o indefinita si giunge a una conclusione con le regole dette precedentemente, se è nulla o semidefinita si ricorre a uno studio diretto di $\nabla f(x_0)$ in un intorno di $x_0$.
\end{enumerate}
Più precisamente, nel caso bidimensionale, per ogni punto critico:
\begin{enumerate}
    \item si calcola l'Hessiana:
        \[
            H_f(x_0, y_0) = \begin{matrix}
                f_{xx}(x_0,y_0) \;\;\; & f_{xy}(x_0,y_0)\\
                f_yx(x_0,y_0) \;\;\; & f_{yy}(x_0,y_0)\\
            \end{matrix}
        \]
    \item se $det(H_f(x_0, y_0)) > 0$ e
        \begin{itemize}
            \item $f_{xx}(x_0,y_0)>0$ allora $(x_0, y_0)$ è di minimo locale forte;
            \item $f_{xx}(x_0,y_0)<0$ allora $(x_0, y_0)$ è di massimo locale forte;
        \end{itemize}
        (si noti che in questo caso $f_{xx}(x_0,y_0)$ e $f_{yy}(x_0,y_0)$ hanno lo stesso segno).
    \item se $det(H_f(x_0, y_0)) < 0$ allora $(x_0, y_0)$ è punto di sella;
    \item se $det(H_f(x_0, y_0)) = 0$ occore un analisi ulteriore. 
\end{enumerate}
\rule{\textwidth}{2pt}
\subsection*{Funzioni convesse di n variabili}
\rule{\textwidth}{0.4pt}
\subsubsection*{Generalità sulle funzioni convesse}
Un insieme $\Omega \subseteq \mathbb{R}^n$ si dice convesso se per ogni coppia di punti $x_1, x_2 \in \Omega$ si ha che $[x_1,x_2]\subseteq \Omega$ (dove col simbolo $[x_1,x_2]$ si denota il segmento con estremi $x_1,x_2$); si dice strettamente convesso se per ogni coppia di punti $x_1,x_2 \in \Omega$ il segmento $(x_1,x_2)$ privato degli estremi è strettamente contenuto in $\Omega$.\newline

Si dice epigrafico di una funzione $f \;\;:\;\; \Omega \subseteq \mathbb{R}^n \rightarrow  \mathbb{R}$ l'insieme
\[
    epi(f)=\{(x,z) \in \mathbb{R}^{n+1} \;\;:\;\; z\geq f(x), x \in\Omega\}
\]
\newline

Si dice che una funzione è convessa se $epi(f)$ è un sottoinsieme convesso, si dice che una funzione è concava se $-f$ è convessa.\newline
Formalmente si dice che una funzione è convessa se e solo se per ogni $x_1,x_2$, $t \in [0,1]$ vale la condizione
\[
    f(tx_2 + (1-t)x_1) \leq tf(x_2) + (1-t)f(x_1).
\]
Si noti che $tx_2 + (1-t)x_1$ percorre il segmento $[x_1,x_2]$ al variare di $t \in[0,1]$\newline

Se $f$ è convessa allora:
\begin{itemize}
    \item $f$ è continua;
    \item $f$ ha derivate parziali destre e sinistre in ogni punto;
    \item nei punti in cui è derivabile, $f$ è anche divverenziabile.
\end{itemize}

Teorema di convessità e piano tangente. Sia $f \;\;:\;\; \Omega \rightarrow \mathbb{R}$ differenziabile in $\Omega$. Allora $f$ è convessa in $\Omega$ se e solo se per ogni coppia di punti $x_0, x \in \Omega$ si ha:
\[
    f(x) \geq f(x_0) + \nabla f(x_0) \cdot (x-x_0).
\]
In due dimensioni:
\[
    f(x,y) \geq f(x_0,y_0) + \frac{\delta f}{\delta x}(x_0,y_0)(x-x_0)+\frac{\delta f}{\delta y}(x_0, y_0)(y-y_0).
\]
che geometricamente signifca che il paino tangente in $x-0, y_0$ sta sotto $f$.\newline

Teorema di convessità e matrice Hessiana. Sia $f \in C^2(\Omega)$, con $\Omega$ aperto convesso in $\mathbb{R}^n$. Se per ogni $x_0$ in $\Omega$ la forma quadratica $d^2f(x_0)$ è semidefinita positiva, allora $f$ è convessa in $\Omega$. \newline
\rule{\textwidth}{0.4pt}
\subsubsection*{Ottimizzazione di funzioni convesse e concave}
Nelle funzioni convesse (concave) i punti stazionari, se esistono, rappresentano minimi (massimi) globali. Inoltre se la funzione è strettamente convessa (concava), il punto critico è di minimo (massimo) globale forte, quindi, in particolare, è unico. \newline
\rule{\textwidth}{2pt}
\subsection*{Funzioni definite implicitamente}
\rule{\textwidth}{0.4pt}
\subsubsection*{Funzione implicita di una variabile}
Teorema di Dini della funzione implicita. Sia $A$ un aperto in $\mathbb{R}^2$ e $f \;\;:\;\; A \rightarrow \mathbb{R}$ una funzione $C^1(A)$. Supponiamo che in un punto $(x_0,y_0) \in A$ sia:
\[
    f(x_0,y_0)= 0 \;\;\;\; e \;\;\;\; f_y(x_0,y_0) \neq 0.
\] 
Allora esiste un intorno $I$ di $x_0$ in $\mathbb{R}$ e un'unica funzione $g \;\;:\;\; I \rightarrow \mathbb{R}$, tale che $y_0=g(x_0)$ e
\[
    f(x,g(x)) = 0 \;\;\;\; \;\forall\;x \in I.
\]
Inoltre, $g \in C^1(I)$ e 
\[
    g'(x) = - \frac{f_x(x,g_x)}{f_y(x,g_x)} \;\;\;\; \;\forall\;x \in I.
\]
Notiamo che se $f(x_0,y_0) = 0$ e $f_y(x_0, y_0) = 0$, ma $f_x(x_0, y_0) \neq 0$, il teorema è ancora applicabile scambiando gli ruoli di $x$ e $y$.\newline
In sostanza i punti in cui il teorema di Dini non è applicabile sono quelli in cui il gradiente di $f$ si annulla, ossia i punti critici. \newline
\rule{\textwidth}{2pt}
\subsection*{Complementi}
\rule{\textwidth}{0.4pt}
\subsubsection*{Topologia e funzioni continue}
Teorema di Bolzano-Weierstrass in $\mathbb{R}^n$. Sia $\{x_k\}_{k=1}^\infty$ una successione limitata. Allora essa ammette una sottosuccessione convergente.\newline

Successione di Cauchy in $\mathbb{R}^n$. Sia $\{x_k\}_{k=1}^\infty$ una successione in $\mathbb{R}^n$. Si dice che la successione soddisfa la condizione di Cauchy se
\[
    \;\forall\;\epsilon > 0 \;\;\exists\;\; n_0 \;\;:\;\;\;\forall\;h,k \geq n_0 \;\;\; si \;\;\; ha \;\;\;|x_h -x_k|<\epsilon.
\]
\newline

Teorema di completezza di $\mathbb{R}^n$. Se $\{x_k\}_{k=1}^\infty$ è una successione di Cauchy in $\mathbb{R}^n$, allora converge.\newline

Teorema dell'uniforme continuità. Si dice che $f \;\;:\;\; \Omega \rightarrow \mathbb{R}$ è uniformemente continua in $\Omega \subseteq \mathbb{R}^n$ se:
\[
    \;\forall\;\epsilon>0,\;\;\exists\;\;\delta>0 \;\;:\;\; \;\forall\;x_1,x_2 \in\Omega,\;\;se \;\; |x_1- x-2| <\delta \;\;allora \;\;|f(x_1) -f(x_2)|<\epsilon
\]
\newline

Teorema di Cantor-Heine. Sia $K\subset \mathbb{R}^n$ un insieme chiuso e limitato e $f \;\;:\;\; K \rightarrow \mathbb{R}$ una funzione continua. Allora $f$ è uniformemente continua in $K$.\newline

Sia $\Omega \subset \mathbb{R}$ un aperto e sia $f \;\;:\;\; \Omega \rightarrow \mathbb{R}$ uniformemente continua in $\Omega$. Allora $f$ è prolungabile con continuità fino alla frontiera di $\Omega$, ossia esiste una funzione $\bar{f} \;\;:\;\; \bar{\Omega} \rightarrow \mathbb{R}$ continua in $\bar{\Omega}$ e tale che in $\Omega$ coincide con $f$.\newline
\rule{\textwidth}{0.4pt} % funzioni di più variabili a valori reali
    %- aggiungere capitolo 3 del libro del prof
    \newpage
    \section*{5-CALCOLO INTEGRALE PER FUNZIONI DI PIU' VARIABILI}
\rule{\textwidth}{2pt}
 % integrali doppi e tripli
    %- finito
    \newpage
    \section*{4- Calcolo differenziale per funzioni di più variabili a valori vettoriali}
\rule{\textwidth}{2pt}
\subsection*{Funzioni di più variabili a valori vettoriali: generalità} % funzioni di più variabili a valori vettoriali
    %- integrare con il capitolo 6 (il successivo) di questo pdf
    \newpage
    \section*{Funzioni di più variabili a valori vettoriali}
\rule{\textwidth}{2pt}
\subsection*{Campi vettoriali}
Un campo vettoriale $F: \mathbb{R}^n \rightarrow  \mathbb{R}^m$ è una funzione che ad ogni  punto dello spazio $\mathbb{R}^n$ associa un vettore di $\mathbb{R}^m$.
\[
    F(\vec{x}) = \left(F_1(\vec{x}), \dots, f_m(\vec{x})\right) = \sum_{h=1}^{m} F_h(\vec{x})e_h
\]
dove le componenti $F_h$ sono funzioni scalari e gli $e_h$ rappresentano i vettori della base canonica di $\mathbb{R}^m$. Se tutte le componenti $F_h$ sono di classe $C^1$, diremo che $F \in C^1(\mathbb{R}^n, \mathbb{R}^m)$\newline
\newline
\textbf{oss.} $n=m=1$ funzione reale di variabile reale; $n=1, m \geq 2$ curva; $n \geq 2, m = 1$ funzione scalare di più variabili.\newline
\newline
Sia $\Omega \in \mathbb{R}^3$ un aperto, dato un campo $F \in C^1(\Omega, \mathbb{R}^3)$, chiameremo \textbf{linea di campo} di $F$ una curva regolare che in ongi punto del suo sostegno sia tangente a $F$.\newline
Con questa definizione si caratterizza una linea di campo di $F \in C^1(\mathbb{R}^3, \mathbb{R}^3)$ con una curva regolare $r = r(t)$ tale che $r'(t)$ sia proporzionale a $F(r(t))$. Dato che il coefficiente di proporzionalità è variabile possiamo scrivere:
\[
    r'(t) = p(t) F(r(t))
\]
dove $p(t)$ rappresenta la proporzionalità. Notiamo che $p(t) \neq 0$ per ogni $t$, quindi o $p(t) > 0$ (verso di percorrenza e flusso coincidono) o $p(t) < 0$ (verso di percorrenza opposto al flusso).
\[
    \left(\begin{matrix}
        x'(t)\\
        y'(t)\\
        z'(t)
    \end{matrix}\right) = p(t) \left(\begin{matrix}
        F_1(x(t),y(t),z(t))\\
        F_2(x(t),y(t),z(t))\\
        F_3(x(t),y(t),z(t))
    \end{matrix}\right)
\]
\[
    \frac{x'(t)}{F_1(x(t),y(t),z(t))} = 
    \frac{y'(t)}{F_2(x(t),y(t),z(t))} = 
    \frac{z'(t)}{F_3(x(t),y(t),z(t))}
\]
\[
    \frac{dx}{F_1(x,y,z)} = 
    \frac{dy}{F_2(x,y,z)} = 
    \frac{dz}{F_3(x,y,z)}
\]
e le linee di campo si ottengono integrando:
\[
    \int \frac{dx}{F_1(x,y,z)} = 
    \int \frac{dy}{F_2(x,y,z)} = 
    \int \frac{dz}{F_3(x,y,z)}
\]
\newline
Chiamiamo \textbf{rotore} di un campo $F \in C^1(\mathbb{R}^3, \mathbb{R}^3)$ il vettore
\[
    rot(F) = \nabla \land F = \left|\begin{matrix}
        \vec{i} & \vec{j} & \vec{k} \\
        \frac{\delta }{\delta x} & \frac{\delta }{\delta y} & \frac{\delta }{\delta z}\\
        F_1 & F_2 & F_3
    \end{matrix} \right|
\]
Un campo a rotore nullo viene chiamato \textbf{irrotazionale}.\newline
Per calcolare il rotore di una funzione $F \in C^1(\mathbb{R}^2, \mathbb{R}^2)$, è sufficiente immergerlo in $\mathbb{R}^3$, ponendo $F(x,Y) = (F_1(x,y), F_2(x,y)) \rightarrow  F(x,y,z) = (F_1(x,y), F_2(x,y), 0)$ e si trova:
\[
    rot(F) = \nabla\land F = \left(\frac{\delta F_2}{\delta x} - \frac{\delta F_1}{\delta y}\right)\vec{k}
\]
Quindi il rotore è perpendicolare al piano del campo.\newline
Anche in tre dimensioni il rotore individue una direzione che è perpendicolare al piano che localmente tende a contenere le linee di campo.\newline
Il modulo del rotore misura la tendenza a ruotare attorno all'asse della sua direzione. Se le linee di campo tendono ad accolgersi attorno all'asse, il rotore abrà il verso della vite che ruota nel verso delle linee di campo. Se il rotore è nullo, non c'è effetto rotatorio.\newline
\newline
Chiamiamo \textbf{divergenza} di un campo $F \in C^1(\mathbb{R}^3, \mathbb{R}^3)$ lo scalare
\[
    div(F) = \nabla \cdot F = \frac{\delta F_1}{\delta x} +\frac{\delta F_2}{\delta y} +\frac{\delta F_3}{\delta z}
\]
Un campo a divergenza nulla viene chiamato solenoidale.\newline
La divergenza in $n=1$ coincide con la derivata di una funzione scalare.\newline
\newline
\[
    \nabla \land (\nabla u) = rot(\nabla u) = 0 \quad \quad \;\forall\;u \in C^2(\Omega, \mathbb{R}),
\]
cioè un gradiente è irrotazionale;
\[
    \nabla \cdot  (\nabla \land F) = div(rot(F)) = 0 \quad \quad \;\forall\; F \in C^2(\Omega, \mathbb{R}^3),
\]
cioè un rotore è solenoidale;
\[
    \nabla \cdot (\nabla u) = div(\nabla u) = \nabla u \quad \quad \;\forall\;u \in C^2(\Omega, \mathbb{R}).
\]
\rule{\textwidth}{2pt}
\subsection*{Lavoro di un campo vettoriale}
\textbf{def.} SIa $\gamma$ il sostegno orientato di una curva regolare a tratti $r: I \rightarrow \mathbb{R}^n$ e sia $F \in C ^0(\mathbb{R}^n, \mathbb{R}^n)$. Si chiama \textbf{lavoro} di $F$ lungo $\gamma$ l'integrale di linea
\[
    L_\gamma (F) = \int_{\gamma} F \cdot dr = \int_{I} F(r(t))\cdot r'(t) dt
\]
Sia $I = [a,b]$, da un punto di vista fisico, $L_\gamma(F)$ rappresenta il lavoro che compie la forza $F$ per spostare il suo punto di applicazione lungo $\gamma$ da $r(a)$ a $r(b)$.\newline
Rispetto al classico integrale di linea che non cambia se riparametriziamo la curva con verso opposto, l'integrale che calcola il lavoro cambia segno se la curva cambia verso.\newline
\newline
Diremo che $F \in C^1(\Omega, \mathbb{R}^n)$ è \textbf{conservativo} se esite una funzione reale $U \in C^2(\Omega, \mathbb{R})$, che chiameremo \textbf{potenziale} di $F$, tale che $\nabla U = F$.\newline
\newline
\[
    F \; \text{conservativo}\; \Longrightarrow F \; \text{irrotazionale}\;
\]
da cui ricaviamo che se $F$ non è irrotazionale, allora non è conservativo.\newline
\newline
Il lavoro di $F$ lungo un qualunque sostegno $\gamma$ si esprime come
\[
    L_\gamma(F) = \int_{a}^{b}\nabla U(r(t)) \cdot r'(t) dt = U(r(b)) - U(r(a))
\]
Questa espressione ci dice che il lavoro di un campo conservativo si calcola come differenza di potenziale e che il lavoro di un campo conservativo non dipende dalla curva di percorrenza, ma solo dai suoi estremi. Inoltre il lavoro di $F$ lungo il sostegno di una qualunque linea chiusa è nullo.\newline
\newline
\textbf{def.} Un insieme $\Omega \subset \mathbb{R}^n$ si dice \textbf{semplicemente connesso} se ogni linea chiusa $\gamma \subset \Omega$ è contraiile in $\Omega$ a un punto di $\Omega$. In altre parole, partengo da una linea chiusa $\gamma \subset \Omega$ deve essere possibile contrarla, fino a ridursi a un punto e questo deve avvenire sempre rimanendo in $\Omega$.\newline
In parole povere, in $\mathbb{R}^2$ basta un buco riducibile a un punto per far perdere questa proprietà, in $\mathbb{R}^3$ basta un buco simile a un segmento.\newline
\newline
\textbf{teor.} sia $\Omega$ un insieme semplicemente connesso, e sia $F$, allora
\[
    F \; \text{conservativo}\; \Longleftrightarrow F \; \text{irrotazionale}\;
\]
Da notare è il fatto che sia una condizione sufficiente, non necessaria.\newline
Notiamo anche che un campo irrotazionale è sempre localmente conservativo.\newline
\newline
Per sapere se un campo è conservativo:
\begin{itemize}
    \item si controlla se è irrotazionale:
    \begin{itemize}
        \item se non lo è, il campo non è conservativo, e per calcolarne il lavoro dobbiamo per forza usare la definizione:
        \[
            L_\gamma (F) = \int_{\gamma} F \cdot dr = \int_{I} F(r(t))\cdot r'(t) dt
        \]
        \item se lo è, dobbiamo vedere se il dominio dove è definito è semplicemente connesso:
        \begin{itemize}
            \item se lo è, il campo è conservativo.
            \item se non lo è, non possiamo concludere nulla a priori e quindi siamo costretti a provare a costruire un potenziale.
        \end{itemize}
    \end{itemize}
\end{itemize}
Come trovare un potenziale di un campo conservativo:\newline
Dato un campo $F(x,y,z) = (F_1(x,y,z), F_2(x,y,z), F_3(x,y,z))$, $F \in C^1(\mathbb{R}^3, \mathbb{R}^3)$, cerchiamo una funzione $U \in C^2(\mathbb{R}^3, \mathbb{R})$ tale che
\[
    U_x(x,y,z) = F_1(x,y,z)
\]
\[
    U_y(x,y,z) = F_2(x,y,z)
\]
\[
    U_z(x,y,z) = F_3(x,y,z)
\]
Analiziamo il procedimento per la prima di queste: integriamo rispetta a $x$, ma trattandosi di un integrale indefinito abbiamo una costante indefinita che potrebbe dipendere da $y$ e $z$:
\[
    U(x,y,z) = \int F_1(x,y,z)dx + \phi(y,z)
\]
Dunque
\[
    U(x,y,z) = \int F_2(x,y,z)dy + \phi(x,z)
\]
\[
    U(x,y,z) = \int F_3(x,y,z)dz + \phi(x,y)
\]
Se riusciamo a trovare un'espressione che si possa esprimere contemporaneamente in queste tre forme, quella sarebbe un potenziale.\newline
\rule{\textwidth}{0,4pt}
\subsubsection*{Forme differenziali lineari}
Nel calcolo del lavoro di un campo $F \in C^1 (\mathbb{R}^n, \mathbb{R}^n)$, ci imbattiamo in integrande del tipo
\[
    F \cdot  dr = F_1 dx + f_2 dy + F_3 dz
\]
nel caso $n=3$ e con solo i primi due addenti nel caso $n=2$. UN'espressione di questo tipo prende il nome di \textbf{forma differenziale lineare}. Il lavoro si esprime quindi come integrale di linea di una forma differenziale lineare:
\[
    \int_\gamma F \cdot dr = \int\left(\begin{matrix}
        F_1(x,y,z)\\
        F_2(x,y,z)\\
        F_3(x,y,z)
    \end{matrix}\right) \cdot \left(\begin{matrix}
        dx\\dy\\dz
    \end{matrix}\right) = \int_\gamma (F_1 dx + F_2 dy + F_3 dz)
\] 
\newline
Se $F$ è conservativo, esiste un potenziale $U \in C^2(\mathbb{R}^n, \mathbb{R})$ tale che $\nabla U = F$ e cioè
\[
    dU = U_x dx + U_y dy + U_z dz = F_1 dx F_2 dy + F_3 dz
\]
In questo caso, la forma differenziale coincide con il differenziale di $U$ e viene chiamata forma differenziale \textbf{esatta}.\newline
\newline
Se 
\[
    \begin{cases}
        (F_1)_y =(F_2)_x\\
        (F_1)_z =(F_3)_x\\
        (F_2)_z =(F_3)_y
    \end{cases}
\]
allora la funzione $F$ è irrotazionale e la forma differenziale lineare corrispondente viene chiamata forma differenziale \textbf{chiusa}.\newline
\newline
Ogni forma differenziale esatta è anche chiusa, mentre il viceversa non è vero.\newline
\rule{\textwidth}{2pt}
\subsection*{Teorema della divergenza}
Consideriamo $F \in C^1(\Omega, \mathbb{R}^n)$ con $\Omega \subset \mathbb{R}^n$ e limitato, inoltre sia $\vec{n}$ un versore definito su un punto del bordo di $\Omega$ (detto $\delta\Omega$ ) e perpendicolare al piano tangente in $\delta \Omega$ e orientato verso l'esterno.\newline
Il prodotto scalare $F \cdot \vec{n}$ rappresenta il flusso uscente (se $> 0$, entrante se $<0$)  da $\delta\Omega$.\newline
\newline
\textbf{teor.} della divergenza.\newline
Sia $\Omega \subset \mathbb{R}^n$ un aperto limitato, semplice rispetto a tutti gli assi cartesiani con versore nomrale uscente $\vec{n}$ in ogni punto di $\delta\Omega$. Sia $F \in C^1(\bar{\Omega}, \mathbb{R}^n)$ un campo vettoriale. Allora
\[
    \int_\Omega \nabla \cdot  F(x) dx = \int_{\delta\Omega} F \cdot  \vec{n} \; dS
\]
Questo teorema afferma che l'integrale su un dominio della divergenza di un campo è pari al flusso del campo che attraversa la sua frontiera.\newline
Con questo teorema spiagamo anche il significato dell'operatore divergenza: misura il grado di comprimibilità di un fluido e, più in generale, quella di un campo vettoriale.\newline
\rule{\textwidth}{2pt}
\subsection*{Note sugli esercizi} % funzioni di più variabili a valori vettoriali
    %- integrare con il capitolo 4 (il precedente) di questo pdf
    \newpage
    \section*{7-Serie di funzioni}
\rule{\textwidth}{2pt}
\subsection*{Serie di potenze}
\rule{\textwidth}{0,4pt}
\subsubsection*{Nel campo complesso}
\textbf{def.} Sia $\{a_n\}$ una successione di numeri complessi e sia $z_0 \in \mathbb{C}$.\newline
La serie
\[
    \sum_{n=0}^{\infty} a_n (z-z_0)^n
\]
si chiama \textbf{serie di potenza centrata} in $z_0$.\newline
\newline
Con la semplice traslazione $z-z_0 \rightarrow z$ possiamo ricondurci al caso $z_0 = 0$:
\[
    \sum_{n=0}^{\infty}a_n z^n
\]
\newline
Una serie di potenze ammette un $R \in [0, +\infty]$ tale che converge se $|z| < R$, non converge se $|z| > R$ e nulla si può dire se $|z| = R$.\newline
\newline
\textbf{Criterio del rapporto:} Se esiste
\[
    R = \lim_{n\rightarrow \infty}\left| \frac{a_n}{a_{n+1}}\right|
\]
allora la serie $\sum_{n=0}^{\infty}a_n z^n$ converge se $|z|< R$ e non converge se $|z|> R$.\newline
\newline
\textbf{Criterio della radice:} Se esiste
\[
    R = \lim_{n\rightarrow \infty} \frac{1}{\sqrt[n]{|a_n|}}
\]
allora la serie $\sum_{n=0}^{\infty}a_n z^n$ converge se $|z| < R$ e non converge se $|z|> R$.\newline
\newline
L'insieme di convergenza di una serie di potenze in $\mathbb{C}$ è un disco.\newline
Se $R = +\infty$ il disco è tutto $\mathbb{C}$, se $R = 0$ il disco è vuoto. Il numero $R \in [0, + \infty]$ si chiama \textbf{raggio di convergenza} della serie di potenza.\newline
\newline
Questi due criteri non dicono nulla sul comportamento della serie nei punti sul bordo del disco, cioè $|z| = R$.\newline
\newline
\textbf{teor.} Sia $\{a_n\}$ una successione di numeri complessi tale che la serie di potenze
\[
    f(z) = \sum_{n=0}^{\infty} a_n z^n
\]
converga per $|z|<R$ (con $R> 0$). Allora le serie ottenute derivando e integrando termine a termine, e cioè
\[
    \sum_{n=1}^{\infty} n a_n z^{n-1} \quad \quad \quad \sum_{n=0}^{\infty} \frac{a_n}{n+1} z^{n+1}
\]
sono rispettivamente la derivata e una primitiva della funzione $f$; inoltre, il loro raggio di convergenza è ancora $R$.\newline
\newline
\rule{\textwidth}{0,4pt}
\subsubsection*{Nel campo reale}
Diremo che una serie di funzioni $\sum_{n}f_n(x)$ \textbf{converge puntualmente} per ogni $x \in I$ se la serie numerica $\sum_{n}f_n(x)$ converge per ogni $x \in I$.
\[
    f(x) = \sum_{n=0}^{\infty} f_n(x) = \lim_{k\rightarrow \infty} \sum_{n=0}^{\infty} f_n(x) \quad \;\forall\;x \in I
\]
cioè
\[
    f_n(x) \rightarrow f(x) \; \text{puntualmente se}\; \lim_{n\rightarrow \infty} f_n(x) = f(x) 
\]
\newline
Diremo che la serie di funzioni $\sum_{n}f_n(x)$ \textbf{converge uniformemente} a $f(x)$ su $I$ se
\[
    \lim_{k\rightarrow \infty} sup_{x \in I}\left| f(x) - \sum_{n=0}^{k}f_n(x) \right| = 0
\]
cioè
\[
    f_n(x) \rightarrow f(x) \; \text{uniformemente se}\; \lim_{n\rightarrow \infty} sup_{x \in I}\left| f_n(x) - f(x) \right| = 0
\]
\newline
Diremo che la serie di funzioni $\sum_{n}f_n(x)$ \textbf{converge totalmente} su $I$ se
\[
    \sum_{n=0}^{\infty} sup_{x \in I}\left| f_n(x) \right| < + \infty
\]
\newline
\[
    \text{convergenza totale}\; \Longrightarrow \text{convergenza uniforme}\; \Longrightarrow \text{convergenza puntuale}
\]
\newline
Serie di potenza nel campo reale:
\[
    \sum_{n=0}^{\infty} a_n(x-x_0)^n
\]
che possiamo traslare nell'origine:
\[
    f(x) = \sum_{n=0}^{\infty} a_n x^n
\]
\newline
\textbf{Criteri del rapporto e della radice:} Si possono ancora usare i criteri della radice e del rapporto specificati nel campo complesso, $R$ rappresenta ancora il raggio del disco di convergenza nel piano complesso, ma essendo interezzati all'asse reale si considera il solo intervallo $(-R, R)$.\newline
La serie di potenza nel campo reale converge \textbf{puntualmente} per ogni $x \in (-R, R)$, dove $R$ è dato dal criterio del rapporto o dal criterio della radice, e non converge se $|x| > R$.\newline
Come nel caso complesso, non possiamo dire nulle sulla convergenza in $x = \pm R$.\newline
Per quanto riguarda la convergenza \textbf{uniforme}, la serie di potenza nel campo reale converge uniformemente in $[-R+\epsilon, R- \epsilon]$ per ogni $\epsilon \in (0,R)$.\newline
\newline
\textbf{Criterio di Abel:} Se la serie di potenza $\sum_{n=0}^{\infty} a_n x^n$ converge per $x = R$, allora converge uniformemente in $[-R + \epsilon, R]$ per ogni $\epsilon \in (0,R)$; analogo risultato se la serie converge per $x = -R$. Se la serie converge per $x = \pm R$ allora converge uniformemente su tutto $[-R, R]$.\newline
\newline
\textbf{teor. (Integrazione per serie)}\newline
Se la serie di potenza $\sum_{n=0}^{\infty} a_n x^n$ converge uniformemente a $f$ su $[c,d]$ allora
\[
    \int_{c}^{d}f(x) dx = \int_{c}^{d}\left(\sum_{n=0}^{\infty}a_n x^n\right)dx = \sum_{n=0}^{\infty}a_n \int_{c}^{d}x^n dx = \sum_{n=0}^{\infty} a_n \frac{d^{n+1}- c^{n+1}}{n+1}
\]
\newline
\textbf{Serie di Taylor}\newline
Introduciamo una vasta classe di funzioni elementari delle quali sappiamo scrivere esplicitamente le serie di potenza che le rappresentano.\newline
Data una funzione $f$ di classe $C^\infty$ in un punto $x_0$, possiamo scrivere formalmente
\[
    f(x) = \sum_{n=0}^{\infty} \frac{f^{(n)(x_0)}}{n!}(x-x_0)^n
\]
che, se poniamo $a_n = f^{(n)}(x_0)/n!$, coincide con una serie di potenza nel campo reale. Se $R > 0$ e la serie converge a $f$, allora la scrittura non e solo formale ma vale per ongi $x \in (x_0 - R, x_0 + R)$. In tale intervallo di ha convergenza putnuale, mentre la convergenza uniforma è garantita negli intervalli $[x_0 - R + \epsilon, x_0 + R -\epsilon]$ per ogni $\epsilon \in (0,R)$.
\[
    e^x = \sum_{n=0}^{\infty} \frac{x^n}{n!} \;\;\;\; (R= \infty)
\]
\[
    log(1-x) = \sum_{n=0}^{\infty} \frac{(-1)^{n+1}}{n} x^n\;\;\;\; (R=1)
\]
\[
    sin(x) = \sum_{n=0}^{\infty} \frac{(-1)^n}{(2n+1)!} x^{2n+1}\;\;\;\; (R= \infty)
\]
\[
    cos(x) = \sum_{n=0}^{\infty} \frac{(-1)^n}{(2n)!}x^{2n}\;\;\;\; (R= \infty)
\]
\[
    \frac{1}{1-x} = \sum_{n=0}^{\infty} x^n\;\;\;\; (R= 1)
\]
\[
    log(1+x) = \sum_{n=0}^{\infty} (-1)^{n+1} \frac{x^n}{n}\;\;\;\; (R= 1)\;\;\;\;\;\;\;\;\;\;\begin{cases}
        \text{se $x=1: $}\; \sum_{n=1}^{\infty}\frac{(-1)^{n+1}}{n} \;\;\text{(converge per Leibniz)}\;\\
        \text{se $x=-1$}\; \sum_{n=1}^{\infty}\frac{-1}{n} \;\;\text{diverge, serie armonica}\;
    \end{cases}
\]
\[
    \frac{1}{1+x^2} = \sum_{n=0}^{\infty} -1^n \cdot x^{2n} \;\;\;\;(R=1)
\]
\[
    arctanx(x) = \sum_{n=0}^{\infty}(-1)^n \frac{x^{2n+1}}{2n+1} \;\;\;\;(R=1)
\]
\rule{\textwidth}{2pt}
\subsection*{Serie di Fourier}
\rule{\textwidth}{0,4pt}
\subsubsection*{Forma trigonometrica}
Nello studio delle serie di potenza $\sum_{n}a_n z^n$ abbiamo visto che i problemi principali riguardano lo studio di $|z| = R$ con $R$ raggio di convergenza. Per tali possiamo scrivere $z = R(cos(\theta) + sin(\theta))$ e ottenere:
\[
    \sum_{n=0}^{\infty} a_n R^n(cos(n \theta) + isin(n \theta))
\]
Se $\{a_n\} \subset \mathbb{R}$, siamo quindi portati a studiare la convergenza di serie trigonometriche del tipo
\[
    \sum_{n=0}^{\infty} \alpha_n cos(n \theta) \;\;\;\;\; \;\;\;\;\; \sum_{n=1}^{\infty} \beta_n sin(n \theta)
\]
Per ogni scelta di $\alpha_n, \beta_n \in \mathbb{R}$ le funzioni
\[
    a_0 + \sum_{n=1}^{\infty} (\alpha_n cos(nx) + \beta_n sin(n x))
\]
si chiamano \textbf{polinomi trigonometrici} di grado $k$; che sono funzioni periodiche di periodo $2\pi$ che hanno valor medio $\alpha_0$ su $[0, 2\pi]$.\newline
Diremo che una serie in fomra trigonometrica converge se converge una successione di polinomi trogonometrici.\newline
\textbf{oss.} 
\[
    \sum_{n=1}^{\infty} (|\alpha_n| + |\beta_n|) < \infty \Rightarrow \text{converge totalmente su } [0,2\pi] \Rightarrow \text{converge uniformemente} \Rightarrow \text{converge puntualmente}
\]
\textbf{oss.} La serie trigonometrica può perdere la convergenza dopo una derivazione.\newline
\newline
\textbf{Criterio di Dirichlet}
\[
    \alpha_n, \beta_n \downarrow 0 \Rightarrow \text{polinomio trigonometrico converge puntualmente su} (0,2\pi)
\]
dove con il simbolo $\downarrow 0$ indichiamo che le successioni descrescono monotonamente a $0$ e che sono positive.\newline
\newline
\textbf{Lemma.} Per ogni $m,n = 1,2,\dots$ risulta 
\[
    \int_{0}^{2\pi} sin(nx) dx = \int_{0}^{2\pi} cos(nx) = 0
\]
\[
    \int_{0}^{2\pi} sin(mx) cos(nx) dx = 0
\]
\[
    \int_{0}^{2\pi} sin^2(nx) dx = \int_{0}^{2\pi} cos^2(nx) dx = \pi
\]
\[
    \int_{0}^{2\pi} sin(mx) sin(nx) dx = \int_{0}^{2\pi} cos(mx) sin(nx) dx = 0 \;\;\;\; \text{se} \; m\neq n
\]
Si trovano gli stessi valori integrando su qualunque intervallo di ampiezza $2\pi$.\newline
\newline
\textbf{teor.} Se una funzione $f$ è $2\pi$-periodica e si può scrivere nella forma trigonometrica
\[
    f(x) = \frac{a_0}{2} + \sum_{n=1}^{\infty}(a_n cos(nx) + b_n sin(nx))
\]
allora
\[
    a_n = \frac{1}{\pi} \int_{0}^{2\pi} f(x) cos(nx) dx \;\;\;\; \;\forall\;n\geq0
\]
\[
    b_n = \frac{1}{\pi} \int_{0}^{2\pi} f(x) sin(nx) dx \;\;\;\;\;\forall\;n\geq 1
\]
Gli $a_n$ e $b_n$ vengono chiamati \textbf{coefficienti di Fourier} di $f$, mentre la serie $f(x) = \frac{a_0}{2} + \sum_{n=1}^{\infty}(a_n cos(nx) + b_n sin(nx))$ viene chiamata \textbf{serie di Fourier} associata a $f$.\newline
\newline
\textbf{oss.} Gli integrali del teorema si possono calcolare se risulta $\int_{0}^{2\pi}|f| < \infty$ e cioè se l'integrale improprio di $|f|$ è convergente.\newline
\newline
Diremo che la serie 
\[
    f(x) = \frac{a_0}{2} + \sum_{n=1}^{\infty}(a_n cos(nx) + b_n sin(nx))
\]
\textbf{converge in media quadratica} alla $f$ se
\[
    \lim_{k\rightarrow \infty} \int_{0}^{2\pi} \left| f(x) - \frac{a_0}{2} - \sum_{n=1}^{k}\left( a_n cos(nx) + b_n sin(nx) \right) \right|^2 dx = 0
\]
\newline
\textbf{teor.} Sia $f$ una funzione $2\pi$-periodica tale che $\int_{0}^{2\pi}f^2 < \infty$. Allora la sua serie di Fourier converge a $f$ in media quadratica.\newline
\newline
Definiamo lo spazio $X$ delle funzioni $f$ tali che $\int_{0}^{2\pi}f^2 < \infty$ e introduciamo il "prodotto scalare" definito come
\[
    (x,g)_{X} = \frac{1}{\pi}\int_{0}^{2\pi} f(x) g(x) dx \;\;\;\;\; \;\forall\;f,g \in X
\]
per cui troviamo che 
\[
    B = \left\{ \frac{1}{\sqrt{2}}, cos(nx), sin(nx) \right\}_{n=1} ^\infty
\]
è ortonormale in $X$.\newline
\newline
\textbf{teor. Identità di Parseval}\newline
Sia $f \in X$ e sia $f(x) = \frac{a_0}{2} + \sum_{n=1}^{\infty}(a_n cos(nx) + b_n sin(nx))$ la sua serie di Fourier. Allora
\[
    \frac{1}{\pi} \int_{0}^{2\pi} f(x)^2 dx = \frac{a_0^2}{2} + \sum_{n=1}^{\infty}(a_n^2 + b_n^2)
\]
Per il teorema di Riemann-Lebesgue sappiamo che $a_n, b_n \rightarrow  0$ per $n \rightarrow  \infty$.\newline
\newline
\[
\text{convergenza uniforme}\; \Rightarrow \text{convergenza in media quadratica}\; \Rightarrow 
\]
\[
    \Rightarrow  \text{convergenza puntuale su sottosuccession in quasi ogni punto}\;
\]
\newline
\textbf{def.} Diciamo che $f : [0,2\pi] \rightarrow \mathbb{R}$ è \textbf{regolare a tratti} se è limitata in $[0,2\pi]$ e se l'intervallo $[0,2\pi]$ si può scomporre in un numero finito di sottointervalli su ciascuno dei quali $f$ è continua e derivabile; inoltre, agli estremi di ogni sottointervallo, esistno finiti i limiti sia di $f$ che di $f'$.\newline
\newline
\textbf{oss.} se $f \in C^1[0,2\pi]$ allora $f$ è regolare a tratti. Ma anche se ci sono punti angolosi o punti di discontinuità a salto, purchè $f$ e $f'$ abbiano limiti finiti in prossimità dei salti, allora $f$ è regolare a tratti. Non devono esserti asintoti verticali o punti a tangenza verticale.\newline
\newline
Se una funzione $f$ è regolare a tratti allora la sua serie di Fourier converge a $f$ in media quadratica. Dunque a meno di pochi possibili punti, la conergenza sarà anche puntuale.\newline
\newline
\textbf{teor.} Sia $f: [0,2\pi] \rightarrow \mathbb{R}$ regolare a tratti. Allora la sua serie di Fourier converge in ogni punto $x_0 \in [0,2\pi]$ alla media dei due limiti $f(x_0^{\pm})$:
\[
    \frac{a_0}{2} + \sum_{n=1}^{\infty} \left( a_n cos(nx_0) + b_n sin(nx_0) \right) = \frac{f(x_0^+) + f(x_0^-)}{2}
\]
con la convenzione che $f(0^\pm) = f(2\pi^\pm)$. In particolare, se $f$ è continua in $x_0$, allora la serie converge a $f(x_0)$.\newline
\rule{\textwidth}{0,4pt}
\subsubsection*{Funzioni pari e dispari, periodi diversi da $2\pi$}
I coefficienti di Fourier di funzioni $2\pi$-periodiche si possono calcolare integrando su un qualunque intervallo di ampiezza $2\pi$. Tipicamente si sceglie l'intervallo $[-\pi, \pi]$ perchè permette di sfruttare eventuali simmetrie della funzione $f$.\newline
Infatti, se la funzione $f$ è \textbf{pari} allora risulta pari anche $f(x)cos(nx)$ mentre risulta dispari $f(x)sin(nx)$, dunque:
\[
    f \; \text{pari}\; \Rightarrow a_n = \frac{2}{\pi} \int_{0}^{\pi}f(x) cos(nx) dx , \;\;\;\;\;b_n = 0
\]
Invece, se la funzione $f$ è \textbf{dispari} allora risulta dispari anche la funzioen $f(x)cos(nx)$ mentre risulta pari $f(x) sin(nx)$, dunque:
\[
    f \; \text{dispari}\; \Rightarrow a_n = 0, \;\;\;\;\;b_n = \frac{2}{\pi} \int_{0}^{\pi} f(x) sin(nx)dx
\]
\newline
\rule{\textwidth}{0,4pt}\newline
Come ci si comporta in presenza di funzioni con periodo $T \neq 2\pi$?\newline
L'unica differenza sta nel calcolo dei coefficienti di Fouriere:
\[
    a_n = \frac{2}{T}\int_{0}^{T}f(x) cos\left( \frac{2\pi n}{T}x \right) dx, \;\;\;\;\; \;\;\;\;\; b_n = \frac{2}{T} \int_{0}^{T}f(x) sin\left( \frac{2\pi n}{T}x \right)
\]
La funzione $f$ si scrive allora:
\[
    f(x) = \frac{a_0}{2} + \sum_{n=1}^{\infty} \left( a_n cos\left( \frac{2\pi n}{T}x \right) + b_n sin \left( \frac{2\pi n}{2}x \right) \right)
\]
Tutti i teoremi valgono allo stesso modo, l'unico da "sistemare" è l'identità di Parseval, che, per fuznioni $T$-periodiche $f$ soddisfacenti $\int_{0}^{T} f^2 < \infty$, diventa:
\[
    \frac{2}{T} \int_{0}^{T}f(x)^2 dx = \frac{a_0^2}{2} + \sum_{n=1}^{\infty} (a_n^2 + b_n^2)
\]
\rule{\textwidth}{0,4pt}
\subsubsection*{Forma esponenziale complessa}
La formula di Eulero, $e^{i \theta} = cos(\theta) + i sin(\theta)$, suggerisce di scrivere una serie di Fourier utilizzando gli esponenziali.
\[
    f_n =\frac{a_n-ib_n}{2} = \frac{1}{2\pi} \int_{0}^{2\pi}f(x)(cos(nx) - i sin(nx))dx = \frac{1}{2\pi} \int_{0}^{2\pi}f(x) e^{-inx}
\]
\[
    f_{-n} =\frac{a_n + ib_n}{2} = \frac{1}{2\pi} \int_{0}^{2\pi}f(x)(cos(nx) + i sin(nx))dx = \frac{1}{2\pi} \int_{0}^{2\pi}f(x) e^{inx}
\]
da cui
\[
    a_n = f_n + f_{-n}
\]
\[
    b_n = i (f_n)
\]
\[
    f(x) = \frac{a_0}{2} + \sum_{n=1}^{\infty} \left( a_n cos(nx) + b_n sin(nx) \right) = \sum_{n=-\infty}^{\infty} f_n e^{inx}
\]
\newline
Identità di Parseval:
\[
    \frac{1}{2\pi}\int_{0}^{2\pi}f(x)^2dx = \sum_{n=-\infty}^{\infty}|f_n|^2
\] % serie
    % - finito
    \newpage
    \section*{Equazioni differenziali}
\rule{\textwidth}{2pt}
\subsection*{Modelli differenziali}
Si dice equazione differenziale di ordine $n$ un equazione del tipo:
\[
    F(t, y, y', y'', \dots, y^{(n)}) = 0
\]
dove $y(t)$ è la funzione incognita.\newline
Si dirà soluzione, o (curva) integrale, di un equazione differenziale, nell'intervallo $I \subset \mathbb{R}$, una funzione $\phi(t)$, definita almeno in $I$ e a valori reali per cui risulti
\[
    F(t, \phi(t), \phi'(t), \phi''(t), \dots, \phi^{(n)}(t))= 0 \;\;\;\;\;\;\;\; \;\forall\;t \in I
\]
Infine si dirà integrale generale di un equazione differenziale una formula che rappresenti la famiglia di tutte le soluzioni.\newline
\rule{\textwidth}{2pt}
\subsection*{Equazioni del primo ordine}
\rule{\textwidth}{0,4pt}
\subsubsection*{Generalità}
L'insieme delle soluzioni di un'equazione differenziale del prim'ordine prende il nome di integrale generale dell'equazione.\newline
Un equazione differenziale si dice in forma normale se è nella forma
\[
    y'(t) = f(t,y(t))
\]
e ha infinite soluzioni del tipo
\[
    y(t) = \int f(t, y(t)) dt + c .
\]
La condizione supplementare
\[
    y(t_0) = y_0
\]
permette di selezionare una soluzione particolare.\newline
Il problema di risolvere queste equazioni prende il nome di problema di Cauchy, e si intende sempre che bisogna trovare come soluzione una funzione:
\begin{itemize}
    \item definita su un intervallo $I$, contenente il punto $t_0$;
    \item derivabile in tutto $I$ e che soddisfa l'equazione in tutto $I$.
\end{itemize}
\rule{\textwidth}{0,4pt}
\subsubsection*{Equazioni a variabili separabili}
Le equazioni a variabili separabili sono equazioni del tipo
\begin{tcolorbox}
\[
    y' =a(t) b(y)
\]
\end{tcolorbox}
con $a$ continua in $I \subset \mathbb{R}$ e b continua in $J \subset \mathbb{R}$.\newline
Per risolverle notiamo che per tutte le $\bar{y}_0$ per cui $b(\bar{y}_0) = 0$, primo e secondo membro si annullano.\newline
Prendendo in considerazione, invece, tutte le $\bar{y}_1$ per cui $b(\bar{y}_1) \neq 0$, otteniamo:
\begin{tcolorbox}
\[
    \int \frac{dy}{b(y)} = \int a(t) dt +c 
\]
\end{tcolorbox}
Quindi se $B(y)$ è una primitiva di $\frac{1}{b(y)}$ e $A(t)$ è una primitiva di $a(t)$, otteniamo:
\[
    B(y) = A(t) + c
\]
e se si può ottenere la funzione inversa di $B$ possiamo scrivere:
\[
    y = B^{-1}(A(t) + c)
\]
\[
    y = F(t,c)
\]
\begin{tcolorbox}
\textbf{teor.} Problema di Cauchy per un'equazione a variabili separabili.\newline
Si consideri il problema di Cauchy:
\[
    \begin{cases}
        y' = a(t) b(y)&\\
        y(t_0) = y_0&
    \end{cases}
\]
Dove $a$ è continua in un intorno $I$ di $t_0$ e $b$ è continua in un intonro $J$ di $t_0$. Allora essite un intorno $t_0$ $I' \subset I$ e una funzione $y \in C^1(I')$ soluzione del problema.\newline
Se inoltre $b \in C^1(J)$, allora tale soluzione è unica.\newline
\end{tcolorbox}
\rule{\textwidth}{0,4pt}
\subsubsection*{Equazioni lineari del prim'ordine}
Sono equazioni la cui forma normale è
\begin{tcolorbox}
\[
    y'(t) + a(t)y(t) = f(t)
\]
\end{tcolorbox}
e viene chiama equazione completa.\newline
Se $f = 0$ l'equazione si dice omogenea.
\[
    z'(t) + a(t)z(t) = 0
\]
\textbf{teor.} L'integrale generale dell'equazione completa si ottiene aggiungengo all'integrale generale dell'omogenea una soluzione particolare della completa.\newline
\textbf{Soluzione dell'equazione omogenea}
\[
    z(t) = c e^{-\int a(t) dt}
\]
\textbf{Ricerca di una soluzione particolare dell'equazione completa}\newline
Se una soluzione particolare non si riesce a trovare facilmente, si può usare il seguente metodo, detto di variazione della costante
\[
    \bar{y}(t) = c(t)e^{-A(t)}
\]
Dove per $A(t)$ si intende una primitiva di $a(t)$ (essendo una primitiva qualunque, si può omettere la $c$).\newline
Sostituendo $\bar{y}$ nella forma completa otteniamo l'integrale generale della forma completa:
\[
    y(t) = ce^{-A(t)} + e^{-A(t)}\int f(t)e^{A(t)}dt
\]
\textbf{oss.} Nella primitiva $A(t)$ e nell'integrale $\int f(t)e^{A(t)}dt$ non c'è bisogno di aggiungere la solita costante di integrazione arbitraria.\newline
\newline
\textbf{Soluzione del problema di Cauchy}\newline
La soluzione
\begin{tcolorbox} 
\[
    y(t) = ce^{-A(t)} + e^{-A(t)}\int f(t)e^{A(t)}dt
\]
\end{tcolorbox}
sarà determinata da una condizione iniziale
\[
    y(t_0) = y_0
\]
scegliendo la primitiva $A(t)$ tale che $A(t_0) = 0$ (cioè $A(t) = \int_{t_0}^{t}a(s)ds$), sarà
\[
    y(t) = ce^{-A(t)} + e^{-A(t)} \int_{t_0}^{t} f(s)e^{A(s)}ds
\]
\begin{tcolorbox}
\textbf{teor.} Problema di Cauchy per un equazione lineare del prim'ordine.\newline
Siano $a, f$ funzioni continue in un intervallo $I$ contenente $t_0$. Allora, per ogni $y_0 \in \mathbb{R}$ il problema di Cauchy
\[
    \begin{cases}
        y'(t) + a(t)y(t)= f(t)\\
        y(t_0) = y_0
    \end{cases}
\]
ha una e una sola soluzione $y \in C^1(I)$ e tale soluzione è
\[
    y(t) = ce^{-A(t)} + e^{-A(t)} \int_{t_0}^{t} f(s)e^{A(s)}ds
\]
\end{tcolorbox}
\rule{\textwidth}{0,4pt}\newline
\begin{tcolorbox}
    \subsubsection*{Note sugli esercizi}
    Per trovare il massimo intervallo della soluzione bisogna prendere un intervallo contenente il punto $t_0$, per cui la soluzione e la funzione di partenza siano continue e definite (si può anche prolungare per continuità).\newline
\end{tcolorbox}
\rule{\textwidth}{0,4pt}
\subsubsection*{Equazione di Bernoulli}
Un equazion differenziale si dice di \textbf{Bernoulli} se si può scrivere nella forma
\[
    y' = a(t)y + b(t) y^\alpha
\]
osserviamo subito che se $\alpha=0$ o $\alpha = 1$ l'equazione è lineare. Se poi $\alpha$ fosse irrazionale o razionale con denominatore pari, non avrebbe senso definire $y^\alpha$ per $y < 0$. Infine, se $\alpha < 0$ non ha senso $y^\alpha$ per $y=0$.\newline
\newline
Per evitare complicazioni ci occuperemo solo di soluzioni non negativa ($y \geq 0$).\newline
\newline
\textbf{teor.} Siano $t_0 \in \mathbb{R}$, $y_0 \geq 0$ e siano $a,b \in C^0$ in un intorno di $t_0$. Allora il problema di Cauchy $y(t_0) = y_0$ per un equazione di Bernoulli ammette una e una sola soluzione nei seguenti casi:
\[
    \alpha> 1 \;\;\;\;\; y_0 \geq 0
\]
\[
    0< \alpha< 1 \;\;\;\;\;y_0 > 0
\]
\[
    \alpha<0 \;\;\;\;\;y_0 > 0
\]
Se $\alpha > 0$ l'equazione di bernoulli ammette anche la soluzione nulla $y=0$, pertanto se  $\alpha > 1$ e $y_0 = 0$, il problema di Cauchy ammette solo la soluzione nulla.\newline
Nel caso $y_0 = 0$ e $0<\alpha < 1$, è garantita l'esistenza di una soluzione per il problema di Cauchy $y(t_0) = 0$ ma non ne è assicurata l'unicità (potrebbe crearsi un pannello di Peano).\newline
Se $y_0 > 0$ la soluzione rimarrà strettamente positiva per $\alpha > 1$, potrenne agganciarsi alla soluzione $y= 0$ con un pannello di peano se $0< \alpha<1$, potrebbe tendere a $0$ con tangente verticale se $\alpha<0$ (in quest'ultimo caso, bisogna verificare il comportamento della funzione $b(t)$ che, annullandosi, potrebbe verificare l'infinito di $y^\alpha$).\newline
\newline
Risolvere un equazione di Bernoulli:\newline
Si divide l'equazione per $y^\alpha$ e si ottiene $y^{-\alpha} y' = a(t) y^{1-\alpha} + b(t)$.\newline
Si pone $z(t) = y(t)^{1-\alpha}$ e si ottiene $z' = (1-\alpha)a(t)z + (1-\alpha) b(t)$, che è un'equazione lineare e quindi possiamo risolvere con i soliti metodi. Una volta trovata la soluzione $z(t)$ (che sarà $\geq 0$ o $>0$ a seconda dei valori di $\alpha$), si determina $y(t) = z(t)^{\frac{1}{1-\alpha}}$\newline
\rule{\textwidth}{0,4pt}
\subsubsection*{Prolungamento delle soluzioni}

\rule{\textwidth}{2pt}
\subsection*{Equazioni lineari del secondo ordine}
\rule{\textwidth}{0,4pt}
\subsubsection*{Spazi di funzioni}
Sia $I$ un intervallo e consideriamo l'insieme $\mathbb{F}$ di tutte le funzioni definite in $I$, a valori reali. Con le operazioni naturali di somma di due funzioni e prodotto per uno scalare:
\[
    (f+g)(x) = f(x) + g(x)
\]
\[
    (\lambda f)(x) = \lambda f()
\]
$\mathbb{F}$ risulta essere uno spazio vettoriale. \newline
\newline
\textbf{def.} definiamo $C^n(I)$ lo spazio delle funzioni dotate di derivata $n$-essima continua.\newline
\rule{\textwidth}{0,4pt}
\subsubsection*{Generalità sulle equazioni lineari. Problema di Cauchy}
Un equazione differenziale del secondo ordine si dice lineare se è del tipo
\[
    a_2(t) y'' + a_1(t) y' +a_0(t) y = g(t)
\]
dove le funzioni $a_i$ e il termine noto $g$ sono funzioni continue in un certo intervallo $I$.\newline
Se il termine noto è nulla l'equazione si dice omogenea, altrimenti si dice completa.\newline
Se i coefficienti $a_i$ sono costanti, l'equazione si dirà a coefficienti costanti, altrimenti a coefficienti variabili.\newline
Se $a_2 = 1$ l'equazione si dirà in forma normale (se in un equazione il coefficiente $a_2$ non si annulla mai possiamo riscrivere l'equazione in fomra normale dividendo per questo).\newline
\newline
Nella soluzione si avranno sempre due coefficienti $c_1$ e $c_2$, per selezionare una soluzione particolare avremo bisogno di due condizioni iniziali:
\[
    \begin{cases}
        y(t_0) = y_0&\\
        y'(t_0) = y_1
    \end{cases}
\]
che insieme all'equazione iniziale prenderà il nome di problema di Cauchy.\newline
\newline
\begin{tcolorbox}
\textbf{teor.} (per funzioni del secondo ordine in forma normale)\newline
Se $a, b, f$ sono funzioni continue in un intervallo $I$ contenente il punto $t_0$, per ogni $y_0, y_1 \in \mathbb{R}$ il problema di Cauchy
\[
    \begin{cases}
        y'' + a(t) y' + b(t) y = f(t)\\
        y(t_0) = y_0\\
        y'(t_0) = y_1
    \end{cases}
\]
ha una e una sola soluzione $y \in C^2(I)$.\newline
Tale soluzione è individuata imponendo le condizioni iniziali nell'espressione che assegna l'integrale generale dell'equazione.\newline
\end{tcolorbox}
\rule{\textwidth}{0,4pt}
\subsubsection*{La struttura dell'integrale generale}
\textbf{teor.} Struttura dell'integrale generale dell'equazione lineare completa \newline
a. $\;\;\;$ L'insieme delle soluzioni dell'equazione omogenea $Lz = 0 $ con $L:C^2(I)\rightarrow C^0(I)$ in un dato intervallo $I$ è uno spazio vettoriale (sottospazio di $C^2(I)$).\newline
b. $\;\;\;$ L'integrale generale dell'equazione completa di ottiene sommando l'integrale generale dell'equazione omogenea e una soluzione particolare dell'equazione completa.\newline
\newline
\textbf{teor.} Proprietà di un equazione omogenea del secondo ordine.\newline
Lo spazio vettoriale delle soluzioni di un'equazione lineare omogenea del secondo ordine ha dimensione due.\newline
Significa che esistono $2$ soluzioni ($z_1, z_2$) tali che:\newline
1. $\;\;\;$ sono linearmente indipendenti \newline
2. $\;\;\;$ ogni altra soluzione è combinazione lineare di queste due soluzioni. \newline
3. $\;\;\;$ L'integrale generale dell'equazione omogenea è assegnato dalla formula
\[
    c_1 z_1(t) + c_2 z_2(t)
\]
\newline
\textbf{teor.} Determinante Wronskiano e indipendenza. \newline
Siano $z_1$ e $z_2$ due funzioni $C^2(I)$, soluzioni di un equazione lineare omogenea di secondo ordine nell'intervallo $I$. Allora esse sono linearmente indipendenti in $C^2(I)$ se e solo se la seguente matrice
\[
    \begin{matrix}
        z_1(t) \;\; & z_2(t)\\
        z_1'(t) \;\; & z_2'(t)
    \end{matrix}
\]
detta matrice Wronskiana, ha determinante diverso da $0$ per ongi $t \in I$. Inoltre, affinché questo accada, è sufficiente che il determinante si adiverso da $0$ in un punto $t_0 \in I$ (il determinante o si annulla in tutti i punti o è diverso da $0$ in tutti i punti, se è diverso da zero in tutti i punti $z_1, z_2$ sono indipendenti).
\begin{tcolorbox}
Per determinare l'integrale generale di un'equazione differenziale completa del secondo ordine si riconduce ai due passi seguenti:\newline
1) $\;\;\;$ determinare l'integrale generale dell'equazione omogenea corrispondente, cioè due soluzioni $z_1(t), z_2(t)$ linearmente indipendenti.\newline
2) $\;\;\;$ determinare una soluzione particolare $\bar{y}(t)$ dell'equazione completa.\newline
L'integrale generale avrà dunque la forma:
\[
    \bar{y}(t) + c_1 z_1(t) + c_2 z_2(t)
\]
\end{tcolorbox}
\rule{\textwidth}{0,4pt}
\subsubsection*{Equazioni omogenee a coefficienti costanti}
\begin{tcolorbox}
\[
    z''(t) + a z'(t) + bz(t) = 0
\]
\end{tcolorbox}
Sostituendo $z(t) = e^{rt}$
\[
    e^{rt}(r^2 + a r + b) = 0
\]
Calcoliamo il $\Delta$ di $r^2 + a r + b$, detta equazione caratteristica:
\begin{itemize}
    \item $\Delta>0$, due radici reali distinte $r_1$ e $r_2$, soluzione:
    \begin{tcolorbox}
    \[
        z(t) = c_1e^{r_1t} + c_2e^{r_2t}
    \]
    \end{tcolorbox}
    \item $\Delta <0$, due radici complesse $r_1 =\alpha + i \beta$, $r_2 = \alpha - i \beta$, soluzione:
    \[
        z_1(t) = e^{(\alpha + i \beta)t} =e^{\alpha t} (cos(\beta t) + i sin(\beta t)) = e^{\alpha t} cos(\beta t)
    \]
    \[
        z_2(t) = e^{(\alpha-i \beta)t} = e^{\alpha t}(cos(\beta t) + i sin(\beta t)) = e^{\alpha t} sin (\beta t)
    \]
    da cui
    \begin{tcolorbox}
    \[
        z(t) = e^{\alpha t}(c_1 cos(\beta t) + c_2 sin (\beta t))
    \]
    \end{tcolorbox}
    quest'ultima espressione può essere riscritta anche come $z(t) = e^{\alpha t} A cos(\beta t + \phi)$, con $A$ e $\phi$ costanti reali arbitrarie.
    \item $\Delta = 0$, unica radice $r=\frac{-a}{2}$, soluzioni:
    \[
        e^{rt} \quad \quad \quad te^{rt}
    \] 
    \begin{tcolorbox}
    \[
        z(t) = e^{rt}(c_1 + c_2 t)
    \]
    \end{tcolorbox}
\end{itemize}
\rule{\textwidth}{0,4pt}
\subsubsection*{Equazioni completa a coefficienti costanti}
\[
    y''(t) + a y'(t) + b y(t) = f(t)
\]
Vediamo per prima cosa il 
\subsubsection*{metodo di somiglianza}
Si analizza il termine noto $f(t)$:
\begin{itemize}
    \item $f(t) = p_r(t)$, dove $p_r(t)$ è un polinomio di grado $r$, si cerca una soluzione di tipo polinomiale:
    \[
        \begin{matrix}
            y(t) = q_r(t)  &se &b\neq0\\
            y(t) = tq_r(t) &se &b=0, a\neq 0\\
            y(t)=t^2q_r(t) &se &b=0, a=0
        \end{matrix}
    \]
    dove $q_r(t)$ è un generico polinomio di grado $r$ di cui occorre determinare i coefficienti.
    \item $f(t) = Ae^{\lambda t}$, con $\lambda \in \mathbb{C}$. Si cerca una soluzione del tipo $y(t) = e^{\lambda t}\gamma(t)$:
    \[
        \gamma'' + \gamma'(2 \lambda + a) + \gamma (\lambda^2 + a \lambda + b) = A
    \]
    Se:
    \begin{itemize}
        \item se $\lambda^2 + a \lambda + b \neq 0$
        \[
            \gamma(t) = costante = \frac{A}{\lambda^2 + a \lambda + b}
        \]
        \[
            y(t) =\frac{Ae^{\lambda t}}{\lambda^2 + a \lambda + b}
        \]
        \item se $\lambda^2 + a \lambda + b = 0$, ma $2 \lambda + a \neq 0$
        \[
            \gamma'(t) = costante = \frac{A}{2 \lambda + a}
        \]
        \[
            \gamma(t) = \frac{At}{2 \lambda + a}
        \]
        \[
            y(t) = \frac{At e^{\lambda t}}{2 \lambda + a}
        \]
        \item se $\lambda^2 + a \lambda + b = 0$, ma $2 \lambda + a = 0$
        \[
            \gamma''= A
        \]z
        \[
            \gamma(t) = \frac{A}{2}t^2
        \]
        \[
            y(t) = \frac{A}{2}t^2e^{\lambda t}
        \]
    \end{itemize}
    Nella classe dei termini noti $e^{\lambda t}$ con $\lambda \in \mathbb{C}$, rientrano anche:
    \[
        cos(\omega t), \quad sin(\omega t), \quad e^{\lambda t}cos(\omega t), \quad e^{\lambda t}sin(\omega t)
    \]
    Ricordiamo la formula di Eulero:
    \[
        re^{i \theta} =r[cos(\theta) + i sin(\theta)]
    \]
    Quindi se si è in presenza di un $sin(\omega t)$ come termine noto, si può studiare l'equazione che ha come termine noto $cos(\omega t) + i \cdot sin(\omega t)$ e poi prenderne la parte immaginaria.\newline
    Viceversa se si è in presenza di una $cos(\omega t)$ come termine noto, si può studiare l'equazione che ha come termine noto $cos(\omega t) + i \cdot sin(\omega t)$ e poi prenderne la parte reale.\newline
    \begin{tcolorbox}
        In ogni caso, per facilitare la derivazione, si trasforma $cos(\omega t) + i \cdot sin(\omega t)$ in $e^{i\omega t}$ e si procede nello studio di un termine noto della forma $f(t) = Ae^{\lambda t}$.
    \end{tcolorbox}
    Fa eccezione a questa tipologia il caso in cui manchi il termine in $y'$.\newline
    \newline
\end{itemize}
\subsubsection*{Metodo di sovrapposizione}
\begin{tcolorbox}
se, per esempio, il termine noto $f(t) =$ polinomio $+$ una funzione trigonometrica, si può trovare una soluzione per $f(t) =$ polinomio, e una per $f(t) =$ funzione trigonometrica e sommando le soluzioni si trova una soluzione dell'equazione di partenza.
\end{tcolorbox}
\subsubsection*{Metodo di variazione delle costanti}
Illustriamo ora un metodo generale che consente di determinare una soluzione particolare qualunque sia la forma del termine noto.\newline
Il metodo è applicabile purchè si conoscando già due soluzioni $z_1, z_2$ indipendenti dell'equazione omogenea associata.\newline
Dovremo quindi trovare le due funzioni $c_1$ e $c_2$ tali che
\[
    \begin{cases}
        c_1'z_1 + c_2'z_2 = 0 \\
        c_1'z_1' + c_2' z_2' = f
    \end{cases}
\]
cioè
\[
    c_1' = \frac{-z_2 f}{z_2'z_1 - z_2 z_1'}
\]
\[
    c_2' = \frac{z-1 f}{z_2'z_1 - z_2 z_1'}
\]
Dobbiamo quindi trovare due primitive di $c_1'$ e $c_2'$ e sostituire in:
\[
    y(t) =(c_1(t)+ k_1)z_1(t) + (c_2(t)+k_2)z_2(t)
\]
con $k_i$ costanti arbitrarie di integrazione.\newline
\rule{\textwidth}{0,4pt}
\subsubsection*{Note sugli esercizi}
\[
    \int e^{ax}cos(bx) dx = Re (\int e^{a+ib}x dx ) 
\]
\[
    \int e^{ax}sin(bx) dx = Im (\int e^{a+ib}x dx ) 
\]
\[
    \int e^{(a+ib)x} dx = \frac{1}{a+ib}e^{(a+ib)x} = \frac{e^{ax}}{a^2+b^2} (a-ib)(cos(bx) + i sin(bx))
\]
\subsubsection*{Equazione di Eulero}
Per risolvere:
\begin{tcolorbox}
\[
    ax^2y''+bxy'+cy= 0
\]
\end{tcolorbox}
si risolve l'equazione di secondo grado:
\[
    ar(r-a)+br+c = 0
\]
\begin{itemize}
    \item $\Delta >0$, soluzioni $r_1, r_2$
    \begin{tcolorbox}
    \[
        y(x) =c_1x^{r_1}+c_2x^{r_2}
    \]
    \end{tcolorbox}
    per $x>0$
    \item $\Delta <0$, soluzioni $r_{1,2} = \alpha \pm i \beta$
    \begin{tcolorbox}
    \[
        y(x) = x^{\alpha}(c_1 cos(\beta log(x)) + c_2 sin(\beta log(x)))
    \]
    \end{tcolorbox}
    per $x>0$
    \item $\Delta = 0$, soluzioni coincidenti $r$
    \begin{tcolorbox}
    \[
        y(x) = x^r(c_1 + c_2log(x))
    \]
    \end{tcolorbox}
    per $x>0$
\end{itemize}
Per $x<0$, si studia ancora la stessa equazione di secondo grado e si hanno soluzioni analoghe, ma con $x$ sostituita da $-x$.\newline
\rule{\textwidth}{0,4pt}\newline
\begin{tcolorbox}
\begin{center}
    \includegraphics[height=250px]{../img/eqdiff2.PNG}
\end{center}
\begin{center}
    \includegraphics[height=250px]{../img/eqdiff2(1).PNG}
\end{center}
\end{tcolorbox}
\rule{\textwidth}{2pt}
\subsection*{Sistemi lineari omogenei}
\[
    y' = A(t) y
\]
viene detto sistema omogeneo.\newline
\[
    \begin{cases}
        y' = A(t) y\\
        y(t_0) = y_0
    \end{cases}
\]
è il problema di cauchy associato.\newline
\newline
Se $y_0 = 0$, allora l'unica soluzione è $y(t) = 0$.\newline
\newline
Se $\phi_1$ e $\phi_2$ risolvono il sistema omogeneo, anche $\alpha \phi_1 + \beta \phi_2$ per ogni $\alpha, \beta \in \mathbb{R}$ lo risolverà.\newline
\newline
Se $A$ è una matrice costante il sistema si dice a coefficienti costanti.\newline
\newline
Sia un matrice $M$, vogliamo calcolare $e^{M}$. \newline
Se $M$ è diagonale si ottiene facilmente:
\[
    M = \left( \begin{matrix}
        \lambda_1  & 0 &\dots & 0\\
        0 & \lambda_2 &\dots &0\\
        0 & 0 &\dots &0\\
        0 & 0 &\dots &\lambda_n
    \end{matrix} \right) \Rightarrow e^M = \left( \begin{matrix}
        e^{\lambda_1}  & 0 &\dots & 0\\
        0 & e^{\lambda_2} &\dots &0\\
        0 & 0 &\dots &0\\
        0 & 0 &\dots & e^{\lambda_n}
    \end{matrix} \right)
\]
Se $M$ non è diagonale, ma è diagonalizzabile, e cioè esiste una matrice non singolare $S$ tale che $\Lambda = S^{-1}MS$ sia diagonale, allora
\[
    M = S\Lambda S^{-1} \Rightarrow  M^k = S\Lambda^kS^{-1} \Rightarrow e^M = S e^{\Lambda}S^{-1}
\]
con $e^{\Lambda}$ che si calcola come abbiamo visto per le matrici diagonali.\newline
\newline
Una matrice quadrata è diagonalizzabile se e solo se tutti i suoi autovalori sono regolari. \newline
\newline
Precisiamo che vengono considerati anche autovalori complessi che, per una matrice a coefficienti reali, possono esserci se e solo se c'è anche il loro coniugato. In tal caso, gli esponenziali dipendenti da tempo vanno interpretati con la formula di Eulero e generano funzioni trigonometriche.\newline
\newline
Abbiamo dunque visto come si trova l'esponenziale di una matrice diagonalizzabile costante. Volendo trovare l'esponenziale della matrice $At$, facciamo un paio di osservazioni elementari:
\begin{itemize}
    \item se A è diagonalizzabile, lo è anche $At$ e si può usare la stessa mtrice di passaggio $S$ per diagonalizzarla
    \item gli autovali di $At$ sono uguali agli autovalori di $A$ moltiplicati per $t$
\end{itemize}
Da queste osservazioni possiamo dedurre che
\[
    e^A  S e^{\Lambda}S^{-1} \Rightarrow  e^{At} = S e^{\Lambda t} S^{-1}
\]
\textbf{teor.} Le colonne della matrice $e^{At}$ formano un sistema fondamentale di soluzioni di $y' = Ay$ e cioè, per ogni $C \in \mathbb{R}^n$ il vettore $e^{At} C$è una soluzione di di $y'=Ay$\newline
\newline
La fuznione $\phi(t) = C e^{\lambda t} $ è soluzione di $y'=At$ se e solo se $\lambda$ è un autovalore di $A$ (possibilmente complesso) e $C$ è un autovettore associato a $\lambda$.\newline
\rule{\textwidth}{0,4pt}
\subsection*{Diagonalizzazione di una matrice}
Una matrice $A$ è diagonalizzabile se
\begin{itemize}
    \item Il numero degli autovalori di $A$ contati con la loro molteplicità è uguale all'ordine della matrice
    \item la molteplicità geometrica di ciascun autovalore coincide  con la realtiva molteplicità algebrica
\end{itemize}

Sia $A$ una matrice, i suoi autovalori si ottengono risolvendo
\[
    det(A-\lambda I ) = \left| \begin{matrix}
        a_{11} -\lambda & a_{12}\\
        a_{21} & a_{22} - \lambda
    \end{matrix} \right| = 0
\]
risolvendo questa equazione per $\lambda$ otteniamo i vari autovalori.\newline
La molteplicità algebrica consiste nel quante volte $\lambda$ appare come soluzione dell'equazione precedente.\newline
Perchè $A$ sia diagonalizzabile, la somma della molteplicità algebrica di ogni autovalore deve essere uguale all'ordine della matrice.\newline
Perchè $A$ sia diagonalizzabile bisogna anche verificare che la molteplicità algebrica di ogni autovalore coincide con la realtiva molteplicità geometrica, che si calcola così:
\[
    m_g(\lambda) = n - rk(A- \lambda I)
\]
dove $n$ è l'ordine di $A$.\newline
\newline
\newline
Se $A$ è diagonalizzabile, allora esiste una matrice $P$  che la diagonalizza e una matrice $D$ a cui $A$ è simile, per cui valga:
\[
    D = P A P^{-1}
\]
\begin{itemize}
    \item la matrice $D$ è una matrice diagonale i cui elementi della diagonale principale sono gli autovalori della matrice $A$. Gli autovalori con molteplicità algebrica maggiore di 1 vanno ripetuti più volte.
    \item la matrice $P$ è la matrice che ha come colonne gli autovettori associati a ogni autovalore, ossia ha come colonne i vettori che fromano le basi degli autospazi relativi a ciascun autovalore.
\end{itemize}
Affinchè tutto funzioni ci deve essere corrispondenza fra le matrici $D$ e $P$: la $j$-esima colonna della matrice $P$ contiene l'autovettore associato all'autovalore presente nella $j$-esima colonna della matrice $D$.\newline
\newline
Il calcolo degli autovettori relativi a un autovalore $\lambda$ si esegue risolvendo il sistema:
\[
    (A- \lambda I ) v = 0
\]
con $v = \binom{x}{y}$, e cioè risolvendo il sistema:
\[
    \left(\begin{matrix}
        a_{11} -\lambda & a_{12}\\
        a_{21} & a_{22} - \lambda
    \end{matrix} \right) \binom{x}{y} = 0
\]
\newline
Invece per calcolare l'inversa della matrice $P$ si seguono i seguenti passaggi:
\begin{itemize}
    \item Calcola la trasposta $A^T$ della matrice $A$ (basta scambiare tra loro le righe con le colonne)
    \item sostituire ogni elemento della matrice trasposta col il proprio complemento algebrico (complemento algebrico: preso l'elemento $a_{h,k}$ della matrice, il suo complemento algebrico si calcola come $(-1)^{(h+k)}\cdot C_{h,k}$, dove con $X_{h,k}$ si intende il determinante della matrice ottenuta da quella di partenza eliminando la riga $h$ e la colonna $k$)
    \item Adesso dividi la matrice dei complementi algebrici per det(A) (cioe' dividi ogni termine per det(A)) e ottieni l'inversa della matrice quadrata di partenza
\end{itemize} % equazioni differenziali:
    %- da confrontare con capitolo 7 del libro del prof
    %- aggiugnere capitolo 8 del libro del prof
\end{document}