\section*{2-Calcolo infinitesimale per le curve}
\rule{\textwidth}{2pt}
\subsection*{Richiami di calcolo vettoriale}
vettore
\[
    \vec{x} = (x_1, x_2, \dots, x_n)
\]
modulo di un vettore
\[
    |\vec{x}| = \sqrt{\sum_{i=1}^{n}x_i^2}
\]
versore
\[
    vers(\vec{x}) = \frac{\vec{x}}{|\vec{x}|}
\]
$\vec{x}, \vec{y}$ sono paralleli se
\[
    \lambda \vec{x} = \mu \vec{y}\;\; \text{ per qualche } \;\;\lambda, \mu \in \mathbb{R}
\]
Somma di vettori: si sommano le componenti simili.\newline
Prodotto fra un vettore e uno scalare: si moltiplica ogni componente per lo scalare.\newline
Prodotto scalare fra vettori: ha come risultato un numero reale ottenuto dalla formula $\vec{u} \cdot \vec{v} = (u_1 v_1 + u_2 v_2 + \dots + u_n v_n)$. Il prodotto scalare può essere espresso anche come
\[
    \vec{u} \cdot \vec{v} = |\vec{u}| |\vec{v}| cos(\theta)
\]
dove $\theta$ rappresenta l'angolo tra i due vettori. Di conseguenza $\vec{u} \cdot  \vec{v} = 0$ solo se i due vettori sono ortogonali.\newline
Dal prodotto scalare si può ricavare l'angolo fra due vettori
\[
    cos(\theta) = \frac{\vec{u} \cdot  \vec{v} }{|\vec{u}| |\vec{v}|}
\]
Inoltre $\vec{v} \cdot  \vec{v} = |\vec{v}|^2$.\newline
Prodotto vettoriale:
\[
    \vec{u}\text{x}\vec{v} = \left|\begin{matrix}
        i \;\; & j \;\;& k \;\;\\
        u_1 & u_2 & u_3\\
        v_1 & v_2 & v_3\\
    \end{matrix}\right|
\]
Il prodotto vettoriale si annulla solo se i vettori sono paralleli.\newline
Inoltre $\vec{v} \text{x} \vec{v} = 0$.\newline
Prodotto misto:
\[
    \vec{u} \cdot (\vec{v} \text{x} \vec{w}) = \left|\begin{matrix}
        u_1 \;\;& u_2 \;\;& u_3\\
        v_1 &v_2 & v_3\\
        w_1 &w_2 & w_3
    \end{matrix}\right|
\]
Il prodotto misto si annulla solo se i tre vettori sono linearmente indipendenti.\newline
\rule{\textwidth}{2pt}
\subsection*{Funzioni a valori vettoriali, limiti e continuità}
Si dice funzione a valori vettoriali una funzione $\vec{f} : \mathbb{R} \rightarrow  \mathbb{R}^n$ con $n > 1$. Le funzioni $\vec{f} : \mathbb{R} \rightarrow \mathbb{R}^2$ o $\mathbb{R}^3$.\newline
Il limite della funzione a valori vettoriali si calcola componente per componente:
\[
    \lim_{t\rightarrow t_0}(r_1(t), r_2(t), \dots, r_n(t)) = \left(\lim_{t\rightarrow t_0}r_1(t), \lim_{t\rightarrow t_0}r_2(t), \dots, \lim_{t\rightarrow t_0}r_n(t)\right)
\]
Valgono allo stesso modo delle funzioni unidimensionali il teorema di unicità del limite e la definizione di continuità (una funzione a valori vettoriali è continua in un punto se lo sono tutte le sue componenti).\newline
\rule{\textwidth}{2pt}
\subsection*{Curve regolari e calcolo differenziale vettoriale}
Nel caso $n = 2$ o $3$, una funzione $\vec{f} : \mathbb{R} \rightarrow  \mathbb{R}^n$ rappresentano curve nel piano o nello spazio tridimensionale.\newline
Sia $I$ un intervallo in $\mathbb{R}$. Si dice arco di curva continua, o cammino, in $\mathbb{R}^n$ una funzione $\vec{r}: I \rightarrow \mathbb{R}^n$ continua.\newline
La curva si dice chiusa se $\vec{r}(a) = \vec{r}(b)$ con $I=[a,b]$.\newline
La curva si dice semplice se non ripassa mai nello stesso punto.\newline
Il sostegno della curva è l'immagine della funzione, cioè l'insieme dei punti di $\mathbb{R}^n$ percorsi dal punto mobile.\newline
Una curva si dice piana se esiste un piano che contiene il suo sostegno.\newline
\rule{\textwidth}{0,4pt}
\subsubsection*{Derivata di una funzione vettoriale. Arco di curva regolare}
\textbf{def.} Sia $\vec{r}: I \rightarrow \mathbb{R}^n$ e $t_0 \in I$, si dice che $\vec{r}$ è derivabile in $t_0$ se esiste finito
\[
    \vec{r}'(t_0) = \lim_{h\rightarrow 0}\frac{\vec{r}(t_0 +h) - \vec{r}(t_0)}{h}
\]
Se $\vec{r}$ è derivabile in tutti $I$ e inoltre $\vec{r}'$ è continuo in $I$, si dice che $\vec{r}$ è di classe $C^1(I)$ ($\vec{r}\in C^1(I)$).\newline
Notiamo che il vettore derivato è il vettore delle derivate delle componeneti:
\[
    \vec{r'}(t_0) = (r'_1(t_0), r'_2(t_0), \dots, r'_n(t_0))
\]
\textbf{def.} Sia $I \subseteq \mathbb{R}$ un intervallo. Si dice araco di curva regolare un arco di curva $\vec{r}: I \rightarrow  \mathbb{R}^n$ tale che $\vec{r} \in C^1(I)$ e $\vec{r}'(t) \neq 0$ per ogni $t \in I$.\newline
Di conseguenza per le curve regolari è ben definito il versore tangente:
\[
    \vec{T} = \frac{\vec{r}'(t)}{|\vec{r}'(t)|}
\]
\textbf{def.} Si dice arco di curva regolare a tratti un arco di curva $\vec{r} : I \rightarrow \mathbb{R}^n$ tale che: $\vec{r}$ è continua e l'intervallo $I$ può essere suddiviso in un numero finito di sottointervalli, su ciascuno dei quali $\vec{r}$ è un arco di curva regolare.\newline
\newline
Alcune proprietà del calcolo differenziale vettoriale:
\[
    (\vec{u} + \vec{v})' = \vec{u}' + \vec{v}'
\]
\[
    (c \vec{u})' = c \vec{u}' \;\;\; \text{con c costante}
\]
\[
    (f \vec{u})' = f' \vec{u} + \vec{u}' f \;\;\;\text{con} \; f \; \text{funzione}
\]
\[
    [\vec{u}(f(t))]' = \vec{u}'(f(t))f'(t)
\]
\[
    (\vec{u} \cdot \vec{v})' = \vec{u}' \cdot  \vec{v} + \vec{u} \cdot  \vec{v}'
\]
\[
    (\vec{u} \text{x} \vec{v})' = \vec{u}' \text{x} \vec{v} + \vec{u} \text{x} \vec{v}' \;\;\; in \;\mathbb{R}^3
\]
\rule{\textwidth}{0,4pt}
\subsubsection*{Integrale di una funzione vettoriale}
\[
    \int_{a}^{b} \vec{r}(t) dt =\left( \int_{a}^{b}\vec{r_1}(t) dt,\int_{a}^{b}\vec{r_2}(t) dt, \dots, \int_{a}^{b}\vec{r_n}(t) dt,\right)
\]
Diremo che $\vec{r}$ è integrabile in $[a,b]$ se lo sono tutte le sue componenti.\newline
\[
    \int_{a}^{b} \vec{r}'(t) dt = \vec{r}(b) - \vec{r}(a)
\]
Può essere utile utilizzare il modulo vettoriale:
\[
    \left| \int_{a}^{b}\vec{r}(t)dt \right| \leq \int_{a}^{b}|\vec{r}(t)|dt
\]
Se $\vec{r}(t)$ è una curva regolare e chiusa in $[a,b]$, allora $\int_{a}^{b}\vec{r}'(t) dt =0$.\newline
\rule{\textwidth}{0,4pt}
\subsubsection*{Classi di curve piane}
\textbf{curve piane, grafico di funzioni}\newline
Curve ottenute da grafici di funzioni in una variabile:
\[
    y = f(x) \quad\text{per x in} \;[a,b]
\]
Forma parametrica:
\[
    \begin{cases}
        x=t \\
        y=f(t)
    \end{cases} \;\;\;\; \text{per t} \;\in[a,b]
\]
\textbf{curve piane, forma polare}\newline
L'equazione
\[
    \rho = f(\theta) \;\;\;\text{per } \; \theta \in [\theta_1, \theta_2]
\]
è una forma abbreviata che rappresenta:
\[
    \begin{cases}
        x = f(\theta) cos(\theta)\\
        y = f(\theta) sin(\theta)
    \end{cases} \;\;\; \text{per } \; \theta \in [\theta_1, \theta_2]
\]
Osserviamo che 
\[
    \vec{r}'(\theta) = (f'(\theta) cos(\theta)- f(\theta)sin(\theta), f'(\theta) sin(\theta) - f(\theta)cos(\theta))
\]
\[
    |\vec{r}'(\theta)| = \sqrt{f'(\theta)^2 + f(\theta)^2}
\]
\textbf{coniche in forma polare}\newline
Equazione polare della conica:
\[
        \rho = \frac{\epsilon p}{1- \epsilon cos(\theta)}
\]
con $\epsilon > 0$ fissato.\newline
Questa equazione rappresenta:
\[
    \begin{cases}
        \text{un'ellissi} \;\; &\text{se} \; \epsilon< 1\\
        \text{una parabola} \;\; &\text{se} \; \epsilon= 1\\
        \text{un'iperbole} \;\; &\text{se} \; \epsilon> 1
    \end{cases}
\]
Notiamo che il segmo $-$ davanti al coseno non è importante, infatti se cambiamo $\theta = t + \pi$ l'equazione si trasforma in $\rho = \frac{\epsilon p}{1 + \epsilon cos(t)}$.\newline
Equazione della circonferenza:
\[
        \rho = R
\]
\rule{\textwidth}{2pt}
\subsection*{Lunghezza di un arco di curva}
\rule{\textwidth}{0,4pt}
\subsubsection*{Curve rettificabili e lunghezza}
