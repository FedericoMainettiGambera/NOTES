\section*{2-Calcolo infinitesimale per le curve}
\rule{\textwidth}{2pt}
\subsection*{Richiami di calcolo vettoriale}
vettore
\[
    \vec{x} = (x_1, x_2, \dots, x_n)
\]
modulo di un vettore
\[
    |\vec{x}| = \sqrt{\sum_{i=1}^{n}x_i^2}
\]
versore
\[
    vers(\vec{x}) = \frac{\vec{x}}{|\vec{x}|}
\]
$\vec{x}, \vec{y}$ sono paralleli se
\[
    \lambda \vec{x} = \mu \vec{y}\;\; \text{ per qualche } \;\;\lambda, \mu \in \mathbb{R}
\]
Somma di vettori: si sommano le componenti simili.\newline
Prodotto fra un vettore e uno scalare: si moltiplica ogni componente per lo scalare.\newline
Prodotto scalare fra vettori: ha come risultato un numero reale ottenuto dalla formula $\vec{u} \cdot \vec{v} = (u_1 v_1 + u_2 v_2 + \dots + u_n v_n)$. Il prodotto scalare può essere espresso anche come
\[
    \vec{u} \cdot \vec{v} = |\vec{u}| |\vec{v}| cos(\theta)
\]
dove $\theta$ rappresenta l'angolo tra i due vettori. Di conseguenza $\vec{u} \cdot  \vec{v} = 0$ solo se i due vettori sono ortogonali.\newline
Dal prodotto scalare si può ricavare l'angolo fra due vettori
\[
    cos(\theta) = \frac{\vec{u} \cdot  \vec{v} }{|\vec{u}| |\vec{v}|}
\]
Inoltre $\vec{v} \cdot  \vec{v} = |\vec{v}|^2$.\newline
Prodotto vettoriale:
\[
    \vec{u}\text{x}\vec{v} = \left|\begin{matrix}
        i \;\; & j \;\;& k \;\;\\
        u_1 & u_2 & u_3\\
        v_1 & v_2 & v_3\\
    \end{matrix}\right|
\]
Il prodotto vettoriale si annulla solo se i vettori sono paralleli.\newline
Inoltre $\vec{v} \text{x} \vec{v} = 0$.\newline
Prodotto misto:
\[
    \vec{u} \cdot (\vec{v} \text{x} \vec{w}) = \left|\begin{matrix}
        u_1 \;\;& u_2 \;\;& u_3\\
        v_1 &v_2 & v_3\\
        w_1 &w_2 & w_3
    \end{matrix}\right|
\]
Il prodotto misto si annulla solo se i tre vettori sono linearmente indipendenti.\newline
\rule{\textwidth}{2pt}
\subsection*{Funzioni a valori vettoriali, limiti e continuità}
Si dice funzione a valori vettoriali una funzione $\vec{f} : \mathbb{R} \rightarrow  \mathbb{R}^n$ con $n > 1$. Le funzioni $\vec{f} : \mathbb{R} \rightarrow \mathbb{R}^2$ o $\mathbb{R}^3$.\newline
Il limite della funzione a valori vettoriali si calcola componente per componente:
\[
    \lim_{t\rightarrow t_0}(r_1(t), r_2(t), \dots, r_n(t)) = \left(\lim_{t\rightarrow t_0}r_1(t), \lim_{t\rightarrow t_0}r_2(t), \dots, \lim_{t\rightarrow t_0}r_n(t)\right)
\]
Valgono allo stesso modo delle funzioni unidimensionali il teorema di unicità del limite e la definizione di continuità (una funzione a valori vettoriali è continua in un punto se lo sono tutte le sue componenti).\newline
\rule{\textwidth}{2pt}
\subsection*{Curve regolari e calcolo differenziale vettoriale}
Nel caso $n = 2$ o $3$, una funzione $\vec{f} : \mathbb{R} \rightarrow  \mathbb{R}^n$ rappresentano curve nel piano o nello spazio tridimensionale.\newline
Sia $I$ un intervallo in $\mathbb{R}$. Si dice arco di curva continua, o cammino, in $\mathbb{R}^n$ una funzione $\vec{r}: I \rightarrow \mathbb{R}^n$ continua.\newline
