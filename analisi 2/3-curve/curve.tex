\section{Funzioni $R \rightarrow R^n$ ("Curve nel piano e nello spazio", "calcolo infinitesimale per le curve")}
\subsection{Introduzione alle funzioni $R \rightarrow R^n$}
\subsubsection{Definizioni e terminologia}
Si dice \textbf{funzione a valori vettoriali} una funzione $\vec{f} : \mathbb{R} \rightarrow  \mathbb{R}^n$ con $n > 1$.\newline
\newline
Il \textbf{limite} della funzione a valori vettoriali si calcola componente per componente:
\[
    \lim_{t\rightarrow t_0}(r_1(t), r_2(t), \dots, r_n(t)) = \left(\lim_{t\rightarrow t_0}r_1(t), \lim_{t\rightarrow t_0}r_2(t), \dots, \lim_{t\rightarrow t_0}r_n(t)\right)
\]
Valgono allo stesso modo delle funzioni unidimensionali il teorema di unicità del limite e la definizione di \textbf{continuità} (una funzione a valori vettoriali è continua in un punto se lo sono tutte le sue componenti).\newline
\newline
Nel caso $n = 2$ o $3$, una funzione $\vec{f} : \mathbb{R} \rightarrow  \mathbb{R}^n$ rappresenta una curva nel piano o nello spazio tridimensionale.\newline
\newline
Sia $I$ un intervallo in $\mathbb{R}$. Si dice \textbf{arco di curva continua}, o \textbf{cammino}, in $\mathbb{R}^n$ una funzione $\vec{r}: I \rightarrow \mathbb{R}^n$ continua.\newline
\newline
Una curva si dice \textbf{chiusa} se $\vec{r}(a) = \vec{r}(b)$ con $I=[a,b]$.\newline
\newline
Una curva si dice \textbf{semplice} se non ripassa mai nello stesso punto, cioè se $r:(a,b) \rightarrow \mathbb{R}^n$ è iniettiva..\newline
\newline
Una curva si dice \textbf{piana} se esiste un piano che contiene il suo sostegno (in particolare tutte le curve  $f: \mathbb{R} \rightarrow \mathbb{R}^n$ con $n=2$).\newline
\newline
Il \textbf{sostegno} della curva è l'immagine della funzione, cioè l'insieme dei punti di $\mathbb{R}^n$ percorsi dal punto mobile.\newline
\newline
Due curve si dicono \textbf{equivalenti} se hanno lo stesso sostegno.
\subsubsection{Calcolo differenziale per funzioni $R \rightarrow R^n$}
Sia $\vec{r}: I \rightarrow \mathbb{R}^n$ e $t_0 \in I$, si dice che $\vec{r}$ è \textbf{derivabile} in $t_0$ se esiste finito
\[
    \vec{r}'(t_0) = \lim_{h\rightarrow 0}\frac{\vec{r}(t_0 +h) - \vec{r}(t_0)}{h}
\]
Se $\vec{r}$ è derivabile in tutto $I$ e inoltre $\vec{r}'$ è continuo in $I$, si dice che $\vec{r}$ è di \textbf{classe} $C^1(I)$ ($\vec{r}\in C^1(I)$).
\newline
\newline
Notiamo che il \textbf{vettore derivato} è il \textbf{vettore delle derivate} delle componeneti:
\[
    \vec{r'}(t_0) = (r'_1(t_0), r'_2(t_0), \dots, r'_n(t_0))
\]
\ \newline
Sia $I \subseteq \mathbb{R}$ un intervallo. Si dice arco di curva \textbf{regolare} un arco di curva $\vec{r}: I \rightarrow  \mathbb{R}^n$ tale che $\vec{r} \in C^1(I)$ e $\vec{r}'(t) \neq 0$ per ogni $t \in I$ (solo nell'intervallo aperto, queste due proprietà possono non verificarsi agli estremi).\newline
Il fatto che $\vec{r}' (t) \neq 0$ è una proprietà il cui significato è vettoriale: le componenti di $\vec{r}'(t)$ non possono annullarsi contemporanemante, e cioè che $|\vec{r}'(t)|\neq 0$, cioè che il punto mobile non si ferma mai ($\vec{r}'(t)$ rappresenta la velocità).\newline
\newline
Dato un vettore $\vec{r}(t)$ si dice
\begin{itemize}
    \item \textbf{velocità}: $\vec{r}'(t)$
    \item \textbf{velocità scalare}: $|\vec{r}'(t)|$
    \item \textbf{accellerazione}: $\vec{r}''(t)$
    \item \textbf{accellerazione scalare}: $|\vec{r}''(t)|$
\end{itemize}
\ \newline
Di conseguenza per le curve regolari è ben definito il \textbf{versore tangente}:
\[
    \vec{T} = \frac{\vec{r}'(t)}{|\vec{r}'(t)|}
\]
\ \newline
Si dice arco di curva \textbf{regolare a tratti} un arco di curva $\vec{r} : I \rightarrow \mathbb{R}^n$ tale che: $\vec{r}$ è continua e l'intervallo $I$ può essere suddiviso in un numero finito di sottointervalli, su ciascuno dei quali $\vec{r}$ è un arco di curva regolare.\newline
\newline
Alcune \textbf{proprietà} del calcolo differenziale vettoriale:
\[
    (\vec{u} + \vec{v})' = \vec{u}' + \vec{v}'
\]
\[
    (c \vec{u})' = c \vec{u}' \;\;\; \text{con c costante}
\]
\[
    (f \vec{u})' = f' \vec{u} + \vec{u}' f \;\;\;\text{con} \; f \; \text{funzione}
\]
\[
    [\vec{u}(f(t))]' = \vec{u}'(f(t))f'(t)
\]
\[
    (\vec{u} \cdot \vec{v})' = \vec{u}' \cdot  \vec{v} + \vec{u} \cdot  \vec{v}'
\]
\[
    (\vec{u} \text{x} \vec{v})' = \vec{u}' \text{x} \vec{v} + \vec{u} \text{x} \vec{v}' \;\;\; in \;\mathbb{R}^3
\]
\subsubsection{Calcolo integrale per funzioni $R \rightarrow R^n$}
\[
    \int_{a}^{b} \vec{r}(t) dt =\left( \int_{a}^{b}\vec{r_1}(t) dt,\int_{a}^{b}\vec{r_2}(t) dt, \dots, \int_{a}^{b}\vec{r_n}(t) dt,\right)
\]
Diremo che $\vec{r}$ è \textbf{integrabile} in $[a,b]$ se lo è ognuna delle sue componenti.\newline
\[
    \int_{a}^{b} \vec{r}'(t) dt = \vec{r}(b) - \vec{r}(a)
\]
Può essere utile utilizzare il seguente lemma:
\[
    \left| \int_{a}^{b}\vec{r}(t)dt \right| \leq \int_{a}^{b}|\vec{r}(t)|dt
\]
Se $\vec{r}(t)$ è una curva regolare e chiusa in $[a,b]$, allora $\int_{a}^{b}\vec{r}'(t) dt =0$.\newline
\subsubsection{Classi di curve piane (non mi piace sto titolo)}
\textbf{curve piane, grafico di funzioni}\newline
\newline
Curve ottenute da grafici di funzioni in una variabile:
\[
    y = f(x) \quad\text{per x in} \;[a,b]
\]
Forma parametrica:
\[
    \begin{cases}
        x=t \\
        y=f(t)
    \end{cases} \;\;\;\; \text{per t} \;\in[a,b]
\]
Proprietà:
\begin{itemize}
    \item è continua se e solo se $f$ è continua
    \item è regolare se e solo se $f$ è derivabile con continuità (le condizioni di non annullamento della derivata prima sono automaticamente verificate perchè $x'(t) = 1$)
    \item è regolare a tratti se e solo se $f$ è continua e a tratti derivabile con continuità
    \item non è mai chiusa
    \item è sempre semplice
\end{itemize}
\ \newline
\textbf{curve piane, forma polare}\newline
\newline
L'equazione
\[
    \rho = f(\theta) \;\;\;\text{per } \; \theta \in [\theta_1, \theta_2]
\]
è una forma abbreviata che, tramite la sostituzione $x= \rho cos(\theta)$ e $y= \rho sin(\theta)$, può essere riscritta:
\[
    \begin{cases}
        x = f(\theta) cos(\theta)\\
        y = f(\theta) sin(\theta)
    \end{cases} \;\;\; \text{per } \; \theta \in [\theta_1, \theta_2]
\]
\ \newline
Ricordiamo che $\rho = \sqrt{x^2 + y^2}$
Osserviamo che 
\[
    \vec{r}'(\theta) = (f'(\theta) cos(\theta)- f(\theta)sin(\theta), f'(\theta) sin(\theta) - f(\theta)cos(\theta))
\]
\[
    |\vec{r}'(\theta)| = \sqrt{\rho(\theta)^2 + \rho'(\theta)^2} = \sqrt{f'(\theta)^2 + f(\theta)^2}
\]
Geometricamente la forma polare $\rho = f(\theta)$ può essere visualizzata come la curva tracciata da una penna posta su un braccio che ruota attorno all'origine a velocità costante (in modo che il tempo $t$ combaci con l'angolo $\theta$). Mentre il braccio ruota la penna si sposta lungo il braccio in modo da essere a distanza $f(\theta)$ dall'origine all'istante $\theta$.\newline
\newline
Proprietà:
\begin{itemize}
    \item è continua se e solo se $f$ è continua
    \item è regolare se e solo se $f$ è derivabile con continuità, inoltra $f$ e $f'$ non si annullano mai contemporaneamente
    \item è chiusa se e solo se $f(\theta_1) = f(\theta_2)$ e $\theta_2 - \theta_1 = 2n\pi$ per qualche intero $n$.
\end{itemize}
\textbf{coniche in forma polare}\newline
Equazione polare della conica:
\[
        \rho = \frac{\epsilon p}{1- \epsilon cos(\theta)}
\]
con $\epsilon > 0$ e $p> 0$ e $\theta$ varia nell'intervallo in cui il secondo membro è definito e positivo.\newline
Questa equazione rappresenta:
\[
    \begin{cases}
        \text{un'ellissi} \;\; &\text{se} \; \epsilon< 1\\
        \text{una parabola} \;\; &\text{se} \; \epsilon= 1\\
        \text{un'iperbole} \;\; &\text{se} \; \epsilon> 1
    \end{cases}
\]
Notiamo che il segmo $-$ davanti al coseno non è importante, infatti se cambiamo $\theta = t + \pi$ l'equazione si trasforma in $\rho = \frac{\epsilon p}{1 + \epsilon cos(t)}$.\newline
Equazione della circonferenza:
\[
        \rho = R
\]
\subsection{Lunghezza di un arco di curva}
\subsubsection{Curve rettificabili e lunghezza}
Si dice che $\gamma$ è \textbf{rettificabile} se 
\[
    sub_\mathcal{P} l(\mathcal{P}) = l(\gamma) < + \infty
\]
Dove l'estremo superiore è calcolato al variare di tutte le possibili partizioni $\mathcal{P}$ di $[a,b]$.\newline
In tal caso $l(\gamma)$ assegna per definizione, la lunghezza di $\gamma$.\newline
\newline
\textbf{teor.} Sia $\vec{r}:[a,b] \rightarrow \mathbb{R}^n$ la parametrizzazione di un arco di curva $\gamma$ regolare. Allora $\gamma$ è rettificabile e la sua \textbf{lunghezza} vale 
\[
    l(\gamma) = \int_{a}^{b}|\vec{r}'(t)| dt
\]
Nello spazio tridimensionale la formula diventa:
\[
    l(\gamma)= \int_{a}^{b}\sqrt{x'(t)^2 + y'(t)^2 + z'(t)^2}dt
\]
(Da un punto di vista fisico queste due formule ci dicono che lo spostamento è l'integrale della velocità, cosa ben nota).
\ \newline
Inoltre:\newline
Sia $\gamma$ l'unione di due curve rettificabili, allora $\gamma$ è rettificabile, la stessa proprietà si estende a un numero finito qualsiasi di curve rettificabili. \newline
Se una curva $\gamma$ è regolare a tratti, allora è rettificabile.\newline
\newline
Vediamo il caso in si vuole calcolare la \textbf{lunghezza di un grafico di una funzione} $f(x)$:\newline
Sia $\gamma$ una curva piana regolare che sia grafico di una funzione, ossia:
\[
    \gamma : \begin{cases}
        x =t \\
        y= f(t)
    \end{cases} \;\; \text{per} \;t \in[a,b]
\]
allora
\[
    l(\gamma) = \int_{a}^{b}\sqrt{1+f'(t)^2}dt
\]
\subsubsection{Riparametrizzazioni e parametro arco (o ascissa curvilinea)}
Sostanzialmente, due curve equivalenti dipendono da un cambio di variabile. Se $r = r(t)$ con $t \in[a,b]$ ha per sostegno $\gamma$ e se $\phi: [c,d] \rightarrow [a,b]$ è una funzione continua e strettamente monotona (crescente o descrescente), allora la curva $\rho(t) := r(\phi(t))$, con $t \in [c,d]$, ha lo stesso sostegno di $\gamma$. Si usa dire che $\rho(t)$ è una \textbf{riparametrizzazione} di $r(t)$.\newline
Notiamo che dopo una riparametrizzazione la lunghezza dell'arco di curva rimane la stessa, anche se il verso o la velocità di percorrenza cambiano (es. per invertire il senso di percorrenza si può riparametrizzare $t$ con $-t$).\newline
\newline
Se invece di calcolare la lunghezza di una curva da un valore $a$ a un valore $b$, cioè in $[a,b]$, la calcolassimo in $[t_0, t]$, con $t_0$ un effettivo valore e $t$ una variabile, otterremmo una funzione di $t$ definita come:
\[
    s(t) = \int_{t_0}^{t}|\vec{r}'(\tau)|d\tau
\]
Inoltre se si è in grado di calcolare esplicitamente tale funzione e poi di invertirla, esprimendo $t$ come funzione di $s$, è possibile riparametrizzare la curva in funzione del parametro $s$, detto \textbf{parametro arco} o \textbf{ascissa curvilinea} (ricordarsi di ricalcolare anche gli estremi dell'intervallo secondo il nuovo parametro).\newline
\newline
Notiamo che se $|\vec{r}'(t)| = 1$, $t$ coincide con il parametro arco, quindi la curva sarebbe già parametrizzata secondo  il parametro arco.\newline
\newline
Se $\vec{r} = \vec{r}(s)$ è un curva parametrizzata mediante il parametro arco, il vettore derivato $\vec{r}'(s)$ coincide col versore tangente $\vec{T}$
\newline
\newline
Se $\vec{r}(t)$ è una curva parametrizzata rispetto a un parametro $t$ qualunque (non necessariamente il parametro arco), si ha:
\[
    \frac{ds}{dt} =|\vec{r}'(t)| = v(t)
\]
che si riscrive anche nella forma
\[
    ds =|\vec{r}'(t)|dt =v(t) dt
\] 
dove il simbolo $ds$ prende il nome di \textbf{lunghezza d'arco elementare}.
\subsection{Inegrali di linea (di prima specie)}
Sia $\vec{r}: [a,b] \rightarrow \mathbb{R}$ un arco di curva regolare di sostegno $\gamma$ e sia $f$ una funzione a valori reali, definita in un sottoinsieme $A$ di $\mathbb{R}^n$ contenente $\gamma$, cioè $f:A\subset \mathbb{R}^n \rightarrow \mathbb{R}$ con $\gamma \subset A$.\newline
\newline
Si dice \textbf{integrale di linea (di prima specie)} di $f$ lungo $\gamma$ l'integrale
\[
    \int_{\gamma} f ds = \int_{a}^{b}f(\vec{r}(t))|\vec{r}'(t)|dt
\]
Lintegrale di $f$ di prima specie lungo $\gamma$ è \textbf{invariante} per \textbf{parametrizzazioni equivalenti} e anche per \textbf{cambiamento di orientazione}.\newline
\newline
Osserviamo che se $f = 1$ (soffitto di altezza costatne), ritroviamo la lunghezza di una linea.
\textbf{Applicazioni fisiche}:
\begin{itemize}
    \item Il \textbf{baricentro} di $\gamma$ è il punto $B = (\bar{x},\bar{y},\bar{z})$ con:
    \[
        \begin{cases}
            \bar{x} = \frac{1}{m}\int_\gamma x\rho ds = \frac{1}{m}\int_{a}^{b}x(t)\rho(\vec{r}(t))|\vec{r}'(t)|dt\\
            \bar{y} = \frac{1}{m}\int_\gamma y\rho ds = \frac{1}{m}\int_{a}^{b}y(t)\rho(\vec{r}(t))|\vec{r}'(t)|dt\\
            \bar{z} = \frac{1}{m}\int_\gamma z\rho ds = \frac{1}{m}\int_{a}^{b}z(t)\rho(\vec{r}(t))|\vec{r}'(t)|dt
        \end{cases}
    \]
    Se il corpo è \textbf{omogeneo} ($\rho =$ costante), il baricentro si dice \textbf{centroide} e ha coordinate:
    \[
        \begin{cases}
            \bar{x} = \frac{\rho}{m}\int_\gamma x ds = \frac{1}{l(\gamma)}\int_{a}^{b}x(t)|\vec{r}'(t)|dt = \frac{\int_{a}^{b}x(t)|\vec{r}'(t)|dt}{\int_{a}^{b}|\vec{r}'(t)|dt}\\
            \bar{y} = \;\text{stesso di sopra}\\   
            \bar{z} = \;\text{stesso di sopra}
        \end{cases}
    \]
    \item \textbf{momento di inerzia} di $\gamma$ rispetto a un asse fissato. Se $\delta(x,y,z)$ indica la distanza del punto $(x,y,z)$ da quest'asse fissato, si ha:
    \[
        I = \int_\gamma \delta^2 \rho ds = \int_{a}^{b}\delta^2(\vec{r}(t))\rho(\vec{r}(t))|\vec{r}'(t)|dt
    \]
    Se il corpo è \textbf{omogeneo} ($\rho =$ costante):
    \[
        I = \frac{m}{l(\gamma)}\int_\gamma \delta^2ds = m \frac{\int_{a}^{b}\delta^2(\vec{r}(t))|\vec{r}'(t)|dt}{\int_{a}^{b}|\vec{r}'(t)|dt}
    \]
    Per gli esercizi: può essere molto utile cercare di ridefinire la curva spostandola in modo tale che l'asse di riferimento sia uno degli assi cartesiani, in questo modo esprimere la distanza $\delta$ di ogni punto dall'asse può essere più facile.
\end{itemize}
\begin{comment}
NON CI SONO IN QUEST'ANNO DI INSEGNAMENTO...
\subsection{Elementi di geometria differenziale delle curve}
\subsubsection{Curvatura e normale principale per una curva in $\mathbb{R}^n$}
Il versore tangente è definito come:
\[
    \vec{T}(t) = \frac{\vec{r}'(t)}{|\vec{r}'(t)|}\text{,} \quad \text{mentre} \;\vec{T}(s) =\vec{r}'(s)
\]
in quanto $|\vec{r}'(s)| = 1$, se $s$ è il parametro arco.\newline
Sia $\vec{u}: I \rightarrow \mathbb{R}^n$, una funzione vettoriale di modulo costante, ossia $|\vec{u}(t)| = c$ per ogni $t$. Allora $\vec{u} \cdot \vec{u}' = 0$, cioè $\vec{u}$ è ortogonale a $\vec{u}'$ in ogni istante.\newline
\newline
\textbf{Normale principale e curvatura, rispetto al parametro arco}\newline
Chiamiamo versone normale principale della curva $\vec{r}(s)$ (parametrizzata rispetto al parametro arco), il versore
\[
    \vec{N}(s) =\frac{\vec{T}'(s)}{|\vec{T}'(s)|}
\]
definito nei punti in cui $\vec{T}'(s) \neq 0$.\newline
Chiamiamo curvatura della curva la funzione scalare
\[
    k(s) = |\vec{T}'(s)|
\]
da cui ricaviamo che $\vec{T}'(s) = k(s) \vec{N}(s)$. \newline
La normale principale ci dice la direzione verso cui sta curvando la nostra curva, mentre la curvatura è uno scalare e rappresenta l'intensità con cui curva.\newline
Definiamo invece come raggio di curvatura
\[
    \rho(s) = \frac{1}{k(s)}
\]
nei punti in cui $k(s) \neq 0$, nei punti in cui invece $k(s) = 0$ si dice che il raggio di curvatura è infinito.\newline
\newline
\newline
\textbf{Normale principale e curvatura, rispetto a un parametro qualsiasi}\newline
Versone normale principale:
\[
    \vec{N}(t) =\frac{\vec{T}'(t)}{|\vec{T}'(t)}
\]
e definendo $\frac{ds}{dt} = |\vec{r}'(t)| = v(t)$ da cui ricaviamo che la curvatura è:
\[
    k(t) = \frac{|\vec{T}'(t)|}{v(t)}
\]
da cui $\vec{T}'(t) = k(t)v(t)\vec{N}(t)$.\newline
Raggio di curvatura:
\[
    \rho(t)= \frac{1}{k(t)}
\]
\newline
Vale la seguente formula:
\[
    \vec{a}(t) = v'(t)\vec{T}(t) + v^2(t) k(t) \vec{N}(t)
\]
dove $\vec{a}(t) = \vec{r}''(t)$.
\subsubsection{Calcolo della curvatura per curve nello spazio $\mathbb{R}^3$ o nel piano}
\[
    k(t) = \frac{|\vec{r}'(t) \text{x} \vec{r}''(t)|}{v^3(t)}
\]
\[
    k(s) = |\vec{r}'(s) \text{x} \vec{r}''(s)|
\]
\newline
\textbf{caso di curve piane}\newline
Sia $\vec{r}(t) =(x(t), y(t))$ per $t \in[a,b]$, possiamo vederla come curva in $\mathbb{R}^3$ così: $\vec{r}(t) = (x(t), y(t), 0)$ per $t \in[a,b]$.\newline
Perciò,
\[
    \vec{r}' \text{x} \vec{r}'' =\left|\begin{matrix}
        i \;\;& j \;\;& k \\
        x'(t) & y'(t) & 0\\
        x''(t) & y''(t) & 0
    \end{matrix} \right|
\]
da cui derivano:
\[
    k(t) = \frac{|x'(t)y''(t) - x''(t) y'(t)|}{(x'(t)^2 + y'(t)^2)^{\frac{3}{3}}}
\]
\[
    k(s) = |x'(s) y''(s) - x''(s) y'(s)|
\]
\newline
\newline
Si dicono vertici di una curva i punti in cui $k'(t) = 0$.
\subsubsection{Torsione(manca) e terna intrinseca per curve nello spazio $\mathbb{R}^3$}
Data una curva $\vec{r} : I \rightarrow \mathbb{R}^3$, regolare di classe $C^3(I)$, definiamo il versore binormale della curva come
\[
    \vec{B} = \vec{T} \text{x} \vec{N}
\]
La terna intrinseca $\vec{B}, \vec{T}, \vec{N}$ costituisce un sistema di riferimento ortonormale.
\end{comment}
\subsection{Note sugli esercizi}
\subsubsection{Ripasso sul calcolo vettoriale}
\textbf{vettore}:
\[
    \vec{x} = (x_1, x_2, \dots, x_n)
\]
\textbf{modulo} di un vettore:
\[
    |\vec{x}| = \sqrt{\sum_{i=1}^{n}x_i^2}
\]
\textbf{versore}:
\[
    vers(\vec{x}) = \frac{\vec{x}}{|\vec{x}|}
\]
$\vec{x}, \vec{y}$ sono \textbf{paralleli} se
\[
    \lambda \vec{x} = \mu \vec{y}\;\; \text{ per qualche } \;\;\lambda, \mu \in \mathbb{R}
\]
\textbf{Somma} di vettori: si sommano le componenti simili.\newline
\newline
\textbf{Prodotto} fra un \textbf{vettore} e uno \textbf{scalare}: si moltiplica ogni componente per lo scalare.\newline
\newline
\textbf{Prodotto scalare} fra vettori: ha come risultato un numero reale ottenuto dalla formula
\[
    \vec{u} \cdot \vec{v} = (u_1 v_1 + u_2 v_2 + \dots + u_n v_n)
\]
Il prodotto scalare può essere espresso anche come
\[
    \vec{u} \cdot \vec{v} = |\vec{u}| |\vec{v}| cos(\theta)
\]
dove $\theta$ rappresenta l'angolo tra i due vettori. Di conseguenza $\vec{u} \cdot  \vec{v} = 0$ solo se i due vettori sono ortogonali.\newline
Dal prodotto scalare si può ricavare l'angolo fra due vettori
\[
    cos(\theta) = \frac{\vec{u} \cdot  \vec{v} }{|\vec{u}| |\vec{v}|}
\]
Inoltre $\vec{v} \cdot  \vec{v} = |\vec{v}|^2$.\newline
\newline
\textbf{Prodotto vettoriale}:
\[
    \vec{u}\text{x}\vec{v} = \left|\begin{matrix}
        i \;\; & j \;\;& k \;\;\\
        u_1 & u_2 & u_3\\
        v_1 & v_2 & v_3\\
    \end{matrix}\right|
\]
Il prodotto vettoriale si annulla solo se i vettori sono paralleli.\newline
\textbf{Regola della mano destra}: il primo fattore va sul pollice, il secondo sull indice, il risultato è nel medio.\newline
Inoltre $\vec{v} \text{x} \vec{v} = 0$.\newline
\newline
\textbf{Prodotto misto}:
\[
    \vec{u} \cdot (\vec{v} \text{x} \vec{w}) = \left|\begin{matrix}
        u_1 \;\;& u_2 \;\;& u_3\\
        v_1 &v_2 & v_3\\
        w_1 &w_2 & w_3
    \end{matrix}\right|
\]
Il prodotto misto si annulla solo se i tre vettori sono linearmente indipendenti.
\subsubsection{Altre cose cambia titolo}
\begin{tcolorbox}
Proprietà delle curve:
\begin{itemize}
    \item continua: se le componenti sono continue.
    \item chiusa: se agli estremi dell'intervallo in cui son definite la curva ha lo stesso valore (se fosse definita su tutto $\mathbb{R}$ si controllano i limiti all'infinito).
    \item asintoti: se i limiti all'infinito hanno un valore finito per una delle due componenti.
    \item semplice: se la curva non passa mai due volte per lo stesso punto. Si verifica per logica, spesso è utile notare se almeno una delle due componenti è strettamente monotona (crescente o decrescente). Spesso per funzioni trigonometriche si cerca di ragionare sulla loro periodicità.
    \item regolare: Si calcola la derivata della curva (derivata delle componenti) e in seguito il modulo della derivata (radice della somma delle componenti alla seconda). Se le derivate delle componenti sono funzioni continue e il modulo della derivata non si annulla mai per i valori di $t$ per cui la curva è definita (estremi esclusi, sempre) allora la curva è regolare. (Ci sono metodi particolari per calcolare il modulo della derivata, per esempio, per funzioni in forma polare il modulo della derivata è: $\rho = f(\theta) \; \Rightarrow \; |\vec{r}'(\theta)| = \sqrt{f'(\theta)^2 + f(\theta)^2}$; ma, inoltre, non bisogna scordarsi di calcolare le derivate delle componenti per controllare che siano continue).\newline
    Infine per individuare i punti di singolarità (cioè quelli in cui la funzione non è regolare) si calcolano i punti in cui $\vec{r}(t) = (x,y,z)$ per cui $t$ annulla la derivata.
    \item piana: \newline
    Primo metodo: si controlla se il versore binormale alla curva è costante. Il vettore binormale della caruva $\gamma(x(t), y(t), z(t))$ si calcola come prodotto vettoriale di $\gamma'(t) \text{x} \gamma''(t)$, una volta normalizzato (dividendo ogni componente per il modulo (componenti al quadrato sommate sotto radice)) si ottiene il versore e si controlla se è costante (curva piana) o non costante (curva non piana) e dunque basta assicurarsi che non sia presente la variabile $t$ (p.s. $\gamma'(t) = (x'(t), y'(t), z'(t)$, $\gamma''(t) = (x'(t), y'(t), z'(t))$ ). Concettualmente, se i calcoli sono troppo complicati, si può ragionare sul fatto che se il vettore binormale (che è sempre ortonormale alla curva) ha sempre la stessa direzione (anche se di modulo diverso) allora la curva è piana.\newline
    Secondo metodo: sostiuiamo le equazioni parametriche della curva $\gamma(x(t), y(t), z(t))$ nella formula del piano generico $ax + by +cz + d = 0$. Se risulta che l'equazione è sempre soddisfatta, allora la curva è complanare.
\end{itemize}
\end{tcolorbox}
\ \newline
Equazione di una retta passante per due punt:\newline
Dati due punti $(x_1, x_2)$ e $(y_1, y_2)$ la retta passante per questi punti è definita dall'equazione
\[
    \frac{x-x_1}{x_2-x_1} = \frac{y-y_1}{y_2-y_1}
\]
Questa formula vale se i due punti non sono allineati verticalmente o orizzontalmente.\newline
\newline
Negli esercizi sul calcolo delle lunghezze di archi di curve e di parametrizzazione secondo il parametro arco è risultato molto utile l'intergale:
\[
    \int_{0}^{2\pi} \sqrt{1 + cos(t)}dt = \left[\text{moltiplicando per} \; \frac{\sqrt{1-cos(t)}}{\sqrt{1-cos(t)}}\right] = \int_{0}^{2\pi}\frac{|sin(t)|}{\sqrt{1-cos(t)}}dt =
\]
\[
    = [2 \sqrt{cos(t) +1}\cdot  tan(\frac{t}{2}) ]_0^{2\pi} = 4 \sqrt{2}
\]
Equazione della retta tangente a una curva $r(t) = (x(t), y(t))$ in un punto $t_0$:\newline
La generica retta ha forma $y = mx +q$. \newline
Il coefficiente $m$ si trova facendo il rapporto $\frac{y'(t_0)}{x'(t_0)}$.\newline
Il valore di $q$ si ricava sostituendo nell'equazione $y = mx +q$ la $m$ trovata, la $y$ con il valore di $y(t_0)$ e la $x$ con il valore di $x(t_0)$.