\section{Serie}
\subsection{serie geometrica}
\[
    \sum_{n=0}^{\infty}q^n = \begin{cases}
        \frac{1}{1-q} & se \;\; -1<q<1\\
        +\infty &se \;\; q \geq 1\\
        irregolare \;\;& se \;\; q\leq-1 
    \end{cases}
\]
\subsection{serie armonica}
\[
    \sum_{n=1}^{\infty}\frac{1}{n} \geq log(n+1) \rightarrow +\infty
\]
\subsection{serie armonica generalizzata}
\[
    \sum_{n=1}^{\infty} \frac{1}{n^{\alpha}}
\]
per $\alpha \leq 1$ 
\[
    \sum_{n=1}^{\infty} \frac{1}{n^\alpha} \geq \sum_{n=1}^{\infty}\frac{1}{n} \rightarrow +\infty \;\;\;\text{diverge}\;
\]
per $\alpha > 1$ 
\[
    \sum_{n=1}^{\infty} \frac{1}{n^\alpha} = converge
\]
per $\alpha = 2$
\[
    \sum_{n=1}^{\infty}\frac{1}{n^2} = \frac{\pi^2}{6} (\sim  \sum_{n=1}^{\infty} \frac{1}{n(n+1)} = serie \;\; di \;\; mengoli)
\]
\subsection{serie di mengoli}
\[
    \sum_{n=1}^{\infty}\frac{1}{n(n+1)} = \sum_{n=1}^{\infty} \frac{1}{n}-\frac{1}{n+1} = 1- \frac{1}{n+1} \rightarrow 1
\]
\subsection{numero e}
\[
    \lim_{n\rightarrow \infty} \left( 1 + \frac{1}{n} \right)^n = e
\]
\subsection{sviluppi di Taylor delle funzioni elementari}
\[
    e^x = \lim_{n\rightarrow +\infty}\sum_{k=0}^{n} \frac{x^k}{k!} = \sum_{k=0}^{+\infty} \frac{x^k}{k!}
\]
\[
    sin (x) =\sum_{k=0}^{\infty}(-1)^k \frac{x^{2k+1}}{(2k+1)!} =  \frac{e^{ix} - e^{-ix}}{2}
\]
\[
    cos (x) =\sum_{k=0}^{\infty}(-1)^k \frac{x^{2k}}{(2k)!} =\frac{e^{ix} + e^{-ix}}{2}
\]
\subsection{Serie di potenza}
Con $a_k$ costanti reali (o complesse) e $x$ variabile reale (o complessa)
\[
    \sum_{k=0}^{\infty}a_k x^k
\]
\[
    Sh (x) = \sum_{k=0}^{\infty} \frac{x^{2k+1}}{(2k+1)!}
\]
\[
    Ch(x) = \sum_{k=0}^{\infty} \frac{x^{2k}}{(2k)!}
\]
\[
    log(1+x)= \sum_{k=1}^{\infty} (-1)^{k+1} \frac{x^k}{k} \;\;\; per \;\; |x|<1
\]
per $\alpha \in \mathbb{R}$
\[
    (1+x)^\alpha = \sum_{k=0}^{\infty} \binom{\alpha}{k}x^k \;\;\; per \;\; |x|<1
\]
\textbf{teor.} Condizione necessaria affinché una serie $\sum_{n=0}^{\infty} a_n$ converga è che il termine generale $a_n$ tenda a zero. (Cioè perchè la serie converga, il termine $a_n$ deve tendere a zero, ma non per forza se il termine $a_n$ tende a zero allora la serie converge)\newline
\newline
\textbf{teor.}supponiamo che una serie $\sum_{n=0}^{\infty} a_n$ converga, allora per ogni $k$ anche risulta convergente anche $\sum_{n=k}^{\infty} a_n$.\newline
\newline
\textbf{Criterio serie a termini non negativi} Una serie $\sum_{n=0}^{\infty}a_n$ a termini non negativi è convergente o divergente a $+\infty$. Essa converge se e solo se la successione delle somme parziali n-esime è limitata.\newline
\newline
\textbf{Criterio del confronto} Siano $\sum an$ e $\sum b_n$ due serie a termini non negativi tali che $a_n<b_n$ definitivamente, allora:
\begin{itemize}
    \item $\sum b_n$ convergente $\Rightarrow \sum a_n$ convergente. 
    \item $\sum a_n$ divergente $\Rightarrow \sum b_n$ divergente. 
\end{itemize}
\textbf{Criterio del confronto asintotico} Se $a_n \sim b_n$, allora le corrispondenti serie $\sum a_n$ e $\sum b_n$ hanno lo stesso carattere (o entrambe divergenti o entrambe divergenti )\newline
\newline
\textbf{Criterio della radice} Sia $\sum a_n$ una serie a termini non negativi. Se esiste il limite 
\[
    \lim_{n\rightarrow +\infty}\sqrt[n]{a_n} = l
\]
\begin{itemize}
    \item $l>1$ la serie diverge $+\infty$
    \item $l<1$ la serie converge
    \item $l=1$ nulla si può concludere
\end{itemize}
Spesso utilizzato con termini che hanno come esponente $n$.\newline
\newline
\textbf{Criterio del rapporto} Sia $\sum a_n$ una serie a termini positivi. Se esiste il limite 
\[
    \lim_{n\rightarrow +\infty} \frac{a_{n+1}}{a_n} = l
\]
\begin{itemize}
    \item $l>1$ diverge $+\infty$
    \item $l<1$ converge 
    \item $l=1$ nulla si può concludere
\end{itemize}
Spesso utilizzato quando si hanno termini come $n^n$ e $n!$.\newline
\newline
\textbf{Criterio serie a termini di segno variabile} Una serie $\sum a_n$ si dice assolutamente convergente se converge la serie $\sum |a_n|$. Se la serie $\sum a_n$ converge assolutamente, allora converge.\newline
\newline
\textbf{Criterio di Leibniz} Sia data la serie 
\[
    \sum_{n=0}^{\infty}(-1)^n a_n \;\; con  \;\; a_n\geq 0 \;\forall\;n
\]
Se la successione $\{a_n\}$ è decrescente e se $a_n \rightarrow 0$ per $n \rightarrow \infty$, allora la serie è convergente.\newline
Il criterio di Leibniz può essere applicato anche se i termini sono definitivamente di segno alterno e la successione $a_n$ è definitivamente decrescente.\newline
Per verificare la decrescenza bisogna dimostrare che $a_{n+1}<a_n$ oppure mediante il limite a $+ \infty$ della derivata prima di $a_n$ o studiano quando la derivata prima di $a_n<0$ .\newline
Per determinare se una serie è decrescente non vanno usati gli asintotici !\newline
\newline
\textbf{Criterio della somma di serie convergenti} Se $\sum_{n=1}^{\infty} a_n$ converge e $\sum_{n=1}^{\infty} b_n$ converge, allora $\sum_{n=1}^{\infty} a_n +b_n$ converge.\newline
\newline
\textbf{Criterio della somma di serie convergenti e divergenti} Se $\sum_{n=1}^{\infty} a_n$ converge e $\sum_{n=1}^{\infty} b_n$ diverge, allora $\sum_{n=1}^{\infty} a_n + b_n$ diverge.\newline
\newline
\textbf{Criterio serie a termini complessi} Sia la serie $\sum_{n=0}^{\infty} a_n$ con $a_n$ complesso, se la serie  $\sum_{n=0}^{\infty} |a_n|$ converge, allora anche $\sum_{n=0}^{\infty} a_n$ converge \newline
\newline
\textbf{Criterio di Dirichlet} Siano $a_n$ e $b_n$ due succesioni tali che:
\begin{itemize}
    \item $a_n$ è a valori complessi e la sua successione delle somme parziali è limitata.
    \item $b_n$ è a valori reali positivi e tende monotonamente a zero
\end{itemize}
allora la serie $\sum a_nb_n$ è convergente.