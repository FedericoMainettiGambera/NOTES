\title{LEZIONE 13 30/04/20}
\textbf{link} \href{https://web.microsoftstream.com/video/9d3a5028-1a7e-471e-95cd-1bcf6ae58d8a?list=user&userId=cfe0965d-9a7c-40e2-be6e-f078296a1914}{clicca qui}
\section{Javascript}
Javascript è un linguaggio interpretato, quindi non ha una fase di compilazione.\newline
\newline
In Javascript è assente la tipizzazione delle variabili, il principale aspetto negativo di un linguaggio di programmazione a tipizzazione debole è che spesso, siccome non c'è un compilatore, gli errori passano inosservati finchè non si esegue il programma.\newline
Inoltre pure la dichiarazione di una variabile non è obbligatoria, una variabile non dichiarata viene automaticamente identificata come una variabile globale (scope globale), cosa da evitare assolutamente.\newline
L'ambiente di sviluppo è estremamente liberale.\newline
\newline
Gli script javascript agiscono su un documento HTML, il cuore della programmazione a lato client consiste nell'interagire con un'interfaccia.
\subsection{Numeri}
Javascript ha un solo tipo di numero (64-bit floating point).\newline
\newline
Esistono inoltre valori speciali: NaN (not a number), col relativo metodo isNaN(), e infinity o -infinity, col relativo metodo isFinite().
\subsection{Stringhe}
Le stringhe possono essere racchiuse dai simboli " e ' indifferentemente: non esiste il tipo char.\newline
\newline
Le stringhe sono immutabili, cioè, una volta create, non possono essere modificate.
\subsection{Istruzioni}
Le istruzioni sono eseguite in ordine di apparizione.\newline
\newline
Un blocco è un insieme di istruzioni racchiuse da parentesi graffe, che in javascript non rappresentano un ambito di scope. Gli unici scope che esistono sono dati dalle funzioni e dal "main" (scope globale). Questo concetto è importante per la programmazione asincrona.\newline
\newline
Scope di livello intermedio possono essere definiti con clausole speciali per la dichiarazione di variabili (let invece di var).\newline
\newline
Il temrinatore di istruzione ";" è opzionale.\newline
\newline
\url{http://www.jslint.com/} è un ottimo sito per scovare cattive pratiche all'interno di codice javascript.
\subsection{Uguaglianza}
Gli operatori di uguaglianza sono "===" e "!==", cioè se due oggetti sono dello stesso tipo ed hanno lo stesso valore, allora "===" ritorna true.\newline
\newline
L'operatore "==" (meno stringente di "===") prova a comparare oggetti di tipo differenti.\newline
\newline
In javascript la nozione di tipo è diversa da quella a cui siamo abituati per i linguaggi fortmente tipati, ovvero l'operatore typeof, che mostra il tipo di una variabile può ritornare solo uno di questi valori: number, string, boolean, undefined, function e object. Da notare è che typeoff null ritorna object.
\subsection{Oggetti}
La nozione di oggetto in javascript è diversa da quella per un linguaggio a oggetti. Per esempio in Java una classe definisce un tipo di dato e permette la creazione di oggetti. In javascript tutto è un oggetto, gli oggetti sono contenitori utilizzati per organizzare dati e possono essere definiti "al volo", è però fortemente consigliato usare funzioni costruttrici che sfruttano il riferimento "this". Esempio: "function Person(name, surname){this.name = name; this.surname = surname;}" e poi "var person = new Person();". Ci sono poi vari modi per ottenere informazioni ulteriori su un oggetto in javascript: abbiamo visto che "typeof" ci ritorna solo che è un "object", invece ".constructor" ci ritorna il costruttore che abbiamo usato per creare tale oggetto.\newline
\newline
Le proprietà degli oggetti possono essere eliminate con il comando "delete".\newline
\newline
E' possibile accedere alle proprietà di un oggetto oltre che con la notazione punto, anche con la notazione a parentesi quadre: oggetto.proprietà, oppure oggetto["proprietà"]. Questo metodo è utile perchè all'interno delle quadre c'è una stringa e di conseguenza si possono usare variabili oppure ciclare fra le proprietà, etc\dots \newline
\newline
Gli oggetti sono sempre passati per riferimento, non vengono mai copiati (!).
\subsection{Funzioni}
Uno dei tipi principali di Javascript sono le funzioni. Oltre ad essere usate nella maniera classica, esse sono dei veri e propri valori e di conseguenza possono essere assegnate a variabili. I metodi degli oggetti sono variabili che contengono funzioni.
\subsection{Prototipi}
In assenza di classi nel senso proprio del termine è possibile definire proprietà che si applicano a gruppi di oggetti con il concetto di prototipo (prototype object). Ogni oggetto è associato a un prototipo dal quale può ereditare proprietà, per default ogni oggetto è associato a Object.prototype, un oggetto standard di javascript. Il prototipo è dinamico, se si aggiunge una nuova proprietà al prototipo, essa sarà visibile in tutti gli oggetti basati su quel prototipo.\newline
Esempio:
\begin{lstlisting}
var property = {value : 1};
function MyObject() {};
MyObject.prototype = property;
var myObject = new MyObject();
myObject.value === 1 //true 
\end{lstlisting}
\ \newline
La ricerca della proprietà di un oggetto parte dalla definizione delle proprietà locali possedute direttamente e risale la catena dei prototipi fino a Object.prototype.
\subsection{Undefined}
Undefined è uno dei tipi fondamentali di javascript. Il tentativo di computare con valori undefined genera un'eccezione TypeError exeption.