\title{LEZIONE 15 07/05/20}
\textbf{link} \href{https://web.microsoftstream.com/video/cc735a4e-c7cb-467d-97bd-910450c2d0a3?list=user&userId=cfe0965d-9a7c-40e2-be6e-f078296a1914}{clicca qui}
\newline
\newline
Function statement: function f()\{\dots;\}; definisce una funzione; la funzione si invoca con l'operatore di chiamata (); deve avere un nome; la funzione è soggetta a "hoisting" per cui è disponibile anche prima di dove è stata definita nel programma.\newline
\newline
Function expression: var f = function ()\{\dots;\};; definisce la variabile f che ha come valore una funzione; non necessita di un nome, ma nel caso in cui venga specificato, il nome è utilizzabile solo all'interno del body della stessa; la funzione è invocata tramite il nome della variabile; la funzione è definita solo quando avviene la valutazione dell'espressione, cioè finchè la riga di codice contenente la variabile f non viene eseguita, la funzione non esiste; la funzione può vedere le variabili dello scope in cui è definita (chiusura).\newline
\newline
Per creare variabili non globali in JavaScript bisogna creare uno scope tramite le funzioni. Mostriamo un esempio:
\begin{lstlisting}
var scope = ( function(){
    var variabileNonGlobale = 0;
    return function(){
        //Possiamo usare la variabile non globale qua per via della chiusura.
    }
})();
\end{lstlisting}
\ \newline
Questa è una cosiddetta "iife", cioè immediately invoked function expression.\newline
Una iife è un'espressione che comporta la dichiarazione e l'esecuzione di una funzione e sono utilizzate per delimitare una porzione di codice evitando che le variabili finiscano nello scope globale.\newline
La presenza delle parentesi tonde esterne (grouping operator) rendono la frase un'espressione valida, altrimenti sarebbe un function statement e non un function expression.\newline
Le iife possono essere usate anche per realizzare un meccanismo di protezione delle variabili, cioè si riesce a imitare il comportamento dei getter e setter di attributi privati:
\begin{lstlisting}
var Sequence = (function sequenceIIFE(){
    var current = 0; //variabile privata
    return { //ritorna un oggetto
        getCurrentValue: function(){
            return current; //rimene vivo per chiusura
        },
        getNextValue: function(){
            current = current +1;
            return current; //rimane vivo per chiusura
        }
    };
})();
\end{lstlisting}
In questo caso la variabile current non è accessibile se non attraverso i metodi Sequence.getCurrentValue() e Sequence.getNextValue().\newline
\newline
Le funzioni possono essere assegnate a proprietà di oggetti (metodi di un oggetto), il parametro this denota l'oggetto su cui la funzione è chiamata, ma un metodo non può usare una funzione innestata che operi direttamente sull'oggetto this, cioè la funzione innestata non condivide il riferimento a this. per permetterle di accedere a this bisogna salvarne un riferimento in una variabile (var self = this;) che sarà visibile nella funzione innestata.\newline
\newline
Invocazione indiretta di funzione: i metodi call e apply di un oggetto di tipo funzione permettono di invocare la funzione, passandogli un valore da usare come this e nel caso di call una lista di argomenti, nel caso di apply un vettore di argomenti.\newline
\newline
L'oggetto arguments è disponibile all'interno delle funzioni quando vengono invocate e contiene i valori dei parametri, include anche i valori opzionali ed è utile per la definizione di funzioni con un numero non specificato di parametri. Notare che arguments non è un vero e proprio vettore, ha la proprietà lenght, ma manca di tutti gli altri metodi tipici dei vettore, si dice essere un oggetto "array-like" o "pseudo-array".\newline
\newline
Una funzione restituisce sempre un valore. Casi speciali:
\begin{itemize}
    \item undefined: se il valore di ritorno non è specificato;
    \item this: se la funzione è un costruttore invocato con l'operatore new.
\end{itemize}
\ \newline
Arrow function: le arrow function sono una notazioen abbreviata per la scrittura di espressioni funzionali. Le uniche differenze sono:
\begin{itemize}
    \item non possiede l'oggetto arguments;
    \item si può usare this, che eredità il valore dalla più vicina funzione normale che la racchiude;
    \item non può essere usata come costruttore con la parola new.
\end{itemize}
\begin{lstlisting}
var add = function(x,y){
    return x + y;
};
//equivale a:
var add = (x,y) => {
    return x + y;
};
\end{lstlisting}
\ \newline
Una funzione di callback è un meccanismo usato per gestire eventi asincroni che causano la chiamata di una funzione:
\begin{itemize}
    \item la funzione da chiamare è passata come parametro alla funzione che gestisce l'evento;
    \item la funzione passata come parametro è invocata quando l'evento si manifesta.
\end{itemize}
Un tipico esempio sono le richieste HTTP asincrone al server.
\newline
\newline