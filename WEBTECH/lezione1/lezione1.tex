\title{LEZIONE 1 11/03/20}\newline
\textbf{link} \href{https://web.microsoftstream.com/video/58568b1d-5fc5-41c0-88f6-608e4b8f9f7a}{clicca qui}
\section{Calendario delle lezioni}
\url{https://docs.google.com/spreadsheets/d/14ShTdkFCZ63MlyKP0LCSLPsPAeiu8D2sYVXSrkw9B6k/edit#gid=0}
\section{Prerequisiti}
Introduzione al corso - Evoluzione architetture \newline
Protocollo HTTP \newline
HTML 3 e 4 \newline
CGI
\section{Introduzione al corso}
Non ci saranno prove in itinere per questo corso.\newline
E' opportuno guardare i video relativi agli argomenti indicati prima di partecipare alla lezione.\newline
Queste lezioni hanno lo scopo di aggiungere informazioni, chiarire dubbi e aiutarci a capire quali siano gli argomenti più importanti e i punti chiave.\newline
I materiali video coprono interamente gli argomenti del corso.
\section{Architetture}
Il web è una piattaforma per sviluppo di applicazioni con un architettura molto particolare. Per architettura si intende l'insieme delle risorse (hardware, connettività, software di base, software applicativo).\newline
Le applicazioni web hanno una particolare conformazione della loro architettura che prevede tre livelli: Client, Middle-tier, Data-tier.\newline 
Le architetture sono  cambiate molto negli anni. Se ne individuano tre grandi famiglie: Mainframe, Client-Server, Multi-tier (per esempio le applicazioni web). \newline
Il compito di questo corso è quello di riuscire a programmare nel Middle-tier con qualche accenno al Client.
In un'applicazione web il Client interagisce con il Middle-tier con un preciso protocollo, che non è il classico tcp-ip, ma il protocollo HTTP.\newline
[Una lettura molto importante per il corso è il documento RequestforComment 1945, Tim Berners Lee (https://www.w3.org/Protocols/rfc1945/rfc1945), che rappresenta l'atto di fondazione del web.]\newline
Il Client manda delle request al Server che invia delle responses. Le request del Client sono gestite tramite un applicazione detta User agent (browser).\newline
Http è rivoluzionario per la sua semplicità: le richieste del Client e le risposte del server sono delle semplici stringhe. \newline
Architettura delle applicazioni web: c'è un client (pc) con un user agent (browser) che emette richieste che vengono servite con delle risposte da un web server, detto anche HTTP server. L'archiettura di cui però ci occupiamo è più complicato di così, perchè vede alle spalle del web server un application server (TomCat), che ha lo scopo di calcolare una risposta "personalizzata" secondo parametri e criteri precisi e di produrre un pagina web appropriata.\newline
[Installare un'archiettura completa sul pc che prevede tutti i livelli appena visti (guida nella cartella strumenti) (si può usare anche IntellyJ invece di Eclipse)]\newline
Altri elementi dell'architettura sono il proxy e il gateway. Il proxy è un intermediario fra un client e un origin server, per definizione può comportarsi sia come server sia come client. Fa copie di risorse e le inoltra ai client e può essere anche utile per limitare gli accessi e gestire autorizzazioni. Il Gateway serve per far interagire un applicazione web con applicazioni che non usano il protocollo HTTP, quindi il gateway traduce richieste HTTP per applicazioni che non lo comprendono, tipicamente usato per SQL.
\section{HTTP}
Come si identifica una risorsa? con l'URL (Uniform Resourse Locator) che è una striga formatta in maniera molto semplice: prefisso (protocollo), indicazione della macchina fisica in cui è presente la risorsa, una eventuale porta in cui la macchina ascolta le richieste e un Path. Gli URL sono estremamenti semplici, e utilizzano concetti già noti in precedenza.\newline
HTTP request: è una banale stringa che contiene una request-line, degli header e un allegato (per gli upload) facoltativi. la request-line contiene tre informazioni: il metodo (funzione) della richiesta, l'URL, la versione del protocollo usata. I metodi di richiesta sono principalmente due, GET e POST; questi metodo possono essere visti come chiamate di funzione.\newline
HTTP response: anche questa è una banale stringa, che, ancora più semplicemente contiene un codice di stato (es. 404 file not found), degli header facoltativi, e un allegato.\newline
La conseguenza di un protocollo così semplice è che con un solo client si può interagire con tutti i back end. La complessità si sposta però nell'application server.\newline
Gli header sono informazioni opzionali a cura del browser, trasmesse come fossero parametri che aggiungono informazioni alla request o alla response.
\section{CGI, Common Gateway Interface}
Tecnologia ormai morta e superata.\newline
CGI è una convenzione che standardiza un certo numero di variabili d'ambiente (condivisibili fra più processi) grazie alle quali il web server può smontare una richiesta HTTP e salvarla in una zona di memoria alla quale un processo di un applicazione esterna può accedere.\newline
Per invocare un applicazione tramite CGI, l'URL della richiesta HTTP deve presentare un path che porta a un applicazione eseguibile. Una volta ricevuta una richiesta HTTP che rimanda a un eseguibile, il web server smonta la richeista, ne preleva le informazioni (parametri) e le salva nelle variabili d'ambiente, successivamente fa eseguire all'application server il programma indicato nel path dell'URL. L'application server preleva le informazioni dalle variabili d'ambiente e costruisce un file HTML che poi verrà mandato come risposta al client.\newline
CGI ha due grandi difetti, il primo riguarda la sicurezza, in quanto i file eseguibili erano direttamente accessibili, il secondo è che siccome i processi dell'application server muoiono dopo ogni esecuzione, non c'è modo di creare un sistema che mantiene una sessione attiva. Inoltre le prestazioni di CGI sono molto basse, per esempio, siccome i processi muoiono continuamente, c'è un continuo bisogno di stabilire connessioni coi database, che è un operazione onerosa.\newline