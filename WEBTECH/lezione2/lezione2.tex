\section{LEZIONE 2 12/03/20}
\textbf{link} https://web.microsoftstream.com/video/58568b1d-5fc5-41c0-88f6-608e4b8f9f7a
\subsubsection{Servlet}
Nasce l'esigenza di generare contenuti dinamici per le applicazioni web, inizialmente, infatti, HTTP era concepito come protocollo per scambio di documenti.\newline
Il gateway è un elemento imprescindibile che aumenta le capacità di un'applicazione web. Abbiamo visto che la versione arcaica di gateway era il CGI, che rappresenta il modo più semplice e che usa le variabili d'ambiente per adempiere al suo compito. Abbiamo però visto che CGI ha diversi problemi, perciò è necessario trovare una nuova soluzione.\newline
Vediamo ora il meccanismo più popolare (almeno fino a qualche anno fa) per realizzare i requisiti di produzione dinamica di contenuti. HTTP è nato per essere semplice, ha solo due metodi, GET e POST, non prevede l'identificazione di un client, tutte le richieste sono uguali. Per un Web Server la nozione di sessione non esiste. I metodi GET e POST sono richieste parametriche, cioè alle quali si possono aggiungere informazioni sotto forma di parametri stringa. Il metodo GET i parametri li mette nella query-string, il metodo POST li mette nel body. HTML ha un costrutto "form" che serve al browser per costruire un richeista dove i parametri sono complicati, per esempio richieste dove un parametro è un file. Il metodo POST è quindi spesso usato con il costrutto HTML form.\newline
Java Servlet è un modo nuovo, rispetto a CGI, di strutturare l'applicazione che riceve la request e formula la response.\newline
Al programmatore di un Servlet si chiede di programmare una classe, egli dovrà lavorare all'interno di un Framework, cioè una soluzione parziale a una serie di problemi con caratteristiche comuni. Il framework è uno schema, uan soluzione parziale che omette la parte variabile di una serie di applicazioni con qualcosa in comune. Java Servlet è appunto un framework, tutte le applicazioni web hanno in comune tutto il protocollo HTTP. Dal framework ci aspettiamo quindi di non dover programmare la gestione delle request e delle response.\newline
Il Servlet container è un ambiente che esegue il tuo programma, nel nostro caso è Tomcat. Il container materializza l'API del framework che stiamo usando, per esempio il metodo doGet() di cui noi facciamo l'override viene chiamato dal Servlet container.\newline
Alle spalle del web server dunque sta la JVM, all'interno del quale risiede il servlet container che gestisce i Servlet che il programmatore scrive.\newline
Un primo beneficio fra che si nota nel programmare in un ambiente così fortemente strutturato è che, diversamente da CGI, i Servlet vengono eseguiti all'interno di un ambiente persistente e quindi può eseguire più di una richiesta senza essere terminato. Non c'è bisogno di preoccuparsi della concorrenza, semplicemente si programma un Servlet pensando a come deve rispondere a una certa request.\newline
Un altro lato positivo