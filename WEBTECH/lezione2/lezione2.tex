\title{LEZIONE 2 12/03/20}\newline
\textbf{link} \href{https://web.microsoftstream.com/video/58568b1d-5fc5-41c0-88f6-608e4b8f9f7a}{clicca qui}
\section{Servlet}
\subsection{Concetti base}
Nasce l'esigenza di generare \textbf{contenuti dinamici} per le applicazioni web, inizialmente, infatti, HTTP era concepito come protocollo per scambio di documenti statici.\newline
\newline
Il gateway è un elemento imprescindibile che aumenta le capacità di un'applicazione web. Abbiamo visto che la versione arcaica di gateway era il CGI, che rappresenta un modo semplice e che usa le variabili d'ambiente per adempiere al suo compito. Ma CGI ha diversi problemi, perciò è necessario trovare una nuova soluzione.\newline
\newline
Vediamo ora il meccanismo più popolare fino a qualche anno fa per la produzione dinamica di contenuti.\newline
\newline
HTTP è nato per essere semplice, ha solo due metodi, GET e POST e non prevede l'identificazione di un client, tutte le richieste sono uguali.\newline
\newline
Per un Web Server la nozione di sessione non esiste.\newline
\newline
I metodi GET e POST sono parametrici, cioè si possono aggiungere informazioni sotto forma di parametri stringa: per il metodo GET i parametri sono inseriti nella query-string, per il metodo POST si mettono nel body.\newline
HTML ha un costrutto \textbf{form}, che viene spesso associato al metodo POST di HTTP. Una form serve al browser per costruire un richeista dove i parametri sono complicati, per esempio richieste dove un parametro è un file.\newline
\newline
\textbf{Java Servlet} è un modo nuovo, rispetto a CGI, di strutturare l'applicazione che riceve la request e formula la response.\newline
\newline
Al programmatore di un \textbf{Servlet} si chiede di programmare una \textbf{classe java}, egli dovrà lavorare all'interno di un \textbf{Framework} (ooo), cioè una soluzione parziale a una serie di problemi con caratteristiche comuni. Il framework è uno schema, una soluzione parziale che omette la parte variabile di una serie di applicazioni con qualcosa in comune. Un famework può anche essere definito come una serie di utils, o come una libreria, che facilita il lavoro al programmatore.\newline
\newline
Java Servlet è appunto un framework, tutte le applicazioni web hanno in comune il protocollo HTTP. Dal framework ci aspettiamo quindi di non dover, per esempio, programmare la gestione delle request e delle response.\newline
\newline
Il Servlet container è un componente dell'application server ed è un ambiente che esegue il tuo programma, nel nostro caso è \textbf{Tomcat}. Il container materializza l'API del framework che stiamo usando e interagisce con le Java servlet, per esempio il metodo doGet() di cui noi facciamo l'override viene chiamato dal Servlet container automaticamente, senza che ce ne dobbiamo preoccupare. Il servlet container è quindi un mediatore fra HTTP server e le Servlet, è responsabile del ciclo di vita delle Servlet e si occupa della mappatura delle richieste HTTP alla specifica servlet.\newline
\newline
(ooo) La gerarchia dunque è che a monte di tutto c'è la JVM, nella quale gira il web server, all'interno del quale risiede il servlet container che gestisce i Servlet che il programmatore scrive.\newline
\newline
Un primo beneficio che si nota nel programmare in un ambiente così fortemente strutturato è che, diversamente da CGI, i Servlet vengono eseguiti all'interno di un ambiente persistente e quindi è possibile eseguire più di una richiesta senza dover essere terminato.\newline
Inoltre non c'è bisogno di preoccuparsi della concorrenza, semplicemente si programma un Servlet pensando a come deve rispondere a una certa request. Siamo quindi in presenza di un ambiente \textbf{stateful}.\newline
\newline
C'è un prezzo da pagare per questo beneficio, ovvero che non siamo noi a gestire, per esempio, la concorrenza e quindi i thread, il modello di concorrenza lo ereditiamo dal contenitore.\newline
Il modello da seguire è chiamato "\textbf{one thread per request}" (ooo) e significa che una volta sviluppata una servlet non siamo noi a invocare la "new" su quell'oggetto, perchè è il contenitore a gestire questo aspetto e lui ne avrà sempre e soltanto uno istanziato. Tutte le request interagiranno sempre con lo stesso oggetto istanza della Servlet programmata da noi.\newline
Quindi: si programma una servlet, ce ne sarà una sola istanza, un solo oggetto, tutte le request che riceviamo (anche tante nello stesso istante) avranno un thread allocato, non da noi, ma dal container, e questi thread agiranno tutti sulla stessa istanza del Servlet.\newline
Questo processo ha ovviamente un grande problema: le variabili della classe Servlet sono condivise per tutte le request.\newline
\newline
Un framework ti da servizi, ma ti impone dei vincoli.\newline
\newline
Dal punto di vista materiale un framework è una libreria e noi useremo tendenzialmente javax.servlet e javax.servlet.http, che sono interfacce implementate dal contenitore.\newline
\newline
Analiziamo il ciclo di vita delle servlet. \newline
Il contenitore mappa automaticamente le request sui metodi doGet() e doPost().\newline
Il metodo init() viene chiamato quando la servlet inizia.
Il metodo destroy() viene chiamato quando il processo dell'applicazione viene stoppato, fatto che è deciso dall'amministratore del server.\newline
\newline
Il file web.xml serve per mappare le richieste http alla corretta servlet.\newline
\newline
La programmazione per Servlet è una programmazione per componenti, nel senso che il programmatore scrive solo i componenti variabili per l'applicazione, tutto il resto è gestito dal framework.\newline
\newline
Il servlet Context è una scatola che contiene diversi componenti che presi nell'insieme rappresentano un'applicazione. L'oggetto java che rappresenta un'applicazione è proprio il servlet context, infatti quando bisogna fare la deploy di una applicazione si distribuisce il servlet context. Il servlet context è un insieme di risorse che contiene i sorgenti delle applicazioni java scritte dal programmatore, i file di configurazione e le risorse (per esempio delle immagini).\newline
\newline
Il file web.xml contiene la configurazione dell'applicazione, fra cui la mappatura delle varie servlet.
\newline
\newline