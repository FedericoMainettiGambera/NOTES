\title{LEZIONE 10 16/04/20}
\textbf{link} \href{https://web.microsoftstream.com/video/23680e25-5ca6-48fc-aeb4-4cdbbee98726?list=user&userId=cfe0965d-9a7c-40e2-be6e-f078296a1914}{clicca qui}
\section{esercizio da un tema d'esame (pt.2)}
\section{Filtri}
La verifica evento epr evento dei diritti di accesso è una funzionalità obligatoria dei siti multiruolo.\newline
La verifica viene fatta in ogni controllore e per evitare di dduplicare codice si utilizzano i filtri.\newline
Un filtro è una classe Java che il servler container interpone fra la richiesta (l'HTTP request) e il servizio (la servlet). I filtri possono essere anche concatenati uno dopo l'altro.\newline
\newline
La classe filtro ha sostanzialmente la stessa struttura di una servlet, ha init(), destroy() e invece di doGet e doPost, ha doFilter(). Ci sono tre interfaccie che vvengono usate per i filtri e sono: Filter, FilterChain e FIlterConfig.\newline
\newline
La catena di filtri si può specificare nel web.xml, inoltre la catena di filtri è unica per ogni app, non si possono specificare catene particolari per ogni servlet. Il comando chain.doFilter() in poche parole passa il comando al prossimo filtro della catena, se ce ne è uno, altrimenti prosegue paassando la richiesta alla servlet che se ne deve occupare.\newline
\newline

