\section*{10-LEZIONE}
18/11/19
\subsection*{Sviluppi di MacLaurin delle funzioni elementari}
Riprendiamo dall'ultima lezione sul polinomio di Taylor:
\[
    T_n(x) = \sum_{k=0}^{n}\frac{f^{(k)}(x_0)}{k!} \cdot  (x-x_0)^k
\]
Il polinomio di Taylor viene chiamato polinomio di \textbf{MacLaurin} se si prende $x_0 = 0$:
\[
    T_n(x) = \sum_{k=0}^{n} \frac{f^{k}(0)}{k!} x^k
\]
\newline
\newline
\textbf{es.} scrivere il polinomio di MacLaurin di grado 9 di $sin(x)$
\[
    f(x) = sin(x) \rightarrow x=0 \rightarrow 0
\]
\[
    f'(x) = cos(x) \rightarrow x=0 \rightarrow 1
\]
\[
    f''(x)= -sin(x) \rightarrow x=0 \rightarrow 0
\]
\[
    f'''(x)= -cos(x) \rightarrow x=0 \rightarrow-1
\]
\[
    f^{iv}(x) = sin(x) = f(x) \rightarrow x=0 \rightarrow 0
\]
e ciclo di derivate riinizia. Quindi $T_9(x) = f(0) +f'(0) x + \frac{f''(0)x^2}{2!} + \frac{f'''(0)}{3!}x^3 + \dots + \frac{f^{ix}(0)}{9!}x^9 =$ tutti i termini pari si annullano  $=$
\[
    = x - \frac{1}{3!}x^3 + \frac{1}{5!}x^5 - \frac{1}{7!}x^7 + \frac{1}{9!}x^9
\]
Notiamo che più si calcola un polinomio di grado maggiore e più troviamo una funziona che approssima meglio la funzione seno. Per esempio $T_1$ approssima il seno solo vicino all'origine, $T_3$, invece, approssima molto meglio di $T_1$ (graficamente notiamo che $T_3$ ha già le prime curvature del seno), etc.\newline
[immagine:mancante (rappresentazione di vari gradi del polinomio di MacLaurin)]
\newline
\newline
\textbf{es.} scrivere il polinomio di MacLaurin di grado 9 di $cos(x)$.\newline
Questo esercizio è estremamente simile a quello precedente, qua invece di annullarsi i termini pari, si annullano quelli dispari. [Fare a casa].\newline
[immagine:mancante (rappresentazione di vari gradi del polinomio di MacLaurin)]
\newline
\newline
\textbf{es.} scrivere il polinomio di MacLauri di grado 9 dell'esponenziale.\newline
\[
    f(x) = e^x
\]
\[
    f^{(k)} = e^k \rightarrow x=0 \rightarrow 1
\]
\[
    T_n(x) = \sum_{k=0}^{n} \frac{1}{k!} \cdot x^k
\]
Quindi per $n=9$ avremo:
\[
    T_9(x) = 1+x+\frac{x^2}{2} + \frac{x^3}{3!} + \frac{x^4}{4!} + \dots + \frac{x^9}{9!}.
\]
[immagine: mancante (rappresentazione di vari gradi del polinomio di MacLaurin)]\newline
\newline
\newline
\subsection*{Teorema del resto secondo Peano}
descrive l'errore nell'approssimazione di $f(x)$ con $T_n(x)$. Dove $x$, che è il punto in cui sto facendo l'approssimazione, è nell'intorno di $x_0$ (ma $x\neq x_0$).\newline
\textbf{teor.}  Se $f: A \rightarrow \mathbb{R}$, $x \longrightarrow  y = f(x)$ e ipotesi:
\begin{itemize}
    \item $A=(a,b)$, $x_0 \in A$
    \item $f \in C^n(A) = \{$continua con tutte le derivate continue fino a $f^{(n)}$ in ogni punto di $A\}$
\end{itemize}
allora
\[
    f(x) = T_n(x) + o( \; (x-x_0)^n \; ) \;\;\;\;\; x \rightarrow x_0
\]
Questa formula è conosciuta anche in questa forma: $f(x) -T_n(x) = o( \; (x-x_0)^n \; )$ per $x \rightarrow x_0$.\newline
\newline
\textbf{Corollario 1} del teorema di Lagrange(che enuncia:$\;\exists\;\; \theta \in (a,b) \;\;:\;\; (b-a) f'(\theta) ) f(b) - f(a)$). \textbf{Teorema di Cauchy}:\newline
Date $f, g: A \rightarrow \mathbb{R}$, $x \rightarrow y=f(x)$ e $y=g(x)$, e ipotesi:
\begin{itemize}
    \item $A = [a,b]$
    \item $f,g $ continue in $A$ e derivabili in $(a,b)$
\end{itemize}
Allora per il teorema di Lagrange sappiamo che
\[
    \begin{cases}
        \;\;\exists\;\; \theta_1 \in (a,b) \;\;:\;\; f'(\theta_1)(b-a) = f(b) - f(a) &\\
        \;\;\exists\;\; \theta_2 \in (a,b) \;\;:\;\; g'(\theta_2)(b-a) = g(b) - g(a) &\\
    \end{cases}
\]
Allora 
\[
    \;\;\exists\;\; x^* \in (a,b) \;\;:\;\; \frac{f'(x^*)}{g'(x^*)} = \frac{f(b) - f(a)}{g(b) - g(a)}
\]
\textbf{dim.} Uso un funzione ausiliaria $h(x) = [f(b) - f(a)] \cdot  g(x) - [g(b) - g(a)] \cdot  f(x)$, che ho ricavato portando tutto a primo termine da : $\frac{f'(x^*)}{g'(x^*)} = \frac{f(b) - f(a)}{g(b) - g(a)}$ e togliendo le "derivate".\newline
$h$ ha la regolarità di $f$ e di $g$ perchè è una combinazione lineare di queste due funzione. Applico ora il teorema di Rolle:
\[
    h(x) = [f(b) - f(a)] \cdot  g(x) - [g(b) - g(a)] \cdot  f(x)
\]
Valutiamo gli estremi: in $a$
\[
    h(a) = f(b)g(a) - g(b)f(a)
\]
in $b$
\[
    h(b) = - f(a)g(b) + g(a)f(b) = h(a)
\]
Quindi per Rolle sappiamo che $h$ ha un punto stazionario. Calcoliamo ora la derivata di $h$
\[
    h'(x) = [f(b) - f(a)] \cdot  g'(x) - [g(b) - g(a) ]\cdot  f'(x)
\]
Ma sappiam oche esiste un $x^*$ per cui la derivata è nulla:
\[
    h'(x^*) = [f(b) - f(a)] \cdot  g'(x^*) - [g(b) - g(a) ]\cdot  f'(x^*) = 0
\]
\newline
\newline
\textbf{Corollario 2} del teorema di Lagrange(che enuncia:$\;\exists\;\; \theta \in (a,b) \;\;:\;\; (b-a) f'(\theta) ) f(b) - f(a)$). \textbf{Teorema di de l'Hopital (di Voldemort, da non usare secondo il prof.)}:\newline
Date $f, g: A \rightarrow \mathbb{R}$, $x \rightarrow y=f(x)$ e $y=g(x)$, e ipotesi:
\begin{itemize}
    \item $A = [a,b]$
    \item $f,g $ continue in $A$ e derivabili in $(a,b)$
    \item entrambe le fuznioni sono infinitesime in $x_0 \in (a,b)$
\end{itemize}
Allora se $l = \lim_{x\rightarrow x_0} \frac{ f'(x)}{g'(x)}$, allora $l = \lim_{x\rightarrow x_0} \frac{f(x)}{g(x)}$\newline
\newline
\textbf{dim.} diretta dal corollario 1 (cauchy), $\;\;\exists\;\; \theta \in (a,b) \;\;:\;\;$
\[
    \frac{f(x)}{g(x)} = \frac{f(x)-f(x_0)}{g(x)-g(x_0)} = \frac{f'(\theta)}{g'(\theta)}
\]
\[
    \lim_{x\rightarrow x_0} \frac{f(x)}{g(x)} = \lim_{x\rightarrow x_0} \frac{f'(\theta)}{g'(\theta)} = l
\]
\newline
\newline
\newline
\subsection*{Dimostrazione del teorema del resto secondo Peano}
se $ f \in C^n (A)$ e $A= (a,b)$ e $x_0 \in A$, allora $f(x) - T_n^{f} (x) = o( \; (x- x_0)^n \;)$\newline
\textbf{dim.} per induzione su $n$
\begin{itemize}
    \item verifico per $n = 1$:\newline
        \[
            f \in C^1 \;\;, \;\; f(x) - [f(x_=) + f'(x_0) (x-x_0)]  \; =? \; o (x-x_0)
        \] 
        per $x \rightarrow x_0$
        \[
            \frac{f(x) - [f(x_0) + f'(x_0) (x-x_0)]}{x-x_0} = \frac{f(x) -f(x_0)}{x-x_0} - f'(x_0)
        \]
        Verificata.
    \item Supponendo che valga per $n-1$ verifico per $n$.\newline
    Scriviamo la validità per $n-1$ con una funzione $g$: $\;\forall\; g \in C^{n-1}(A)$ allora so che vale
    \[
        g(x) -T_{n-1}^g (x) = o( \; (x-x_0)^n-1\;)
    \]
     Ora dobbiamo verificare che presa $f \in C^n(A)$ allora vale che 
     \[
        f(x) - T_n^f(x) = o(\; (x-x_0)^n \;)
     \] 
     e per farlo ci basta dimostrare che 
     \[
        \frac{f(x) - T_n^f(x)}{(x-x_0)^n}
     \]
     tenda a $0$ per $x \rightarrow x_0$. Per calcolare questo limite usiamo il corollario 2: 
    \[
        \lim_{x \rightarrow x_0} \frac{[f(x) - T_n^f(x)]'}{[(x-x_0)^n]'}
    \]
    Analiziamo prima:
    \[
        [T_n^f(x)]' = [f(x_0) + f'(x_0)(x-x_0) + \frac{f''(x_0)}{2!}(x-x_0)2 + \dots + \frac{f^{(n)}(x_0)}{n!}(x-x_0)^n] =
    \]
    \[
        = f'(x_0) + f''(x_0)(x-x_0) + \dots + \frac{f^{(n)}(x_0)}{(n-1)!}(x-x_0)^{n-1} = T_{n-1}^{f'}(x)
    \]
    Quindi: $[T_n^f(x)]' = T_{n-1}^{f'}(x)$.\newline
    Ora possiamo tornare al limite:
    \[
        \lim_{x \rightarrow x_0} \frac{[f(x) - T_n^f(x)]'}{[(x-x_0)^n]'} = \lim_{x \rightarrow x_0} \frac{f'(x) - T_{n-1}^{f'}(x)}{n \cdot  (x-x_0)^{n-1}} =
    \]
    che per ipotesi di induzione (quella con la funzione $g$) è
    \[
        = 0
    \]
\end{itemize}
Fine dimostrazione.\newline
\newline
\newline
\subsection*{Teorema del resto secondo Lagrange}
Questo teorema, rispetto a quello del resto secondo Peano, presenta ipotesi più restrittive, che consentono di scrivere il resto in modo pi dettagliato.\newline
$f \in C^{n+1}(A)$, $A=(a,b)$, $x_0 \in A$, Allora $\;\;\exists\;\; \theta \in (x_0, x) \;\;:\;\; $
\[
    f(x) -T_n(x) = \frac{f^{n+1}(\theta)}{(n+1)!}(x-x_0^{(n+1)})
\]
Notiamo che $\frac{f^{n+1}(\theta)}{(n+1)!}(x-x_0^{(n+1)})$ è $o( \; (x-x_0)^n \;)$, quindi la forma di Lagrange è più dettagliata di Peano.\newline
\newline
\textbf{dim.} Considero due funzioni ausiliarie:
\[
    g(x) = f(x) - T_n(x)
\]
\[
    w(x) = (x-x_0)^{n+1}
\]
$w(x)$ ha regolarità $ \in C^\infty ( \mathbb{R})$ e $g(x)$ ha regolarità $in C^{n+1}(A)$.\newline
Calcoliamo $g(x_0)$, $g'(x_0)$, \dots, $g^{n+1}(x_0)$ e $w(x_0)$, $w'(x_0)$, \dots, $w^{n+1}(x_0)$.\newline
Dato che $f$ e $T_n$ hanno un contatto di ordine $n$ in $x_0$:
\[
    g(x_0) = g'(x_0) = \dots = g^{n}(x_0) = 0
\]
Ci rimane quindi da calcolare solo $g^{n+1}(x)$.\newline
Anche per $w$ notiamo che in $x_0$ le derivate sono nulle:
\[
    w(x_0) = w'(x_0) = \dots = w^{n}(x_0) = 0
\]
Anche qui quindi ci rimane da calcolare solo $w^{n+1}(x)$.
\[
    g^{n+1}(x) = f^{n+1}(x)
\]
\[
    w^{n+1}(x) = (n+1)!
\]
A questo punto vogliamo dimostrare che
\[
    f(x) = T_n(x) + \frac{f^{n+1}(\theta)}{(n+1)!} \cdot (x-x_0)^{n+1}
\] 
Per farlo portiamo $T_n(x)$ a primo menbro otteniamo:
\[
    f(x) - T_n(x) = \frac{f^{n+1}(\theta)}{(n+1)!} \cdot (x-x_0)^{n+1}
\]
Ora dividiamo per $(x-x_0)^{n+1}$
\[
    \frac{f(x) - T_n(x)}{(x-x_0)^{n+1}} = \frac{f^{n+1}(\theta)}{(n+1)!}
\]
Dimostreremo il teorema in questa forma.\newline
\newline
Riprendiamo le due funzioni ausiliarie
\[
    \frac{g(x)}{w(x)} = \frac{ g(x) - g(x_0)}{ w(x) - w(x_0)} = 
\]
dove $g(x_0) = w(x_0) = 0$ e col Corollario 1 ( Cauchy ), otteniamo che $\exists\;\; x_1 \in(x_0, x)$:
\[
    = \frac{g'(x_1)}{w'(x_1)} = \frac{g'(x_1) -g'(x_0)}{w'(x_1)- w'(x_0)} =
\]
riapplicando ora nuovamente il Corollario 1 ( Cauchy ), otteniamo che $\exists\;\; x_2 \in(x_0, x_1)$:
\[
    = \frac{g''(x_2)}{w''(x_2=} = \frac{ g''(x_2) - g''(x_0)}{w''(x_2)- w''(x_0)} =
\]
riapplichiamo per la terza volta il Corollario 1 ( Cauchy ) restringendo sempre di più l'intervallo, quindi otteniamo $\exists\;\; x_3 \in (x_0, x_2)$:
\[
    = \frac{g'''(x_3)}{w'''(x_3)} = \dots
\]
iterando $n$ volte otteniamo:\newline
$\;\;\exists\;\; \theta \in (x_0, X_n)$:
\[
    \dots = \frac{g^{n+1}(\theta)}{w^{n+1}(\theta)} = \frac{f^{n+1} ( \theta)}{(n+1)!}
\]
\newpage