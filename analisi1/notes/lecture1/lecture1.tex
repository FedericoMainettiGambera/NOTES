\section*{30/09/19 - LEZIONE}
\subsection*{Formula di Newton [mancano primi 20 minuti]}

\subsubsection*{es. trovare il coefficiente di $b^7$ nell espressione $(a^3b^2-b)^5$}
\[
    (a^3b^2-b)^5 = b^5(a^3b-1)^5
\] 
il coefficiente di $b^7$ e uguale a quello di $b^2$ in $(a^3b-1)^5$
\[
    (a^3b^2-b)^5 = \sum_{k=0}^5{\binom{5}{k}[a^3b]^k(-1)^5-k}
\] 
per ottenere il coefficiente di $b^2$ devo prendere $k=2$
\[
    \binom{5}{2}(a^3)^2b^2(-1)^{5-2}
\] 
\[
    -\binom{5}{2}a^6b^2
\] 
il coefficiente cercato è $-\binom{5}{2} = -\frac{5!}{2!3!} = 10$.

\subsubsection*{es. dimostrare che la seguente uguaglianza vale}
\[
    \sum_{k=0}^{n}\binom{n}{k}s^k = 3^n
\] 
é possibile dimostrare questa formula per induzione, ma in realtà è molto più semplice usare direttamente la formula di Newton $\sum_{k=0}^n\binom{n}{k}a^kb^{n-k}=(a+b)^n$. Infatti se poniamo $a=2$ e $b=1$ otteniamo esattamente l'equazione della consegna.

\subsubsection*{es dimostrare che la seguente uguaglianza vale}
\[
    \sum_{k=0}^{2n}\binom{2n}{k}(-2)^k=1
\] 

\subsection*{Calcolo combinatorio}
\textbf{dim.} dimostrazione combinatoria della formula di Newton
\[
    (a+b)^n = \sum_{k=0}^n\binom{n}{k}a^kb^{n-k}
\]
n.b. $\binom{n}{k} = P^{*}_{k,n-k}$ dove $P^*_{k,n-k}$ rappresenta il numero delle permutazioni con ripetizione di n oggetti k di un tipo e n-k dell'altro.\\
$(a+b)^n = (a+b)_1(a+b)_2\dots(a+b)_n$. Nello svolgere questi prodotti avrò scelto $k$ volte $a$ e $n-k$ volte $b$ per ottenere $a^kb^n-k$. E' come avere n caselle di cui le prime $k$ occupate da $a$ e le restanti $n-k$ occupate da $b$:
\[
    [a_1]_1[a_2]_2\dots[a_k]_k[b_1]_{k+1}[b_2]_{k+2}\dots[b_{n-k}]_n
\]
Ma una configurazione così può presentarsi $P^*_{k,n-k}$ (cioè $\binom{n}{k}$) volte.
\newline
\newline
\textbf{def.} dato un insieme $X$ di $n$ oggetti distinti, chiamo combinazioni semplici (senza ripetioni) di classe $K$ un qualsiasi sottoinsieme (il cui ordine non importa) di $k$ oggetti estratti da $X$.\newline
$C_{n,k}$ è il simbolo che rappresenta il numero di combinazioni semplici di $k$ oggetti estratti senza ordine.
\newline
\newline
\textbf{teor.} $C_{n,k} = \binom{n}{k}$
\newline
\textbf{dim.} Immaginiamo di dover inserire elementi in delle caselle per ottenere uno specifico allineamento. abbiamo k caselle e n elementi. Nella prima casella posso scegliere fra n elementi da inserire, nella seconda potrò scegliere fra n-1 elementi e così via. 
\[
    n(n-1)\dots(n-k+1)
\] 
Ma questo modo di ragionare rappresenta un allineamento con questo ordine, noi non vogliamo considerare l'ordine. Allora dividiamo il risultato trovato prima per le $k!$ permutazioni possibili.
\[
    \frac{n(n-1)\dots(n-k+1)}{k!} = \binom{n}{k} = \frac{n!}{k!(n-k)!}
\] 
n.b. $n(n-1)\dots(n-k+1) = \frac{n!}{(n-k)!}$

\subsection*{Cardinalità di un insieme}
E' lo studio del numero di oggetti appartenenti a un certo insieme.
dato un insieme $X$ di $n$ oggetti, qual è la cardinalità dell'insieme delle parti di $X$?
L'insieme delle parti è l'insime di tutti i sottoinsiemi ed è rappresentato dalla lettera $\mathcal{P}(X)$
\newline
\textbf{teor.} L'insieme delle parti $\mathcal{P}(X)$ di un insieme $X$ di cardinalità $n$ ha cardinalità $2^n$
\newline
\textbf{dim.} X ha certamente come sottoinsiemi quelli banali, cioè $\O$ e $X$ stesso.
\newline
Quanti sottoinsiemi di 1 elemento ha $X$? $n$ (riscrivibile come $C_{n,1}$)
\newline
Quanti sottoinsiemi di 2 elemento ha $X$? $C_{n,2}$
\newline
\dots
\newline
Quanti sottoinsiemi di n-1 elemento ha $X$? $C_{n,n-1}$
\newline
Dal teorema precedentemente visto sappiamo che $C_{n,x} = \binom{n}{x}$
\[
    1+C_{n,1}+C_{n,2+\dots+C_{n,n-1}+1}
,\] 
dove il primo $1$ rappresenta l'insieme vuoto e l'ultimo $1$ rappresetna $X$ stesso.
\[
    1+\binom{n}{1}+\binom{n}{2}+\dots++\binom{n}{n-1}+1
\]
\[
    \sum_{k=0}^n\binom{n}{k} = (1+1)^n
\]  
(l'ultimo passaggio è stato ottenuto tramite l'utilizzo della formula di Newton)
\newline
\textbf{cons.} L'insieme delle parti di un insieme finito(!) ha cardinalità sempre maggiore dell insieme stesso.

\subsection*{Teologia in $\mathbb{R}$}
\textbf{def.} definizione fondamentale: \textbf{Intorno} di un ponto $x_0$
\newline
in $\mathbb{R}$: IMMAGINE
\newline
in $\mathbb{R}^2$: IMMAGINE
\newline
$B_r(x_0)$ è il simbolo che viene rappresenta l'intorno di raggio $r$ del punto $x_0$.
$B_r(x_0)$ è l'insieme dei punti con distanza inferiore $(<)$ di $r$ dal centro $x_0$, i punti sul bordo/confine dell'intorno non appartengono all'intorno.
\newline
definizioni formali:
\newline
in $\mathbb{R}$:
\[
    B_r(x_0) = \{x\in\mathbb{R}:|x-x_0|<r\}
\] 
\newline
in $\mathbb{R}^2$: 
\[
    B_r(p_0) = \{p\in\mathbb{R}^2:dist(p,p_0)<r\}
\]
dove $dist(p,p_0) = \sqrt{(x-x_0)^2+(y-y_0)^2}$
\newline
n.b. in $\mathbb{R}$ una qualunque semiretta è detta "intorno di $\pm\infty$". in $\mathbb{R}^2$, invece, non si parla di "intorno di $\pm\infty$".
\newline
\newline
\textbf{def.} Dato un insieme $A\subset \mathbb{R}$ oppure $\mathbb{R}^2$, $x^*$ è un punto \textbf{interno} ad $A$ se (non solo se $\subset A$, ma anche se è circondato solo da punti di $A$):
\[
    \exists B_r(x^*):B_r(x^*)\subset A
\] 
ovviamente se $x^*$ è un punto interno allora $x^*\subset A$
\newline
n.b. questa definizione non si occupa di definire l'operatore "$\in$", ma definise il termine "interno".
\newline
\newline
\textbf{def.} $x^*$ è un punto \textbf{esterno} ad $A$ (non solo se non è in $A$ ($\notin A$), ma anche se è circondato solo da punti di A):
\[
    \exists B_r(x^*):B_r(x^*)\subset A^c
\] 
ovviamente se $x^*$ è un punto esterno allora $x^*\in A^c$ ovvero $x^*\notin A$
\newline
\textbf{def.} $x^*$ è un punto di \textbf{frorntiera} (se non è né interno né esterno):
\[
    \forall B_r(x^*), B_r(x^*) \cap a\neq\O \land b_r(x^*) \cap A^c \neq \O
\] 
\textbf{es.}
\[
    a = \{(x,y)\in \mathbb{R}^2 : x^2+y^2 \leq 4 \land xy >0\}
\] 
\begin{figure}[ht]
    \centering
    \incfig{image1}
\end{figure}
\newline
\textbf{def.} Dato un insieme $A$ (in $\mathbb{R}$ o in $\mathbb{R}^2$)
\begin{itemize}
    \item è \textbf{aperto} se è fatto solo da punti interni;
    \item è \textbf{chiuso} se il suo complementare è aperto oppure se contiene tutti i suoi punti di frontiera.
\end{itemize}
\textbf{es.} l'insieme"aperto" in $\mathbb{R}$
\[
    A = (a,b)
\] 
IMMAGINE
\newline
I punti $a$ e $b$ sono di frontiera\dots
\newline
\textbf{es.} \dots se, però, cambiamo ambiente e andiamo in $\mathbb{R}^2$:
\[
    A = \{(x,y) \in \mathbb{R}^2 : a < x < b, y = 0\}
\] 
IMMAGINE
\newline
non è aperto, è costituito solo da punti di frontiera, ma non li contiene tutti, quindi non è neanche chiuso.

\newpage