\section*{1-LEZIONE}
30/09/19
\subsection*{Formula di Newton [mancano primi 20 minuti]}
\textbf{es.} Trovare il coefficiente di $b^7$ nell espressione $(a^3b^2-b)^5$.
\[
    (a^3b^2-b)^5 = b^5(a^3b-1)^5
\] 
il coefficiente di $b^7$ e uguale a quello di $b^2$ in $(a^3b-1)^5$
\[
    \sum_{k=0}^5{\binom{5}{k}[a^3b]^k(-1)^5-k}
\] 
per ottenere il coefficiente di $b^2$ devo porre $k=2$, quindi ottengo:
\[
    \binom{5}{2}(a^3)^2b^2(-1)^{5-2} = -\binom{5}{2}a^6b^2
\]
il coefficiente cercato è $-\binom{5}{2} = -\frac{5!}{2!3!} = 10$.
\newline
\newline
\textbf{es.} Dimostrare che la seguente uguaglianza vale:
\[
    \sum_{k=0}^{n}\binom{n}{k}s^k = 3^n.
\] 
E' possibile dimostrare questa uguaglianza per induzione, ma in realtà è molto più semplice usare direttamente la formula di Newton  $\sum_{k=0}^n\binom{n}{k}a^kb^{n-k}=(a+b)^n$. Infatti se poniamo $a=2$ e $b=1$ otteniamo esattamente l'equazione della consegna.
\newline
\newline
\textbf{es.} Dimostrare che la seguente uguaglianza vale:
\[
    \sum_{k=0}^{2n}\binom{2n}{k}(-2)^k=1.
\] 
Per dimostrare questa uguaglianza usiamo ancora una volta la formmula di Newton con $a=-2$ e $b=1$ e, invece di $n$, usiamo $2n$.
\[
    \sum_{k=0}^{2n} a^kb^{2n-k} = ( a+b)^{2n}.
\]

\subsection*{Calcolo combinatorio}
\textbf{dim. Dimostrazione combinatoria} della \textbf{formula di Newton}
\[
    (a+b)^n = \sum_{k=0}^n\binom{n}{k}a^kb^{n-k}
\]
Utiliziamo il termine $P^{*}_{k,n-k} = \binom{n}{k}$ per rappresentare il numero delle permutazioni con ripetizione di $n$ oggetti $k$ di un tipo e $n-k$ dell'altro.
\newline
Partiamo dalla definizione di esponenziale $(a+b)^n = (a+b)_1(a+b)_2\dots(a+b)_n$. Nello svolgere questi prodotti avrò scelto $k$ volte $a$ e $n-k$ volte $b$ per ottenere $a^kb^{n-k}$. E' come avere n caselle di cui le prime $k$ occupate da $a$ e le restanti $n-k$ occupate da $b$:
\[
    [a_1]_1[a_2]_2\dots[a_k]_k[b_1]_{k+1}[b_2]_{k+2}\dots[b_{n-k}]_n
\]
Ma una configurazione così può presentarsi $P^*_{k,n-k}$ (cioè $\binom{n}{k}$) volte.
\newline
Da qui quindi arriviamo alla forma:
\[
    \sum_{k=0}^n\binom{n}{k}a^kb^{n-k}
\]
\newline
\newline
\textbf{def.} dato un insieme $X$ di $n$ oggetti distinti, chiamo \textbf{combinazioni semplici (senza ripetioni) di classe $K$} un qualsiasi sottoinsieme (il cui ordine non importa) di $k$ oggetti estratti da $X$.\newline
\textbf{$C_{n,k}$} è il simbolo che rappresenta il numero di combinazioni semplici di $k$ oggetti estratti senza ordine.
\newline
\newline
\textbf{teor.}
\[
    C_{n,k} = \binom{n}{k}
\]
\textbf{dim.} Immaginiamo di dover inserire elementi in delle caselle per ottenere uno specifico allineamento. abbiamo $k$ caselle e $n$ elementi. Nella prima casella posso scegliere fra $n$ elementi da inserire, nella seconda potrò scegliere fra $n-1$, poi fra $n-2$ elementi e così via. 
\[
    n(n-1)\dots(n-k+1)
\] 
Seguendo questo ragionamento abbiamo disposto in maniera \textbf{ordinata} gli elementi lungo un allineamento. Se non si vuole considerare l'ordine, dobbiamo dividere il risultato trovato per le $k!$ permutazioni possibili:
\[
    \frac{n(n-1)\dots(n-k+1)}{k!} = \binom{n}{k} = \frac{n!}{k!(n-k)!}
\] 
n.b. $\;\;\;n(n-1)\dots(n-k+1) = \frac{n!}{(n-k)!}$

\subsection*{Cardinalità di un insieme}
E' lo studio del numero di oggetti appartenenti a un certo insieme.
\newline
Dato un insieme $X$ di $n$ oggetti, qual è la cardinalità dell'insieme delle parti di $X$? L'insieme delle parti è l'insime di tutti i sottoinsiemi ed è rappresentato dalla lettera $\mathcal{P}(X)$
\newline
\newline
\textbf{teor.} L'insieme delle parti $\mathcal{P}(X)$ di un insieme $X$ di cardinalità $n$ ha cardinalità $2^n$.
\newline
\textbf{dim.} X ha certamente come sottoinsiemi quelli banali, cioè $\O$ e $X$ stesso.
\newline
Quanti sottoinsiemi di 1 elemento ha $X$? $n$ (riscrivibile come $C_{n,1}$, cioè come combinazione semplice senza ripetizioni di classe $1$)
\newline
Quanti sottoinsiemi di 2 elemento ha $X$? $C_{n,2}$
\newline
\dots
\newline
Quanti sottoinsiemi di n-1 elemento ha $X$? $C_{n,n-1}$
\newline
Dal teorema precedentemente visto sappiamo che $C_{n,x} = \binom{n}{x}$, quindi:
\[
    1+C_{n,1}+C_{n,2}+\dots+C_{n,n-1}+1 \;\;=\;\; 1+\binom{n}{1}+\binom{n}{2}+\dots++\binom{n}{n-1}+1
\] 
dove il primo $1$ rappresenta $/O$ e l'ultimo $1$ rappresetna $X$ stesso.
\newline
Ora questa espressione può essere raccolta in una sommatoria e tramite la formula di Newton possiamo scrivere:
\[
    \sum_{k=0}^n\binom{n}{k} = (1+1)^n = 2^n
\]  
\newline
\textbf{oss.} L'\textbf{insieme delle parti} di un insieme \textbf{finito} ha cardinalità sempre \textbf{maggiore} dell insieme stesso.

\subsection*{Topologia in $\mathbb{R}$}
\subsubsection*{Intorno}
\textbf{def.} concetto fondamentale è quello di \textbf{intorno} di un punto $x_0 \in \mathbb{R}$ o $(x_0,y_0) \in \mathbb{R}^2$.
\newline
in $\mathbb{R}$: IMMAGINE
\newline
in $\mathbb{R}^2$: IMMAGINE
\newline
$B_r(x_0)$ è il simbolo che rappresenta l'intorno di raggio $r$ del punto $x_0$.
\newline
$B_r(x_0)$ è l'insieme dei punti con distanza inferiore $(<)$ di $r$ dal centro $x_0$. Da notare è il fatto che i punti sul bordo/confine dell'intorno non appartengono all'intorno.
\newline
Vediamo ora una definizione formale:
\begin{itemize}
    \item in $\mathbb{R}$:
        \[
            B_r(x_0) = \{x\in\mathbb{R}:|x-x_0|<r\}
        \] 
    \item in $\mathbb{R}^2$: 
        \[
            B_r(p_0) = \{p\in\mathbb{R}^2:dist(p,p_0)<r\}
        \]
        dove $dist(p,p_0) = \sqrt{(x-x_0)^2+(y-y_0)^2}$
\end{itemize}
\textbf{oss.} in $\mathbb{R}$ una qualunque semiretta è detta "intorno di $\pm\infty$". in $\mathbb{R}^2$, invece, non si parla di "intorno di $\pm\infty$".
\subsubsection*{Punti interni, esterni e di frontiera}
\textbf{def.} Dato un insieme $A\subset \mathbb{R}$ oppure $\mathbb{R}^2$, $x^*$ è un punto \textbf{interno} ad $A$ se 
\[
    \exists B_r(x^*):B_r(x^*)\subset A
\] 
\textbf{oss.}  Ovviamente se $x^*$ è un punto interno allora $x^*\subset A$, ma per essere un punto interno non deve solo essere $\subset$ in $A$, ma anche se è circondato solo da punti di $A$).
\newline
\textbf{oss.}  questa definizione non si occupa di definire l'operatore "$\in$", ma definise il concetto di punto "interno".
\newline
\newline
\newline
\textbf{def.} $x^*$ è un punto \textbf{esterno} ad $A$ se
\[
    \exists B_r(x^*):B_r(x^*)\subset A^c
\] 
\textbf{oss.} Ovviamente se $x^*$ è un punto esterno allora $x^*\in A^c$ ovvero $x^*\notin A$, ma per essere un punto esterno ad $A$ non deve $\in$ ad $A^c$ e, inoltre, deve essere circondato solo da punti di $A^c$
\newline
\newline
\newline
\textbf{def.} $x^*$ è un punto di \textbf{frontiera} se
\[
    \forall B_r(x^*), \; (B_r(x^*) \cap a )\neq \O \;\;\land \;\; (B_r(x^*) \cap A^c) \neq \O
\] 
\textbf{oss.} Un punto è di frontiera se non é nè interno nè esterno.
\newline
\newline
\subsubsection*{Insiemi aperti e chiusi}
\textbf{def.} Dato un insieme $A$ (in $\mathbb{R}$ o in $\mathbb{R}^2$)
\begin{itemize}
    \item si dice \textbf{aperto} se è fatto solo da punti interni;
    \item si dice \textbf{chiuso} se il suo complementare è aperto oppure se contiene tutti i suoi punti di frontiera.
\end{itemize}
\textbf{es.} l'insieme in $\mathbb{R}$
\[
    A = (a,b)
\] 
è un insieme aperto e i punti $a$ e $b$ sono di frontiera.
\newline
\textbf{es.} se, però, cambiamo l'ambiente da $\mathbb{R}$ a $\mathbb{R}^2$ otteniamo l'insieme
\[
    A = \{(x,y) \in \mathbb{R}^2 \;\;:\;\; a < x < b ,\;\; y = 0\}
\] 
che non è aperto in quanto è costituito solo da punti di frontiera, ma non li contiene tutti, quindi non è neanche chiuso.
\newline
IMMAGINE
\newpage