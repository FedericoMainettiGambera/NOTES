\section*{7-LEZIONE}
31/10/19
\subsection*{[manca primi 20 minuti]}
\subsubsection*{Teorema di Bolzano o teorema degli zeri [dimostrazione iterativa, manca]}
presa una funzione $f \;\;:\;\; A \subseteq \mathbb{R} \longrightarrow \mathbb{R}$:\newline
ipotesi:
\begin{itemize}
    \item $A=[a,b]$ è un intervallo compatto
    \item $f$ è continua su $A$
    \item $f(a) \cdot  f(b) < 0$ 
\end{itemize}
\textbf{teor.} $\;\;\exists\;\; x^* \in (a,b) \;\;:\;\; f(x^*)=0$\newline
\newline
\textbf{dim.} Per fissare le idee posiamo $f(a)>0$ e $f(b)<0$\newline
Procediamo per bisezione: si prende... 
\subsubsection*{Teorema di Darboux o teorema dei valori intermedi}
E' una conseguenza del teorema di Bolzano e Weierstrass.\newline
Enunciato:
\[
    f:A \subseteq \mathbb{R} \longrightarrow \mathbb{R} 
\]
\[
    x \longrightarrow y=f(x)
\]
ipotesi:
\begin{itemize}
    \item $A=[a,b]$ intervallo compatto
    \item f è continua su $A$
\end{itemize}
Tesi:\newline
Con $m$ minimo e $M$ massimo valori di massimo della funzione in $A$, possiamo dire che $\;\forall\;\lambda $ con $ m < \lambda< M \;\;\exists\;\; x_\lambda \in A \;\;:\;\; f(x_\lambda) = \lambda$.\newline
\newline
\textbf{dim.} Valendo Weierstrass, so che esistono $M$ e $m$ e almeno un valore di massimo $x_M$ e uno di minimo $x_m$.
\[
    f(x_m) = m \leq f(x) \leq M = f(x_M)
\]
Introduco ora una funzione ausiliaria $g(x) = f(x) - \lambda$ che è la funzione $f$ traslata "in giù" di $\lambda$ (se $\lambda > 0$). \newline
Nota: $g$ ha la stessa regolarità di $f$ (è continua).\newline
Inoltre $g$, studiata nell'intervallo $[x_m,x_M]$ soddisfa le ipotesi del teorema di Bolzano, quindi le valutazioni di $g$ agli estremi dell'intervallo hanno segno opposto.
\[
    g(x_m) = f(x_m) - \lambda = m-\lambda < 0
\]
\[
    g(x_M) = f(x_M) - \lambda = M-\lambda > 0
\]
Per il teorema di Bolzano, quindi, $\;\;\exists\;\; x^* \;\;:\;\; g(x^*) = 0 = f(x^*)-\lambda = 0$.\newline
\newline
\newline
\subsubsection*{Cardinalità degli insiemi infiniti}
\textbf{def.} due insiemi infiniti hanno lo stesso numero di elementi (la \textbf{stessa cardinalità}) se e solo se riesco a costruire una \textbf{corrispondenza biunivoca} tra i due insiemi.\newline
\newline
\textbf{def.} un'insieme $X$ è \textbf{numerabile} se posso scrivere i suoi elementi in una successione (ovvero se può essere messo in corrispondenza biunivoca con $\mathbb{N}$).\newline
\begin{itemize}
    \item $\mathbb{N}$ è numerabile? sì
    \item $\mathbb{Z}$ è numerabile? è chiaro che $\mathbb{N} \subset \mathbb{Z}$, ma si può costruire una successione ordinata che copra tutti gli elementi di $\mathbb{Z}$? sì, per esempio: $ \;0, \;1, -1, \;2, -2, \;3, -3, \;\;\dots$, quindi $\mathbb{Z}$ è numerabile.
    \textbf{oss.} Da notare è che un insieme infinito può essere messo in corrispondenza biunivoco con una sua parte.
    \item $\mathbb{Q}$ è numerabile? è chiaro, anche qui, che $\mathbb{N} \subset \mathbb{Q}$, ma può essere messo in corrispondenza biunivoca con $\mathbb{N}$? \newline
    Vediamo come si può scrivere una successione che elenca tutti i razionali:
    \[
        \begin{bmatrix}
            0& 1& 2& 3& 4&\\
            0& -1& -2& -3& -4&\\
            0& \frac{1}{2}& \frac{2}{2}& \frac{3}{2}& \frac{4}{2}&\\
            0& -\frac{1}{2}& -\frac{2}{2}& -\frac{3}{2}& -\frac{4}{2}&\\
            0& \frac{1}{3}& \frac{2}{3}& \frac{3}{3}& \frac{4}{3}&\\
            0& -\frac{1}{3}& -\frac{2}{3}& -\frac{3}{3}& -\frac{4}{3}&\\
        \end{bmatrix}
    \]
    Ora per dare un ordine a questa tabella, che copre tutti i razionali, ci basta percorrerla in diagonali che vanno verso il basso a sinistra.\newline
    La prima diagonale contiene solo l'elemento $0$, la seconda contiene gli elementi $1$ e $0$, poi abbiamo $2, -1, 0$\dots e così via.\newline
    Quindi sì, $\mathbb{Q}$ è numerabile.
\end{itemize}
$\mathbb{R}$ è numerabile? il problema è che $\mathbb{R}$ è continua e di conseguenza non  numerabile.\newline
\textbf{def.} si dice che $\mathbb{R}$ ha cardinalità superiore a quella numerabile, detta \textbf{cardinalià del continuo}.\newline
\textbf{dim.} dimostriamo che non è possibile numerare tutti gli elementi di $\mathbb{R}$.\newline
Per semplicità dimostriamo che non posso numerare nemmeno una parte di $\mathbb{R}$, per esempio nell'intervallo $[0,1]$.\newline
Dimostrazione per assurdo:\newline
Se l'intervallo $[0,1]$ fosse numerabile, potrei scrivere i suoi elementi in una succesione, per esempio:\newline
$x_0$: 0.1327401101279 \dots\newline
$x_1$: 0.00010104 \dots\newline
$x_2$: 0.1110140510000 \dots\newline
$x_3$: 0.31313131 \dots\newline
$x_4$: 0.00500010 \dots\newline
$x_5$: \dots \newline
Ma per quanto grande sia la lista, sarà sempre incompleta. Trovo in fretta un numero reale compreso fra $0$ e $1$ che in questa successione non appare, costruiamolo:\newline
Prendiamo la prima cifra da $x_0$, che è un uno, poi prendiamo la seconda cifra dell'elemento $x_1$, poi il terzo dell'elemtno $x_2$ e così via ottengo:\newline
$0.101104 \dots$\newline
Ora sommiamo uno a ogni cifra ottenuta $1$:\newline
$0.212215 \dots$\newline
E così ottengo sicuramente un elemento che non appartiene alla lista, perchè ogni elemento ha almeno una cifra diversa da quella del numero appena costurito.
\newline
\newline
\newline
Anche i numeri \textbf{irrazionali} ($\mathbb{I}$) hanno la cardinalità del \textbf{continuo}.\newline
\newline
\newline
Presa una retta, essa ha cardinalità del continuo, come $\mathbb{R}$, ma un piano? che cardinalità ha? Si riesce a costruire una corrispondenza biunivoca fra $\mathbb{R}$ e $\mathbb{R}^2$? [Riflettere a casa.]
\newpage