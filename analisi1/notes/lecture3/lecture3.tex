\section*{3-LEZIONE}
10/10/19
\subsection*{[perso i primi 20 minuti]}
Sta parlando di limiti\dots
\newline
\[
    \forall B_s(l) \;\exists\; B_q(x_0) \;\;:\;\; \forall x\in A \cap B_q(x_0)-\{x_0\} \;\;,\;\;f(x) \in B_s(l)
\]
"per ogni intorno del valore limite $l$" definitivamente vicino a $x_0$ la funzione sta nell'intorno del valore limite.
\subsection*{Algebra dei limiti}
Per poter fare dei confronti rigorosi fra limiti, ovvero fra funzioni, ho bisogno di introdurre due simboli:
\begin{itemize}
    \item asintotico: $\sim$
    \item o-piccolo: $o$
\end{itemize}
\subsection*{o-piccolo}
\textbf{def.} $f = o(g)$ per $x \rightarrow x_0$ se $f$ è trascurabile rispetto $g$. Cioè se, confrontando $f$ e $g$, $f$ perde.
\newline
\newline
\textbf{def.} Definizione formale:
\[
    f = o(g) \;\; se \;\; f(x) = g(x)h(x) \;\; e \;\; h(x) \xrightarrow[x\rightarrow x_0] \; 0
\]
\textbf{oss.} conseguenza è che
\[
    \frac{f(x)}{g(x)}\xrightarrow[x\rightarrow x_0] \; 0
\]
\newline
\textbf{es.} Per $x\longrightarrow 0$ dire se le seguenti uguaglianze sono vere o false
\[
    x^2 = o(x) \rightarrow vera
\]
\[
    x=o(x^2) \rightarrow falsa
\]
\textbf{es.} Per $x\longrightarrow +\infty$ dire se le seguenti uguaglianze sono vere o false
\[
    x^2 = o(x) \rightarrow  falsa
\]
\[
    x=o(x^2) \rightarrow  vera
\]
\newline
\textbf{regola.} Nell'intorno dell'origine (tendendo a $\rightarrow  0$) potenze alte della variabili sono trascurabili, cioè sono o-piccolo rispetto a potenze più basse.
\newline
\textbf{regola.} allontanandosi dall'origine (tendendo a $\rightarrow \pm \infty$)  potenze basse sono trascurabili rispetto a potenze più alte, cioè sono o-piccolo rispetto a potenze più alte.
\newline
\subsubsection*{Proprietà di o-piccolo}
\begin{itemize}
    \item Costanti in o-piccolo.
        \newline
        Con $k \in \mathbb{R}$ e costante:
        \[
            o(k \cdot  g) = o(g) = k \cdot  o(g)
        \]
        \textbf{dim.} 
        \[
            f = o( k \cdot  g) \rightarrow f = o(g)
        \] 
        \[
            f= k \cdot  g \cdot  h
        \]
        ma $h\rightarrow0$, quindi $k \cdot  h \rightarrow 0$
        \[
            f = o(g)
        \]
    \item Somma di o-piccoli.
        \[
            o(g) \pm o(g) = o(g)
        \]
        \textbf{dim.} conseguenza della proprietà precedente.
        \newline
        \textbf{oss.} errore tipico: $o(g) - o(g) = 0$. SBAGLIATISSIMO.
        \newline
        \textbf{es.} per $z\longrightarrow + \infty$
        \[
            f_1=x^3 -2x +4 = x^3 + o(x^3)
        \]
        \[
            f_2 = x^3 + x^2 -7 = x^3 + o(x^3)
        \]
        \[
            f_1 - f_2 = x^3 + o(x^3) -( x^3 + o(x^3)) = o(x^3) - o(x^3) \neq 0
        \]
        \[
            f_1 - f_2 = -2x +4 - x^2 + 7 
        \]
    \item Prodotto di funzioni e o-piccolo.
        \newline
        Con $f$ una funzione
        \[
            f \cdot o(g) = o(f \cdot g)  
        \]
        \newline
        \textbf{dim.} 
        \[
            F = o(g) = g \cdot h \;\; e \;\; h\rightarrow0
        \] 
        moltiplico entrambe le parti per $f$
        \[
            f \cdot F = f \cdot g \cdot h
        \]
    \item Potenze di o-piccolo.
        \newline
        Con $k \in \mathbb{R}^{+}$
        \[
            [o(g)]^k = o(g^k)  
        \]
        \textbf{dim.}
        \[
            G = o(g)
        \] 
        \[
            G = g \cdot h \;\; e \;\; h\rightarrow0
        \]
        elevo tutto alla $k$
        \[
            G^k = g^k \cdot h ^k \;\;\;\;\; H = h^k \rightarrow 0
        \]
\end{itemize}
\subsection*{Asintotico}
\textbf{def.} $f$ è \textbf{asintotico} a $g$ se tendono allo stesso valore e inoltre ci tendono allo stesso modo.
\[
    f \sim g \;\;\; x \rightarrow x_0
\]
formalemente:
\[
    f \sim g \;\;\; se \;\;\; f=g \cdot  h \;\; e \;\; h\rightarrow 1
\]
\textbf{oss.} conseguenza:
\[
    \frac{f}{g} \rightarrow 1
\]
\newline
\newline
\textbf{teor.} \textbf{teorema fondamentale} che lega $ \sim $ e $o()$  
\newline
per $x\rightarrow x_0$
\[
    f \sim g \Longleftrightarrow f = g + o(g)
\]
Due funzioni sono asintotiche se hanno lo stesso termine dominante.
\newline
\newline
\textbf{dim.} dimostrazione da sinistra a destra $(\Rightarrow)$:
\newline
ipotesi: $ f= g \cdot  h$ e $ h \rightarrow 1$. Sottraggo $g$ da entrambi i membri:
\[
    f-g = g \cdot h -g = g(h-1) \;\;\; H = h-1  \longrightarrow 0
\]
\[
    f-g = o(g)
\]
\[
    f=g+o(g)
\]
\textbf{dim.} dimostrazione da destra a sinistra $(\Leftarrow)$
\newline
ipotesi: $f-g = o(g)$
\[
    f-g = g \cdot h \;\;\;h \rightarrow 0
\]
\[
    f = g + g \cdot h = g (1+h) \;\;\; H = h+1 \rightarrow 1
\]
\[
    f \sim g
\]
\subsubsection*{Proprietà di asintotico}
\begin{itemize}
    \item Potenza di funzioni asintotiche:
        \[
            f \sim g \Longleftrightarrow    f^k \sim g^k 
        \] 
        \newline
        \textbf{dim.} [manca la dimostrazione]
        \[
            f = g \cdot  h \;\;\; \dots
        \] 
    \item Prodotti e rapporti di funzioni asintotiche.
        \newline
        $f_1 \sim g_1$ e $f_2 \sim  g_2$ allora
        \[
            f_1 \cdot  f_2 \sim  g_1 \cdot  g_2    
        \]
        \[
            \frac{f_1}{f_2} \sim \frac{g_1}{g_2}
        \]
        prodotti e rapporti di funzioni asintotiche sono asintotici fra loro.
        \newline
        \newline
        \textbf{dim.} [manca la dimostrazione]
        \[
            f_1 \sim  g_1
        \] 
        \[
            f-1 = g_1 \cdot  h-1 \;\;\; e \;\;\; h_1 \rightarrow 1
        \]
        \[
            f_2 \sim  g_2
        \] 
        \[
            f-1 = g_1 \cdot  h-1 \;\;\; e \;\;\; h_1 \rightarrow 1
        \]
        \[
            \dots
        \]
        \[
            \frac{f_1}{f_2} = \frac{g_1}{g_2} \cdot H \;\;\; H = \frac{h_1}{h_2}\rightarrow 1
        \] 
\end{itemize}
\textbf{oss.} Notare che non c'è una proprietà per la somma di asintotici, che invece nell'o-piccolo c'è.
\subsection*{Limiti notevoli}
[Stiamo guardando simulazioni su MATLAB. Osserviamo che il seno nell'intorno dell origine è approssimabile con la bisettrice, il cosendo con la parabola, la funzione esponenziale traslata in giù di un unità con la bisettrice, il logaritmo traslato a sinistra di un unità con la bisettrice,etc.]
\newline
Vediamo ora in formule questi risultati:
\begin{itemize}
    \item \textbf{Seno} \newline
        per $x \rightarrow 0$
        \[
            sin (x) = x +o(x)
        \]
        questa uguaglianza si può riscrivere come $sin(x) - x = o(x)$, cioè o(x) è l'errore che sto facendo nell'approssimare $sin(x)$ come $x$.
        \newline
        img1
        \newline
        Vediamo altre due forme utili della stessa uguaglianza:
        \[
            sin(x) \sim x
        \]
        \[
            \lim_{x\rightarrow0} \frac{sin(x)}{x} = 1
        \]
    \item \textbf{Coseno} \newline
        per $x \rightarrow 0$
        \[
            cos (x) = 1- \frac{1}{2}x^2 +o(x^2)
        \]
        \[
            cos(x) -1 = -\frac{1}{2}x^2 + o(x^2)
        \]
        img2
        \newline
        Vediamo altre due forme utili della stessa uguaglianza:
        \[
            cos(x) -1 \sim -\frac{1}{2}x^2
        \]
        \[
            \lim_{x\rightarrow 0} \frac{cos(x) -1}{-\frac{1}{2}x^2} = 1
        \]
    \item \textbf{Esponenziale} \newline
        per $x \rightarrow 0$
        \[
            e^x -1 = x+o(x)
        \]
        img3
        \newline
        Vediamo altre due forme utili della stessa uguaglianza:
        \[
            e^x -1 \sim x
        \]
        \[
            \lim_{x\rightarrow 0} \frac{e^x-1}{x} = 1
        \]
    \item \textbf{Logaritmo} \newline
        per $x \rightarrow 0$
        \[
            ln(1+x) = x +o(x) 
        \]
        img4
        \newline
        Vediamo altre due forme utili della stessa uguaglianza:
        \[
            ln(1+x) \sim x 
        \]
        \[
            \lim_{x\rightarrow0}\frac{ln(1+x)}{x} = 1
        \]
\end{itemize}
Per ognuno di questi limiti notevoli sono state fornite tre versioni che rappresentano la stessa cosa, la più importante e più ricca di significato è sempre la prima, quella con o-piccolo.
\newline
\newline
\newline
\textbf{es.} Studiare il seguente limite
\[
    \lim_{x\rightarrow + \infty} \frac{\sqrt{3x^4-x}}{x^2-1}
\]
\[
    f(x) = \frac{\sqrt{3x^4-x}}{x^2-1} \;\;\;\;\;\; x\rightarrow + \infty
\]
numeratore:
\[
    3x^4-x = 3x^4 + o(3x^4) = 3x^4 + o(x^4)
\]
\[
    3x^4-x  \sim  3 x^4
\]
\[
    (3x^4-x)^{\frac{1}{2}}  \sim  (3 x^4)^{\frac{1}{2}}
\]
\[
    \sqrt{3x^4-x} \sim  \sqrt{3 x^4}
\]
\[
    \sqrt{3x^4-x} = \sqrt{3x^4} + o(x^2)
\]
\[
    x^2-1 = x^2 +o(x^2)
\]
\[
    x^2 -1 \sim x^2
\]
usiamo ora la proprietà dei rapporti di funzioni asintotiche:
\[
    \frac{\sqrt{3x^4-x}}{x^2-1} \sim \frac{\sqrt{3x^4}}{x^2} = \sqrt{3}
\]
\[
    \lim_{x\rightarrow+\infty} = \sqrt{3}
\]
\subsubsection*{Limite notevole della pontenza $\alpha$-esima con $0<\alpha<1$}
\[
    y = x^{\alpha} \;\;\; con \;0<\alpha<1
\] 

$per \;\; x \rightarrow 0$
\newline
\[
    (1+x)^{\alpha} = 1+\alpha x + o(x)
\]
\[
    (1+x)^{\alpha} -1 = \alpha x +o(x)
\]
img5
\newline
Questo limite notevole viene usato per risolvere le radici.
\newline
Forniamo, come per gli altri limiti notevoli visti, le altre due forme notevoli:
\[
    (1+x)^{\alpha} -1 \sim \alpha x
\]
\[
    \lim_{x \rightarrow 0} \;\; \dots [manca]
\]
\newline
\newline
\textbf{es.} Studiare il seguente limite:
\[
    \lim_{x\rightarrow 0} \frac{\sqrt{1+2x^2-x^3}-cos(x)}{sin(x)}
\]
per studiare il limite analiziamo l'andamento asintotico della seguente funzione:
\[
    f(x) = \frac{\sqrt{1+2x^2-x^3}-cos(x)}{sin(x)} \;\;\; per \;\; x \rightarrow 0
\]
occupiamoci della radice con la proprietà appena vista:
\[
    (1+t)^{\alpha} -1 = \alpha t +o(t) \;\;\; per \;\; t \rightarrow 0
\]
se prendo $t = 2x^2 -x^3$ e $\alpha = \frac{1}{2}$
\[
    (1+2x^2-x^3)^{\frac{1}{2}} = 1 + \frac{1}{2} (2x^2-x^3) + o(2x^2 - x^3)
\]
Analizzando $o(2x^2 - x^3)$, noto che $x^3$ è trascurabile rispetto a $2x^2$ (ricorda che $x \rightarrow 0$), Inoltre la costatne $2$ non conta nell'o-piccolo, quindi:
\[
    1 + \frac{1}{2} (2x^2-x^3) + o(x^2)
\]
Inoltre anche in $(2x^2-x^3)$ posso ignorare $x^3$ per lo stesso motivo, quindi:
\[
    (1+2x^2-x^3)^{\frac{1}{2}} = 1 + \frac{1}{2} (2x^2) + o(x^2) = 1 + x^2 + o(x^2)
\]
Quindi tornando alla funzione originale
\newline
Numeratore:
\[
    1 + x^2 + o(x^2) -[1-\frac{1}{2}x^2 +o(x^2)]= \frac{3}{2}x^2 +o(x^2)
\]
Denominatore:
\[
    x+o(x)
\]
Numeratore/denominatore:
\[
    \frac{\sqrt{1+2x^2-x^3}-cos(x)}{sin(x)} = \frac{\frac{3}{2}x^2 +o(x^2)}{x+o(x)} \sim \frac{\frac{3}{2}x^2}{x} \sim \frac{3}{2}x
\]
Quindi la funzione tende a $0$ con la pendenza di $\frac{3}{2}x$
\newline
\newline
\newline
\textbf{es.} Studiare il seguente limite:
\[
    \lim_{x\rightarrow + \infty} \frac{2\sqrt{x-1}-\sqrt[4]{16x^2-2}}{\sqrt{x+1}}
\]
Analiziamo $\sqrt[4]{16x^2-2}$, vorremmo usare il limite notevole della potenza $\alpha$-esima con $0<\alpha<1$, in questo caso $\alpha = \frac{1}{4}$:
\[
    (1+t)^{\frac{1}{4}} = 1+\frac{1}{4}t +o(t) \;\;\; t \rightarrow 0
\]
ma noi nell'espressione della consegna abbiamo $x\rightarrow + \infty$ e non possiamo quindi usare il limite notevole, perciò:
\[
    = \sqrt[4]{16x^2-2} = \sqrt[4]{16x^2(1-\frac{2}{16x^2})} = 2 \sqrt{|x|} (1- \frac{1}{8x^2})^{\frac{1}{4}} = 
\]
Ora possiamo usiamo il limite notevole con $t = -\frac{1}{8x^2}$, perchè ora $\frac{1}{x^2}$ tende a $0$ per $x\rightarrow + \infty$
\[
    = 2\sqrt{x}(1+\frac{1}{4} \cdot (-\frac{1}{8x^2})) + o(\frac{1}{x^2}) =
\]
\[
    = 2 \sqrt{x} - \frac{1}{16} \frac{1}{x^{\frac{3}{2}}} + o(\frac{1}{x^{\frac{3}{2}}})
\]
Analiziamo ora $\sqrt{x-1}$:
\[
    \sqrt{x-1} = \sqrt{x(1-\frac{1}{x})} = \sqrt{x} (1-\frac{1}{x})^{\frac{1}{2}} =
\]
\[
    =\sqrt{x}[ 1 + \frac{1}{2} (-\frac{1}{x}) + o(-\frac{1}{x}) ] = 
\]
essendo il $-$ dentro all'o-piccolo, lo considero come una costante ($-1$) e quindi lo posso togliere:
\[
    =\sqrt{x}  - \frac{1}{2\sqrt{x}}  + o(\frac{1}{\sqrt{x}}) 
\]
Ora, quindi, ottengo:
\[
    -\frac{1}{\sqrt{x}} + \frac{1}{16} \cdot \frac{1}{x^{\frac{3}{2}}} + o(\frac{1}{x^{\frac{3}{2}}}) + o(\frac{1}{\sqrt{x}})
\]
Fra i due o-piccolo, $\frac{1}{x^{\frac{1}{2}}}$ è più grande di $\frac{1}{x^{\frac{3}{2}}}$
\[
    -\frac{1}{\sqrt{x}} + o(\frac{1}{\sqrt{x}})
\]
\newpage