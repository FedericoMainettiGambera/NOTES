\section*{2-ESERCITAZIONE}
08/10/19
\newline
\textbf{es.} Studiare il seguente insieme:
\[
    B =\{X \in \mathbb{R} \;\;:\;\; -3 < x \leq 3\}
\]
L'insieme dei minoranti è $(-\infty, -3]$, inoltre $-3$ è il massimo dei minoranti, quindi $inf(B)= -3$, ma non è il $min(B)$ perchè non $\in B$.
\newline
Linsieme dei maggioranti è $[3, + \infty)$, inoltre $3$ è il $sub(B)$ e $max(B)$.
\newline
$-3$, $3$ sono punti di frontiera.
\newline
$[-3,3]$ punti di accumulazione.
\newline
Non esistono punti isolati.
\newline
$B$ non é nè aperto nè chiuso.
\newline
\newline
\textbf{es.} studiare i seguenti insiemi:
\[
    A =\{x \in \mathbb{R} \;\;:\;\; |x-1|\leq 2\}
\]
\[
    B =\{x \in \mathbb{R} \;\;:\;\; x^2 \in A\}
\]
\[
    A =\{-2 \leq x-1 \leq 2\} = \{-1\leq x \leq 3\}
\]
\[
    B=\{-1 \leq x^2 \leq 3\} = \{-\sqrt{3} \leq x \leq \sqrt{3}\} = [-\sqrt{3}, \sqrt{3}]
\]
$B$ è chiuso, $infB) = -\sqrt{3} = min(B)$, $\sqrt{3} = sub(B) = max(B)$
\newline
\newline
\textbf{es.} 
\[
    B=\{x \in \mathbb{R} \;\;:\;\; |x^3|-x > 0\}
\]
Riscrittura algebrica:
\[
    \begin{cases}
        per \;\; x \geq 0 \;\;\;\; & x^3-x^3 > 0 \;\;\; \nexists x \in \mathbb{R} \\
        per \;\; x < 0 \;\;\;\; & -2x^3 >0 \;\;\; \forall x < 0
    \end{cases}
\]
\[
    \Rightarrow B = \{x in \mathbb{R} \;\;:\;\; x<0\}
\]
\newline
\newline
\textbf{es.} 
\[
    f(x,y) = \frac{arccos(x)}{arcsin(y)}
\]
\[
    D=\{-1  leq x \leq 1, -1 \leq y \leq 1, y \neq 0\}
\]
L'insieme è limitato perchè
\[
    \exists k \;\;:\;\; \sqrt{x^2 + y^2} < K
\]
L'insieme non è aperto (il punto (1,1) non è interno).
\newline
L'insieme non è chiuso (il punto (1,0) è di accumulazione e $\notin A$)
\newline
\newline        
\textbf{es.} TDE
\[
    A = \{z \in \mathbb{C} \;:\; |z-9| > |z-9i|\}
\]
\[
    B=\{w \in \mathbb{C} \;:\; arg(w) = \frac{\pi}{4}\}
\]
 $A\cup B$ è:
\begin{itemize}
    \item chiuso
    \item aperto
    \item nè chiuso nè aperto
    \item numerabile (cioè se può essere messo in corrispondenza biunivoca con l'insieme dei numeri naturali)
\end{itemize}
img 22
\newline
I punti che appartengono ad $A$ sono quelli al di sopra della bisettrice, bisettrice non inclusa. L'insieme $B$ è invece la bisettrice che parte dall'origine e taglia il I quadrante.
\newline
I punti della semiretta bisettrice che parte dall'origine e taglia il III quadrante è costituita da punti di accumulazione.  
\newline
La risposta corretta è che l'insieme $A\cup B$ non è nè chiuso nè aperto.
\newline
\newline
\textbf{es.} TDE
\[
    A = \{z = x+iy \; \in \mathbb{C}\;:\; \frac{|z-2|}{|z|}<1 \}
\]
$A$ è:
\begin{itemize}
    \item senza punti di accumulazione
    \item aperto
    \item chiuso
    \item limitato
\end{itemize}
\[
    z-2 = x-2+iy
\]
\[
    \frac{|z-2|}{|z|} = \frac{\sqrt{(x-2)^2+y^2}}{\sqrt{x^2+y^2}}<1
\]
\[
    x^2 +4x +4 +y^2 < x^2+y^2
\]
\[
    x>1
\]
Ha punti di accumulazione, per esempio quelli sull'asse delle ascisse, non è limitato, non è chiuso perchè non contiene tutti i suoi punti di accumulazione, per esempio quelli sull'asse delle ordinate. La risposta giusta è che è aperto.
\newline
\newline
\textbf{es.} 
\[
    A = \{x \in \mathbb{R} \;:\; x = 5 -\frac{2}{n} \;,\; n \in \mathbb{N}-\{0\}\}
\]
per $n=1$ abbiamo $x_1 =3$, poi per $n=2$ abbiamo $x_2=4$, poi $x_3 = 5-\frac{2}{3}$ etc.
\newline
L'idea dell'insieme è che al crescere di $n$ ci avviciniamo sempre di più al valore $5$.
\newline
\[
    f(x) = 5-\frac{2}{x}
\] 
img23
\newline
Troviamo $inf$ e $sup$ di $A$:
\[
    inf(A)=3 = min(A)
\]
Per verificare che $3$ sia l'$inf$ devo verificare che sia un minorante e sia il massimo dei minoranti.
\newline
Verifichiamo che $3$ sia un minorante:
\[
    3 \leq x_n
\]
\[
    3 \leq 5 - \frac{2}{n}    
\]
\[
    \frac{2}{n} \leq 2
\]
\[
    \frac{1}{n} \leq 1
\]
Ora verifichiamo che sia il massimo dei minoranti, cioè che se salgo sopra il $3$, non trovo più un minorante:
\[
    \forall \epsilon > 0 \;\exists\; x_n \;:\; x_n < 3+\epsilon
\]
\[
    5-\frac{2}{n} < 3 +\epsilon    
\]
\[
    \frac{2}{n} > 2 -\epsilon
\]
\[
    n < \frac{2}{2-\epsilon}
\]
essendo $\frac{2}{2-\epsilon}$ maggiore di 1 mi basta prendere $n=1$.
\newline
Inoltre $3$ è anche minimo perchè appartiene all'insieme.
\newline
Verifichiamo ora che $5$ è il sup ma non il max di $A$.
\[
    sup(A)=5 \neq max(A)
\]
Per essere il sup deve essere maggiorante e il minimo dei maggioranti.
\[
    5 \geq X_n \;  \forall n \in \mathbb{N}-\{0\}
\]
\[
    5\geq 5 - \frac{2}{n} \;  \forall n \in \mathbb{N}-\{0\}
\]
\[
    0\geq -\frac{2}{n}
\]
Ora voglio verificare che sia il minimo dei maggioranti, cioè:
\[
        \forall \epsilon>0 \;\exists\; x_n \;:\; x_n > 5-\epsilon
\]
\[
    5-\frac{2}{n}> 5-\epsilon
\]
\[
    -\frac{2}{n}> -\epsilon
\]
\[
    \epsilon > \frac{2}{n}
\]
\[
    n > \frac{2}{\epsilon}
\]
\[
    n = \frac{2}{\epsilon} +3
\]
Ora dobbiamo dimostrare che non è il massimo, cioè che non appartiene all'insieme.
\newline
Tutti gli elementi dell'insieme hanno la forma 
\[
    5-\frac{2}{n}
\]
Ma siccome non esiste $n$ per cui $5-\frac{2}{n} = 5$, l'elemento non appartiene all'insieme.
\newline
L'insieme è limitato? sì.
\newline
Tutti i punti sono isolati? sì, dimostriamolo:
\[
    x_{\bar{n}} = 5 - \frac{2}{\bar{n}}
\]
\[
    d = dist(x_{\bar{n}}, x_{\bar{n}+1}) = \frac{1}{\bar{n}(\bar{n}+1)}
\]
se quindi prendiamo l'intorno
\[
    B(x_{\bar{n}},\frac{1}{\bar{n}(\bar{n}+1)}) \;\;\;(?)
\]
abbiamo dimostrato che tutti i punti sono isolati.
\newline
Troviamo ora il limite per $n \rightarrow \infty$ e dimostriamolo:
\[
    A = \{x \in \mathbb{R} \;:\; x_n = 5-\frac{2}{n} \;\;, n\in \mathbb{N} -{0}\}
\]
\[
    \lim_{n \rightarrow \infty} x_n = 5
\]
\[
    \forall \epsilon > 0 \;\exists\; \bar{n} \;:\; \forall n > \bar{n} \;\;vale\;\; |x_n-5| < \epsilon 
\]
\[
    |x_n-5| <\epsilon 
\]
\[
    |(3-\frac{2}{n}- 5)| < \epsilon
\]
\[
    \frac{2}{n}< \epsilon
\]
\[
    n> \frac{2}{\epsilon}
\]
\newline
\newline
\textbf{es.} dimsotrare
\[
    lim_{n\rightarrow+\infty} \frac{3n^2 +n}{n^2-1} = 3
\]
\[
    \forall \epsilon > 0 \;\exists\; \bar{n} \;:\; \forall n > \bar{n} \;\;vale\;\; |\frac{3n^2 +n}{n^2-1}| <\epsilon 
\]
\[
    |\frac{n+3}{n^2-1}|<\epsilon
\]
\[
    -\epsilon< \frac{n+3}{n^2-1}< \epsilon
\]
\[
    \begin{cases}
        \frac{n+3}{n^2-1}>-\epsilon \\
        \frac{n+3}{n^2-1}< \epsilon \\
    \end{cases}
\]
La prima disequazione è sempre verificata, per la seconda:
\[
    \epsilon n^2 - n -(\epsilon + 3) > 0
\]
\[
    n_{1,2} = \frac{1 \pm \sqrt{1+4\epsilon(\epsilon+3)}}{2\epsilon}
\]
\[
    n \leq \frac{\-\alpha ??}{2\epsilon} \lor \frac{1+\sqrt{1+4\epsilon(\epsilon+3)}}{2\epsilon}
\]
\newline
\newline
\textbf{es.} descrivere il carattere della seguente successione:
\[
    a_n = n-2 cos n \frac{\pi}{2} \;\rightarrow n\in \mathbb{N}- \{0\}
\]
notiamo un pattern:
\[
    cos(1+ \frac{\pi}{2}) = 0 \;\;,\;\;
    cos(2+ \frac{\pi}{2}) = -1 \;\;,\;\;
    cos(3+ \frac{\pi}{2}) = 0 \;\;,\;\;
    cos(4+ \frac{\pi}{2}) = 1
\]
al crescere di n il pattern si ripete. La nostra successione quindi si muove così:z
\[
    a_1 = 1 \;\;,\;\;
    a_2 = 4 \;\;,\;\;
    a_3 = 3 \;\;,\;\;
    a_4 = 2
\]
quindi $a_n$ è:
\[
    a_n = 
    \begin{cases}
        4k+1 \;\;\;&se \;n= 4k+1 \\
        4k+4 \;\;\;&se \;n = 4k+2\\
        4k+3 \;\;\;&se \;n= 4k+3\\
        4k+2 \;\;\;&se \;n= 4k+4\\
    \end{cases}
\]
con $k \in \mathbb{N}$
\newline
Il limite di $a_n$ per $n\rightarrow +\infty$ é $+\infty$.
\newline
dimostriamolo:
\[
    \forall k>0 \;\exists\; \bar{n} \;:\; \forall n \geq \bar{n} \;\;vale\;\; a_n> k
\]
qualunque $k$ fissato riesco a trovare un $\bar{n}$ per cui tutti i successivi elementi sono maggiori di $k$.
\newline
[manca dimostrazione] \dots
\newline
\newline
\newline
\textbf{es.} 
\[
    a_n = (1-cos \frac{9}{10}\pi )^n
\]
E' simile alla successione geometrica ($a_n = q^n$).
\newline
Analiziamo la sua ragione:
\[
    -1 < cos \frac{9}{10} < 0
\]
quindi
\[
    1<q<2  
\]
diverge a $+\infty$