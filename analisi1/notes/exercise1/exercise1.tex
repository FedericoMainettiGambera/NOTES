\section*{01/10/19 - ESERCITAZIONE}
\subsection*{Numeri complessi}
\textbf{es.} risolvere
\[
    |z|^2 -2z = 0
\] 
Due soluzioni possibili. La prima:
\[
    z=x+iy\\
\]
\[
    |z| =\sqrt{x^2+y^2}
\] 
\[
    x^2 +y^2-2x-2iy = 0
\] 
\[
    \begin{cases}
        x^2+y^2 -2x = 0 \\
        2y = 0 
    \end{cases} 
\]
\[
    \begin{cases}
        x = 0 \lor x=2 \\
        y=0
    \end{cases} 
\]
La seconda:
\[
    z = \rho e*{i\theta}
\] 
\[
    |z| = \rho^2
\] 
\[
    \rho^2 -2\rho e ^{i\theta} = 0
\] 
raccogliamo $\rho$
\[
    \rho = 0 \Rightarrow z =0
\] 
\[
    \rho -2 e^{i\theta} = 0  
\]
\[
    \rho = 2e^{i\theta}
\] 
$\rho$ è il modulo
\[
    \rho e^{i\theta} = 2 e^{i\theta}
\] 
\[
    \theta = 0 \land \rho = 2
\] 
\newline
\textbf{es.} determina $z_1$ e $z_2$ tali che $|z_1| = |z_2| = \sqrt{5} \; \land \; Im(z_1) = Im(z_2) = 2 $
\[
    z = x+iy = Re(z) + iIm(z)
\] 
\[
    \begin{cases}
        \sqrt{x^2 + y^2} = \sqrt{5}\\
        y=-2
    \end{cases}
\] 
\[
    \begin{cases}
        x^2 + y^2 = 5\\
        y=-2
    \end{cases}
\] 
\[
    \begin{cases}
        x = \pm 1
        y=-2
    \end{cases}
\] 
\[
    z_1 = -1-2i \;\;\; z_2 = +1-2i
\] 
ora verifica che $\bar{z_1} = z_2$
\[
    \bar{z_1}= -1+2i
\]  
\[
    -z_2 = -1+2i
\] 
\newline
\textbf{es.}
\[
    A =\{z \in \mathbb{C} \;:\; |z| < 3\}
\] 
\[
    B = \{w \in \mathbb{C} \;:\; Re(w) +Im(w) +1 = 0\}
\] 
(IMMAGINE di $A$ e di $B$)
\[
    C=\{v \in \mathbb{C} \;:\; v = z-3+i , z \in A \cap B\}
\] 
il termine $v = z-3+i$ rappresenta un tranlazione di $(-3,1)$ da applicare all'insieme che è l'intersezione di $A$ e $B$:
\[
    z = x+iy
\] 
\[
    z = -3+i = x +iy-3+i
\] 
\[
    v = (x-3) +i (1+y)
\] 
\newline
\textbf{es.} trovare le soluzioni di
\[
    z^2-z\bar{z}-\frac{2}{i}z = 0
\] 
Non si può applicare il teorema fondamnetale dell'algebra per via della presenza di $\bar{z}$.
\newline
Iniziamo togliendo $i$ dal denominatore
\[
    -\frac{2}{i}\frac{i}{i} = -\frac{2i}{1} = 2i
\] 
\[
    z^2 -z \bar{z} + 2iz = 0
\] 
\[
    z(z-\bar{z} + 2i) = 0
\] 
prima soluzione è $z=0$
\newline
Ora poniamo $z= x+iy$ e $\bar{z}=x-iy$
\[
    2iy +2i = 0
\] 
\[
    2i(y+1) = 0
\] 
Da cui ricaviamo $y = -1$
\newline
Definire l'insieme B (A rappresenta le soluzioni del punto precedente):
\[
    B = \{w \in \mathbb(C) \;:\; w = z+3i, z \in A \}
\] 
Notiamo che $w = z+3i$ rappresenta una traslazione verso l'alto. Quindi il punto $(0,0)$ diventa $(0,3i)$, invece la retta $y = -1$, cioè $Im(z) = -1$, diventa la retta $Im(w) = 2$.
\newline
\newline
\textbf{es.} 
\[
    i ^ {255} z^3 = \bar{z}
\] 
Per risolvere $i^{225}$ si può notare che gli esponenti di $i$ seguono un pattern: $i^0 = 1 \;\; i^1 = i \;\; i^2 = -1 \;\; i^3 = -i$. Inoltre da ricordarsi che $i$ ha modulo $1$ e argomento $\frac{\pi}{2}$
\newline
IMMAGINE
\newline
\[
    i^{225} = i^{224+1} = i^{224} i^{1} = 1i = i
\] 
\[
    iz^3 = \bar{z}    
\] 
Ora risolviamo questa equazione usando la forma esponenziale dei numeri complessi: $z=\rho e ^ {i\theta}$, $i=e^{i\frac{\pi}{2}}$, $\bar{z} = \rho e^{-i\theta}$
\[
    e ^{i\frac{\pi}{2}}\rho^3 e^{3i\theta} = \rho e^{-i\theta}
\] 
\[
    \rho (\rho ^2 e^{i(3\theta + \frac{\pi}{2})} - e ^{-i\theta})
\] 
Che da origine a due soluzioni. La prima:
\[
    \rho = 0
\] 
Accettata perchè $\rho$ è un numero reale positivo. La seconda:
\[
    \rho ^ 2 e ^{i(3\theta + \frac{\pi}{2})}= e^{-i\theta}
\] 
Modulo:
\[
    \rho ^ 2 = 1
\] 
\[
    \rho = \pm 1
\] 
Ma essendo $\rho$ un numero reale positivo rifiutiamo $-1$ come soluzione. Quindi $\rho = 1$.
\newline
Argomento:
\[
    3 \theta + \frac{\pi}{2} = -\theta + 2k\pi
\] 
\[
    4\theta = -\frac{\pi}{2} + 2k\pi
\] 
\[
    \theta = -\frac{\pi}{8} +k\frac{\pi}{2}    
\] 
per $k = 0,\dots \;,3$.
\newline
\textbf{es.} TDE
\[
    A =\{z \in \mathbb{C} \;:\; Re(z) = \sqrt{3|Im(z)|}\}
\] 
\[
    B=\{w\in \mathbb{C} \;:\; i w^2 \in A\}
\] 
partendo dal fatto che $z=x+iy$
\[
    A = \{z\in \mathbb{C} \;:\; x = \sqrt{3}|y|\}
\] 
\[
    \begin{cases}
        y = \frac{\sqrt{3}}{3} x & \;  \;\; y \geq 0 \\
        y = -\frac{\sqrt{3}}{3} x & \;  \;\; y < 0
    \end{cases} 
\] 
\[
    \begin{cases}
        \rho_{z} e ^{i\frac{\pi}{6}} & \;  \;\; y \geq 0 \\
        \rho_{z} e ^{-i\frac{\pi}{6}} & \;  \;\; y < 0
    \end{cases} 
\] 
\[
    w = \rho_{w} e ^{i\theta_{w}}
\] 
\[
    iw^2 = z
\] 
moltiplico per $-i$
\[
    w^2 = -iz
\] 
Per $A^+$:
\[
    \rho_{w}^{2} e^{2i\theta_{w}} = (-i)(\rho_{z}e^{i\frac{\pi}{6}})
\]
\[
    \rho_{w}^2e^{2i\theta_{w}}= \rho_{z}e^{-i\frac{\pi}{3}}
\]
\[
    \rho_{w} =\sqrt{\rho_{z}}
\] 
\[
    arg(w) = 2\theta_{w} = -\frac{\pi}{3}+2k\pi \theta_{w} = -\frac{\pi}{6}+2k\pi
\]
per $k=0,1$.
\newline
Per $A^-$:
\[
    \rho_{w}^2e^{2i\theta w} = (-i)(\rho_{z}e^{-i\frac{\pi}{6}}) = \rho_{z}e^{-i\frac{2}{3}\pi}
\] 
\[
    \rho_{w} = \sqrt{\rho_{z}}
\] 
\[
    \theta_{w} = -\frac{1}{3}\pi + k\pi
\] 

\subsection*{Permutazioni con e senza ripetizioni}
Definiamo il fattoriale di un numero $n\in \mathbb{N}$
\[
    n! = \begin{cases}
        1 & n=0 \\
        n(n-1)\dots2\;1 &n\geq1 \\
        n(n-1)! &n\geq1 \\
    \end{cases} 
\]
Il fattoriale è tipicamente usato per calcolare il numero di possibili permutazioni. Per esempio il numero di possibili permutazioni (anagrammi) di una parola si ottiene con il fattoriale del numero di lettere.
\newline
\textbf{es.} ROMA $\rightarrow 4!$
\newline
\textbf{es.} FARFALLA $\rightarrow 8!$, ma se per esempio volessimo eliminare la possibilità di permutare lettere identiche, dovremmo togliere a $8!$ le possibili permutazioni delle F ($2!$), delle A ($3!$), e delle L ($2!$) e quindi otteremmo: 
\[
    \frac{8!}{2!2!3!}
\]
\newline
\textbf{es.} In quante configurazioni diverse si possono porre 9 persone in fila indiana? $9!$
\newline
\textbf{es.} Se le 9 persone dell'esercizio fossero 5 maschi e 4 femmine e noi volessimo avere sempre per prima i tutti i maschi e poi tutte le femmine? $5!4!$

\subsection*{Esercizi sui Fattoriali}
\textbf{es.} TDE. 3 uomini e 3 donne devono sedersi alternati a un tavolo rotondo, quante sono le diverse possibili configurazioni?
\newline
Per risolvere questo esercizio ragioniamo a "coppie" di persone (uomo-donna)  che possiamo creare: $3!$.
\newline
Queste $3!$ coppie possibili possono essere disposte sul tavolo in $3!$ modi diversi.
\newline
Il numero fino ad ora ottenuto va moltiplicato per due perchè abbiamo solo lavorato con le coppie uomo-donna, ma possiamo rifare lo stesso ragionamento anche per le coppie donna-uomo.
\newline
Ultimo fattore da considerare è il fatto che il tavolo sia rotondo, infatti le permutazioni possibili di elementi su un tavolo rotondo non sono $n!$ ma $\frac{n!}{n} = (n-1)!$. Questo accade perchè se avessimo un fila indiana da riempire con gli elementi A,B e C otterremmo tre possibili configurazioni: ABC, CAB, BCA. Ma se disposte su un tavolo rotondo queste tre disposizioni sono esattamente la stessa disposizione.
\newline
Risposta:
\[
    \frac{3!2!2!}{6}
\] 
($6$ sono i posti a tavola)
\newline
\newline
\textbf{es.} TDE
\newline
Possibili anagrammi di ESAME?
\[
    \frac{5!}{2!} = 5\;4\;3 = 60
\] 
\newline
\newline
\textbf{es.} TDE
\newline
Con 14 partite di una schedina di calcio con 3 pareggi e 2 vittorie in casa, quante possibilità di compilare la schedina ci sono?
\newline
\[
    \frac{14!}{9!2!3!}
\] 
\newline
\textbf{es.} TDE
\newline
Quante password di $6$ cifre e composte solo dai caratteri "0" "1" "2" esistono?
\[
    3^6
\] 

\subsection*{Coefficienti binomiali}
\textbf{def.} Coefficiente binomiale con $n, k \in \mathbb{N}$ e $n \geq k$
\[
    \binom{n}{k} = \frac{n!}{k!(n-k)!}
\] 
Vediamo alcuni coefficienti binomiali notevoli:
\newline
caso $k = 0$ :
\[
    \binom{n}{0} = 1
\] 
\newline
\newline
casp $k=n$ :
\[
    \binom{n}{n} = 1
\]
\newline
\newline
caso k piccolo, comodo perchè risolve il binomiale trasformandolo in una frazione con k fattori sopra e sotto.
\[
    \binom{n}{k} = \frac{n(n-1)\dots(n-k+1)}{k!}  
\]
\newline
\textbf{dim.}
\[
    \binom{n}{k} = \frac{n!}{k!(n-k)!} = \frac{n(n-1) = (n-k+1) (n-k)!}{k!(n-k!)}
\]
\newline
\newline
caso n e k sono numeri molto simili, comodo perchè riconduce il binomiale alla formula precedente
\[
    \binom{n}{k} = \binom{n}{n-k}
\]
\textbf{dim.} 
\[
    \binom{n}{n-k} = \frac{n!}{(n-k)!k!} = \binom{n}{k}
\]
\newline
\textbf{def.} potenza del binomio di Newton
\[
    (a+b)^n = \sum_{k=0}^n \binom{n}{k} a^k b^{n-k}
\]
\newline
\textbf{es.} ceofficiente di $x^7y^3$ nello sviluppo di $(2x-y)^{10}$
\[
    \sum_{k=0}^{10} \binom{10}{k} (2x)^k (-y)^{10-k} = \sum_{k=0}^{10}2^k(-1)^{10-k}x^ky^{10-k}
\]
per $x^7y^3$ devo prendere $k=7$:
\[
    \binom{10}{7}2^7(-1)^3 = \dots = -30 \; 2^9
\]
\newline
\textbf{es.} TDE. Coefficiente di $a^5b^7$ nello sviluppo di $(2\sqrt{a}b+3ab)^7$
\newline
Si potrebbe applicare direttamente Newton, ma per semplificare i calcoli sarebbe meglio prima raccogliere $ab$
\[
    a^{\frac{7}{2}}b^7(2+3a^{\frac{1}{2}})^7 = a^{\frac{7}{2}}b^7 \left[ \sum_{k=0}^7\binom{7}{k}2^k(3a^{\frac{1}{2}})^{7-k} \right]
\]
l'intera sommatoria è moltiplicata per $b^7$ e $a^{\frac{7}{2}}a^{\frac{7}{2}-\frac{k}{2}}$, quindi per ottenere $b^7a^5$ devo prendere $k=4$.
\[
    k=4 \rightarrow \binom{7}{4}2^4(3)^3 = \dots
\]
\newline
\textbf{es.} risolvere la seguente equazione
\[
    2 \binom{x-1}{1}+3 \binom{x+1}{3} = \binom{x}{2}
\]
per $x-1 \geq 1$, per $x+1\geq3$ e $x\geq 2$, quindi solo $x\geq2$
\[
    2 \frac{(x-1)!}{(x-2)!} +3 \frac{(x+1)!}{3!(x-2)!} - \frac{x!}{2!(x-2)!} = 0
\]
\[
    2 (x-1) \frac{(x-2)!}{(x-2)!} + 3 \frac{x+1}{6}- \frac{x(x-1)}{2} = 0
\]
\[
    (x-1)(2+\frac{1}{2}x^2 + \frac{x}{2} -\frac{x}{2}) = 0
\]
$x-1 = 0$ non si accetta.
\newline
$\frac{1}{2}x^2 +2 = 0$ è impossibile.
\newline
non ci sono soluzioni.

\subsection*{Topologia in $\mathbb{R}$}
\textbf{def.}
\newline
Preso un insieme $A\in \mathbb{R}$, k é \textbf{maggiorante} di $A$ se $k\geq x, \forall x \in A$.
\newline
Preso un insieme $A\in \mathbb{R}$, k é \textbf{minorante} di $A$ se $k\leq x, \forall x \in A$.
\newline
Un insieme è \textbf{limitato superiormente} se ne esiste almeno un maggiorante.
\newline
Un insieme è \textbf{limitato inferioremente} se ne esiste almeno un minorante.
\newline
L'\textbf{estremo inferiore} $inf(A)$ è il massimo dei minoranti (non deve per forza appartenere ad $A$).
\newline
L'\textbf{estremo superiore} $sup(A)$ è il minimo dei maggioranti (non deve per forza appartenere ad $A$).
\newline
Il \textbf{minimo} $min(A)$ è uguale all'$inf(A)$ se esso appartiene ad $A$. Notare che se esiste il $min(A)$ esso è anche l'$inf(A)$, ma non vale il viceversa.
\newline
\newline
\textbf{es.} consideriamo l'insieme $A=(0,1]$.
\newline
$-1$ è un minorante, pure $-2$, etc. L'insieme dei minoranti di $A$ è: $(-\infty,0]$, il più grande è lo $0$, che quindi è l'$inf(A)$, ma non è il $min(A)$, perchè non appartiene ad $A$.
\newline
Se invece l'insieme fosse stato $A=[0,1]$, l'insime dei minoranti sarebbe ancora $(-\infty,0]$, l'$inf(A)$ sarebbe ancora $0$, ma in questo caso sarebbe anche il $min(A)$.
\newline
Linsieme dei maggioranti è invece $[1,+\infty]$, $sup(A)=1$, $max(A)=1$.
\newline
\newline
\textbf{def.} Un punto è detto di \textbf{accumulazione} se:
\begin{itemize}
    \item qualunque intorno di quel punto contiene almeno un punto di $A$
    \item ogni intorno di $x_0$ contiene un punto in A diverso da $x_0$
\end{itemize}
\textbf{def.} un punto è detto di \textbf{frontiera} se:
\begin{itemize}
    \item in ogni intorno cadono punti di $A$ e di $A^c$
\end{itemize}
\textbf{def.} un punto è detto \textbf{isolato} se:
\begin{itemize}
    \item per qualunque intorno non ci sono altri punti di $A$
\end{itemize}
\textbf{es.}
\[
    A = \{\{-2\}\cup(1,3]\}
\]
$-2$ è di frontiera e isolato, $1$ e $3$ sono di frontiera.
\newline
\newline
\textbf{def.} un insieme è detto \textbf{interno} se:
\begin{itemize}
    \item esiste almeno un intorno con solo punti di $A$
\end{itemize}
\textbf{def.} un insieme è detto \textbf{aperto} se:
\begin{itemize}
    \item tutti i punti di $A$ sono punti interni
\end{itemize}
\textbf{def.} un insieme è detto \textbf{chiuso} se:
\begin{itemize}
    \item tutti i punti di $A$ sono punti di accumulazione
\end{itemize}
\textbf{es.}
\[
    B = \{x \in \mathbb{Q} \;:\; x^2 \in [1,5]\}
\]
che equivale all'insime
\[
    \{- \sqrt{5} \leq x \leq -1 \;\; \lor \;\; 1\leq x\leq\sqrt{5} \} \cap \mathbb{Q}
\]
maggioranti: $(\sqrt{5}, +\infty)$
\newline
$sup(B) = \sqrt{5}$, ma non esiste perchè siamo in $\mathbb{q}$
\newline
$inf(B) = \dots$
\newline
$max(B) =$ non esiste
\newline
$min(B) = \dots$
\newpage