\section*{4-LEZIONE}
14/10/19
\subsection*{Gerarchia degli infiniti}
\textbf{def.} Si dice \textbf{infinito} una qualunque successione o funzione con limite infinito, cioè divergente.
\newline
\newline
Vediamo alcune successioni particolari:
\[
    a_n = ln(n)
\]
\[
    b_n = n
\]
\[
    c_n = e^n
\]
\[
    d_n = n!
\]
Tutte queste successioni divergono a $+ \infty$, ma non sono asintotiche fra loro, perchè non tendono a $+ \infty$ allo stesso modo.
\newline
\newline
Osserviamo (tramite grafici su Matlab) che $a_n = ln(n)$ e $b_n = n$ tendono a infinito in maniera diversa, infatti possiamo dire che $ln(n) = o(n)$.
\newline
Lo stesso vale per $n$, $e^n$ e $n!$: $n = o(e^n)$ e $e^n = o(n!)$
\[
    \lim_{n\rightarrow +\infty} \frac{ln(n)}{n} = 0
\] 
\[
    \lim_{n\rightarrow +\infty} \frac{n}{e^n} = 0
\]
\[
    \lim_{n\rightarrow +\infty} \frac{e^n}{n!} = 0
\]
\newline
Gerarchia degli infiniti:
\[
    ln(n) = o(n) \;\;\;\;\;\;\;\;\;\;\;\; n=o(e^n) \;\;\;\;\;\;\;\;\;\;\;\; e^n = o(n!)
\]
Questa gerarchia si mantiene anche con potenze positive diverse fra di loro.
\newline
\textbf{es.} $ln(n)^k = o(n^\frac{1}{k})$ con $k>0$.
\newline
\newline
\subsection*{Gerarchia degli infinitesimi}
\textbf{def.} Si dice \textbf{infinitesimo} una qualunque successione o funzione con limite zero, cioè convergente a $0$.
\newline
\newline
\[
    a_n = \frac{1}{ln(n)}
\]
\[
    b_n = \frac{1}{n}
\]
\[
    c_n = \frac{1}{e^n} = e ^{-n}
\]
\[
    d_n = \frac{1}{n!}
\]
Tutte queste successioni convergono a $0$, ma non sono asintotiche fra loro, perchè non tendono a $0$ allo stesso modo.
\newline
\newline
Osserviamo (tramite grafici su Matlab) che $a_n = \frac{1}{ln(n)}$ e $b_n = \frac{1}{n}$ tendono a $0$ in maniera diversa, infatti possiamo dire che $\frac{1}{n} = o(\frac{1}{ln(n)})$, ha un comportamento opposto agli infiniti.
\newline
Lo stesso vale per $\frac{1}{n}$, $e ^{-n}$ e $\frac{1}{n!}$: $e^{-n} = o(\frac{1}{n})$ e $\frac{1}{n!} = o(e^{-n})$.
\newline
\newline
Gerarchia degli infinitesimi:
\[
    \frac{1}{n!} = o(e^{-n}) \;\;\;\;\;\;\;\;\;\;\;\; e^{-n} = o(\frac{1}{n}) \;\;\;\;\;\;\;\; \frac{1}{n} = o(\frac{1}{ln(n)})
\]
Questa gerarchia si mantiene anche con potenze positive diverse fra loro.
\newline
\newline
\newline
\newline
\textbf{es.} Trova l'adamento asintotico del seguente limite:
\[
    \lim_{x\rightarrow -\infty} \frac{\sqrt{x^2+1}+x}{(x^{\frac{3}{5}}-x^{\frac{1}{7}})^{\frac{5}{7}}}
\]
\[
    f(x)= \frac{\sqrt{x^2+1}+x}{(x^{\frac{3}{5}}-x^{\frac{1}{7}})^{\frac{5}{7}}}
\]
Analiziamo il numeratore:
\[
    N(x) = \sqrt{x^2+1}+x
\]
\[
    N(x) = [\infty - \infty]
\]
è una forma di indeterminazione, quindi dobbiamo risolverla
\[
    x^2+1 = x^2+0(x^2)
\]
\[
    \sqrt{x^2+1} \sim \sqrt{x^2} = |x|\;\; (= -x) =
\]
$-x$ per valori negativi e dato che $x \rightarrow - \infty$ possiamo scriverlo.
\[
    = |x| + o(|x|) = -x + o(x)
\]
\[
    N(x) = \sqrt{x^2+1} +x = -x + o(x) + x = o(x)
\]
Questi passaggi non mi permettono di uscire dalla forma di indeterminazione.
\newline
Quindi potremmo provare a usare:
\[
    (1+t)^{\frac{1}{2}} = 1+ \frac{1}{2} t + o(t) \;\;\;\;per \;\; t \rightarrow 0
\]
\[
    \sqrt{x^2+1} = \sqrt{x^2(1+\frac{1}{x^2})} = \sqrt{x^2} (1 +\frac{1}{x^2})^{\frac{1}{2}} = |x| \cdot [1+\frac{1}{2}\cdot \frac{1}{x^2}+o(\frac{1}{x^2})] =
\]
\[
    = -x \cdot [1+\frac{1}{2}\cdot \frac{1}{x^2}+o(\frac{1}{x^2})] =
\]
\[
    -x - \frac{1}{2x} + o(\frac{1}{x})
\]
\[
    N(x) = \sqrt{x^2 +1} +x = -x -\frac{1}{2x} + o(\frac{1}{x}) +x = -\frac{1}{2x} + o(\frac{1}{x})
\]
Analiziamo il denominatore:
\[
    D(x) = (x^{\frac{3}{5}}-x^{\frac{1}{7}})^{\frac{5}{7}} 
\]
\[
    x^{\frac{3}{5}} - x^{\frac{1}{7}} = 
\]
Essendo $\frac{3}{5} > \frac{1}{7}$ :
\[
    = x^{\frac{3}{5}} + o(x^{\frac{3}{5}}) \sim  x^{\frac{3}{5}}
\]
\[
    (x^{\frac{3}{5}} - x^{\frac{1}{7}})^{\frac{5}{3}} \sim  (x^{\frac{3}{5}})^{\frac{5}{3}} = x^1
\]
\[
    D(x) = x + o(x)
\]
Vediamo ora numeratore/denominatore:
\[
    \frac{N(x)}{D(x)} = \frac{-\frac{1}{2x} + o(\frac{1}{x})}{x-o(x)} \sim  - \frac{1}{2x^2} \xrightarrow[x \rightarrow - \infty] \; 0^{-} 
\]
Ricorda di fare sempre i conti con o-piccolo, e di toglierlo solo all'ultimo passaggio, quando hai fatto già tutti i conti necessari.
\newline
\newline
\newline
\newline
\textbf{es.} TDE(parziale) determinare il carattere della seguente successione:
\[
    a_n = \frac{ln(|(e^{-\frac{1}{ln(n)}}-1)^2 - 1|)}{sin (\frac{1}{n} - \frac{\pi}{2})}
\]
Analiziamo il numeratore:
\[
    N(x) = ln(|(e^{-\frac{1}{ln(n)}}-1)^2 - 1|)
\]
\[
    t = -\frac{1}{ln(n)} \xrightarrow[n \rightarrow  + \infty] \; 0^{-} 
\]
Ora usiamo questa formula:
\[
    e^{t} = 1+t+o(t)
\]
\[
    e^{t} -1 = t+o(t) \;\;\;\;\; per \;\;\; \rightarrow  0
\]
\[
    e ^{-\frac{1}{ln(n)}} -1 = -\frac{1}{ln(n)} + o(\frac{1}{ln(n)}) \sim -\frac{1}{ln(n)}
\]
potenze di  funzioni asintotiche sono ancora asintotiche fra loro
\[
    (e ^{-\frac{1}{ln(n)}} -1) ^2 \sim (-\frac{1}{ln(n)})^2 = \frac{1}{ln^2(n)}
\]
\[
    (e ^{-\frac{1}{ln(n)}} -1) ^2 = \frac{ 1}{ln^2(n)} + o(\frac{ 1}{ln^2(n)} )
\]
\[
    |(e ^{-\frac{1}{ln(n)}} -1) ^2 -1| = |-1 + \frac{1}{ln^2(n)} + o(\frac{1}{ln^2(n)})| =
\]
Il segno del valore assoluto è negativo ($|-1 + \frac{1}{ln^2(n)} + o(\frac{1}{ln^2(n)})|$), quindi posso toglierlo e cambiare il segno al suo intenro:
\[
    = 1 - \frac{1}{ln^2(n)} + o(\frac{1}{ln^2(n)})
\]
\[
    ln(|(e^{-\frac{1}{ln(n)}}-1|) = ln(1- \frac{1}{ln^2(n)} + o(\frac{1}{ln^2(n)})
\]
\[
    ln(1+t) = t + o(t)
\]
con $t = -\frac{1}{ln^2(n)}$
\[
    -\frac{1}{ln^2(n)} + o (\frac{1}{ln^2(n)}) + o(-\frac{1}{ln^2(n)} + o(\frac{1}{ln^2(n)}))=
\]
Questo $o(-\frac{1}{ln^2(n)} + o(\frac{1}{ln^2(n)}))$ si può eliminare:
\[
    = N(x) = -\frac{1}{ln^2(n)} + o (\frac{1}{ln^2(n)})
\]
Analiziamo il denominatore:
\[
    D(x) = sin (\frac{1}{n} - \frac{\pi}{2}) = cos(\frac{1}{n})
\]
Imponiamo $\frac{1}{n} = t$
\[
    -cos(t) = 1-\frac{t^2}{2} + o(t^2)
\]
\[
    -cos(\frac{1}{n}) = 1- \frac{1}{2n^2} + o(\frac{1}{n^2})
\]
Vediamo numeratore/denominatore:
\[
    \frac{N(x)}{D(x)} = a_n = \frac{-\frac{1}{ln^2(n)} + o (\frac{1}{ln^2(n)})}{-(1- \frac{1}{2n^2} + o(\frac{1}{n^2}))} \sim -\frac{1}{ln^2(n)} \xrightarrow \; 0^+
\]
\newline
\newline
\newline
\newline
\textbf{Continuità (puntuale) di una funzione}
\textbf{def.} Si dice che
\[
    f: A \subseteq \mathbb{R} \rightarrow \mathbb{R} \;\;\; x \rightarrow y=f(x)
\] 
è \textbf{continua} in $x_0 \in A$, se 
\begin{itemize}
    \item $x_0$ è un punto isolato di $A$
    \item $x_0$ è un punto di accumulazione di $A$ e $\lim_{x\rightarrow x_0} f(x) = f(x_0)$
\end{itemize}
\textbf{oss.} le successioni sono continue in tutto il loro dominio (son ocostituite da soli punti isolati)
\newline
\newline
\newline
\textbf{Continuità di una funzione in un insieme}
\textbf{def.} Si parla di \textbf{continuità in tutto l'insieme} $A$ se è continua in tutti i punti di $A$.
\newline
\newline
\newline
\textbf{teor.} le funzioni elementari sono funzioni continue in tutto il loro dominio naturale.
\newline
Vediamo alcuni esempi di funzioni elementari:
\begin{itemize}
    \item $y = sin(x)$
    \item $y= x^2 + x$
    \item $y = arctg(x)$
    \item $y= e^x$
    \item $y = a^x$
    \item $y = ln(x)$
    \item $y=\frac{1}{x}$
    \item $y=tg(x)$
\end{itemize}
Se prendiamo la funzione $y=tg(x)$ notiamo che c'è un salto, ma essa non è definita in quel punto, quindi possiamo dire che sono continue nel loro dominio naturale.
\newline
\newline
\newline
\textbf{classificazione delle discontinuità}
\textbf{es.} 
\[
    f(x) =
    \begin{cases}
        \frac{sin(x)}{x} & se \;\;\; x \neq 0 \\
        0 & se x = 0
    \end{cases}
\]
Analiziamola in $x_0 = 0$:
\[
    f(0) = 0 
\]
\[
    \lim_{x\rightarrow 0} \frac{sin(x)}{x} = 1
\]
C'è un salto, questa discontinuità si dice del III tipo o eliminabile.
\newline
\newline
\textbf{def.} \textbf{discontinuità del III tipo o eliminabile}
\[
    \lim_{x\rightarrow x_0^{-}} f(x) = \lim_{x\rightarrow x_0^{+}} f(x) \neq f(x_0) 
\]
\textbf{oss.} la funzione $y = \frac{sin(x)}{x}$ di dominio $\mathbb{R} - \{0\} = A$ è una funzione continua su $A$ può anche essere \textbf{estesa o prolungata per continuità} si tutto $\mathbb{R}$.
\newline
\newline
\textbf{def.} \textbf{discontinuità del II tipo}
Questa discontinuità non è eliminabile e appare se anche solo uno dei due limiti (o da destra o da sinistra o entrambi) sia infinito o non esista.
\newline
\textbf{es.} 
\[
    f(x) = \begin{cases}
        e^{-\frac{1}{x}} & x \neq 0 \\
        0 & x = 0
    \end{cases}
\]
analiziamo la continuità in $x_0 = 0$:
\[
    f(0) = 0
\]
\[
    \lim_{x\rightarrow 0^+} e^{-\frac{1}{x}} = \lim_{t=\frac{1}{x}\rightarrow + \infty} e^{-t} = 0^{+}
\]
$f$ risulta continua da destra nell'origine
\[
    \lim_{x\rightarrow 0^-} e^{-\frac{1}{x}} =\lim_{t=\frac{1}{x}\rightarrow -\infty} e^{-t} = + \infty
\]
$f$ tende a $+ \infty$ da sinistra nell'origine.
\newline
Siamo in presenza di una discontinuità di II specie.
\newline
\newline
\textbf{def.} \textbf{discontinuità di I specie o a salto}
\[
    x_0 \in A \;\;\;\;\;\; \exists f(x_0)), \;\;\;\;\;\; \exists \lim_{x\rightarrow x_0^-} f(x), \;\;\;\;\;\; \exists \lim_{x\rightarrow x_0^+} f(x)
\]
ma i limiti destri e sinistri sono diversi.
\newline
Anche questa discontinuità non è eliminabile.
\newline
\newline
\newline
\textbf{es.} trovare i valori di a e b per avere continuità in questa funzione:
\[
    f(x) = \begin{cases}
        -2 sin(x) & x\leq - \frac{\pi}{2} \\
        a \cdot  sin (x) + b & -\frac{\pi}{2}<x< \frac{\pi}{2}\\
        cos(x) & x \geq \frac{\pi}{2}
    \end{cases}
\]
in $\mathbb{R}-\{-\frac{\pi}{2}, \frac{\pi}{2}\}$ la funzione $f$ è continua.
Analiziamo ora i due punti $\{-\frac{\pi}{2}, \frac{\pi}{2}\}$, devo verificare queste due cose:
\[
    \lim_{x\rightarrow -(\frac{\pi}{2})^+} a \cdot sin(x) + b = -2 sin(-\frac{\pi}{2})
\]
\[
    \lim_{x\rightarrow \frac{\pi}{2}} = ( a \cdot  sin (x) + b ) = cos(\frac{\pi}{2})
\]
Svolgiamo i limiti:
\[
    \lim_{x\rightarrow -\frac{\pi}{2}} a \cdot sin(x) + b = +2 = -a +b
\]
\[
    \lim_{x\rightarrow -(\frac{\pi}{2})^-} a \cdot sin(x) + b = 0 = a+b
\]
Da cui 
\[
    b = 1
\]
\[
    a = -1
\]
\newline
\newline
\textbf{es.} Vediamo un esempio di funzione discontinua in tutti i suoi punti:
\[
    f(x) 0 \begin{cases}
        1 & se \; x \in \mathbb{Q} \\
        -1 & se \; x \in \mathbb{R}-\{\mathbb{Q}\}
    \end{cases}
\]
è costituito da solo discontinuità di seconda specie.
\newpage 