\section*{2-LEZIONE}
07/10/19
\subsection*{Topologia in $\mathbb{R}$}
\subsubsection*{Punti isolati e di accumulazione}
\textbf{def.} Un punto $x_0 \in A$ si dice \textbf{isolato} per $A$ se 
\[
    \exists B_r(x_0) \;\;:\;\; B_r(x_0) \cap A = \{x_0\}
\]
img1
\newline
\newline
\textbf{es.} Prendiamo l'insieme
\[
    A = {x \in \mathbb{R} \;:\; x = n \in \mathbb{N}}
\]
img2
\newline
Notiamo che $A$ è fatto di soli punti isolati.
\newline
\newline
\textbf{def.} Un punto $x_0$ è un punto di \textbf{accumulazione} per $A$ se
\[
    \forall B_r(x_0) \exists x \in A \; e \; x \neq x_0
\]
cioè se esiste una successione di punti di $A$ che raggiunge $x_0$.
\newline
img3
\newline
il disegno mostra come ci sia un percorso che raggiunge $x_0$.
\newline
\textbf{oss.} Da notare è il fatto che il punto può $\in A$ come può $\in A^c$.
\newline
\textbf{oss.} tutti i punti interni ad un insieme sono di accumulazione.
\newline
\textbf{oss.} i punti di frontiera sono di accumulazione purchè non siano isolati
\newline
\newline
\textbf{es.} 
\[
    A = \{x \in \mathbb{R}, x \;:\; -4 < x \leq e \;\lor\; x = \pi \;\lor\; x > 10\}
\]
\[
    (-4,e] \cup {\pi} \cup (10, +\infty)
\]
I punti di frontiera sono $-4, e, \pi, 10$. Ma $\pi$ non è un punto di accumulazione.
\subsubsection*{Insieme limitato, convesso, non convesso e compatto}
\textbf{def.} Un insieme $A$ è \textbf{limitato} se occupa una porzione con area finita dell'ambiente. Formalmente si dice che è limitato se
\[
    \exists y \in \mathbb{R}^2 \;e\; B_r(y) \subset A.
\]
Spesso per $y$ si prende l'origine degli assi.
\newline
\newline
\textbf{es.} Analizza il seguente insieme
\[
    A = \{x \in \mathbb{R}, x \;:\; -4 < x \leq e \;\lor\; x = \pi \;\lor\; x > 10\}
\]
L'insieme corrispone a questo:
\[
    (-4,e] \cup {\pi} \cup (10, +\infty)
\]
che non è limitato per via di $(10, +\infty)$.
\newline
\newline
\newline
\textbf{def.} un insieme $A$ è \textbf{convesso}  se $\forall x,\; y \in A$ il segmento di estremi $x$ e $y$ appartiene ad $A$.
\newline
Esempi di insiemi convessi:
\newline
img4
\newline
\newline
\newline
\textbf{def.} un insieme $A$ è \textbf{non convesso} se esiste un segmento con estremi $x,\; y \;\;\in \; A$ tale per cui parte di esso non sia contenuto nell'insieme $A$. 
\newline
Esempi di insiemi non convessi:
\newline
img5
\newline
\newline
\textbf{es.} In $\mathbb{R}$ gli insiemi "convessi" che insiemi sono?
\newline $\mathbb{N}$ è convesso in $\mathbb{R}$? no.
\newline $\mathbb{Q}$ è convesso in $\mathbb{R}$? no.
\newline
Gli unici insimei "convessi" in $\mathbb{R}$ sono gli insiemi:
\begin{itemize}
    \item singoletti;
    \item intervalli;
    \item semirette.
\end{itemize}
\textbf{def.} un insieme $A$ è \textbf{compatto} se è chiuso e limitato.
\newline
\newline
\textbf{es.} esempio di insieme compatto:
\[
    A =\{ (x,y) \in \mathbb{R}^2 \;:\; xy \leq1 \;e\; x^2+y^2 \leq 1 \}
\]
img6
\newline
l'insieme è chiuso e limitato, quindi compatto.
\newline
\newline
\newline
\subsection*{Funzioni}
\subsubsection*{Definizioni principali}
\textbf{def.} cos'è una \textbf{funzione}?
\newline
\newline
Una funzione ha tre "ingredienti" principali che la compondono:
\begin{itemize}
    \item un insieme di partenza detto \textbf{dominio} $V$, un elemento di questo insieme lo simboleggiamo con la lettera $v$.
    \item un insieme di arrivo detto \textbf{codominio} $W$, un elemento del secondo insieme lo simboleggiamo con la lettera $w$.
    \item una legge che definisce la funzione, simboleggiata dal simbolo $f()$.
\end{itemize}
\textbf{def.} Si dice \textbf{dominio naturale} l'insieme $V'$, contenuto o uguale $V$, il più grande sottoinsieme del dominio dove la legge è completamente definita.
\newline
\newline
\textbf{def.}  Si dice che $w$ è l' \textbf{immagine} di $v$ attraverso $f$.
\[
    w = f(v)
\]
\newline
\textbf{def.}  Si dice che $v$ è la \textbf{controimmagine} di $w$ attraverso $f$.
\newline
\newline
\textbf{def.} L'\textbf{insieme immagine} è la totalità delle immagini e si indica come $im(f)$, spesso si dice anche "immagine della funzione".
\[
    im(f) = \{w \in W \;:\; \exists \; v \rightarrow  f(v) = w\}
\]
\subsubsection*{Funzioni iniettive, suriettive e biettive}
\textbf{def.} una funzione $f$ si dice \textbf{iniettiva} se preserva elementi distinti.
\[
    se \;\; \forall v_1 \neq v_2 \;\rightarrow\;\; f(v_1) \neq f(v_2)
\]
\textbf{def.} una funzione è \textbf{non iniettiva} se ci sono elementi diversi con la stessa immagine.
\newline
\newline
\textbf{def.} una funzione è \textbf{suriettiva} se "invade" tutto il codominio.
\[
    im(f) \;\equiv\; W
\] 
\newline
\textbf{es.} Definire se la seguente funzione è iniettiva o suriettiva:
\[
    \mathbb{R} \rightarrow \mathbb{R} \;\;\;\;\;\;\;\;\; x \rightarrow y=sin(x)
\]
non è iniettiva e non è suriettiva.
\newline
\newline
\textbf{es.} Definire se la seguente funzione è iniettiva o suriettiva:
\[
    [-\frac{\pi}{2}, \frac{\pi}{2} ] \rightarrow [-1,1]
\]
\[
    x\rightarrow y=sin(x)
\]
img8
\newline
è iniettiva e suriettiva.
\newline
\newline
\newline
\textbf{def.} una funziona che è sia iniettiva sia suriettiva si dice \textbf{biiettia}.
\subsection*{Successioni}
\subsubsection*{definizione}
\textbf{def.} Le successioni sono funzioni particolari il cui dominio è $\mathbb{N}$ e il codominio $\mathbb{R}$
\[
    f:\;\; V = \mathbb{N} \rightarrow \mathbb{R}
\]
\[
    n \rightarrow y= f(n)
\]
Spesso si scrive: $n \rightarrow y= f_n$, oppure $n \rightarrow y= a_n$, dove $a_n$ è l'immagine dell'elemento n-esimo.
\newline
img9
\subsubsection*{Successioni monotone e limitate}
\textbf{def.} Una successione si dice \textbf{monotona} se ha un andamento con un trend costante.
\newline
\newline
\textbf{def.} Una successione di dice monotona \textbf{crescente} se
\[
    \forall n_1 < n_2 \;\; \Rightarrow \;\; a_{n_{1}} \leq a_{n_{2}}
\]
\newline
\newline
\textbf{def.} Una successione di dice monotona \textbf{strettamente crescente} se
\[
    \forall n_1 < n_2 \;\; \Rightarrow \;\; a_{n_{1}} < a_{n_{2}}
\]
\newline
\newline
\textbf{def.} Una successione di dice monotona \textbf{decrescente} se
\[
    \forall n_1 < n_2 \;\; \Rightarrow \;\; a_{n_{1}} \geq a_{n_{2}}
\]
\newline
\newline
\textbf{def.} Una successione di dice monotona \textbf{strettamente decrescente} se
\[
    \forall n_1 < n_2 \;\; \Rightarrow \;\; a_{n_{1}} > a_{n_{2}}
\]
\newline
\textbf{oss.} una successione \textbf{costante} è una successione monotona crescente e decrescente.
\newline
\newline
\textbf{def.} una successione è \textbf{limitata} se il suo insieme immagine $im(f)$ è un insieme limitato di $\mathbb{R}$
\[
    im(f) \;\;\subseteq\;\; B_r(x)
\]
\newline
\newline
Vediamo alcuni esempi di successioni che contengono alcuni casi notevoli.
\newline
\newline
\textbf{es.} Analizza la seguente successione:
\[
    a_n = (-1)^n \frac{1}{n}
\]
img10
\newline
Notiamo che $n$ non può assumere il valore $0$, che quindi è escluso dal suo dominio. La successione è limitata, ma oscilla, quindi non è monotona.
\newline
\newline
\textbf{es.} Analizza la seguente successione:
\[
    a_n= \left(1+\frac{1}{n}\right)^n
\]
img11
\newline
\[
    a_1 = 2
\]
\[
    a_2 = (1 + \frac{1}{2})^2 = \frac{9}{4} > a_1
\]
\[
    a_3 = (1 + \frac{1}{3})^3 = \frac{64}{27} > a_2
\]
Questa successione è famosa perchè converge al valore $e$. E' limitata e monotona strettamente crescente.
\newline
\newline
\textbf{es.} Analizza la seguente successione:
\[
    a = ln(n)
\]
Non è limitata, ma è limitata solamente inferiormente, è monotona strettametne crescente.
\newline
\newline
\textbf{es.} Analizza la seguente successione:
\[
    a_n = (-1)^n
\]
img13
\newline
è limitata e periodica con periodo $T=2$.
\newline
\newline
\textbf{es.} Analizza la seguente successione:
\[
    a_n = sin(n)
\]
img14
\newline
è limitata, non è monotona, può sembrare periodica, ma non lo è perchè il periodo sarebbe $2\pi \notin \mathbb{N}$.
\newline
\newline
\textbf{es.} successione potenza
\[
    a_n = n ^{\frac{1}{2}}
\]
img15
\newline
monotona strettamente crescente, limitata inferiormente.
\newline
\newline
\textbf{es.} Analizza la seguente successione:
\[
    a_n=(-1)^nn^{(\frac{1}{2})}
\]
img16
\newline
\newline
\textbf{es.} Analizza la seguente successione:
\[
    a_n=(-1)^nn^{2}
\]
img17
\subsubsection*{Natura delle successioni}
\textbf{def.} La \textbf{natura} (o \textbf{comportamento}) di una successione è l'andamento osservato per grandi valori del dominio.
\newline
La natura di una successione è di tre tipi:
\begin{itemize}
    \item convergente
    \item divergente (positivamente o negativamente)
    \item irregolare
\end{itemize}
\textbf{def.} Si dice che $\{a_n\}$ \textbf{converge} a $L$ e lo scrivo
\[
    a_n \; \xrightarrow[+ \infty] \; L
\]
\[
    \lim_{n\rightarrow + \infty} a_n = L
\]
img18
\newline
definizione formale:
\[
    \lim_{n\rightarrow + \infty} a_n = L \;\;se\;\; \forall B_r(L) \;\exists \; M \;\;:\;\; \forall n>M \;\;\; a_n \in B_r(L)
\]
definizione a parole:
\newline
per ogni intorno $B_r$ del limite $L$ esiste una posizione $M$ al di là della quale la successione sta tutta nell'intorno del valore L.
\newline
img19
\newline
\newline
\newline
\textbf{def.} $a_n$ è \textbf{positivamente divergente} o \textbf{divergente a $+\infty$} e lo scrivo
\[
    a_n \; \xrightarrow[+ \infty] \; + \infty
\]
\[
    \lim_{n\rightarrow +\infty} a_n = + \infty
\]
\newline
definizione formale:
\[
    \lim_{n\rightarrow + \infty} a_n = +\infty \;\;se\;\; \forall B_r(+\infty) \;\exists \; M \;\;:\;\; \forall n>M \;\;\; a_n \in B_r(+\infty)
\]
definizione a parole:
\newline
per ogni intorno $B_r$ di $+\infty$ esiste una posizione $M$ al di là della quale la successione sta tutta nell'intorno di $+\infty$.
\newline
img20
\newline
\newline
\newline
\textbf{def.} una successione è \textbf{negativamente divergente} o \textbf{divergente a $-\infty$} e lo scrivo
\[
    a_n \; \xrightarrow[+ \infty] \; - \infty
\]
\[
    \lim_{n\rightarrow +\infty} a_n = - \infty
\]
\newline
definizione formale:
\[
    \lim_{n\rightarrow - \infty} a_n = -\infty \;\;se\;\; \forall B_r(-\infty) \;\exists \; M \;\;:\;\; \forall n>M \;\;\; a_n \in B_r(-\infty)
\]
definizione a parole:
\newline
per ogni intorno $B_r$ di $+\infty$ esiste una posizione $M$ al di là della quale la successione sta tutta nell'intorno di $+\infty$.
\newline
img21
\newline
\newline
\newline
\textbf{def.} Una successione ha una \textbf{proprietà definitiva} se la proprietà è valida per la successione da un certo valore in poi.
\newline
\newline
Le definizioni di convergenza e divergenza possono essere riscritte usando la definizione della proprietà definitiva.
\newline
\newline
\newline
\textbf{def.} Una successione che non è né divergente né convergente è \textbf{irregolare}.
\newline
\newline
\newline
\textbf{es.} Analizza la seguente successione:
\[
    a_n = (-1)^n
\]
questa successione è irregolare e limitata.
\newline
\newline
\textbf{es.} Analizza la seguente successione:
\[
    a_n = (-1)^nn^2
\]
questa successione è irregolare e illimitata.
\newpage


