\documentclass[a4paper, 9pt]{report}
  
\author{Federico Mainetti Gambera}
\usepackage{amsmath}
\usepackage{amssymb}
\usepackage{graphicx}
\usepackage[italian]{babel}
\usepackage{import}
\usepackage{xifthen}
\usepackage{pdfpages}
\usepackage{transparent}
\usepackage{xcolor}
\usepackage{cancel}
\usepackage[a4paper,left=35mm,top=26mm,right=26mm,bottom=15mm]{geometry}
\usepackage{color}
\usepackage{tcolorbox}
\usepackage{hyperref}
\usepackage{makeidx}
\usepackage{soul}
\usepackage{multicol}
\usepackage{pdflscape}
\makeindex
\definecolor{lightgray}{gray}{0.75}
\renewcommand{\familydefault}{\sfdefault}
\newenvironment{rcases}
  {\left.\begin{aligned}}
  {\end{aligned}\right\rbrace}
\newcommand{\incfig}[1]{%
    \def\svgwidth{\columnwidth}
    \import{../images/}{#1.pdf_tex}
}
\begin{document}
\subsection*{TRIGONOMETRIA}
\[
    sin^2(x) + cos^2(x) = 1
\]
\[
    sin(2x) = 2sin (x)cos(x)
\]
\[
    sin(x) cos(x) = \frac{1}{2}sin(2x)
\]
\[
    cos(2x) = \begin{cases}
        cos^2(x) -sin^2(x)\\
        1-2sin^2(x)\\
        2cos^2(x)-1
    \end{cases}
\]
\[
    sin^2(x) = \frac{1}{2} (1-cos(2x)) \;\;\;\; ottenuta \; da \;[cos(2x) = cos^2(x) - sin^2(x) = 1 - 2 sin^2(x)]
\]
\[
    cos^2(x) = \frac{1}{2}(1+cos(2x)) \;\;\;\; ottenuta \; da \; [cos(2x) = cos^2(x) - sin^2(x) = 2cos^2(x) - 1]
\]
\[
    Ch^(x) = \frac{e^x + e^{-x}}{2}
\]
\[
    Sh^(x) = \frac{e^x - e^{-x}}{2}
\]
\[
    Ch^2(x) - Sh^2(x) = 1
\]
\[
    Sh(2x) = 2Sh(x)Ch(x)
\]
\[
    Ch(2x) = Sh^2(x) + Ch^2(x)
\]
\[
    SettSh(x) = log(+ + \sqrt{x^2+1})
\]
\[
    SettCh(x) = log(x + \sqrt{x-1} \cdot \sqrt{x+1})
\]
\[
    Sh(SettCh(a))= \sqrt{a^2-1} \;\;\;\; ottenuta \; da \; [Ch^2(x) -Sh^2(x) = 1] \rightarrow [Sh(x) = \sqrt{Ch^2(x) -1}] \rightarrow [x = SettCh(a)]
\]
\[
    Ch(SettSh(a))=\sqrt{a^2+1} \;\;\;\; ottenuta \; da \; [Ch^2(x) -Sh^2(x) = 1] \rightarrow [Ch(x) = \sqrt{1 + Sh^2(x)}] \rightarrow [x = SettSh(a)]
\]
\[
    sin(a)cos(b)=\frac{1}{2}sin(a+b)+sin(a-b)
\]
\[
    cos(a)sin(b)=\frac{1}{2}sin(a+b)-sin(a-b)
\]
\[
    cos(a)cos(b)=\frac{1}{2}cos(a+b)+cos(a-b)
\]
\[
    sin(a)sin(b)=-\frac{1}{2}cos(a+b)- cos(a-b)
\]
\[
    sin(a+b) =sin(a)cos(b) + sin(b) cos(a)
\]
\[
    sin(a-b) = sin(a)cos(b) - sin(b)cos(a)
\]
\[
    cos(a+b) = cos(a)cos(b) - sin(a)sin(b)
\]
\[
    cos(a-b)=cos(a)cos(b) + sin(a)sin(b)
\]
\[
    sin(\alpha) + sin(\beta) = 2 sin\left(\frac{\alpha + \beta}{2}\right) cos \left(\frac{\alpha - \beta}{2}\right)
\]
\[
    sin(\alpha) - sin(\beta) = 2 cos\left(\frac{\alpha + \beta}{2}\right) sin\left(\frac{\alpha - \beta}{2}\right)
\]
\[
    cos(\alpha) + cos(\beta) = 2 cos\left(\frac{\alpha + \beta}{2}\right) cos\left(\frac{\alpha - \beta}{2}\right)
\]
\[
    cos(\alpha) - cos(\beta) = -2sin\left(\frac{\alpha + \beta}{2}\right) sin\left(\frac{\alpha - \beta}{2}\right)
\]
\begin{figure}[h!]
    \includegraphics[width=300px]{../dim/trigonometria.PNG}
\end{figure}
\newpage
\subsection*{ASINTOTICI}
\begin{figure}[h!]
    \includegraphics[width=\linewidth]{../dim/asintotici.PNG}
\end{figure}
\newpage
\subsection*{DERIVATE}
\begin{figure}[h!]
    \includegraphics[width=\linewidth]{../dim/derivate1.PNG}
\end{figure}
\newpage
\begin{figure}[h!]
    \includegraphics[width=\linewidth]{../dim/derivate2.PNG}
\end{figure}
\newpage
\subsection*{SVILUPPI}
\begin{figure}[h!]
    \includegraphics[width=\linewidth]{../dim/mclaurin.PNG}
\end{figure}
\subsection*{SERIE}
\subsection*{serie geometrica}
per $q \neq 1$
\[
    \sum_{n=0}^{\infty}q^n = \frac{1-q^{n+1}}{1-q}
\]
per $q = 1$
\[
    \sum_{n=0}^{\infty}q^n = \begin{cases}
        \frac{1}{1-q} & se \;\; -1<q<1\\
        +\infty &se \;\; q \geq 1\\
        irregolare \;\;& se \;\; q\leq-1 
    \end{cases}
\]
\subsection*{serie armonica}
\[
    \sum_{n=1}^{\infty}\frac{1}{n} \geq log(n+1) \rightarrow +\infty
\]
per $\alpha \leq 1$ 
\[
    \sum_{n=1}^{\infty} \frac{1}{n^\alpha} \geq \sum_{n=1}^{\infty}\frac{1}{n} \rightarrow +\infty
\]
per $\alpha > 1$ 
\[
    \sum_{n=1}^{\infty} \frac{1}{n^\alpha} = converge
\]
per $\alpha = 2$
\[
    \sum_{n=1}^{\infty}\frac{1}{n^2} = \frac{\pi^2}{6} (\sim  \sum_{n=1}^{\infty} \frac{1}{n(n+1)} = serie \;\; di \;\; mengoli)
\]
\subsection*{serie di mengoli}
\[
    \sum_{n=1}^{\infty}\frac{1}{n(n+1)} = \sum_{n=1}^{\infty} \frac{1}{n}-\frac{1}{n+1} = 1- \frac{1}{n+1} \rightarrow 1
\]
\subsection*{sviluppi di Taylor delle funzioni elementari}
\[
    e^x = \lim_{n\rightarrow +\infty}\sum_{k=0}^{n} \frac{x^k}{k!} = \sum_{k=0}^{+\infty} \frac{x^k}{k!}
\]
\[
    sin (x) =\sum_{k=0}^{\infty}(-1)^k \frac{x^{2k+1}}{(2k+1)!} =  \frac{e^{ix} - e^{-ix}}{2}
\]
\[
    cos (x) =\sum_{k=0}^{\infty}(-1)^k \frac{x^{2k}}{(2k)!} =\frac{e^{ix} + e^{-ix}}{2}
\]
\subsection*{Serie di potenza}
Con $a_k$ costanti reali (o complesse) e $x$ variabile reale (o complessa)
\[
    \sum_{k=0}^{\infty}a_k x^k
\]
\[
    Sh (x) = \sum_{k=0}^{\infty} \frac{x^{2k+1}}{(2k+1)!}
\]
\[
    Ch(x) = \sum_{k=0}^{\infty} \frac{x^{2k}}{(2k)!}
\]
\[
    log(1+x)= \sum_{k=1}^{\infty} (-1)^{k+1} \frac{x^k}{k} \;\;\; per \;\; |x|<1
\]
per $\alpha \in \mathbb{R}$
\[
    (1+x)^\alpha = \sum_{k=0}^{\infty} \binom{\alpha}{k}x^k \;\;\; per \;\; |x|<1
\]
\textbf{teor.} Condizione necessaria affinché una serie $\sum_{n=0}^{\infty} a_n$ converga è che il termine generale $a_n$ tenda a zero. (Cioè perchè la serie converga, il termine $a_n$ deve tendere a zero, ma non per forza se il termine $a_n$ tende a zero allora la serie converge)\newline
\newline
\textbf{teor.}supponiamo che una serie $\sum_{n=0}^{\infty} a_n$ converga, allora per ogni $k$ anche risulta convergente anche $\sum_{n=k}^{\infty} a_n$.\newline
\newline
\textbf{Criterio serie a termini non negativi} Una serie $\sum_{n=0}^{\infty}a_n$ a termini non negativi è convergente o divergente a $+\infty$. Essa converge se e solo se la successione delle somme parziali n-esime è limitata.\newline
\newline
\textbf{Criterio del confronto} Siano $\sum an$ e $\sum b_n$ due serie a termini non negativi tali che $a_n<b_n$ definitivamente, allora:
\begin{itemize}
    \item $\sum b_n$ convergente $\Rightarrow \sum a_n$ convergente. 
    \item $\sum a_n$ divergente $\Rightarrow \sum b_n$ divergente. 
\end{itemize}
\textbf{Criterio del confronto asintotico} Se $a_n \sim b_n$, allora le corrispondenti serie $\sum a_n$ e $\sum b_n$ hanno lo stesso carattere (o entrambe divergenti o entrambe divergenti )\newline
\newline
\textbf{Criterio della radice} Sia $\sum a_n$ una serie a termini non negativi. Se esiste il limite 
\[
    \lim_{n\rightarrow +\infty}\sqrt[n]{a_n} = l
\]
\begin{itemize}
    \item $l>1$ la serie diverge $+\infty$
    \item $l<1$ la serie converge
    \item $l=1$ nulla si può concludere
\end{itemize}
Spesso utilizzato con termini che hanno come esponente $n$.
\textbf{Criterio del rapporto} Sia $\sum a_n$ una serie a termini positivi. Se esiste il limite 
\[
    \lim_{n\rightarrow +\infty} \frac{a_{n+1}}{a_n} = l
\]
\begin{itemize}
    \item $l<1$ diverge $+\infty$
    \item $l<1$ converge 
    \item $l=1$ nulla si può concludere
\end{itemize}
Spesso utilizzato quando si hanno termini come $n^n$ e $n!$.
\textbf{Criterio serie a termini di segno variabile} Una serie $\sum a_n$ si dice assolutamente convergente se converge la serie $\sum |a_n|$. Se la serie $\sum a_n$ converge assolutamente, allora converge.\newline
\newline
\textbf{Criterio di Leibniz} Sia data la serie 
\[
    \sum_{n=0}^{\infty}(-1)^n a_n \;\; con  \;\; a_n\geq 0 \;\forall\;n
\]
Se la successione $\{a_n\}$ è decrescente e se $a_n \rightarrow 0$ per $n \rightarrow \infty$, allora la serie è convergente.\newline
Il criterio di Leibniz può essere applicato anche se i termini sono definitivamente di segno alterno e la successione $a_n$ è definitivamente decrescente.\newline
Per verificare la decrescenza bisogna dimostrare che $a_{n+1}<a_n$ oppure mediante il limite a $+ \infty$ della derivata prima di $a_n$ o studiano quando la derivata prima di $a_n<0$ .\newline
\newline
\textbf{Criterio della somma di serie convergenti} Se $\sum_{n=1}^{\infty} a_n$ converge e $\sum_{n=1}^{\infty} b_n$ converge, allora $\sum_{n=1}^{\infty} a_n +b_n$ converge.\newline
\newline
\textbf{Criterio della somma di serie convergenti e divergenti} Se $\sum_{n=1}^{\infty} a_n$ converge e $\sum_{n=1}^{\infty} b_n$ diverge, allora $\sum_{n=1}^{\infty} a_n + b_n$ diverge.\newline
\newline
\textbf{Criterio serie a termini complessi} Sia la serie $\sum_{n=0}^{\infty} a_n$ con $a_n$ complesso, se la serie  $\sum_{n=0}^{\infty} |a_n|$ converge, allora anche $\sum_{n=0}^{\infty} a_n$ converge \newline
\newline
\textbf{Criterio di Dirichlet} Siano $a_n$ e $b_n$ due succesioni tali che:
\begin{itemize}
    \item $a_n$ è a valori complessi e la sua successione delle somme parziali è limitata.
    \item $b_n$ è a valori reali positivi e tende monotonamente a zero
\end{itemize}
allora la serie $\sum a_nb_n$ è convergente.



\subsection*{INTEGRALI}
\subsection*{Teorema fondamentale del calcolo integrale:}
\[
    \int_{a}^{b} f(x) dx = G(b) - G(a)
\]
\subsection*{Proprietà degli integrali:}
\[
    \int_{a}^{b} f(x)dx = \int_{a}^{r} f(x) dx + \int_{r}^{b} f(x) dx
\]
\[
    \left| \int_{a}^{b} f(x) dx \right| \leq \int_{a}^{b} |f(x)| dx
\]
\[
    \int k \cdot f(x) dx = k \cdot \int f(x) dx
\]
\[
    \int [f_1(x) + f_2(x) ] dx= \int f_1(x) dx + \int f_2(x) dx
\]
\subsection*{Integrali fondamentali:}
\[
    \int f'(x) dx = f(x) +c
\]
\[
    \int a \; dx = ax +c
\]
\[
    \int x^n dx = \frac{x^{n+1}}{n+1} +c
\]
\[
    \int \frac{1}{x}dx = ln(|x|) +c
\]
\[
    \int sin(x) dx = -cos(x) +c
\]
\[
    \int cos(x) dx = sin(x)+c
\]
\[
    \int tan(x) dx = -log|cos(x)| +c
\]
\[
    \int log(x) dx \;\; = xlog(x) - \int x \cdot  \frac{1}{x} dx - \int 1 dx = x log(x) -x + c
\]
\[
    \int arctg (x) dx \;\; = x arctg(x) -\int \frac{x}{1+x^2}dx = x arctg(x) -\frac{1}{2}log(1+x^2) + c 
\]
\[
    \int cotg(x) dx = log|sin(x)| +c
\]
\[
    \int (1+tg^2(x))dx = \int \frac{1}{cos^2(x)} dx = tg(x) +c
\]
\[
    \int (1+ctg^2(x))dx = \int \frac{1}{sin^2(x)} dx = -cotg(x) +c
\]
\[
    \int Sh(x) dx = Ch(x) +c
\]
\[
    \int Ch(x) dx = Sh(x) +c
\]
\[
    \int Th(x) dx = log(Ch(x))+c
\]
\[
    \int Coth(x) dx = log|Sh(x)| +c
\]
\[
    \int e^x dx = e^x+c
\]
\[
    \int e^{kx} dx = \frac{e^{kx}}{k} +c
\]
\[
    \int a^x dx = \frac{a^x}{ln(a)}+c
\]
\subsection*{Integrali notevoli:}
\[
    \int sin^2(x) dx = [integrato \; una \; volta \; per \; parti \; e \; sostituzione \; con \; cos^2(x)+ sin^2(x) = 1] = \frac{1}{2}(x-sin(x)cos(x)) +c
\]
\[
    \int cos^2(x) dx = [integrato \; una \; volta \; per \; parti \; e \; sostituzione \; con \; cos^2(x)+ sin^2(x) = 1] = \frac{1}{2}(x+sin(x)cos(x)) +c
\]
\[
    \int tan^2(x) dx = tan(x) -x +c
\]
\[
    \int cotan^2(x) dx = -x -cot(x) +c 
\]
\[
    \int Sh^2(x) dx = [integrato \; una \; volta \; per \; parti \; e \; sostituzione \; con \; Ch^2(x) - Sh^2(x) = 1] = \frac{1}{4}(Sh(2x)-2x) +c
\]
\[
    \int Ch^2(x) dx = [integrato \; una \; volta \; per \; parti \; e \; sostituzione \; con \; Ch^2(x) - Sh^2(x) = 1] = \frac{1}{2}(x + Sh(x)Ch(x)) +c
\]
\[
    \int Th^2(x) dx = x - Th(x) +c 
\]
\[
    \int Coth^2(x) dx = x - Coth(x) +c 
\]
\[
    \int \frac{1}{sin^2(x)} dx = [1 = cos^2(x) +sin^2(x)] = \int 1 + tan^2(x) dx = -cotan(x) + c
\]
\[
    \int \frac{1}{cos^2(x)} dx = [1 = cos^2(x) +sin^2(x)] = \int 1 + cotan^2(x) dx = tan(x) + c
\]
\[
    \int \frac{1}{tan^2 (x)} dx = \int cotan^2(x) dx 
\]
\[
    \int \frac{1}{cotan^2(x) }dx = \int tan^2(x) dx 
\]
\[
    \int \frac{1}{Ch^2(x)} dx = \int (1-Th^2(x))dx= Th(x)+c
\]
\[
    \int \frac{1}{Sh^2(x)} = \int (-1 Coth^2(x)) dx = -Coth(x) +c
\]
\[
    \int \frac{1}{1+x^2} dx = arctg(x) +c
\]
\[
    \int \frac{1}{1-x^2} dx = \frac{1}{2}log\left|\frac{1+x}{1-x}\right|+c
\]
\[
    \int \frac{1}{\sqrt{1+x^2}} dx = arcSh(x) +c = log(x + \sqrt{1+x^2}) +c
\]
\[
    \int \frac{1}{\sqrt{1-x^2}} dx = arcsin(x) +c
\]
\[
    \int \frac{1}{\sqrt{-1 + x^2}} dx = log|x+ \sqrt{x^2-1}| +c
\]
\[
    \int \frac{1}{\sqrt{\pm a^2 + x^2}} dx = log|x + \sqrt{x^2 \pm a^2}|+c
\]
\[
    \int \sqrt{x^2 \pm a^2} dx = \frac{x}{2} \sqrt{x^2 \pm a^2} \pm \frac{a^2}{2} log(x + \sqrt{x^2 \pm a^2}) +c
\]
\[
    \int \sqrt{a^2 - x^2}dx = \frac{1}{2}(a^2arcsin(\frac{x}{a}) + x \sqrt{a^2 - x^2} )+c
\]
\subsection*{Integrali riconducibili:}
\[
    \int f^n(x) \cdot f'(x) dx = \frac{f^{n+1}(x)}{n+1} +c
\]
\[
    \int \frac{f'(x)}{f(x)} dx = log|f(x)|+c
\]
\[
    \int f'(x) \cdot cos(f(x)) dx =  sin(f(x))+c
\]
\[
    \int f'(x) \cdot sin(f(x)) dx = -cos(f(x)) +c
\]
\[
    \int e^{(f(x)} \cdot f'(x) dx = e^{f(x)}+c
\]
\[
    \int a^{f(x)} \cdot f'(x) dx = \frac{a^{f(x)}}{ln(a)} +c
\]
\[
    \int \frac{f'(x)}{1+f^2(x)} dx = arctg(f(x))+c
\]
\subsection*{Integrazione per sostituzione:}
Sostituire alla variabile $x$ una funzione di un'altra variabile $t$, purchè tale funzione sia derivabile e invertibile.\newline
Ponendo $x = g(t)$ da cui deriva $dx = g'(t) dt$ si ha che:
\[
    \int f(x) dx = \int f[g(t)] \cdot g'(t) dt
\]
Da ricordare è che se si è in presenza di un integrale definito bisogna aggiornare anche gli estremi di integrazione. Se non si volesse cambiare l'intervall odi integrazione si può risostituire il vecchio valore di $t$.
\subsection*{Integrazione delle funzioni razionali:}
\[
    \int \frac{P_n(x)}{Q_m(x)} dx
\]
Per prima cosa se il grado del numeratore è $\geq$ del grado del denominatore, si esegue la divisione di polinomi:
\begin{itemize}
    \item Si dispongono i polinomi dal termine di grado maggiore a quello minore nella seguente maniera:
    \[
        P(x) \;\; | \;\; Q(x)    
    \]
    badando al fatto che se nel polinomio $P(x)$ mancasse qualche termine bisognerebbe scrivere $0$ nella sua posizione.
    \item Si dividono il termine di grado massimo di $P(x)$ con quello di grado massimo di $Q(x)$, riportando il risultato al di sotto di $Q(x)$.
    \item Moltiplichiamo il termine appena scritto per ogni termine di $Q(x)$, ne invertiamo il segno e lo trascriviamo al di sotto dei termini con lo stesso grado di $P(x)$
    \item Sommiamo termine per termine $P(x)$ con i valore appena scritti e li riportiamo sotto.
    \item Ripetiamo questo procedimento finchè il grado più alto fra i termini dell'ultima riga scritta a sinistra è minore (non minore uguale) del termine di grado massimo di $Q(x)$
    \item Il polinomio a destra è il risultato della divisione $S(x)$, mentre ciò che rimane sulla sinistra è il resto $R(x)$. Possiamo ora riscrivere il numeratore:
    \[
        P(x) = S(x) \cdot Q(x) + R(x)
    \]
\end{itemize}
Vediamo ora i vari casi possibili:
\begin{itemize}
    \item \textbf{denominatore di primo grado:} integrale immediato tramite il logaritmo
    \item \textbf{denominatore di secondo grado:} si calcola il segno del discriminante:
    \begin{itemize}
        \item \textbf{due radici distinte:} si scompone in fratti semplici
        \[
            \frac{N(x)}{D_1(x) \cdot D_2(x)} = \frac{a}{D_1(x)} + \frac{b}{D_2(x)}
        \]
        \[
            \frac{a \cdot D_2(x) + b \cdot D_1(x)}{D_1(x) \cdot  D_2(x)} = \frac{N(x)}{D_1(x) \cdot D_2(x)}
        \]
        \[
            a \cdot D_2(x) + b \cdot D_1(x)= N(x)
        \]
        Una volta determinate $a$ e $b$ si riscrive l'integrale come $\frac{a}{D_1(x)} + \frac{b}{D_2(x)}$ e si integra come somma di logaritmi
        \item \textbf{denominatore quadrato perfetto:} (due soluzioni coincidenti), si procede per sostituzione:
        \[
            \int \frac{N(x)}{D(x)^2} dx = [D(x) = t, \; \dots] = \; \dots
        \] 
        L'utilità della sostituzione è quella di spezzare la frazione in una somma di frazion ida integrare una ad una.
        \item \textbf{denominatore non si annulla mai:}\newline
        Casi semplici:
        \[
            \int \frac{1}{1 + x^2}dx = arctg (x) +c
        \]
        \[
            \int \frac{1}{x^2 + a^2}dx = \frac{1}{a} arctg \frac{x}{a} +c
        \]
        \[
            \int \frac{1}{a^2 + (x+b)^2}dx = \frac{1}{a} arctg \frac{x+b}{a} +c
        \]
        Caso generico: Si cerca di dividere l'integrale in una somma di integrali, il primo deve contenere al numeratore la derivata del denominatore, il secondo non deve contenere la $x$ al numeratore, cioè deve essere una costante e quindi riconducibile ai casi semplici sopra riportati. Il denominatore non cambia. Ci si arriva a logica. 
    \end{itemize}
    \item \textbf{denominatore di grado maggiore di due:} è sempre possibile scomporlo in prodotti di fattori di primo grado o di secondo grado irriducibili, per farlo si usa Ruffini (o altrimenti si va a tentoni ricordando che PROBABILMENTE una radice della funzione è un dividendo (positivi e negativi) del numero che si ricava moltiplicando il coefficiente del termine massimo e il termine noto).\newline
    Fatto questo si scompone la frazione in fratti semplici con la stessa logica del caso di due radici distinte, ricordando che il numeratore deve essere un espressione di un grado minore del denominatore, per esempio se il denominatore è di grado $2$, allora si userà $ax+b$ che è di grado $1$.
\end{itemize}
\subsection*{Funzioni razionali di $e^x$}
Si pone $e^x = t$, $x= log(t)$, $dx = \frac{dt}{t}$ e ci si riconduce a una funzione razionale classica.
\subsection*{Integrazione per parti:}
\[
    \int f'(x) \cdot  g(x) dx = f(x) \cdot g(x) - \int f(x) \cdot g'(x)dx
\]
La formula deriva dalla formula di derivazione della moltiplicazioni di due funzioni:
\[
    (fg)' = f'g + fg'
\]
\[
    fg' = (fg)' - f'g
\]
Si può vedere la formula di integrazione per parti più facilmente così:
\[
    \int integranda \cdot derivanda \; dx = primitiva \cdot derivanda - \int primitiva \cdot derivata \; dx
\]
L'integrazione per parti si usa:
\begin{itemize}
    \item dovendo calcolare integrali della forma
    \[
        \int x^n \cdot f(x) dx \;\;\; f(x) = \begin{cases}
            cos(x)\\ 
            sin(x)\\ 
            e^x\\ 
            Sh(x)\\ 
            Ch(x)        
        \end{cases}
    \]
    si integra per parti derivando $x^n$ e integrando $f(x)$. Per $n=1$ l'integrale si riduce a uno immediato, per $n>1$ si itera il procedimento fino al caso $n=1$. Si possono svolgere allo stesso modo anche integrali del tipo:
    \[
        \int P_n(x) f(x) dx
    \]
    \item dovendo calcolare integrali della forma
    \[
        \int f(x) g(x) dx \;\;\;\; \begin{cases}
            f(x) = e^{\alpha x}, Sh(\alpha x), Ch( \alpha x), a^{bx} \\
            g(x) = cos( \beta x), sin(\beta x)
        \end{cases}
    \]
    si eseguono due integrazioni per parti consecutive, nella prima la scelta della funzione da integrare o derivare è indifferente, nella seconda però la scelte deve essere coerente alla prima. Chiamando $I$ l'integrale di partenza si ottiene una funzione della forma
    \[
        I = h(x) - \frac{\beta}{\alpha}^2 I
    \]
    da cui si ricava $I$.\newline
    Se entrambe le funzioni $f(x)$ e $g(x)$ sono del tipo $cos(x)$ o $sin(x)$ si usano le formule di duplicazione o prostaferesi (vedi più avanti).
    \item L'integrale del logaritmo, derivando $log(x)$ e integrando $1$
    \[
        \int log(x) dx \;\; = xlog(x) - \int x \cdot  \frac{1}{x} dx - \int 1 dx = x log(x) -x + c
    \]
    Più in generale, dovendo calcolare integrali della forma
    \[
        \int x^m log^n(x) dx
    \]
    e ponendo $g' = x^m$ e $f = log^n(x)$ ed eseguendo iterativamente $n$ integrazioni per parti si riesce a calcolare l'integrale del logaritmo. Ancora più in generale si possono risolvere integrali della forma:
    \[
        \int P_m(x) \cdot Q_n(log(x)) dx 
    \]
    \item l'integrale dell'arcotangente, derivando $arctg(x)$ e integrando $1$
    \[
        \int arctg (x) dx \;\; = x arctg(x) -\int \frac{x}{1+x^2}dx = x arctg(x) -\frac{1}{2}log(1+x^2) + c 
    \]
    Più in generale
    \[
        \int x^n arctg(x) dx  = \frac{x^{n+1}}{n+1}arctg(x) - \int\frac{x^{n+1}}{n+1} \frac{dx}{1+x^2}
    \]
\end{itemize}
\subsection*{integrazione delle funzioni trigonometriche}
\begin{itemize}
    \item dovendo calcolare
    \[
        \int f(sin(x)) \cdot  cos(x) dx \;\;\; \Rightarrow  \;\;\; sin(x) = t, cos(x) dx =dt
    \]
    \[
        \int f(cos(x)) \cdot  sin(x) dx\;\;\; \Rightarrow  \;\;\; cos(x) ) t, -sin(x) dx = dt
    \]
    In particolare per calcolare 
    \[
        \int sin^n(x) cos^m(x)
    \]
    se almeno uno degli esponenti è dispari si riesce a riscrivere l'integrale in una delle forme viste sopra utilizzando: $sin^2(x) + cos^2(x) = 1$. Se entrambi gli esponenti sono pari si usano le formule trigonometriche per abbassarne il grado: $cos^2(x) = \frac{1}{2}(1+cos(2x))$ e $sin^2(x) = \frac{1}{2} (1-cos(2x))$
    \item per integrali del tipo
    \[
        \int cos(\alpha x) sin(\beta x) dx, \;\;\;\int cos(\alpha x) cos(\beta x) dx, \;\;\;\int sin(\alpha x) sin(\beta x) dx,
    \]
    si usano le regole di prostaferesi che riconducono a somme di integrali immediati
    \item integrali di funzioni razionali di $sin(x)$ e $cos(x)$ possono sempre essere ricondotti a integrali di funzioni razionali generiche tramite la sostituzione:
    \[
        t = tg\left(\frac{x}{2}\right), \;\;\; x= 2 arctg(t), \;\;\;dx = \frac{2}{1+t^2}dt
    \]
    ne derivano le seguenti identità trigonometriche:
    \[
        \begin{cases}
            cos(x) = \frac{1-t^2}{1+t^2}\\
            sin(x) = \frac{2t}{1+t^2}
        \end{cases}
    \]
    \item integrali definiti notevoli:
    \[
        \int_{0}^{n \frac{\pi}{2}}cos^2(kx) dx = \int_{0}^{n \frac{\pi}{2}}sin^2(kx) dx = n \frac{\pi}{4}
    \]
    \item per calcolare integrali razionali con $Sh(x)$ e $Ch(x)$ o si trovano scorciatoie con trasformazioni oppure si usa la sostituzione $e^x = t, x= log(t), dx = \frac{dt}{t}$
\end{itemize}
\subsection*{Integrazione delle funzioni irrazzionali}
\begin{itemize}
    \item se l'integranda è una funzione razionale di $x$ moltiplicata per solo una delle seguenti
    \[
        \int R(x) \sqrt{a^2-x^2} dx = [x=a \cdot  sin(t), dx =a \cdot  cos(t)dt] = \int \sqrt{a^2(1-sin^2(t))}dx = \int|a \cdot  cos(t)|dx
    \]
    \[
        \int R(x) \sqrt{a^2 + x^2} = [x=a \cdot  Sh(t), dx =a \cdot  Ch(t)dt] = \int \sqrt{a^2(1-Sh^2(t))}dx = \int a \cdot  Ch(t)dx
    \]
    \[
        \int R(x) \sqrt{x^2 -a^2} = [x=a \cdot  Ch(t), dx =a \cdot Sh(t)dt] = \int \sqrt{a^2(Ch^2(t)-1)}dx = \int|a \cdot Sh(t)|dx
    \]
    Negli ultimi due casi per tornare alla variabile x occorre usare le funzioni iperobliche inverse:
    \[
        \begin{cases}
            x = a \cdot  Ch(t) \Rightarrow t = SettCh(\frac{x}{a}) = log\left(\frac{x}{a}+\sqrt{\frac{x^2}{a^2}-1}\right)\\
            x= a \cdot Sh(t) \Rightarrow t = SettSh\left(\frac{x}{a}\right) = log\left(\frac{x}{a}+\sqrt{\frac{x^2}{a^2}+1}\right)
        \end{cases}
    \]
    è utile anche ricordare che $Sh(SettCh(a))= \sqrt{a^2-1}$ e $Ch(SettSh(a))=\sqrt{a^2+1}$
    \item integrale di una funzione razionale di $x, x^{\frac{n_1}{m_1}}, x^{\frac{n_2}{m_2}}$, etc. \newline
    Si pone $x = t^n$ con $n=$ minimo comune multiplo di $m_1$, $m_2$, etc. Si ha quindi $dx = n \cdot  t^{n-1} dt$ e si ottiene una funzione razionale di $t$.
    \item Se l'integranda è una funzione del tipo $R(x^{2n+1}, \sqrt{x^2 \pm a^2})$
    \[
        \int x^{2n+1}R(\sqrt{x^2 \pm a^2})dx = [\sqrt{x^2 \pm a^2} = t, xdx= tdt, x^{2n+1} \cdot dx = (t^2 \mp a^2)^n t \cdot  dt]
    \]
\end{itemize}
\subsection*{Simmetrie e valori assoluti nel calcolo di integrali definiti}
\begin{itemize}
    \item se $f(x)$ è \textbf{pari}:
    \[
        \int_{-k}^{k}f(x)dx = 2 \int_{0}^{k} f(x) dx
    \]
    \item se $f(x)$ è \textbf{dispari}:
    \[
        \int_{-k}^{k}f(x)dx = 0
    \]
\end{itemize}
\subsection*{Osservazione. Integrale generalizzato di una funzioen dispari su un intervallo simmetrico}
Non è corretto affermare l'annullarsi di un integrale dispari per motivi di simmetria in un intervallo simmetrico senza prima verificare la convergenza dell'integrale stesso.
\subsection*{INTEGRALI GENERALIZZATI}
\subsection*{Integrazione di funzioni non limitate}
Metodo generale di risoluzione:
\[
    \lim_{x\rightarrow b^-} f(x) = \pm \infty \;\;\;\;\;\;\;\;\;\;\;\;\;\;\;\;\;\;\;\;\;\;\;\;\;\;\;\;\;\;\;\;\;\;\;\;\;\;\;\;\lim_{x\rightarrow a^+}f(x) = \pm \infty
\]
\[
    \int_{a}^{b}f(x)dx = \lim_{\epsilon\rightarrow 0^+}\int_{a}^{b-\epsilon}f(x)dx \;\;\;\;\;\;\;\;\;\;\;\;\; \int_{a}^{b}f(x)dx = \lim_{\epsilon\rightarrow 0^+}\int_{a+\epsilon}^{b}f(x)dx 
\]
\subsection*{Criteri di integrabilità al finito}
Siano $\lim_{x\rightarrow b^-}f(x) = \lim_{x\rightarrow b^-}g(x) = + \infty$:
\begin{itemize}
    \item confronto: se $0\leq f(x) \leq g(x)$, allora $g$ integrabile $\Rightarrow f$ integrabile e $f$ non integrabile $\Rightarrow g$ non integrabile.
    \item confronto asintotico: se $f>0$ e $g>0$ e $f \sim g$ per $x \rightarrow b^-$, allora $f$ integrabile $\Leftrightarrow g$ integrabile.
    \item \textbf{teor.} (da usare per studiare per esempio funzioni seno e coseno per $x \rightarrow \infty$)
    \[
        \int_{a}^{b}|f(x)|dx \;\; convergente \;\;\Rightarrow \int_{a}^{b}f(x)dx \;\;convergente
    \]
\end{itemize}
\subsection*{Integrazione su intervalli illimitati}
Metodo generale di risoluzione:
\[
    \int_{a}^{+\infty}f(x) dx = \lim_{\omega\rightarrow +\infty}\int_{a}^{\omega}f(x)dx
\]
\[
    \int_{- \infty}^{b}f(x) dx = \lim_{\omega\rightarrow -\infty}\int_{\omega}^{b}f(x)dx
\]
\[
    \int_{-\infty}^{+\infty}f(x) dx = \int_{-\infty}^{c} f(x) dx + \int_{c}^{+\infty}f(x) dx
\]
\textbf{def.} Se il limite dell'integrale di $f$ esiste finito allora $f$ si dice integrabile oppure che l'integrale è convergente. \newline
\textbf{def.} Se il limite dell'integrale è $\pm \infty$, l'integrale si dice divergente.\newline
\textbf{def.} Se il limite non esiste, l'integrale non esiste.\newline
\textbf{per essere integrabile deve avere limite finito.}
\subsection*{Criteri di integrabilità all'infinito}
\begin{itemize}
    \item confronto: se $0\leq f(x) \leq g(x)$ in $[a,+\infty)$, allora $g$ integrabile $\Rightarrow f$ integrabile e $f$ non integrabile $\Rightarrow g$ non integrabile.
    \item confronto asintotico: se $f>0$, $g>0$ e $f \sim g$ per $x \rightarrow + \infty$, allora $f$ integrabile $\Leftrightarrow g$ integrabile
    \item \textbf{teor.} (da usare per studiare per esempio funzioni seno e coseno per $x \rightarrow \infty$)
    \[
        \int_{a}^{+\infty}|f(x)| dx \;\; convergente \;\; \Rightarrow \int_{a}^{+\infty} f(x) dx \;\;convergente
    \]
\end{itemize}
\subsection*{Osservazione. Ordine di annullamento di una funzione derivabile.}
Se $f$ è una funzione derivabile in un intervallo $I$, la formula di Taylor ci dice che se $f$ si annulla in un punto $\alpha \in I$, si annulla almeno del prim'ordine. Precisamente poichè
\[
    f(x) - f(\alpha) = f'(\alpha)(x-\alpha) + o(x-\alpha)
\]
se $f'(\alpha) \neq 0$ allora $f$ ha uno zero del prim'ordine in $\alpha$. Se $f'(\alpha) = 0$ ma, ad esempio, $f''(\alpha) \neq 0$, si può concludere che $f$ si annulla del $2^0$ ordine, e così via. In ogni caso non può annullarsi di un ordine inore di $1$.
\subsection*{Integrali generalizzati notevoli}
Caso 1:
\[
    \int_{a}^{b}\frac{1}{(x-a)^p}dx \rightarrow \begin{cases}
        converge \;\;&se \;\;p<1\\
        diverge \;\;&se \;\; p\geq 1
    \end{cases}
\]
\[
    \int_{a}^{b}\frac{dx}{(b-x)^p} \rightarrow  \begin{cases}
        converge \;\;&se \;\;p<1\\
        diverge \;\;&se \;\; p\geq 1
    \end{cases}
\]
Caso 2:
\[
    \int_{a}^{+\infty}\frac{1}{x^p}dx \rightarrow \begin{cases}
        converge \;\; & se \;\; p > 1\\
        diverge \;\; &  se \;\; p \leq 1
    \end{cases}
\]
Caso 3: con $0< \alpha < 1$
\[
    \int_{0}^{\alpha} \frac{1}{x^a \cdot  |ln(x)|^b} \rightarrow \begin{cases}
        converge \;\; se \;\; & \begin{cases}
            a < 1 \;\; e \;\; b \in \mathbb{R}\\
            oppure \\
            a=1 \;\; e \;\; b > 1
        \end{cases}\\
        diverge \;\; se \;\; & \begin{cases}
            a>1 \;\; e \;\; b \in \mathbb{R}\\
            oppure\\
            a=1 \;\; e \;\; b \leq 1
        \end{cases}
    \end{cases}
\]
Caso 4: con $\alpha > 1$
\[
    \int_{\alpha}^{+\infty} \frac{1}{x^\alpha \cdot ln^b(x)}dx \rightarrow \begin{cases}
        converge \;\; se \;\; & \begin{cases}
            a > 1 \;\; e \;\; b \in \mathbb{R}\\
            oppure \\
            a=1 \;\; e \;\; b > 1
        \end{cases}\\
        diverge \;\; se \;\; & \begin{cases}
            a<1 \;\; e \;\; b \in \mathbb{R}\\
            oppure\\
            a=1 \;\; e \;\; b \leq 1
        \end{cases}
    \end{cases}
\]
Caso 5: con $\alpha> 1$
\[
    \int_{1}^{\alpha}\frac{1}{ln^p(x)} dx \rightarrow \begin{cases}
        converge \;\; se \;\; &p<1\\
        diverge \;\; se \;\; &p\geq 1
    \end{cases}
\]
\subsection*{FUNZIONI INTEGRALI}
\textbf{teor.} Secondo teorema fondamentale del calcolo integrale \newline
Sia $f: [a,b] \rightarrow \mathbb{R}$ una funzione integrabile e sia $x_0 \in [a,b]$ e sia 
\[
    F(x) = \int_{x_0}^{x} f(t) dt
\]
Allora:
\begin{itemize}
    \item La funzione $F$ è continua in $[a,b]$
    \item Se inoltre $f$ è continua in $[a,b]$, allora $F$ è derivabile in $[a,b]$ e vale
    \[
        F'(x) = f(x) \;\; per \; ogni \; x \in[a,b]
    \]
    (Se $f(t)$ non è continua su tutto $I$, ma è integrabile in senso generalizzato, in tutti i punti in cui $f(t)$ è continua, $F(x)$ è derivabile e $F'(x) = f(x)$)\newline
    $F$ ha punti di non derivabilità dove $f$ è discontinua.
\end{itemize}
Conseguenze:
\begin{itemize}
    \item se $f$ è continua, $F$ è derivabile con continuità
    \item se $f$ è continua e derivabile con continuità, anche $F'$ è derivabile con continuità, quindi $F$ è due volte derivabile con continuità. Iterando: la funzione integrale ha sempre un grado di regolarità in più rispetto alla funzione integranda
    \item ogni fuzione continua su $I$ ha una primitiva su $I$
\end{itemize}
Logica degli esercizi in cui bisogna trovare l'intervallo di definizione:
\[
    F(x) = \int_{x_0}^{x}f(x)dx
\]
\begin{itemize}
    \item lo scopo è determinare dove la funzione integranda è integrabile.
    \item Vedere dove la funzione integranda è continua, una funzione continua è integrabile. Analizzare i punti di discontinuità:
    \item Se una funzione ha un numero finito di discontinuità limitate in un intervallo, allora è integrabile in quell'intervallo. In poche parole se è una discontinuità a salto è integrabile.
    \item Per gli altri punti di discontinuità la funzione integranda è illimitat, quindi bisogna studiarla (con i criteri del confronto, del confronto asintotico, col teorema del modulo, calcolando effettivamente la primitiva e il limite, o riducendosi al caso particolare delle funzioni non limitate con gli asintotici o gli sviluppi di Taylor).
    \item Se la funzione itegranda non è integrabile nel punto $x_0$ allora l'insieme di definizione di $F$ è vuoto. Ma se $x_0$ fosse un punto di accumulazione bisogna studiare l'integrale della funione per $t \rightarrow x_0$ e vedere se è effettivamente integrabile o meno.
\end{itemize}
Logica degli esercizi sulla regolarità delle funzioni integrali:
\[
    F(x) = \int_{x_0}^{x} f(x)dx
\]
\begin{itemize}
    \item si determina l'insieme di definizione. (vedi sopra)
    \item per determinare i punti di non derivabilità di $F(x)$ studiamo la sua derivata $F'(x) = f(x)$. I punti di non derivabilità sono quelli quelli dove $f(x)$ non è definita, e in $F(x)$ corrispondono a:
    \begin{itemize}
        \item discontinuità a salto in $f$ è un punto angoloso in $F$
        \item punti di asintoto verticale di $f$ sono cuspidi (verso l'alto o il basso) o flessi a tangente verticale (ascendente o discendente) di $F$
    \end{itemize}
    \item Notiamo che tangenti verticali o discontinuità a salto o buchi nella funzione di $F$ non possono essere presenti nel dominio di $F$, perchè essendo punti di discontinuità non sono derivabili e dunque non presenti nell'intervallo di integrazione di $f$.\newline
    Dunque la funzione $F$ è (sempre) continua nel suo intervallo di definizione.
\end{itemize}
Logica degli esercizi sui grafici qualitativi della funzione integrale $F(x)$ a partire dalla funzione integranda $g(x)$
\begin{itemize}
    \item $F$ è crescente sugli intervalli in cui $g$ è positiva, $F$ è decrescente sugli intervalli in cui $g$ è negativa.
    \item punti in cui $g$ incrocia l'asse delle $x$ sono punti di massimo o minimo
    \item discontinuità a salto in $g$ sono punti angolosi
    \item $F$ è concava verso l'alto (il basso) negli intervalli in cui $g$ è crescente (decrescente)
    \item punti di cambio massimo e minimo in $g$ sono punti di cambio di concavità in $F$
\end{itemize}
Limite all'infinito di una funzione integrale:
\[
    \lim_{x\rightarrow +\infty}F(x) = \; integrale \;\;generalizzato \;\;= \int_{x_0}^{+\infty}f(t)dt
\]
se l'integrale generallizzato converge esiste limite finito (anche se non si riesce a calcolare), se non converge o è divergente o non esiste.\newline
Caso particolare è quello in cui $f(t) \rightarrow m$, costante non nulla, per cui $F(x) \sim  mx$. Quindi $F(x)$ tende a infinito con crescita lineare e potrebbe avere asintoto obliquo calcolabile come 
\[
    \lim_{x\rightarrow \infty}[F(x)-mx] = \lim_{x\rightarrow \infty}\int_{x_0}^{x}[f(t)-m]dt + mx_0
\]
Ossia esiste asintoto obliquo se l'integrale generalizzato
\[
    \int_{x_0}^{\infty}[f(t)-m]dt
\]
converge.





\newpage
\subsection*{TEORIA}
\subsection*{Derivabilità implica continuità}
\textbf{enunciato}\newline
Sia
\[
    f: A \in \mathbb{R} \rightarrow  \mathbb{R}
\]
definita come $x \longrightarrow y = f(x)$, allora in un punto $x_0 \in A$ derivabilità $\Rightarrow $ continuità. \newline
\textbf{dimostrazione}\newline
Per definizione di derivabilità in $x_0$ sappiamo che esiste finito il limite
\[
    \lim_{h\rightarrow 0}\frac{f(x_0 +h)- f(x_0)}{h} = f'(x_0) = m
\]
allora possiamo dire che
\[
    \lim_{h\rightarrow 0} \frac{f(x_0 + h) - f(x_0)}{h \cdot m} = 1
\]
Che rispecchia la definizione di asintotico, quindi per $h \rightarrow  0$
\[
    f(x_0+h) - f(x_0) \sim m \cdot h
\]
\[
    f(x_0 + h ) - f(x_0) = m \cdot h  + o(m \cdot h)
\]
\[
    f(x_0 + h) = f(x_0) + m \cdot h + o(m \cdot h)
\]

\[
    \lim_{x\rightarrow x_0} f(x) = \lim_{h\rightarrow 0} f(x_0 +h) = \lim_{h \rightarrow 0}[f(x_0) + m \cdot h + o(m \cdot h)] = f(x_0)
\]
Dunque il limite per $x \rightarrow x_0$ di $f(x)$ vale $f(x_0)$ che soddisfa la definizione di continuità di $f(x)$ in $x_0$.







\newpage
\subsection*{Teorema di Fermat}
\textbf{enunciato}:\newline
Sia $f:[a,b]\rightarrow \mathbb{R}$. Se $f$ è derivabile in $x_0 \in(a,b)$ e $x_0$ è punto stazionario allora
\[
    f'(x_0) = 0
\]
\textbf{dimostrazione}:\newline
Dimostriamo per il caso in cui $x_0$ sia un punto di massimo locale (analogamente si procede per dimostrare il caso in cui sia un punto di minimo). \newline
Per ipotesi $f(x)$ è derivabile nel punto $x_0$, dunque vale la condizione
\[
    \lim_{x\rightarrow x_0^-} f'(x) = \lim_{x\rightarrow x_0^+} f'(x)
\]
Essendo $x_0$ punto di massimo relativo, dato un incremento $h$ vale 
\[
    f(x_0 +h) - f(x_0) \leq 0.
\]
Dividiamo i casi in cui:
\[
    \begin{cases}
        \frac{f(x_0 +h) - f(x_0)}{h} \leq 0 \;\;\;\; &(h>0)\\
        \\
        \frac{f(x_0 +h) - f(x_0)}{h} \geq 0 &(h<0)
    \end{cases}
\]
Passiamo ora ai limiti per $h \rightarrow 0$:
\[
    \begin{cases}
        \lim_{h\rightarrow 0}\frac{f(x_0 +h) - f(x_0)}{h} \leq 0 \;\;\;\; &(h>0)\\
        \\
        \lim_{h\rightarrow 0}\frac{f(x_0 +h) - f(x_0)}{h} \geq 0 &(h<0)
    \end{cases}
\]
Questi due limiti sono rispettivamente limite destro e sinistro della derivata prima,
\[
    f'_+(x_0) = f'_-(x_0)
\]
Per l'ipotesi di derivabilità di $f$ in $x_0$ i due limiti devono coincidere, quindi essendo
\[
    f'_+(x_0) \leq 0
\]
\[
    f'_-(x_0) \geq 0
\]
L'unico caso possibile è che
\[
    f'_+(x_0) = f'_-(x_0) = 0
\]
Cioè
\[
    f'(x_0) = 0
\]






\newpage
\subsection*{Teorema di Rolle}
\textbf{enunciato}\newline
Sia $f:[a,b] \in \mathbb{R} \rightarrow  \mathbb{R}$. Se $f$ è continua in $[a,b]$, derivabile in $(a,b)$ e se vale $f(a) = f(b)$, allora esiste un punto $c \in(a,b)$ tale che $f'(c) = 0$.\newline
\newline
\textbf{dimostrazione}\newline
Poiché $f$ è continua, in virtù del teorema di Weierstrass la funzione sull'intervallo $[a,b]$ ammette massimo e minimo assoluti (che indichiamo rispettivamente con $M$ e $m$). Si danno due casi: o il massimo e il minimo sono entrambi raggiunti negli estremi dell'intervallo $[a,b]$, oppure almeno uno dei due è raggiunto in un punto appartenente all'intervallo $(a,b)$.\newline

1. Il massimo e il minimo sono entrambi raggiunti negli estremi e quindi, poiché per ipotesi si ha che $f(a)=f(b)$, ne segue che $M=m$. Questo implica che la funzione è costante sull'intervallo $[a,b]$ e quindi la derivata è nulla in ciascun punto $c$ dell'intervallo $(a,b)$.


2. Il massimo o il minimo sono raggiunti all'interno dell'intervallo. Per fissare le idee, consideriamo il caso in cui il massimo è raggiunto in un punto $c$ dell'intervallo aperto $(a,b)$, cioè $f(c)=M$. Per il teorema di Fermat allora la derivata è nulla nel punto $c$.






\newpage
\subsection*{Teorema di Lagrange o del valor medio}
\textbf{enunciato}\newline
Sia $f$ derivabile in $(a,b) in \mathbb{R}$ e continua in $[a,b] \in \mathbb{R}$. Allora esiste almeno un punto $c \in (a,b)$ tale che 
\[
    f'(c) = \frac{f(b)-f(a)}{b-a}
\]
\textbf{dimostrazione}\newline
È possibile dimostrare l'asserto mediante un'applicazione del teorema di Rolle.\newline
Sia $g$ la seguente funzione ausiliare:
\[
    g(x)=f(a)+\frac{f(b)-f(a)}{b-a}(x-a)
\] 
Si tratta della retta passante per i punti $(a,f(a))$ e $(b,f(b))$.\newline
Sia ora $h$ la differenza tra le due funzioni $f$ e $g$:
\[
    h(x) = (f-g)(x) = f(x) - g(x).
\]
Si verifica immediatamente che
\[
    h(a) = f(a) - g(a) = f(a) - f(a) - \frac{f(b) - f(a)}{b-a}(a-a)= 0
\]
\[
    h(b) = f(b) - g(b) = f(b) - f(a) - \frac{f(b) - f(a)}{b-a}(b-a) = 0
\]
La funzione $h$ è continua perché somma di funzioni continue ($f$ per ipotesi e $g$ perché è un polinomio di primo grado); inoltre è derivabile perché somma di funzioni derivabili ($f$ per ipotesi e $g$ in quanto polinomio di primo grado).\newline
Per il teorema di Rolle, se una funzione è continua in un intervallo $[a,b]$, derivabile in $(a,b)$ e assume valori uguali agli estremi dell'intervallo, esiste almeno un punto $c\in (a,b)$ in cui la sua derivata sia $0$.\newline
Applichiamo quindi il teorema di Rolle alla funzione $h$, dal momento che ne soddisfa tutte le ipotesi:
\[
    \exists \;\; c\in (a,b) \;\;\;\;tale \;\; che \;\;\;\; h'(c)=0.
\]
Segue che
\[
    h'(c)=f'(c)-g'(c)=0
\]
Ora si osserva che
\[
    g'(c) = \frac{f(b) - f(a)}{b-a}  
\]
quindi
\[
    f'(c) = g'(c) =  \frac{f(b) - f(a)}{b-a}
\]
e il teorema è così dimostrato.





\newpage
\subsection*{Test di monotonia su un intervallo}
\textbf{enunciato}\newline
Sia $f:(a,b) \in \mathbb{R} \rightarrow \mathbb{R}$, derivabile. Allora
\[
    f \;\; crescente \Longleftrightarrow f'(x)\geq 0
\]
\[
    f \;\; decrescente \Longleftrightarrow  f'(x) \leq 0
\]
$\;\forall\; x \in (a,b)$\newline 
\newline
\textbf{dimostrazione}\newline 
Consideriamo $f: (a,b) \rightarrow \mathbb{R}$ e due punti qualunque $x,z \in (a,b)$. Allora se
\[
    f \;\; crescente \;\;\; \Longleftrightarrow \frac{f(z) - f(x)}{z-x} \geq 0
\]
\[
    f \;\; decrescente \Longleftrightarrow \frac{f(z) - f(x)}{z-x} \leq 0
\]
Passando al limite per $z \rightarrow x$, per il teorema della permanenza del segno, dalle due precedenti relazioni si ottiene
\[
    f \;\; crescente \;\;\; \Longleftrightarrow \lim_{z\rightarrow x}\frac{f(z) - f(x)}{z-x} \geq 0
\]
\[
    f \;\; decrescente \Longleftrightarrow \lim_{z\rightarrow x}\frac{f(z) - f(x)}{z-x} \leq 0
\] 
E quindi $\;\forall\; x \in (a,b)$
\[
    f \;\; crescente \;\;\; \Rightarrow f'(x) \geq 0
\]
\[
    f \;\; decrescente \Rightarrow  f'(x) \leq 0
\]
Viceversa , sia, ad esempio, $f'(x) \geq 0$ (si procede in modo analogo per $f'(x) \leq 0$) per ogni $x \in (a,b)$, e proviamo che allora $f$ è crescente in $(a,b)$. Prendiamo dunque due punti $x_1, x_2 \in (a,b)$ tali che $x_1 < x_2$, mostriamo che $f(x_1) < f(x_2)$. Infatti, applicando il teorema di Lagrange ad $f$ sull'intervallo $[x_1, x_2]$ abbiamo che esiste un $c \in (x_1, x_2)$ tale che
\[
    f'(c) = \frac{f(x_2) - f(x_1)}{x_2 - x_1}
\]
Poichè per ipotesi $f'(c) \geq 0$ e $x_2 - x_1 > 0$, ne segue che anche $f(x_2) - f(x_1) \geq 0$, cioè la tesi.\newline





\newpage
\subsection*{Teorema di Cauchy}
\textbf{enunciato}\newline
Siano $f,g : [a,b] \rightarrow \mathbb{R}$ due funzioni continue su $[a,b]$ e derivabili in $(a,b)$. Allora esiste almeno un punto $x_0$ interno ad $(a,b)$, tale che 
\[
    [f(b) -f(a)] g'(x_0) = [g(b) -g(a)] f'(x_0)
\]
\textbf{dimostrazione}\newline
Consideriamo la funzione ausiliaria $h$
\[
    h(x) = [f(b) -f(a)]g(x) - [g(b) - g(a)]f(x)
\]
e teniamo presente che $[f(b) - f(a)]$ e $[g(b) - g(a)]$ sono valori costanti e che $h(x)$ è continua su $[a,b]$ e derivabile su $(a,b)$, poichè differenza di due funzioni continue.\newline
Valutiamo $h(x)$  sugli estremi dell'intervallo $[a,b]$:
\[
    h(a) = [f(b)- f(a)]g(a) - [g(b)- g(a)] f(a) = f(b)g(a) - g(b)f(a) 
\]
\[
    h(b) = [f(b) - f(a)]g(b)- [g(b)- g(a)]f(b) = -f(a) g(b) + g(a)f(b)
\]
Dunque
\[
    h(a) = h(b)
\]
Si vede ora che $h(x)$ ha la stessa valutazione agli estremi e quindi soddisfa le ipotesi del teorema di Rolle, applichiamolo: esiste almeno un punto $x_0 \in (a,b)$ tale che 
\[
    h'(x_0) = 0
\]
Calcoliamone ora la derivata di $h(x)$
\[
    [f(b) - f(a)]g'(x) - [g(b) - g(a)]f'(x) = 0
\]
e valutandola nel punto $x_0$ fornitoci dal teorema di Rolle risulta che
\[
    [f(b)- f(a)]g'(x_0) = [g(b) - g(a)]f'(x_0)
\]






\newpage
\subsection*{Teorema derivata l'Hospital}
\textbf{enunciato (LIBRO)}\newline
Siano $f, g$ funzioni derivabili in un intervallo $(a,b)$ con $g, g' \neq 0$ in $(a,b)$. Se
\[
    \lim_{x\rightarrow a^+} f(x) = \lim_{x\rightarrow a^-} g(x) = 0 \;\; oppure \;\; \pm \infty
\]
e
\[
    \lim_{x\rightarrow a^+} \frac{f'(x)}{g'(x)} = L \in \mathbb{R}^*
\]
Allora
\[
    \lim_{x\rightarrow a^+} \frac{f(x)}{g(x)} =L
\]
Il teorema continua a valere se $a = -\infty$ oppure se si considera il limite per $x \rightarrow b^-$ (anziché per $x \rightarrow a^+$),con $b\leq+\infty$\newline
\newline
\textbf{dimostrazione (LIBRO)}\newline
Caso $f(x), g(x) \rightarrow 0$. Sia $x_n$ una successione tendente ad $a^+$, prolunghiamo per continuità $f$ e $g$ in $a$ ponendo $f(a) = g(a) = 0$. Allora
\[
    \frac{f(x_n)}{g(x_n)} = \frac{f(x_n) -f(a)}{g(x_n)-g(a)}
\]
Se applichiamo a $f,g$ separatemente il teorema di lagrange sull'intervallo $[a, x_n]$, otteniamo che l'ultimo quoziente scritto è uguale a:
\[
    \frac{f'(t_n)(x_n -a)}{g'(t_n^*)(x_n-a)} = \frac{f'(t_n)}{g'(t_n^*)}
\]
dove $t_n, t_n^*$ sono due punti opportuni che cadono nell'intervallo $(a,x_n)$. Poichè quando $x_n \rightarrow 0$ anche $t_n$ e $t_n^*\rightarrow 0$, sembra "ragionevole" che il limite del quoziente $f'/g'$ sia uguale al limite del quoziente $f/g$. Tuttavia questo non si può affermare rigorosamente, perchè le successioni $t_n, t_n^*$ sono a priori diverse tra loro. Per aggirare il problema occorre modificare leggermente l'argomentazione seguita. Riprendiamo dunque la dimostrazione della $\frac{f(x_n)}{g(x_n)} = \frac{f(x_n) -f(a)}{g(x_n)-g(a)}$, e definiamo
\[
    h(x) = f(x_n) g(x) - g(x_n) f(x)
\]
Notiamo che $h(a) = h(x_n) = 0$. La funzione $h$ soddisfa le ipotesi del teorema di lagrange sull'intervallo $[a,x_n]$, dunque esiste $t_n \in(a,x_n)$ tale che 
\[
    h'(t_n) = \frac{f(x_n) - h(a)}{x_n-a} = 0
\]
ovvero, calcolando
\[
    h'(x) = f(x_n)g'(x) -g(x_n) f'(x), \;\;\;\;\;\; f(x_n)g'(t_n)-g(x_n)f'(t_n) = 0
\]
Dunque per ogni $x_n$ esiste un punto $t_n \in (a,x_n)$ tale che 
\[
    \frac{f(x_n)}{g(x_n)} = \frac{f'(t_n)}{g'(t_n)}
\]
Per $n \rightarrow  \infty$, $t_n \rightarrow a^+$, perciò $\frac{f'(t_n)}{g'(t_n)} \rightarrow L$, e di conseguenza anche $\frac{f(x_n)}{g(x_n)} \rightarrow L$, che è quanto volevamo dimostrare.






\newpage
\subsection*{Formula di Taylor con resto secondo Peano}
\textbf{enunciato (LIBRO)}\newline
Sia $f: (a,b) \in \mathbb{R} \rightarrow  \mathbb{R}$ derivabile $n$ volte in $x_0 \in (a,b)$. Allora
\[
    f(x) = T_{n,x_0}(x) + o((x-x_0)^n) \;\;\;\; per x \rightarrow x_0
\]
dove
\[
    T_{n,x_0}(x) = \sum_{k=0}^{n}\frac{f^{(k)}(x_0)}{k!}(x-x_0)^n
\]
\textbf{dimostrazione (APPUNTI)}\newline
Dimostriamo per induzione su $n$. \newline
Per $n=1$. Se $f \in C^1((a,b))$, allora
\[
    f(x) = f(x_0) + f'(x_0)(x-x_0) + o (x-x_0)
\]
\[
    f(x) - f(x_0) - f'(x_0)(x-x_0) = 0(x-x_0)
\]
Che è vero se il rapporto incrementale tende a $0$, quindi verifichiamo:
\[
    \frac{f(x)-f(x_0)-f'(x_0)(x-x_0)}{x-x_0} = \frac{f(x)- f(x_0)}{x-x_0} - f'(x) \rightarrow 0
\]
Verificato.\newline
Assunto vero per $n-1$ verifichiamo per $n$. \newline
Essendo vera per $n-1$, abbiamo che per ogni funzione $g \in C^{n-1}((a,b))$, $\frac{g(x) - T_{n-1}^g(x)}{(x-x_0)^{n-1}} \rightarrow 0$. Per verificare per $n$ devo verificare se $f \in C^n((a,b))$, $\frac{f(x)-T_n^f(x)}{(x-x_0)^n} \rightarrow_? 0$. Per farlo ci avvaliamo del teorema di De l'Hopital:
\[
    \frac{[f(x) - T_n^f(x)]'}{[(x-x_0)^n]'} = \frac{f'(x)- [t_n^f(x)]'}{n(x-x_0)^{n-1}} = \frac{f'(x)-T_{n-1}^f(x)}{(x-x_0)^{n-1}} \cdot \frac{1}{n} \rightarrow 0
\]
Dove il termine 
\[
    \frac{f'(x)-T_{n-1}^f(x)}{(x-x_0)^{n-1}} \rightarrow 0
\]
per ipotesi di induzione.\newline
Torniam ora alla derivata
\[
    [t_n^f(x)]' = \left[f(x_0) + f'(x_0)(x-x_0) + \frac{f''(x_0)}{2!}(x-x_0)^2 + \dots + \frac{f^{(n)}(x_0)}{n!}(x-x_0)^n\right]'=
\]
\[
    =f'(x_0) + f''(x_0) + \dots + \frac{f^{(n)}(x_0)}{(n-1)!}(x-x_0)^{n-1} = T_{n-1}^{f'} \;\;\;\;\;\;f' \in C^{n-1}((a,b))
\]
Quindi anche 
\[
    \frac{f(x)-T_n^f(x)}{(x-x_0)^n} \rightarrow 0
\]






\newpage
\subsection*{Formula di Taylor con resto secondo Lagrange}
\textbf{enunciato (LIBRO)}\newline
Sia $f:[a,b] \rightarrow \mathbb{R}$ derivabile $n+1$ volte in $[a,b]$, e sia $x_0 \in [a,b]$. Allora esiste un punto $c$ compreso tra $x_0$ e $x$ tale che
\[
    f(x) = T_{n,x_0}(x) + \frac{f^{(n+1)(c)}}{(n+1)!}(x-x_0)^{n+1}
\]
dove
\[
    T_{n,x_0}(x) = \sum_{k=0}^{n}\frac{f^{(k)}(x_0)}{k!}(x-x_0)^n
\]
\textbf{dimostrazione (LIBRO)}\newline
Proviamo il teorema per $n=1$. Ponendo per comodità $x_0 = a, x =b$ l'enunciato diventa: se $f : [a,b] \rightarrow \mathbb{R}$ è derivabile $2$ volte in $[a,b]$, allora esiste un punto $c \in [a,b]$ tale che
\[
    f(b)=f(a) +f'(a)(b-a) + \frac{1}{2}f''(c)(b-a)^2
\]
Poniamo $f(b)- \{f(a) + f'(a)(b.-a)\} = k(b-a)^2$ e cerchiamo di determinare la forma di $k$.\newline
Consideriamo la funzione
\[
    g(x) = f(b) - f(x) -f'(x)(b-x) - k(b-x)^2
\]
e applichiamo ad essa il teorema del valor medio. Poichèé $g(b) = g(a) = 0$ (la seconda uguaglianza segue dalla definizione di $k$) si trova che esiste $c \in (a,b)$ tale che
\[
    0 = \frac{g(b) - g(a)}{b-a} = g'(c)
\]
Ma $g'(x) = - f'(x) - f''(x)(b-x) + f'(x) + 2k(b-x) = (b-x)[2k-f''(x)]$ e quindi $g'(x) = 0$ implica, essendo $c \neq b$:
\[
    k = \frac{1}{2}f''(c)
\]
Con lo stesso metodo possiamo ampliare al caso $n$ qualsiasi.\newline
Definiamo
\[
    g(x) = f(b) - \sum_{j=0}^{n}\frac{f^{(j)}(x)(b-x)^j}{j!}(b-x)^{n+1}
\]
con $k$ definito implicatamente dall'identità
\[
    f(b)-T_{n,a}(b) = k (b-a)^{n+1}.
\]
Ora si procede così:
\begin{enumerate}
    \item Si verifica che $g(b) = g(a) = 0$:
    \[
        g(b) = f(b) - f(b) = 0 \quad \quad \quad g(a) = f(b) - T_{n,a}(b) - k(b-a)^{n+1} = 0 \;\;\;\;\; per \;\; definizione \;\; di \;\; k.
    \]
    \item Si applica il teorema di lagrange a $g$ in $[a,b]$, e si mostra che l'affermazione "esiste $c \in (a,b)$ tale che $g'(c) = 0$" è esattamente la tesi. Cominciamo a calcolare $g'$:
    \[
        g'(x)= \sum_{j=0}^{n}\frac{f^{(j+1)}(x)(b-x)^j}{j!} + \sum_{j=1}^{n}\frac{f^{(j)}(x)(b-x)^{(j-1)}}{(j-1)!} + k(n+1)(b-x)^n =
    \]
    \[
        = - \sum_{j=0}^{n} \frac{f^{(j+1)}(x)(b-x)^j}{j!}+ \sum_{j=0}^{n-1} \frac{f^{(j+1)}(x)(b-x)^{j}}{j!} + k(n+1)(b-x)^n =
    \]
    \[
        - \frac{f^{(n+1)}(x)(b-x)^n}{n!}+ k(n+1)(b-x)^n
    \]
    dove nel secondo passaggio abbiamo eseguito una traslazione di indici nella sommatoria).\newline
\end{enumerate}
Allora $g'(c) = 0$ significa:
\[
    -\frac{f^{(n+1)(c)(b-c)^n}}{n!}+ k(n+1)(b-c)^n = 0
\]
ossia
\[
    k = \frac{f^{(n+1)(c)}}{(n+1)!}
\]
che combinata col risultato precedente ($f(b)-T_{n,a}(b) = k (b-a)^{n+1}$) dimostra la tesi





\newpage
\subsection*{Primo teorema fondamentale del calcolo integrale}
\textbf{enunciato (APPUNTI)}\newline
Sia $f : [a,b] \in \mathbb{R} \rightarrow \mathbb{R}$. Se $f$ è Riemann-integrabile su $[a,b]$ e sia $G$ una primitiva di $f$ su $[a,b]$, allora
\[
    \int_{a}^{b}f(x) dx = G(b)-G(a)
\]
\textbf{dimostrazione (LIBRO = APPUNTI)}\newline
Siano $x_0 (=a), x_1, \dots, x_n (=b)$ punti che suddividono l'intervallo $[a,b]$ in modo equilibrato, allora aggiungendo e togliendo $G(x_j)$ per $j = 1,2,\dots,n-1$ si ha:
\[
    G(b) - G(a) = G(x_n) - G(x_0) =
\]
\[
    = [G(x_n) - G(x_{n-1})] + [G(x_{n-1}) - G(x_{n-2})] + \dots + [G(x_2) - G(x_1)] + [G(x_1) - G(x_0)] =
\]
\[
    = \sum_{j=1}^{n} [G(x_j) - G(x_{j-1})]
\]
Applichiamo ora il teorema di Lagrange alla funzione $G(x)$ su ciascuno degli intervalli $[x_{j-1}, x_j]$. Esiste allora $c_j \in (x_{j-1}, x_j)$ tale che
\[
    G(x_j) - G(x_{j-1}) =(x_j - x_{j-1})G'(c_j) = (x_j - x_{j-1})f(c_j)
\]
Perchè per ipotesi $G$ è una primitiva di $f$ e perciò $G'(c_j) = f(c_j)$. Ne segue che 
\[
    G(b)-G(a) = \sum_{j=1}^{n}(x_j - x_{j-1})f(c_j) = S_n
\]
dove $S_n$ è una somma $n$-esima di Cauchy-Riemman di $f$. L'identità scritta vale per ogni $n$, possiamo allora far tendere $n$ a $+\infty$, trovando 
\[
    G(b) - G(a) = \int_{a}^{b}f(x)dx
\]
Si osservi che questa dimostrazione mostra che una certa (non tutte quelle possibili!) somme di Cachy-Riemann di $f$ tende a $G(b) - G(a)$. Poichè però sappiamo già che $f$ è integrabile, in quanto continua, questo è sufficiente a concludere che ogni altra successione di Cauchy-Riemann converge allo stesso limite, e quindi vale la formula enunciata.





\newpage
\subsection*{Teorema del valor medio integrale}
\textbf{enunciato (LIBRO)}\newline
Sia $f:[a,b] \rightarrow  \mathbb{R}$ continua. Allora esiste $c \in [a,b]$ tale che
\[
    \frac{1}{b-a}\int_{a}^{b}f(x) dx =f(c)
\]
\textbf{dimostrazione (LIBRO)}\newline
Essendo $f$ continua in $[a,b]$, per Weierstrass, essa è dotata di massimo($=M$) e minimo ($=m$). Dalla proprietà di monotonia dell'integrale si ha
\[
    m = \frac{1}{b-a} \int_{a}^{b} m dx \leq \frac{1}{b-a} \int_{a}^{b}f(x)dx \leq \frac{1}{b-a} \int_{a}^{b}M dx = M
\]
Essendo quindi il valore $\frac{1}{b-a} \int_{a}^{b}f(x)dx$ è compreso fra il minimo e il massimo, per il teorema di Darboux, tale valore è uguale a $f(c)$ per qualche $c \in[a,b]$.







\newpage
\subsection*{Secondo teorema fondamentale del calcolo integrale}
\textbf{enunciato (LIBRO)}
Sia $f : [a,b] \in \mathbb{R} \rightarrow \mathbb{R}$ una funzione integrabile (in senso proprio o generalizzato), sia $x_0 \in [a,b]$ e sia
\[
    F(x) = \int_{x_0}^{x} f(t) dt
\]
Allora
\begin{itemize}
    \item La funzione $F$ è continua in $[a,b]$
    \item Se inoltre $f$ è continua in $[a,b]$, allora $F$ è derivabile in $[a,b]$, e vale \[
        F'(x) = f(x)
    \]
    per ogni $x \in [a,b]$
\end{itemize}
\textbf{dimostrazione (LIBRO)}\newline
Proviamo il primo punto prima nell'ipotesi in cui $f$ è integrabile in senso proprio, e quindi, in particolare, è limitata. Perciò esiste $k>0$ per cui è
\[
    |f(t)| \leq k \;\;\;\;\forall\;t \in[a,b]
\]
Per provare che $F$ è continua in un generico punto $\bar{x} \in[a,b]$, consideriamo ora:
\[
    |F(\bar{x}+h) - F(\bar{x})| =
\]
\[
    = \left|\int_{x_0}^{x_0+h}f(t)dt - \int_{x_0}^{\bar{x}}f(t)dt\right| =
\]

\[
    = \left| \int_{\bar{x}}^{\bar{x}+h}f(t)dt \right|  \leq \left| \int_{\bar{x}}^{\bar{x}+h}|f(t)|dt \right| \leq
\]
\[
    \leq \left|\int_{\bar{x}}^{\bar{x}+h}kdt\right| = k|h|.
\]
(procedimenti ottenuti con l'additività dell'integrale e la disuguaglianza del valore assoluto). Perciò per $h \rightarrow 0$ si ha $F(\bar{x}+h) \rightarrow  F(\bar{x})$, e $F$ è continua in $\bar{x}$\newline
Supponiamo ora che $f$ sia integrabile solo in senso generalizzato, e che $\bar{x}$ sia proprio un punto in cui $f$ è illimitata. Abbiamo ancora che:
\[
    F(\bar{x}+h) -F(\bar{x}) = \int_{\bar{x}}^{\bar{x}+h}f(t) dt
\]
Scriviamo ora:
\[
    \int_{\bar{x}}^{b}f(t) dt = \int_{\bar{x}}^{\bar{x}+h}f(t) dt + \int_{\bar{x}+h}^{b}f(t) dt
\]
Poichè per ipotesi $f$ è integrabile in senso generalizzato, si ha, per definizione:
\[
    \int_{\bar{x}+h}^{b} f(t)dt \rightarrow \int_{\bar{x}}^{b}f(t) dt
\]
per $h \rightarrow 0^+$.\newline
E quindi per differenza 
\[
    \int_{\bar{x}}^{\bar{x}+h} f(t) dt \rightarrow 0
\]
per $h \rightarrow 0^+$. \newline
Ossia $F(\bar{x}+h) \rightarrow  f(\bar{x})$ per $h \rightarrow 0^+$. (Si ragiona in modo analogo per $0^-$)\newline
\newline
Proviamo ora il secondo punto, partendo dall'uguaglianza, già ottenuta, 
\[
    F(\bar{x}+h) \rightarrow F(\bar{x}) = \int_{\bar{x}}^{\bar{x}+h}f(t)dt
\]
Poichè ora per ipotesi $f$ è continua, per il teorema della media si ha 
\[
    \int_{\bar{x}}^{\bar{x}+h}f(t) dt = hf(x_h)
\]
per qualche $x_h \in[\bar{x},\bar{x}+h]$ (supponendo, per fissare le idee, $h>0$). Ne segue 
\[
    \frac{F(\bar{x}+h)- F(\bar{x})}{h} = f(x_h)
\]
Per $h \rightarrow 0$ si avrà $x_h \rightarrow \bar{x}$ e quindi, ancora per la continuità di $f$, $f(x_h) \rightarrow f(\bar{x})$. Pertanto esiste 
\[
    \lim_{h\rightarrow 0} \frac{F(\bar{x}+h) - F(\bar{x})}{h}= f(\bar{x})
\]
e il teorema è dimostrato.






\newpage
\subsection*{Condizione necessaria per la convergenza delle serie}\textbf{enunciato}\newline
\textbf{dimostrazione}\newline






\newpage
\subsection*{Criterio del rapporto per la convergenza delle serie a termini positivi}
\textbf{enunciato}\newline
\textbf{dimostrazione}\newline





\newpage
\subsection*{Criterio del confronto per la convergenza di una serie a termini positivi}
\textbf{enunciato}\newline
\textbf{dimostrazione}\newline






\newpage
\subsection*{Criterio della radice per la convergenza della serie a termini positivi}
\textbf{enunciato}\newline
\textbf{dimostrazione}\newline






\newpage
\subsection*{Giustificazione della formula di  con l’esponenziale complesso}
\textbf{enunciato}\newline
\textbf{dimostrazione}\newline






\newpage
\end{document}