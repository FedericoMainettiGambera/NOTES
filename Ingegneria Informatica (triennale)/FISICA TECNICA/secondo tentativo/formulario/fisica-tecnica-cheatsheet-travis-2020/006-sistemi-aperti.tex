\section{Sistemi aperti}

\subsection{Bilancio di massa}
\[ \dv{m}{t} = \sum_i \dot{m}_i^\sla \]

\subsubsection{Equazione di continuità}
\[ \dot{m} = \rho w \Omega \]
$\rho$: massa volumica \newline
$w$: velocità media \newline
$\Omega$: area della sezione di passaggio.

\subsection{Bilancio di energia}
\[ \dv{E}{t} = \sum_i \dot{E}_i^\sla = \dot{Q}^{\leftarrow } - \dot{L}^{\rightarrow } + \sum_{k} \dot{E}_{m,k}^{\leftarrow } + sorgenti \]
dove \newline
\subsubsection{calore scambiato}
calore scambiato attraverso le sezioni non attraversate dalla massa (il calore scambiato attraverso le sezioni di ingresso e uscita è trascurabile)

\subsubsection{lavoro d'elica}
$-\dot{L}_e^{\rightarrow }$ lavoro scambiato nelle sezioni non attraversate dalla massa

\subsubsection{lavoro di pulsione} 
$\dot{L}_P^{\leftarrow }$ (quello scambiato nell'ingresso e nell'uscita, è il lavoro necessario per immettere nel sistema la massa che il sistema scambia con l’estern) 
\[ L_P^\sla = - \int^{V}_{V+V_m} P\dd{V} = PV_m = m_i P v_i \]

\subsubsection{Energia associata al trasporto di massa}
\[ E_m = \sum_i m_i^\sla \qty(u + gz + \frac{w^2}{2}) \]

\subsubsection{Sorgenti}
Ogni altra fonte di energia (dissipata o aggiunta) va inserita nel bilancio, varia da esercizio a esercizio.

\subsection{Bilancio energetico}
\[ \dv{E}{t} = \dot{E}_m + \dot{Q}^\sla - \dot{L_e}^\sra + \sum_{k} \left(\dot{L}_P^{\leftarrow }\right)_ k = \]
\[= \sum_i \dot{m}_i^\sla \qty(h_i + gz_i + \frac{w_i^2}{2}) + \dot{Q}^\sla - \dot{L_e}^\sra\]

\subsection{Bilancio di entropia}
\[ \dv{S}{t} = \sum_i \dot{m}_i^\sla s_i + \dot{S}_{Q^\sla} + \dot{S}_{irr} \]

\subsection{Regime stazionario}
\[ \dot{m}_{ingresso}^\sla = -\dot{m}_{uscita}^\sla \]

\[ \dot{m}^\sla\qty[(h_i - h_u) + g(z_i-z_u) + \frac{(w_i^2-w_u^2)}{2}] + \dot{Q}^\sla - \dot{L}_e^\sra = 0 \]

\[ \dot{m}^\sla (s_i - s_u) + \dot{S}_{Q^\sla} + \dot{S}_{irr} = 0 \]

\subsection{Regime non stazionario}
si scrivono i bilanci, si semplificano, e poi si integrano
\[
    \frac{dM}{dt} = M_f - M_i = \dots \text{formula integrata}
\]
\[
    \frac{dE}{dt} = U_f - U_i + Mgz_f - Mgz_i + M \frac{w_f^2}{2} - M \frac{w_i^2}{2}= \dots \text{formula integrata}
\]
\[
    \frac{dS}{dt} = S_f - S_i = \dots \text{formula integrata}
\]
dove con $\dots$ si intende che anche la parte dopo l'uguale nei bilanci si deve integrare

\section{Macchina aperta}
\subsubsection{Turbina, compressore e pompa}
\[ \dot{m}^\sla (h_i - h_u) - \dot{L}_e^\sra = 0 \]
\[ \dot{m}^\sla (s_i - s_u) + \dot{S}_{irr} = 0 \]

\subsubsection{Scambiatore di calore}
\[ \dot{m}^\sla (h_i - h_u) + \dot{Q}^\sla = 0 \]
\[ \dot{m}^\sla (s_i - s_u) + \dot{S}_{Q^\sla} + \dot{S}_{irr} = 0 \]

\subsubsection{Diffusore (w↓) e ugello (w↑)}
\[  (h_i - h_u) + \frac{(w_i^2 - w_u^2)}{2} = 0 \]
\[ \dot{m}^\sla (s_i - s_u) + \dot{S}_{irr} = 0 \]

\subsubsection{Valvola di laminazione}
\[  (h_i - h_u) = 0 \]
\[ \dot{m}^\sla (s_i - s_u) + \dot{S}_{irr} = 0 \]

\subsection{Rendimento isoentropico}
\subsubsection{Turbina}
\[ \eta_{is,turbina} = \frac{\dot{L}_{reale}^\sra}{\dot{L}_{ideale}^\sra} = \frac{h_1 - h_{2,reale}}{h_1 - h_{2,ideale}} \]

\subsubsection{Compressore}
\[ \eta_{is,compressore} = \frac{\dot{L}_{ideale}^\sra}{\dot{L}_{reale}^\sra} = \frac{h_1 - h_{2,ideale}}{h_1 - h_{2,reale}} \]

\subsection{Irreversibilità interna associata al moto del fluido}
..dopo vari passaggi si ottiene che
\[ -\delta l_e^\sra = v\dd{P} + T\dd{s_{irr}} \]
Integrando fra la sezione di ingresso e quella di uscita:
\[ l_e^\sra = -\int_i^u v\dd{P} -\int_i^u T\dd{s_{irr}} \]
dove il secondo termine è l'energia dissipata per irreversibilità interna.

In generale l’irreversibilità interna associata al moto si traduce in una spesa energetica per movimentare il fluido che può essere espressa come
\[
    \dot{L} = \dot{V} \Delta P 
\]
dove $\dot{L}$ è la potenza meccanica necessaria a movimentare il fluido, $\dot{V} = w \Omega$ è la portata volumetrica e $\Delta P$ sono le perdite di carico originate da attrito, cambi di direzione, ostacoli, etc.\newline

Le perdite di carico \textbf{concentrate} sono $\Delta P = K \rho \frac{w^2}{2}$ (con $K$ costante di proporzionalità, vedi slide).

Le perdite di carico \textbf{distribuite} sono $\Delta P = f \frac{L}{D} \rho \frac{w^2}{2}$ (vedi slide per i singoli termini).
