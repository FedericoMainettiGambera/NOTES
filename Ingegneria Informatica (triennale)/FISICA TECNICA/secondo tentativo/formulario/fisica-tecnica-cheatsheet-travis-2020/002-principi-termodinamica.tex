\section{Principi della termodinamica}

\subsection{Principi di conservazione}
\begin{itemize}
    \item Conserazione della \textbf{massa};
    \item Conserazione della \textbf{energia} (I principio);
    \item Conserazione della \textbf{entropia} (II principio);
\end{itemize}

\subsection{Primo principio della termodinamica per sistemi chiusi}
Per un sistema semplice all’equilibrio è definita una proprietà intrinseca (funzione di stato) detta energia interna $U$ la cui variazione è il risultato di interazioni del sistema con l’ambiente esterno.

L'energia che è immagazinata in un sistema e che non va a cambiare né l'energia cinetica del centro di massa, né quella potenziale (e neanche l'energia elastica o chimica o elettrica) è chiamata energia interna.

\[\Delta U = Q^\sla - L^\sra \]

\begin{tabular}{ll}
    $U$ & energia interna del sistema \\
    $Q^\sla$ & calore entrante nel sistema\\
    $L^\sra$ & lavoro uscente dal sistema\\
\end{tabular}

L'energia interna è una grandezza estensiva (additiva):
\[U_z = U_1 + U_2 + \ldots\]

Per una trasformazione ciclica:
\[\Delta U_{\text{ciclo}} = 0 \]

Per un sistema isolato:
\[\Delta U_{\text{isolato}} = 0\]

Primo principio in forma differenziale:
\[\dd{u} = \var{q^\sla} - \var{l^\sra}\]

\subsection{Secondo principio della termodinamica per sistemi chiusi}

Per un sistema all'equilibrio esiste una funzione di stato detta \textbf{entropia} $S$ la cui variazione per una trasformazione \emph{reversibile} è data da:
\[\Delta S = \int \frac{\var{Q_{rev}^\sla}}{T}\]

L'entropia è una grandezza estensiva (additiva):
\[
    S_z = S_1 + S_2 + \ldots \qquad \Delta S_z = \Delta S_1 + \Delta S_2 + \ldots
\]

La variazione di entropia totale di un sistema isolato è sempre maggiore di zero e tende a zero con il tendere dei processi alla reversibilità:
\[
    \Delta S_{\text{isolato}} \geq 0
\]  

In un sistema chiuso il \textbf{bilancio entropico} è dato da:
\[
    \Delta S = S_Q^\sla + S_{irr} \qquad \text{$S_Q^\sla$ dovuta dallo scmabio di calore $Q$}
\]
\[
    S_{irr} \ge 0 \qquad \text{segno}~S_Q^\sla = \text{segno}~Q^\sla
\]