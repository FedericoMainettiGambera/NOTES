\section{Trasformazioni}

\subsection{Lavoro termodinamico}

\begin{align*}
    &\delta L^\sra = PA\dd{s} = P\dd{V} \; \text{(sistema cilindro-pistone)}\\
    &\delta l^\sra = P\dd{v} \; (\text{grandezze specifiche})\\
    &l^\sra = \int_i^f P\dd{v} \; (\text{calcolabile conoscendo $P = P(v)$})
\end{align*}
Il lavoro (specifico) è pari all'area sottesa (integrale) dal grafico della trasformazione nel piano P-v.\newline

Il lavoro reversibile è maggiore del lavoro irreversibile.\newline

Il lavoro dipende dal percorso: \textbf{non} è funzione di stato.

\subsection{Calori specifici}

\begin{align*}
    \text{Capacità termica} &\qquad C_x = \left(\frac{\delta Q^\sla}{\dd{T}}\right)_x \\
    \text{Calore specifico} &\qquad c_x = \frac{1}{M} \left(\frac{\delta Q^\sla}{\dd{T}}\right)_x
\end{align*}

\begin{align*}
    \text{A pressione cost.} &\qquad c_P = \frac{1}{M} \left(\frac{\delta Q^\sla}{\dd{T}}\right)_P = \left(\frac{\delta q^\sla}{\dd{T}}\right)_P \\
    \text{A volume cost.} &\qquad c_v = \frac{1}{M} \left(\frac{\delta Q^\sla}{\dd{T}}\right)_v = \left(\frac{\delta q^\sla}{\dd{T}}\right)_v
\end{align*}

Inoltre si ha che
\begin{align*}
    \text{A pressione cost.} &\qquad c_P = \qty(\pdv{h}{T})_P\\
    \text{A volume cost.} &\qquad c_v = \qty(\pdv{u}{T})_v
\end{align*}

\subsection{Entalpia}

L'entalpia è una funzione di stato.
\[ h = u + Pv \]
\[ \dd{h} = \dd{u} + v\dd{P} + P\dd{v} = \delta q^\sla + v\dd{P} \]
\[ \delta q^\sla = \dd{h} - v\dd{P} \]
