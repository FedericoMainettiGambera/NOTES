\section{Cicli a vapore}
\subsection{Ciclo Carnot a vapore}
\begin{tikzpicture}[thick,>=stealth']
    \coordinate (O) at (0,0);
    \draw[->] (0,0) -- (3.5,0) coordinate[label = {below:$s$}] (xmax);
    \draw[->] (0,0) -- (0,3.5) coordinate[label = {left:$T$}] (ymax);

    \gaussiana{-1.2}{1.2}{0}{0.7}

    \draw (1,1) node[below left] {1} -- (1,1.8) node[above left] {2} -- (2.42,1.8) node [above right] {3} -- (2.42,1) node [below right] {4} -- cycle;

\end{tikzpicture}

Vantaggi:
\begin{itemize}
    \item L'isobara nel bifase è anche isoterma (meno irreversibilità);
    \item La transizione di fase aumenta l'energia specifica scambiata lungo le isoterme;
\end{itemize}

Svantaggi:
\begin{itemize}
    \item Compressione $1-2$ di un bifase difficile da realizzare e soggetta a molte irreversibilità;
    \item L'espansione $3-4$ conviene se $x_4<0.9$.
\end{itemize}
