\section{Macchina termodinamica}
La macchina termodinamica è un sistema termodinamico composto ed isolato che nel caso più semplice è realizzato da
\begin{itemize}
\item due serbatoi di calore
\item un serbatoio di lavoro
\item una macchina ciclica che è in grado di interagire con continuità con i serbatoi di calore e lavoro
\end{itemize}

\begin{description}
\item[Serbatoio di calore]
sistema termodinamico che scambia solo calore con l'esterno senza alterare il suo stato interno;
gli scambi avvengono con trasformazioni quasi-statiche internamente reversibili.
\item[Serbatoio di lavoro]
sistema termodinamico che scambia solo lavoro con l'esterno senza alterare il suo stato interno;
gli scambi avvengono con trasformazioni quasi-statiche internamente reversibili.
\end{description}

\subsubsection{Risoluzione problemi macchine termiche}
Bisogna impostare e risolvere il sistema contente le equazioni di bilancio
\[
    \begin{cases}
    \Delta U_Z = 0 \\
    \Delta S_Z = S_{irr} \\
    \end{cases} \qquad \begin{cases}
        \Delta U_C + \Delta U_F + \Delta U_{SL} + \Delta U_M = 0\\
        \Delta S_C + \Delta S_F + \Delta_{SL} + \Delta S_M = S_{irr}
    \end{cases}
\]

Per macchina motrice con masse infinite:
\[
    \begin{cases}
        -Q_C + Q_F + L = 0 \\
        -\frac{Q_C}{T_C} + \frac{Q_F}{T_F} = S_{irr} \\
    \end{cases}
\]

Per macchina operatrice con masse infinite:
\[
    \begin{cases}
        Q_C - Q_F - L = 0 \\
        \frac{Q_C}{T_C} - \frac{Q_F}{T_F} = S_{irr} \\
    \end{cases}
\]

\subsection{Rendimenti macchine termiche}
{\renewcommand{\arraystretch}{1.5}
\begin{tabular}{p{3.5cm}r}
\textbf{Macchina motrice} & $\eta = \frac{L}{Q_C} \quad \eta_{II} = \frac{\eta}{\eta_{rev}} = \frac{L}{L_{rev}} $ \\
→ serbatoi a massa infinita & $\eta = 1 - \frac{T_F}{T_C} - \frac{T_F}{Q_C}S_{irr}$ \\
\phantom{→}$\put(3,2){\line(0,1){5}}\to$ reversibile ($S_{irr} = 0$) & $\eta_{rev} = 1 - \frac{T_F}{T_C}$ \\
\\
\textbf{Macchina operatrice} & (serbatoi a temp. cost.) \\
→ frigorifera & $\varepsilon_f = \frac{Q_F}{L} \quad \eta_{II} = \frac{\varepsilon}{\varepsilon_{rev}} = \frac{L_{rev}}{L}$ \\
\phantom{→}$\put(3,2){\line(0,1){5}}\to$ reversibile & $\varepsilon_{f,rev} = \frac{T_F}{T_C - T_F}$ \\
→ pompa di calore & $\varepsilon_{pdc} = \frac{Q_C}{L}$ \\
\phantom{→}$\put(3,2){\line(0,1){5}}\to$ reversibile & $\varepsilon_{pdc,rev} = \frac{T_C}{T_C - T_F}$ \\
\end{tabular}
}
