\subsection{Ciclo Rankine con surriscaldamento}
\begin{tikzpicture}[thick,>=stealth']
    \coordinate (O) at (0,0);
    \draw[->] (0,0) -- (3.5,0) coordinate[label = {below:$s$}] (xmax);
    \draw[->] (0,0) -- (0,3.5) coordinate[label = {left:$T$}] (ymax);

    \gaussiana{-1.2}{1.2}{0}{0.7}

    \draw (0.54,1) node[below] {1} -- (0.54,1.3) node[left]{2} -- (0.98,1.8) node[above left] {3} -- (2.42,1.8) node [below left] {4} parabola (2.8,3) node[right]{5} -- (2.8,1) node [below] {6} -- cycle;
\end{tikzpicture}

\begin{tabular}{p{2.5cm}l}
Trasformazione 1-2: & compressione isoentropica (pompa)\\
Trasformazione 2-5: & riscaldamento isobaro (bruciatore)\\
Trasformazione 5-6: & espansione isoentropica (turbina)\\
Trasformazione 6-1: & raffreddamento isobaro (condensatore)\\
\end{tabular}

Rendimento ciclo Rankine:
\[ \eta = 1 - \frac{\dot{Q}_F}{\dot{Q}_C} = 1 - \frac{\dot{m}(h_6-h_1)}{\dot{m}(h_5-h_2)} = 1 - \frac{h_6-h_1}{h_5-h_2} \]
