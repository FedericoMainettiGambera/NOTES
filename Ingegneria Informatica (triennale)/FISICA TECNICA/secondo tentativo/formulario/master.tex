\documentclass[a4paper, 9pt]{article}
\title{COMPUTER SCIENCE AND ENGINEERING}
  
\author{Federico Mainetti Gambera}
\usepackage{amsmath}
\usepackage{amssymb}
\usepackage{graphicx}
\usepackage[italian]{babel}
\usepackage{import}
\usepackage{xifthen}
\usepackage{pdfpages}
\usepackage{transparent}
\usepackage{xcolor}
\usepackage{cancel}
\usepackage[a4paper,left=35mm,top=26mm,right=26mm,bottom=15mm]{geometry}
\usepackage{color}
\usepackage{tcolorbox}
\usepackage{hyperref}
\usepackage{makeidx}
\usepackage{soul}
\usepackage{multicol}
\usepackage{pdflscape}
\makeindex
\definecolor{lightgray}{gray}{0.75}
\renewcommand{\familydefault}{\sfdefault}
\newenvironment{rcases}
  {\left.\begin{aligned}}
  {\end{aligned}\right\rbrace}
\newcommand{\incfig}[1]{%
    \def\svgwidth{\columnwidth}
    \import{../images/}{#1.pdf_tex}
}
\begin{document}
    \maketitle
    \tableofcontents
    \newpage
    \section{ARGOMENTI}

    \textbf{Lezione 1: INTRODUZIONE}\newline
    \url{../lezioni/L01-Introduzione.pdf}
    \begin{itemize}
        \item \textbf{Sistema termodinamico}: \textit{contorno, ambiente, serbatoio, sistema composto, sistema mono e pluricomponenti, sistema semplice}
        \item \textbf{Stato di equilibrio}: \textit{grandezze intensive ed estensive, legge di Duhem, regola di Gibbs, equazione di stato}
        \item \textbf{Tipologie di sistemi termodinamici}: \textit{contorno del sistema, sistema chiuso e aperto}
        \item \textbf{Trasformazioni termodinamiche}: \textit{internamente reversibile, reversibile, irreversibile, ciclica, elementare}
        \item \textbf{Equazione di stato}: \textit{gas ideali, costante R, gas reali, liquidi e solidi}
    \end{itemize}
    \ \newline
    \newline
    \textbf{Lezione 2: PRINCIPI DI CONSERVAZIONE}\newline
    \url{../lezioni/L02-Principi+di+conservazione.pdf}
    \begin{itemize}
        \item \textbf{Primo principio della termodinamica per sistemi chiusi}: \textit{formulazione assiomatica, lavoro L, calore Q, proprietà e casi particolari, formulazione classica, esperienze di Joule}
        \item \textbf{Secondo principio della termodinamica per sistemi chiusi}: \textit{formulazione assiomatica, entropia S, proprietà e casi particolari, bilancio di entropia}
        \item \textbf{Osservazioni sul primo e secondo principio della termodinamica}
    \end{itemize}
    \ \newline
    \newline
    \textbf{Lezione 3: TRASFORMAZIONI}\newline
    \url{../lezioni/L03-Trasformazioni.pdf}
    \begin{itemize}
        \item \textbf{Lavoro termodinamico}: \textit{calcolo, trasformazione reversibile vs irreversibile, funzione di stato, lavoro in un ciclo}
        \item \textbf{Calori specifici}: \textit{capacità termica, calore specifico, calori specifici a pressione costante e a volume costante e per i gas ideali e perfetti e per i liquidi incomprimibili ideali e perfetti, entalpia, relazione di Mayer}
        \item \textbf{Trasformazioni politropiche}: \textit{indice della politropica, equazione della politropica, politropiche per i as perfetti, trasformazioni elementari, lavoro scambiato in una politropica}
        \item \textbf{Diagramma T-S}
        \item \textbf{Calcolo delle grandezze termodinamiche}: \textit{tabella gas perfetti, variazione di entropia per i gas ideali e perfetti e per liquidi incomprimibili perfetti, ..., note aggiuntive}
    \end{itemize}
    \ \newline
    \newline
    \textbf{Lezione 4: SISTEMI BIFASE}\newline
    \url{../lezioni/L04-Sistemi+bifase.pdf}
    \begin{itemize}
        \item \textbf{Sistema eterogeneo}: \textit{omogeneo vs eterogeneo, monocomponente vs multicomponente, grndezze estensive in sistemi eterogenei bifase, frazione massica, regola di Gibbs, transizione di fase}
        \item \textbf{Sistema eterogeneo monocomponente}: \textit{nomenclatura}
        \item \textbf{Diagramma di stato P-v-T}
        \item \textbf{Proprietà termodinamiche dei sistemi eterogenei}: \textit{entalpia di transizione di fase, titoli}
        \item \textbf{Utilizzo delle tabelle}: \textit{Tabella di saturazione in pressione e in temperatura, tabella del vapore surriscaldato, interpolazione lineare, interpolazione bilineare, formule pre l'acqua sottoraffreddata}
        \item \textbf{Relazioni semplificate vicino al punto triplo per l'acqua}
    \end{itemize}
    \ \newline
    \newline
    \textbf{Lezione 5: MACCHINE TERMODINAMICHE}\newline
    \url{../lezioni/L05-Macchine+termodinamiche.pdf}
    \begin{itemize}
        \item \textbf{Macchine termodinamiche}: \textit{Servatoio di calore superiore e inferiore, serbatoio di lavoro, macchina motrice e operatrice}
        \item \textbf{Macchina motrice}: \textit{bilancio macchina motrice, rendimento}
        \item \textbf{Macchina operatrice}: \textit{bilancio macchina operatrice, efficienza o COP}
        \item \textbf{Macchina motrice con serbatoio caldo a massa finita
        contenente liquido incomprimibile perfetto}
        \item \textbf{Rendimento di secondo principio}
    \end{itemize}
    \ \newline
    \newline
    \textbf{Lezione 6: SISTEMI APERTI}\newline
    \url{../lezioni/L06-Sistemi+aperti.pdf}
    \newline
    \newline
    \textbf{Lezione 7: CICLI A GAS}\newline
    \url{../lezioni/L07-Cicli+a+gas.pdf}
    \newline
    \newline
    \textbf{Lezione 8: CICLI A VAPORE}\newline
    \url{../lezioni/L08-Cicli+a+vapore.pdf}
    \newline
    \newline
    \textbf{Lezione 9: TRASMISSIONE DEL CALORE}\newline
    \url{../lezioni/L09-Trasmissione+del+calore.pdf}
    \newline
    \newline
    \textbf{Lezione 10: CONDUZIONE}\newline
    \url{../lezioni/L10-Conduzione.pdf}
    \newline
    \newline
    \textbf{Lezione 11: CONVEZIONE}\newline
    \url{../lezioni/L11-Convezione.pdf}
    \newline
    \newline
    \textbf{Lezione 12: IRRAGIAMENTO}\newline
    \url{../lezioni/L12-Irraggiamento.pdf}
    \newpage
    \section{ESERCITAZIONI}
    \begin{tabular}{llll}
        \textbf{Esercitazione} & \textbf{Numero} & \textbf{First try} & \textbf{Soluzione}\\
        \hline
        1 & 1 & \textcolor{green}{corretto} & esercitazione\\
        1 & 2 & \textcolor{red}{sbagliato} & no\\
        1 & 3 & \textcolor{red}{sbagliato} & cercare "rame" su telegram\\
        1 & 4 & \textcolor{green}{corretto} & tutorato 1\\
        1 & 5 & \textcolor{red}{sbagliato} importante & esercitazione\\
        1 & 6 & \textcolor{green}{corretto} & esercitazione\\
        1 & 7 & \textcolor{green}{corretto} & esercitazione\\
        \hline
        2 & 1 & \textcolor{green}{corretto} & no\\
        2 & 2 & \textcolor{green}{corretto} & no\\
        2 & 3 & \textcolor{green}{corretto} & esercitazione\\
        2 & 4 & \textcolor{green}{corretto} & esercitazione\\ 
        2 & 5 & \textcolor{green}{corretto} & esercitazione\\
        2 & 6 & \textcolor{green}{corretto} & tutorato 1\\
        \hline
        3 & 1 & \textcolor{green}{corretto}: (1,2,3); \textcolor{red}{sbagliato}: (6) importante & esercitazione\\
        3 & 2 & & \\
        3 & 3 & \textcolor{red}{sbagliato} & esercitazione\\
        3 & 4 & & \\
        3 & 5 & & \\
        3 & 6 & & \\
        3 & 7 & & \\
        3 & 8 & \textcolor{red}{sbagliato} & esercitazione\\
        3 & 9 & \textcolor{green}{corretto} & esercitazione\\
        3 & 10 & & \\
        \hline
        4 & 1 & & \\
        4 & 2 & & \\
        4 & 3 & \textcolor{green}{corretto} & esercitazione\\
        4 & 4 & & \\
        4 & 5 & \textcolor{green}{corretto} & esercitazione\\
        4 & 6 & & \\
        4 & 7 & \textcolor{green}{corretto} & esercitazione\\
        4 & 8 & \textcolor{green}{corretto} & esercitazione\\
        4 & 9 & & \\
        4 & 10 & & \\
        \hline
        5 & 1 & & \\
        5 & 2 & \textcolor{red}{sbagliato} & esercitazione\\
        5 & 3 & & \\
        5 & 4 & \textcolor{red}{sbagliato} & esercitazione\\
        5 & 5 & & \\
        5 & 6 & \textcolor{green}{corretto} & esercitazione\\
        5 & 7 & \textcolor{red}{sbagliato} & esercitazione\\
        5 & 8 & & \\
        5 & 9 & & \\
        5 & 10 & & \\
        5 & 11 & & \\
        5 & 12 & & \\
        5 & 13 & \textcolor{red}{sbagliato} importante & esercitazione\\
        5 & 14 & & \\
        5 & 15 & & \\
        5 & 16 & & esercitazione\\
        5 & 17 & \textcolor{red}{sbagliato} importante & esercitazione\\
        5 & 18 & & esercitazione\\
        5 & 19 & & \\
        5 & 20 & & \\
        \hline
        6 & 1 & & \\
        6 & 2 & \textcolor{green}{metà si} e \textcolor{red}{metà no}& esercitazione\\
        6 & 3 & \textcolor{green}{corretto} & esercitazione\\
        6 & 4 & & \\
        6 & 5 & \textcolor{gree}{corrett} (con aiutino iniziale) & esercitazione\\
        6 & 6 & & esercitazione\\
        6 & 7 & & \\
        6 & 8 & & \\
        6 & 9 & & \\
    \end{tabular}
    \section{Formulario}
    \url{./fisica-tecnica-cheatsheet-travis-2020/fisica-tecnica.pdf}
    \section{Riassunto slides}
    \url{file:///C:/Users/Federico/Desktop/NOTES/Ingegneria%20Informatica%20(triennale)/FISICA%20TECNICA/primo%20tentativo/riassunto/riassunto.pdf}
    \section{Utilities}
    \textbf{Linear interpolation}: \url{https://www.ajdesigner.com/phpinterpolation/linear_interpolation_equation.php}\newline
    \newline
    \textbf{Bilinear interpolation}: \url{https://www.ajdesigner.com/phpinterpolation/bilinear_interpolation_equation.php}
\end{document}