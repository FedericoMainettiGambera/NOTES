\section{ARGOMENTI}
\textbf{Lesson 1: INTRODUZIONE}\newline
\url{../lezioni/L01-Introduzione.pdf}
\begin{itemize}
    \item \textbf{Sistema termodinamico}: \textit{contorno, ambiente, serbatoio, sistema composto, sistema mono e pluricomponenti, sistema semplice}
    \item \textbf{Stato di equilibrio}: \textit{grandezze intensive ed estensive, legge di Duhem, regola di Gibbs, equazione di stato}
    \item \textbf{Tipologie di sistemi termodinamici}: \textit{contorno del sistema, sistema chiuso e aperto}
    \item \textbf{Trasformazioni termodinamiche}: \textit{internamente reversibile, reversibile, irreversibile, ciclica, elementare}
    \item \textbf{Equazione di stato}: \textit{gas ideali, costante R, gas reali, liquidi e solidi}
\end{itemize}
\ \newline
\newline
\textbf{Lesson 2: PRINCIPI DI CONSERVAZIONE}\newline
\url{../lezioni/L02-Principi+di+conservazione.pdf}
\begin{itemize}
    \item \textbf{Primo principio della termodinamica per sistemi chiusi}: \textit{formulazione assiomatica, lavoro L, calore Q, proprietà e casi particolari, formulazione classica, esperienze di Joule}
    \item \textbf{Secondo principio della termodinamica per sistemi chiusi}: \textit{formulazione assiomatica, entropia S, proprietà e casi particolari, bilancio di entropia}
    \item \textbf{Osservazioni sul primo e secondo principio della termodinamica}
\end{itemize}
\ \newline
\newline
\textbf{Lesson 3: TRASFORMAZIONI}\newline
\url{../lezioni/L03-Trasformazioni.pdf}
\begin{itemize}
    \item \textbf{Lavoro termodinamico}: \textit{calcolo, trasformazione reversibile vs irreversibile, funzione di stato, lavoro in un ciclo}
    \item \textbf{Calori specifici}: \textit{capacità termica, calore specifico, calori specifici a pressione costante e a volume costante e per i gas ideali e perfetti e per i liquidi incomprimibili ideali e perfetti, entalpia, relazione di Mayer}
    \item \textbf{Trasformazioni politropiche}: \textit{indice della politropica, equazione della politropica, politropiche per i as perfetti, trasformazioni elementari, lavoro scambiato in una politropica}
    \item \textbf{Diagramma T-S}
    \item \textbf{Calcolo delle grandezze termodinamiche}: \textit{tabella gas perfetti, variazione di entropia per i gas ideali e perfetti e per liquidi incomprimibili perfetti, ..., note aggiuntive}
\end{itemize}
\ \newline
\newline
\textbf{Lesson 4: SISTEMI BIFASE}\newline
\url{../lezioni/L04-Sistemi+bifase.pdf}
\begin{itemize}
    \item \textbf{Sistema eterogeneo}: \textit{omogeneo vs eterogeneo, monocomponente vs multicomponente, grndezze estensive in sistemi eterogenei bifase, frazione massica, regola di Gibbs, transizione di fase}
    \item \textbf{Sistema eterogeneo monocomponente}: \textit{nomenclatura}
    \item \textbf{Diagramma di stato P-v-T}: \textit{}
\end{itemize}
\ \newline
\newline
\textbf{Lesson 5: MACCHINE TERMODINAMICHE}\newline
\url{../lezioni/L05-Macchine+termodinamiche.pdf}
\newline
\newline
\textbf{Lesson 6: SISTEMI APERTI}\newline
\url{../lezioni/L06-Sistemi+aperti.pdf}
\newline
\newline
\textbf{Lesson 7: CICLI A GAS}\newline
\url{../lezioni/L07-Cicli+a+gas.pdf}
\newline
\newline
\textbf{Lesson 8: CICLI A VAPORE}\newline
\url{../lezioni/L08-Cicli+a+vapore.pdf}
\newline
\newline
\textbf{Lesson 9: TRASMISSIONE DEL CALORE}\newline
\url{../lezioni/L09-Trasmissione+del+calore.pdf}
\newline
\newline
\textbf{Lesson 10: CONDUZIONE}\newline
\url{../lezioni/L10-Conduzione.pdf}
\newline
\newline
\textbf{Lesson 11: CONVEZIONE}\newline
\url{../lezioni/L11-Convezione.pdf}
\newline
\newline
\textbf{Lesson 12: IRRAGIAMENTO}\newline
\url{../lezioni/L12-Irraggiamento.pdf}