\section{L02-Principi di conservazione}
\subsection{Principi di conservazioni}
\begin{itemize}
    \item Conservazione della \textbf{massa};
    \item Conservazione dell'\textbf{energia} (\textbf{primo principio della termodinamica});
    \item Conservazione dell'\textbf{entropia} (\textbf{secondo principio della termodinamica}).
\end{itemize}
\subsection{Principio di conservazione della massa}
Un sistema chiuso no nscambia massa e quindi la massa totale del sistema è sempre costante.\newline
\newline
Per i sistema aperti il discorso è differente (vedremo più avanti).
\subsection{Primo principio della termodinamica per sistemi chiusi}
\textbf{Formulazione assiomatica}\newline
Per un sistema semplice all’equilibrio è definita una proprietà intrinseca (funzione di stato) detta \textbf{energia interna} $U$ la cui variazione è il risultato di interazioni del sistema con l’ambiente esterno.
\[
    \Delta U = Q^\leftarrow - L^\rightarrow
\]
che in forma differenziale diventa
\[
    du = \delta q^\leftarrow - \delta l^\rightarrow 
\]
dove $d$ è un differenziale esatto e $\delta$ indica il differenziale di una grandezza che non è una funzione
di stato (dal punto di vista matematico hanno lo stesso significato, è solo una notazione usata per
indicare se si sta parlando di funzioni di stato o meno).\newline
\newline
\textbf{Lavoro} $L$: energia fornita ad un sistema termodinamico semplice che sia riconducibile alla variazione di quota di un grave.\newline
\newline
\textbf{Calore} $Q$: energia fornita ad un sistema termodinamico semplice che non è riconducibile alla variazione di quota di un grave.\newline
\newline
\textbf{Osservazioni}:
\begin{itemize}
    \item L’energia interna totale di un sistema, cioè l’energia interna riferita alla intera massa del sistema, $M$, è una quantità estensiva e perciò additiva:
    \[
        U = M \cdot u
    \]
    \item In un sistema isolato isolato il bilancio energetico diviene 
    \[
        \Delta U_{\text{isolato}} = 0
    \]
    \item Per un sistema che suubisce una trasformazione ciclica si ha
    \[
        \Delta U_{\text{ciclo}} = 0
    \]
    \item Per un sistema $Z$ composto da due (o più) sottosistemi $A, B$ l'energia interna totale è
    \[
        U_Z = U_A + U_B
    \]
    \[
        \Delta U_Z = \Delta U_A + \Delta U_B
    \]
    \item Se $Z$ è un sistema composto da più sottosistemi ($Z=A+B+..$) e non è isolato la variazione della sua energia interna risulta
    \[
        \Delta U_Z = \Delta U_A + \Delta U_B + \dots = Q_Z^\leftarrow - L_Z^\rightarrow 
    \]
\end{itemize}
\textbf{Formulazione classica}\newline
L’energia che è immagazzinata in un sistema e che non va a cambiare né l’energia cinetica del centro di massa, né quella potenziale (e neanche l’energia elastica, o chimica, o elettrica) è chiamata energia interna.
\subsection{Secondo principio della termodinamica per sistemi chiusi}
\textbf{Formulazione assiomatica}\newline
In un sistema termodinamico all’equilibrio esiste una funzione intrinseca dello stato del sistema (funzione di stato) detta \textbf{entropia} $S$ la cui variazione per una trasformazione reversibile è data data
\[
    \Delta S = \int \frac{\delta Q_{\text{rev}}^\leftarrow }{T}
\]
\textbf{Osservazioni}:
\begin{itemize}
    \item L’entropia totale di un sistema, cioè l’entropia riferita all’intera massa del sistema, $M$, è una quantità estensiva
    \[
        S = M \cdot s
    \]
    \item La variazione di entropia totale di un sistema isolato sede di trasformazioni termodinamiche è sempre maggiore di zero e tende a zero con il tendere dei processi alla reversibilità
    \[
        \Delta S_{\text{isolato}} \geq 0
    \]
    \item Essendo $S$ una quantità estensiba (additiva), se il sistema $Z$ è composto da due (o più) sottosistemi $A, B, \dots$ l'entropia totale è
    \[
        S_Z = S_A + S_B
    \]
    \[
        \Delta S_Z = \Delta S_A + \Delta S_B
    \]
    \item In un sistema chiuso sede di trasformazioni termodinamiche il \textbf{bilancio entropico} può essere scritto come
    \[
        \Delta S = S_Q^\leftarrow + S_{\text{irr}}
    \]
    dove il termine $S_Q^\leftarrow$ rappresenta l'\textbf{entropia entrante} attraverso i confini del sistema come conseguenza dello scambio di calore $Q$, mentre $S_{\text{irr}}$ è il termine di \textbf{generazione entropica per irreversibilità}.\newline
    \newline
    Notiamo che $S_{\text{irr}} \geq 0$ è sempre maggiore di zero.\newline
    Notiamo che il segno di $S_Q^\leftarrow$ è uguale al segno di $Q^\leftarrow $.
\end{itemize}
\subsection{Osservazioni sul primo e secondo principio}
\begin{itemize}
    \item Il primo principio non individua il verso delle trasformazioni spontanee. Il primo principio non precisa per esempio che il calore fluisce nel verso delle temperature decrescenti.
    \item Il primo principio non pone alcun limite alla possibilità di trasformazione di calore in lavoro, ma
    si limita a postularne la equivalenza metrologica.
    \item Il primo principio non stabilisce le condizioni di equilibrio termico e meccanico non vincolato.
    \item Il secondo principio colma le lacune individuate nei punti precedenti.
\end{itemize}