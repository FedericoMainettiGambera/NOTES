\section{Equazioni di bilancio per sistemi aperti}
Un \textbf{sistema aperto} è un sistema dal quale può \textbf{uscire} ed \textbf{entrare massa}.\newline
\newline
E' un \textbf{sistema fluente} è di conseguenza è necessario introdurre la \textbf{variabile di tempo} ($t$) e di conseguenza i \textbf{flussi di massa, di energia, di entropia,} etc.
\[
    \dot{m} \;\;\;\;\; \dot{E}\;\;\;\;\; \dot{U} \;\;\;\;\; \dot{H} \;\;\;\;\; \dot{S} \;\;\;\;\; \dot{Q} \;\;\;\;\; \dot{L}
\]
\ \newline
I \textbf{bilanci di massa, energia ed entropia} per i sistemi aperti sono la base per lo studio dei più
comuni componenti di interesse tecnico: pompe, compressori, turbine e scambiatori di calore. \newline
\newline
Le equazioni di bilancio, nella loro formulazione generale, sono:
\[
    \begin{cases}
        \frac{dM}{dt} = \sum_{k=1}^{n}\dot{m}_k^\leftarrow \\
        \frac{dE}{dt} = \sum_{k=1}^{n}\dot{m}_k^\leftarrow  \left(h + gz + \frac{w^2}{2}\right) + \dot{Q}^\leftarrow  - L_e^\rightarrow \\
        \frac{dS}{dt} = \sum_{k=1}^{n} \dot{m}_k^\leftarrow  s_k + \dot{S}_Q^\leftarrow + \dot{S}_{irr}
    \end{cases}
\]
Con $k$ si intendono le varie sezioni di passaggio.\newline
Con $\dot{L}_e^\rightarrow $ si intende il \textbf{lavoro d'elica}, cioè il lavoro scambiato per unità di tempo attraverso le sezioni non attraversate dalla massa.\newline 
Esiste anche il \textbf{lavoro di pulsione} $\dot{L}_P^\leftarrow $ che è il lavoro scambiato per unità di tempo attraverso le sezioni attraversate dalla massa e (per l'ingresso) si calcola come
\[
    L_{P,i}^\leftarrow  = M_i P v_i
\]
con $M_i$ la massa immessa nel sistema, $P$ la pressione costante che agisce su $M_i$, e $v_i$ il volume specifico della sessione di ingresso. Allo stesso modo si calcola il lavoro di pulsione per l'uscita.\newline
\newline
Un approccio tipico prevede come primo passo la stesura dei tre bilanci seguiti dalla semplificazione dei bilanci di energia e di entropia tramite il bilancio di massa.
\newline
\newline
\newline
\textbf{Regime stazionario con un ingresso e con un'uscita}\newline
Per regime stazionario o regime permanente si intende una condizione in cui le variazioni di massa, di energia e di entropia nel sistema sono nulle nel tempo:
\[
    \frac{dM}{dt} = 0 \;\;\;\;\;\;\;\;\;\; \frac{dE}{dt} = 0 \;\;\;\;\;\;\;\;\;\; \frac{dS}{dt} = 0
\]
Considerando inoltre una porzione di condotto delimitata da una sezione di \textbf{ingresso} (i) ed una di
\textbf{uscita} (u), i bilanci di massa, di energia e di entropia, in regime stazionario, si scrivono: 
\[
    \begin{cases}
        \dot{m}_i^\leftarrow = -\dot{m}_u^\leftarrow  = \dot{m}\\
        \frac{dE}{dt} = \dot{m}\left[ (h_i - h_u) + g(z_i-z_u) + \frac{w_i^2}{2} - \frac{w_u^2}{2}\right] + \dot{Q}^\leftarrow -L_e^\rightarrow = 0\\
        \dot{m}(s_i-s_u) + \dot{S}_Q^\leftarrow  + \dot{S}_{irr} = 0
    \end{cases}
\]
dove $\dot{m} = \rho w \Omega$ è la \textbf{portata} in massa del sistema ($\rho$ è la massa volumica, $\omega$ è la velocità media, $\Omega$ è l'area della sezione di passaggio, da notare che $\omega \Omega = \Gamma$ è la portata volumica e che $rho = VM$).
\subsection{Sistemi aperti notevoli}
\subsubsection{Macchina aperta}
La \textbf{machcina aperta} è un \textbf{dispositivo adiabatico} destinato a \textbf{scambiare lavoro} per il quale si ipotizzano trascurabili le variazioni di energia potenziale e di energia cinetica tra le sezioni di ingresso e di uscita.\newline
\newline
I bilanci diventano:
\[
    \begin{cases}
        \dot{m} (h_i - h_u) - \dot{L}_e^\rightarrow  = 0\\
        \dot{m} (s_i-s_u) + \dot{S}_{irr} = 0
    \end{cases}
\]
\begin{itemize}
    \item \textbf{Turbina}:\newline
    Una \textbf{turbina} produce lavoro, cioè genera potenza meccanica.\newline
    \newline
    Una turbina riduce il suo contenuto entalpico, la sua pressione e temperatura e aumenta il suo volume specifico.\newline
    \newline
    Nel caso di turbina il \textbf{lavoro d'elica uscente} è positivo, quindi l'entalpia in ingresso è maggiore dell'antalpia in uscita.\newline
    \newline
    Si chiama \textbf{rendimento isoentalpico} di una \textbf{macchina motrice aperta} (turbina) il rapporto fra la potenza realmente ottenuta e la potenza massima ottenibile in condizioni ideali a parità di condizioni in ingresso e a parità di pressione di fine espansione:
    \[
        \eta_T = \frac{\dot{L}_{reale}^\rightarrow }{\dot{L}_{ideale}^\rightarrow } = \frac{h_1 - h_{2'}}{h_1-h_2}
    \]
    dove per $\dot{L}_{reale}^\rightarrow $ si intende $\dot{L}_e$ e per $\dot{L}_{ideale}^\rightarrow $ si intende la potenza meccanica scambiata con un processo ideale (isoentropico).
    \item \textbf{Compressore e pompa}:\newline
    Se la macchina aperta assorbe lavoro dall'esterno, si parla di compressore o pompa a seconda del fluido di lavoro.\newline
    \newline
    Il meccanismo è opposto a quello della turbina.\newline
    \newline
    Nel caso di compressore o pompa il \textbf{lavoro d'elica uscente} è negativo.\newline
    \newline
    Si chiama \textbf{rendimento isoentropico} di una macchina operatrice aperta (compressore e pompa) il rapporto fra la potenza minima spesa in condizioni ideali e la potenza realmente spesa a parità di condizioni in ingresso e a parità di pressione di fine espansione:
    \[
        \eta_C = \frac{\dot{L}_{reale}^\rightarrow }{\dot{L}_{ideale}^\rightarrow } = \frac{h_1-h_2}{h_1-h_{2'}} 
    \]
    dove per $\dot{L}_{reale}^\rightarrow $ si intende $\dot{L}_e$ e per $\dot{L}_{ideale}^\rightarrow $ si intende la potenza meccanica scambiata con un processo ideale (isoentropico).
\end{itemize}
\subsubsection{Scambiatori di calore}
Lo \textbf{scambiatore di calora} è un dispositivo destinato a \textbf{scambiare calre} e che \textbf{non scambia lavoro} per il quale si ipotizzano trascurabili le variazioni di energia potenziale e di energia cinetica tra le sezioni di ingresso e di uscita.\newline
\newline
I bilanci diventano:
\[
    \begin{cases}
        \dot{m}(h_i - h_u) + \dot{Q}^\leftarrow  = 0\\
        \dot{m}(s_i-s_u) + \dot{S}_{Q}^\leftarrow + \dot{S}_{irr} =0 
    \end{cases}
\]
\subsubsection{Casi minori}
\begin{itemize}
    \item I \textbf{diffusori} ($w \searrow$) e gli \textbf{ugelli} ($w \nearrow$) sono sistemi aperti stazionari che operano \textbf{senza scambio di lavoro nè calore} per i quali si ipotizzano trascurabili le variazioni di energia potenziale tra le sezioni di ingresso e di uscita.\newline
    La differenza fra diffusore e ugello dipende se lo stato di ingresso si trova a una velocità di ingresso maggiore (diffusore) rispetto a quella d'uscita o viceversa (ugello).\newline
    \newline
    I bilanci diventano:
    \[
        \begin{cases}
            \left[(h_i-h_u) + \frac{w_i^2 - w_u^2}{2}\right] = 0\\
            \dot{m} (s_i-s_u) + \dot{S}_{irr} =0 
        \end{cases}
    \]
    \item La \textbf{valvola di laminazione} è un \textbf{dispositivo adiabatico} che \textbf{non scambia lavoro} per il quale si ipotizzano trascurabili le variazioni di energia potenziale e di energia cinetica tra le sezioni di ingresso e di uscita. Si ottiene un processo detto di \textbf{laminazione isoentalpica} (cioè in cui l'entalpia in ingresso e in uscita sono identiche).\newline
    A livello pratico ciò che succede in questo tipo di sistemi è che la pressione scende fra ingresso e uscita (come se ci fosse uno strozzamenteo).\newline
    \newline
    I bilanci diventano:
    \[
        \begin{cases}
            (h_i- h_u) = 0\\
            \dot{m} (s_i-s_u) + \dot{S}_{irr} = 0
        \end{cases}
    \]
\end{itemize}
