\section*{MATERIALE DEL CORSO}
Link a una cartella one drive con materiale di beep e video lezioni e esercitazioni in alta qualità:
\href{https://polimi365-my.sharepoint.com/personal/10304796_polimi_it/_layouts/15/onedrive.aspx?id=%2Fpersonal%2F10304796%5Fpolimi%5Fit%2FDocuments%2FFisicaTecnica2019%2D2020%5FINF&ct=1585366140917&or=OWA-NTB&cid=7ae5a48e-6f76-321c-d57b-c21d3b4d10f6&originalPath=aHR0cHM6Ly9wb2xpbWkzNjUtbXkuc2hhcmVwb2ludC5jb20vOmY6L2cvcGVyc29uYWwvMTAzMDQ3OTZfcG9saW1pX2l0L0VtS2lKTFJsS0RSUGxkSTJDVXlfMWg0QjFDSFhsM2p5MENHWEt5MzNrQXc0cnc_cnRpbWU9Q2wtaUxjalMxMGc}{clicca qui}
\section*{LEZIONE 1 04/03/20}
\textbf{link} \url{https://web.microsoftstream.com/video/b69d7ddf-1b39-40d5-a435-597b73f156a3}
\subsection*{Slide: L01}
\subsubsection*{Modalità del corso}
\textbf{[1-2]}\;\newline
\textbf{[3]}\; Il corso è diviso in due parti, termodinamica (60-70\% del corso) e trasmissione del calore. La parte di termodinamica si divide anch'essa in due: termodinamica principale e termodinamica di processo.\newline
\textbf{[4]}\; su beep ci sono due dispense (termodinamica e trasmissione del calore) e coprono l'intero corso. In certe lezioni faremo approfondimenti o chiarimenti di cultura generale che non sono presenti nelle dispense, ma non saranno richieste all'esame. Le dispense hanno argomenti in più (sono tarate per un corso da 10 crediti). Per capire cosa fa parte del programma e cosa possiamo trovare in esame bisogna seguire il filo logico delle slide. Le dispense e le slide sono molto utili per la parte teorica del corso, ma la parte più difficile è rappresentata dall'imparare come usare gli strumenti insegnati in maniera pratica negli esercizi. Bisogna fare molti esercizi. A supporto delle lezioni avremo un attività di progetto che servirà per dare una cultura più pratica dei concetti visti a lezione.\newline
Riguardo l'esame: la modalità di verifica è condivisa con un altro scaglione, l'esame è a libro aperto, non ci sarà teoria, c'è un documento su beep che spiega in dettaglio come si svolgerà l'esame.
\subsubsection*{Introduzione alla termodinamica}
\textbf{[5]}\;\newline
\textbf{[6]}\; Una buona prima parte dele corso tratterà la tematica del trasformare il calore in energia meccanica.\newline
\textbf{[7]}\; Informazioni sulla situazione generale della produzione di energia generale.\newline
\textbf{[8]}\; La principale trasformazione di energia è quella da combustibile a energia elettrica. L'energia elettrica è detta energia nobile perchè è facile da trasformare ed utilizzare in qualunque ambito. La maggior parte di energia viene prodotta da combustibili fossili. Se ci limitiamo a guardare il rinnovabile, notiamo che la maggiorparte viene dall'idroelettrico.\newline
\textbf{[9]}\; Informazioni sulla situazione generale dei consumi di energia.\newline
\textbf{[10-11]}\; Informazioni sulla situazione generale delle emissioni di $CO_2$.\newline
\textbf{[12]}\; Previsioni sulla situazione energetica futura.\newline
\textbf{[13]}\; Impatto delle tecnologie informatiche sul consumo energetico. Progetto: negli ultimi anni la parte di maggiore consumo elettrico si è spostata verso il backend, per esempio il data center che analizzeremo per il progetto.
\subsection*{Slide: P01}
\textbf{[1-2-3]}\; Il tema del progetto è quello di dimensionare da un punto di vista termodinamico un data center e di darci una cultura su quali sono le soluzioni impiantistiche più tipiche.\newline
\textbf{[4]}\; articoli da leggere autonomamente (soprattutto il secondo, il primo è un po' scritto male) (sono facoltativi)\newline
\textbf{[5]}\; informazioni su server e rack