\title{LEZIONE 6 26/03/20}\newline
\textbf{link} \href{https://web.microsoftstream.com/video/ee54067c-9686-4347-ba02-4b8a9715c174}{clicca qui}
\section{JSTL}
JSP ha una serie di tag attivi (useBean, setProperty, getProperty) che hanno lo scopo di arricchire il template di funzionalità nascondendone il codice dietro a degli elementi pseudo html. In JSP per programmare un custom tag bisogna costruire una classe (tag handler) che implementa un'interfaccia con una serie di metodi (per esempio doStartTag() e doEndTag()) che operano sul contenuto  del tag. Così la programmazione Java diventa completamente trasparente al template. Questa tecnica prende il nome di Code-behind.\newline
\newline
Un primo svantaggio di questo metodo è che nel momento in cui un grafico va ad aprire un file con questi custom tag, non potrà visualizzarlo correttamente in quanto queste marche non possono essere interpretate da un comune editor di html.\newline
Un altro aspetto svantaggioso di permettere di creare tag custom al programmatore è che molti di questi tag finiscono per eseguire la stessa cosa. Per questo motivo si è passati alla standardizzazione di una libreria di tag "universale": JSTL (JSP Standard Tag Library).\newline
\newline
JSTL è un set di standard tag che possono essere usati in modo da evitare la proliferazione di codice duplicato.\newline
\newline
JSTL contiene due componenti:
\begin{itemize}
    \item un linguaggio di espressione EL;
    \item una libreria di tag.
\end{itemize}
Il linguaggio di espressione è EL (expression language) ed è stato inizialmente concepito come sussidio di JSP, ma ora è stato generalizzato per molti linugaggi di programmazione. Il linguaggio è abbastanza articolato, ma noi lo useremo a livello basico (quello che impariamo guardando esercizi e progetti è più che sufficiente).\newline
EL è un linguaggio per la scrittura di espressioni, che sono costrutti la cui valutazione da luogo a un valore (mentre un istruzione è un costrutto che produce un effetto collaterale ed eventualmente produce un valore).\newline
E' particolarmente importante che un linguaggio di espressione sia conciso e facilmente leggibile. EL permette l'accesso diretto agli attributi di uno scope JSP che ha il ruolo di mappa. EL permette anche la navigazione all'interno di una complessa struttura di un oggetto con la dot-notation.\newline
Il contenitore a cui si fa riferimento non è necessariamente da esprimere, EL utilizza un sistema default in cui cerca l'attributo in tutti gli scope, dal più specifico al più generale (page, request, session, application).\newline
EL ha anche una serie di oggetti impliciti che rappresentano degli shortcut per raggiungere dati spesso utilizzati (es.param, header, cookie, etc).\newline
\newline
JSTLContructs ed ELexamples sono progetti in cui possiamo vedere tutti questi concetti applicati, con varie varianti sintattiche.\newline
\newline
Il secondo componente è la libreria di tag, che è una famiglia di tag attivi che assolvono a determinati compiti. JSTL prevede 4 famigli, noi ne vedremo solo due, perchè le altre due sono poso utilizzare (XML Processing, ormai si usa Json) o addirittura quasi deprecate (SQL, secondo il nostro schema il template non si occupa di accedere al database). Le due librerie che useremo sono Core e Internationalization.\newline
La libreria Core si occuppa dell'output, dell'accesso alle variabili e agli scope, di logica condizionale, di loop, di URLs e di error handling.\newline
La famiglia Internationalization ha lo scopo di risolvere tutti i problemi di formattamento dei dati a seconda dei vari paesi del mondo di provenienza della richiesta (es. distribuire contenuto in lingue diverse, convenzioni sui numeri con la virgola, il formato delle date, etc.) costruendo uno e un solo template. L'internationalization non può però automatizzare il contenuto dinamico, perchè dipende dal database che non è gestito nel web tier.\newline
\newline
Ci sono numerosi progetti che mostrano l'utilizzo di tutti tag.\newline
\newline
