\section{Serie di funzioni (analisi II)}
\subsection{Serie di potenze}
\subsubsection{Nel campo complesso}
\textbf{def.} Sia $\{a_n\}$ una successione di numeri complessi e sia $z_0 \in \mathbb{C}$.\newline
La serie
\[
    \sum_{n=0}^{\infty} a_n (z-z_0)^n
\]
si chiama \textbf{serie di potenza centrata} in $z_0$.\newline
\newline
Con la semplice traslazione $z-z_0 \rightarrow z$ possiamo ricondurci al caso $z_0 = 0$:
\[
    \sum_{n=0}^{\infty}a_n z^n
\]
\newline
Una serie di potenze ammette un $R \in [0, +\infty]$ tale che converge se $|z| < R$, non converge se $|z| > R$ e nulla si può dire se $|z| = R$.\newline
\newline
\textbf{Criterio del rapporto:} Se esiste
\[
    R = \lim_{n\rightarrow \infty}\left| \frac{a_n}{a_{n+1}}\right|
\]
allora la serie $\sum_{n=0}^{\infty}a_n z^n$ converge se $|z|< R$ e non converge se $|z|> R$.\newline
\newline
\textbf{Criterio della radice:} Se esiste
\[
    R = \lim_{n\rightarrow \infty} \frac{1}{\sqrt[n]{|a_n|}}
\]
allora la serie $\sum_{n=0}^{\infty}a_n z^n$ converge se $|z| < R$ e non converge se $|z|> R$.\newline
\newline
L'insieme di convergenza di una serie di potenze in $\mathbb{C}$ è un disco.\newline
Se $R = +\infty$ il disco è tutto $\mathbb{C}$, se $R = 0$ il disco è vuoto. Il numero $R \in [0, + \infty]$ si chiama \textbf{raggio di convergenza} della serie di potenza.\newline
\newline
Questi due criteri non dicono nulla sul comportamento della serie nei punti sul bordo del disco, cioè $|z| = R$.\newline
\newline
\textbf{teor.} Sia $\{a_n\}$ una successione di numeri complessi tale che la serie di potenze
\[
    f(z) = \sum_{n=0}^{\infty} a_n z^n
\]
converga per $|z|<R$ (con $R> 0$). Allora le serie ottenute derivando e integrando termine a termine, e cioè
\[
    \sum_{n=1}^{\infty} n a_n z^{n-1} \quad \quad \quad \sum_{n=0}^{\infty} \frac{a_n}{n+1} z^{n+1}
\]
sono rispettivamente la derivata e una primitiva della funzione $f$; inoltre, il loro raggio di convergenza è ancora $R$.\newline
\subsubsection{Nel campo reale}
Diremo che una serie di funzioni $\sum_{n}f_n(x)$ \textbf{converge puntualmente} per ogni $x \in I$ se la serie numerica $\sum_{n}f_n(x)$ converge per ogni $x \in I$.
\[
    f(x) = \sum_{n=0}^{\infty} f_n(x) = \lim_{k\rightarrow \infty} \sum_{n=0}^{k} f_n(x) \quad \;\forall\;x \in I
\]
cioè
\[
    f_n(x) \rightarrow f(x) \; \text{puntualmente se}\; \lim_{n\rightarrow \infty} f_n(x) = f(x) 
\]
\ \newline
Diremo che la serie di funzioni $\sum_{n}f_n(x)$ \textbf{converge uniformemente} a $f(x)$ su $I$ se
\[
    \lim_{k\rightarrow \infty} sup_{x \in I}\left| f(x) - \sum_{n=0}^{k}f_n(x) \right| = 0
\]
dove $f(x) = \sum_{n=1}^{\infty} f_n(x)$ è il limite puntuale della serie e $\sum_{n=0}^{k} f_n(x)$ è la somma parziale ennesima della serie.\newline
Cioè
\[
    f_n(x) \rightarrow f(x) \; \text{uniformemente se}\; \lim_{n\rightarrow \infty} sup_{x \in I}\left| f_n(x) - f(x) \right| = 0
\]
\ \newline
Diremo che la serie di funzioni $\sum_{n}f_n(x)$ \textbf{converge totalmente} su $I$ se
\[
    \sum_{n=0}^{\infty} sup_{x \in I}\left| f_n(x) \right| < + \infty
\]
\newline
\[
    \text{convergenza totale}\; \Longrightarrow \text{convergenza uniforme}\; \Longrightarrow \text{convergenza puntuale}
\]
\newline
Serie di potenza nel campo reale:
\[
    \sum_{n=0}^{\infty} a_n(x-x_0)^n
\]
che possiamo traslare nell'origine:
\[
    f(x) = \sum_{n=0}^{\infty} a_n x^n
\]
\newline
\textbf{Criteri del rapporto e della radice:} Si possono ancora usare i criteri della radice e del rapporto specificati nel campo complesso, $R$ rappresenta ancora il raggio del disco di convergenza nel piano complesso, ma essendo interessati all'asse reale si considera il solo intervallo $(-R, R)$.\newline
La serie di potenza nel campo reale converge \textbf{puntualmente} per ogni $x \in (-R, R)$, dove $R$ è dato dal criterio del rapporto o dal criterio della radice, e non converge se $|x| > R$.\newline
Come nel caso complesso, non possiamo dire nulle sulla convergenza in $x = \pm R$.\newline
Per quanto riguarda la convergenza \textbf{uniforme}, la serie di potenza nel campo reale converge uniformemente in $[-R+\epsilon, R- \epsilon]$ per ogni $\epsilon \in (0,R)$.\newline
\newline
\textbf{Criterio di Abel:} Se la serie di potenza $\sum_{n=0}^{\infty} a_n x^n$ converge per $x = R$, allora converge uniformemente in $[-R + \epsilon, R]$ per ogni $\epsilon \in (0,R)$; analogo risultato se la serie converge per $x = -R$. Se la serie converge per $x = \pm R$ allora converge uniformemente su tutto $[-R, R]$.\newline
\newline
\textbf{teor. (Integrazione per serie)}\newline
Se la serie di potenza $\sum_{n=0}^{\infty} a_n x^n$ converge uniformemente a $f$ su $[c,d]$ allora
\[
    \int_{c}^{d}f(x) dx = \int_{c}^{d}\left(\sum_{n=0}^{\infty}a_n x^n\right)dx = \sum_{n=0}^{\infty}a_n \int_{c}^{d}x^n dx = \sum_{n=0}^{\infty} a_n \frac{d^{n+1}- c^{n+1}}{n+1}
\]
\newline
\textbf{teor. (Derivazione per serie)}\newline
Date le serie
\[
    (1) \;\; \sum f_n(x) \quad \quad \quad \quad (2) \;\; \sum f_n'(x),
\]
se sono verificate le seguenti ipotesi:
\begin{itemize}
    \item le $f_n$ sono continue nell'intervallo $(a,b)$,
    \item la serie $(2)$ converge uniformemente in $(a,b)$,
    \item la serie $(1)$ converge per $x = x_0 \in (a,b)$  
\end{itemize}
allora valgono le seguenti tesi:
\begin{itemize}
    \item la serie $(1)$ converge uniformemente in $(a,b)$ (quindi converge a una funzione continua),
    \item la serie $(1)$ converge a una funzione derivabile in $(a,b)$,
    \item è possibile derivare la $(1)$ termine a termine, cioè
    \[
        \frac{d}{dx}\left(\sum f_n(x)\right) = \sum f_n'(x).
    \]
\end{itemize}
\subsubsection{Serie di Taylor / MacLaurin}
Introduciamo una vasta classe di funzioni elementari delle quali sappiamo scrivere esplicitamente le serie di potenza che le rappresentano.\newline
Data una funzione $f$ di classe $C^\infty$ in un punto $x_0$, possiamo scrivere formalmente
\[
    f(x) = \sum_{n=0}^{\infty} \frac{f^{(n)}(x_0)}{n!}(x-x_0)^n
\]
che, se poniamo $a_n = f^{(n)}(x_0)/n!$, coincide con una serie di potenza nel campo reale. Se $R > 0$ e la serie converge a $f$, allora la scrittura non è solo formale ma vale per ongi $x \in (x_0 - R, x_0 + R)$. In tale intervallo si ha convergenza puntuale, mentre la convergenza uniforme è garantita negli intervalli $[x_0 - R + \epsilon, x_0 + R -\epsilon]$ per ogni $\epsilon \in (0,R)$.
\[
    e^x = \sum_{n=0}^{\infty} \frac{x^n}{n!} \;\;\;\; (R= \infty)
\]
\[
    Ch(x) = \text{termini pari dello sviluppo di $e^x$}\;= \sum_{n=0}^{\infty} \frac{x^{2n}}{(2n)!} \;\;\;\; (R= \infty)
\]
\[
    Sh(x) = \text{termini dispari dello sviluppo di $e^x$}\;=  \sum_{n=0}^{\infty} \frac{x^{2n+1}}{(2n+1)!} \;\;\;\; (R= \infty)
\]
\[
    sin(x) = \sum_{n=0}^{\infty} \frac{(-1)^n}{(2n+1)!} x^{2n+1}\;\;\;\; (R= \infty)
\]
\[
    cos(x) = \sum_{n=0}^{\infty} \frac{(-1)^n}{(2n)!}x^{2n}\;\;\;\; (R= \infty)
\]
\[
    \frac{1}{1-x} = \sum_{n=0}^{\infty} x^n\;\;\;\; (R= 1)
\]
\[
    log(1+x) = \sum_{n=1}^{\infty} (-1)^{n+1} \frac{x^n}{n}\;\;\;\; (R= 1)\;\;\;\;\;\;\;\;\;\;\begin{cases}
        \text{se $x=1: $}\; \sum_{n=1}^{\infty}\frac{(-1)^{n+1}}{n} \;\;\text{(converge per Leibniz)}\;\\
        \text{se $x=-1$}\; \sum_{n=1}^{\infty}\frac{-1}{n} \;\;\text{(diverge, serie armonica)}\;
    \end{cases}
\]
\[
    \frac{1}{1+x^2} = \sum_{n=0}^{\infty} -1^n \cdot x^{2n} \;\;\;\;(R=1)
\]
\[
    arctan(x) = \sum_{n=0}^{\infty}(-1)^n \frac{x^{2n+1}}{2n+1} \;\;\;\;(R=1)
\]
\[
    (1+x)^\alpha = \sum_{n=0}^{\infty} \binom{\alpha}{n}x^n \;\;\;\; (R=1)
\]
\subsection{Serie di Fourier}
\subsubsection{Forma trigonometrica}
Nello studio delle serie di potenza $\sum_{n}a_n z^n$ abbiamo visto che i problemi principali riguardano lo studio di $|z| = R$ con $R$ raggio di convergenza. Per tali possiamo scrivere $z = R(cos(\theta) + sin(\theta))$ e ottenere:
\[
    \sum_{n=0}^{\infty} a_n R^n(cos(n \theta) + isin(n \theta))
\]
Se $\{a_n\} \subset \mathbb{R}$, siamo quindi portati a studiare la convergenza di serie trigonometriche del tipo
\[
    \sum_{n=0}^{\infty} \alpha_n cos(n \theta) \;\;\;\;\; \;\;\;\;\; \sum_{n=1}^{\infty} \beta_n sin(n \theta)
\]
Per ogni scelta di $\alpha_n, \beta_n \in \mathbb{R}$ le funzioni
\[
    a_0 + \sum_{n=1}^{\infty} (\alpha_n cos(nx) + \beta_n sin(n x))
\]
si chiamano \textbf{polinomi trigonometrici} di grado $k$; che sono funzioni periodiche di periodo $2\pi$ che hanno valor medio $\alpha_0$ su $[0, 2\pi]$.\newline
Diremo che una serie in forma trigonometrica converge se converge una successione di polinomi trogonometrici.\newline
\textbf{oss.} 
\[
    \sum_{n=1}^{\infty} (|\alpha_n| + |\beta_n|) < \infty \Rightarrow \text{converge totalmente su } [0,2\pi] \Rightarrow \text{converge uniformemente} \Rightarrow \text{converge puntualmente}
\]
\textbf{oss.} La serie trigonometrica può perdere la convergenza dopo una derivazione.\newline
\newline
\textbf{Criterio di Dirichlet}
\[
    \alpha_n, \beta_n \downarrow 0 \Rightarrow \text{polinomio trigonometrico converge puntualmente su} (0,2\pi)
\]
dove con il simbolo $\downarrow 0$ indichiamo che le successioni descrescono monotonamente a $0$ e che sono positive.\newline
\newline
\textbf{Lemma.} Per ogni $m,n = 1,2,\dots$ risulta 
\[
    \int_{0}^{2\pi} sin(nx) dx = \int_{0}^{2\pi} cos(nx) = 0
\]
\[
    \int_{0}^{2\pi} sin(mx) cos(nx) dx = 0
\]
\[
    \int_{0}^{2\pi} sin^2(nx) dx = \int_{0}^{2\pi} cos^2(nx) dx = \pi
\]
\[
    \int_{0}^{2\pi} sin(mx) sin(nx) dx = \int_{0}^{2\pi} cos(mx) sin(nx) dx = 0 \;\;\;\; \text{se} \; m\neq n
\]
Si trovano gli stessi valori integrando su qualunque intervallo di ampiezza $2\pi$.\newline
\newline
\textbf{teor. calcolo delle serie di Fourier per funzioni $2\pi$-periodiche} 
Se una funzione $f$ è $2\pi$-periodica ed è sviluppabile in serie di Fourier, si ha che
\[
    f(x) \sim  \frac{a_0}{2} + \sum_{n=1}^{\infty}(a_n cos(nx) + b_n sin(nx))
\]
allora
\[
    a_n = \frac{1}{\pi} \int_{I} f(x) cos(nx) dx \;\;\;\; \;\forall\;n\geq0
\]
\[
    b_n = \frac{1}{\pi} \int_{I} f(x) sin(nx) dx \;\;\;\;\;\forall\;n\geq 1
\]
dove l'intervallo $I$ è un qualunque intervallo di ampiezza $2\pi$ (tipicamente si prende da $-\pi$ a $\pi$, per sfruttare eventuali simmetrie\dots) e il termine $a_0$ si calcola come $a_0 = \frac{1}{\pi} \int_{I} f(x) dx$.\newline
Gli $a_n$ e $b_n$ vengono chiamati \textbf{coefficienti di Fourier} di $f$, mentre la serie $f(x) = \frac{a_0}{2} + \sum_{n=1}^{\infty}(a_n cos(nx) + b_n sin(nx))$ viene chiamata \textbf{serie di Fourier} associata a $f$.\newline
\newline
\textbf{oss.} Gli integrali del teorema si possono calcolare se risulta $\int_{0}^{2\pi}|f| < \infty$ e cioè se l'integrale improprio di $|f|$ è convergente.\newline
\newline
Diremo che la serie 
\[
    f(x) = \frac{a_0}{2} + \sum_{n=1}^{\infty}(a_n cos(nx) + b_n sin(nx))
\]
\textbf{converge in media quadratica} alla $f$ se
\[
    \lim_{k\rightarrow \infty} \int_{0}^{2\pi} \left| f(x) - \frac{a_0}{2} - \sum_{n=1}^{k}\left( a_n cos(nx) + b_n sin(nx) \right) \right|^2 dx = 0
\]
\newline
\textbf{teor.} Sia $f$ una funzione $2\pi$-periodica tale che $\int_{0}^{2\pi}f^2 < \infty$. Allora la sua serie di Fourier converge a $f$ in media quadratica.\newline
\newline
Definiamo lo spazio $X$ delle funzioni $f$ tali che $\int_{0}^{2\pi}f^2 < \infty$ e introduciamo il "prodotto scalare" definito come
\[
    (x,g)_{X} = \frac{1}{\pi}\int_{0}^{2\pi} f(x) g(x) dx \;\;\;\;\; \;\forall\;f,g \in X
\]
per cui troviamo che 
\[
    B = \left\{ \frac{1}{\sqrt{2}}, cos(nx), sin(nx) \right\}_{n=1} ^\infty
\]
è ortonormale in $X$.\newline
\newline
\textbf{teor. Identità di Parseval}\newline
Sia $f \in X$ e sia $f(x) = \frac{a_0}{2} + \sum_{n=1}^{\infty}(a_n cos(nx) + b_n sin(nx))$ la sua serie di Fourier. Allora
\[
    \frac{1}{\pi} \int_{0}^{2\pi} f(x)^2 dx = \frac{a_0^2}{2} + \sum_{n=1}^{\infty}(a_n^2 + b_n^2)
\]
Per il teorema di Riemann-Lebesgue sappiamo che $a_n, b_n \rightarrow  0$ per $n \rightarrow  \infty$.\newline
\newline
\[
\text{convergenza uniforme}\; \Rightarrow \text{convergenza in media quadratica}\; \Rightarrow 
\]
\[
    \Rightarrow  \text{convergenza puntuale su sottosuccession in quasi ogni punto}\;
\]
\newline
\textbf{def.} Diciamo che $f : [0,2\pi] \rightarrow \mathbb{R}$ è \textbf{regolare a tratti} se è limitata in $[0,2\pi]$ e se l'intervallo $[0,2\pi]$ si può scomporre in un numero finito di sottointervalli su ciascuno dei quali $f$ è continua e derivabile; inoltre, agli estremi di ogni sottointervallo, esistno finiti i limiti sia di $f$ che di $f'$.\newline
\newline
\textbf{oss.} se $f \in C^1[0,2\pi]$ allora $f$ è regolare a tratti. Ma anche se ci sono punti angolosi o punti di discontinuità a salto, purchè $f$ e $f'$ abbiano limiti finiti in prossimità dei salti, allora $f$ è regolare a tratti. Non devono esserti asintoti verticali o punti a tangenza verticale.\newline
\newline
Se una funzione $f$ è regolare a tratti allora la sua serie di Fourier converge a $f$ in media quadratica. Dunque a meno di pochi possibili punti, la conergenza sarà anche puntuale.\newline
\newline
\textbf{teor.} Sia $f: [0,2\pi] \rightarrow \mathbb{R}$ regolare a tratti. Allora la sua serie di Fourier converge in ogni punto $x_0 \in [0,2\pi]$ alla media dei due limiti $f(x_0^{\pm})$:
\[
    \frac{a_0}{2} + \sum_{n=1}^{\infty} \left( a_n cos(nx_0) + b_n sin(nx_0) \right) = \frac{f(x_0^+) + f(x_0^-)}{2}
\]
con la convenzione che $f(0^\pm) = f(2\pi^\pm)$. In particolare, se $f$ è continua in $x_0$, allora la serie converge a $f(x_0)$.
\subsubsection{Funzioni con periodi diversi da $2\pi$}
Come ci si comporta in presenza di funzioni con periodo $T \neq 2\pi$?\newline
L'unica differenza sta nel calcolo dei coefficienti di Fouriere:
\[
    a_n = \frac{2}{T}\int_{0}^{T}f(x) cos\left( \frac{2\pi n}{T}x \right) dx, \;\;\;\;\; \;\;\;\;\; b_n = \frac{2}{T} \int_{0}^{T}f(x) sin\left( \frac{2\pi n}{T}x \right)
\]
La funzione $f$ si scrive allora:
\[
    f(x) = \frac{a_0}{2} + \sum_{n=1}^{\infty} \left( a_n cos\left( \frac{2\pi n}{T}x \right) + b_n sin \left( \frac{2\pi n}{2}x \right) \right)
\]
Tutti i teoremi valgono allo stesso modo, l'unico da "sistemare" è l'identità di Parseval, che, per fuznioni $T$-periodiche $f$ soddisfacenti $\int_{0}^{T} f^2 < \infty$, diventa:
\[
    \frac{2}{T} \int_{0}^{T}f(x)^2 dx = \frac{a_0^2}{2} + \sum_{n=1}^{\infty} (a_n^2 + b_n^2)
\]
\subsubsection{Forma esponenziale complessa}
La formula di Eulero, $e^{i \theta} = cos(\theta) + i sin(\theta)$, suggerisce di scrivere una serie di Fourier utilizzando gli esponenziali.
\[
    f_n =\frac{a_n-ib_n}{2} = \frac{1}{2\pi} \int_{0}^{2\pi}f(x)(cos(nx) - i sin(nx))dx = \frac{1}{2\pi} \int_{0}^{2\pi}f(x) e^{-inx}
\]
\[
    f_{-n} =\frac{a_n + ib_n}{2} = \frac{1}{2\pi} \int_{0}^{2\pi}f(x)(cos(nx) + i sin(nx))dx = \frac{1}{2\pi} \int_{0}^{2\pi}f(x) e^{inx}
\]
da cui
\[
    a_n = f_n + f_{-n}
\]
\[
    b_n = i (f_n)
\]
\[
    f(x) = \frac{a_0}{2} + \sum_{n=1}^{\infty} \left( a_n cos(nx) + b_n sin(nx) \right) = \sum_{n=-\infty}^{\infty} f_n e^{inx}
\]
\newline
Identità di Parseval:
\[
    \frac{1}{2\pi}\int_{0}^{2\pi}f(x)^2dx = \sum_{n=-\infty}^{\infty}|f_n|^2
\]
\subsection{Note sugli esercizi}
\subsubsection{Serie di potenze nel campo reale}
Sono dette serie di potenza le serie di funzioni della forma:
\[
    \sum_{n} a_nx^n \;\;\;\;\;\;\;\;\;\;\;\;\;\;\;\sum_{n}a_n(x-x_0)^n
\]
in cui la prima si dice centrata nell'origine e la seconda centrata in $x_0$ (con la semplice traslazione di $x-x_0$ a $x$ ci si può sempre ricondurre al caso $x_0 = 0$).\newline
\newline
Una volta calcolato il raggio di convergenza $R$ come
\[
    R = \lim_{n\rightarrow +\infty} \frac{|a_n|}{|a_{n+1}|} \;\;\;\;\;\;\;\;\;\;\;\;\;\;\;R = \lim_{n\rightarrow +\infty} \frac{1}{\sqrt[n]{|a_n|}}
\] dove l'uso della prima o della seconda è suggerito dalle circostanze, ci sono tre casi possibili:
\begin{itemize}
    \item $R = 0$, la serie converge se e solo se $x = x_0$;
    \item $R = +\infty$, la serie converge puntualmente per ogni $x \in \mathbb{R}$;
    \item $0 < R < \infty$, \begin{itemize}
        \item la serie converge puntualmente se $|x-x_0| < R$, ossia nell'intervalo $(x_0-R, x_0 + R)$
        \item la serie non converge se $|x-x_0| > R$
        \item nulla si può dire riguardo al comportamento della serie nei punti $x = x_0 - R$ e $x= x_0+R$
    \end{itemize}
\end{itemize}
Per calcolare il raggio di convergenza $R$ si usano due formule:
\[
    R = \lim_{n\rightarrow +\infty} \frac{|a_n|}{|a_{n+1}|}
\]
\[
    R = \lim_{n\rightarrow +\infty} \frac{1}{\sqrt[n]{|a_n|}}
\]
\subsubsection{Convergenza uniforme per serie di potenza}
Se una serie di potenze converge per $x \in (a,b)$, si dimostra che tale serie converge uniformemente in ogni intervallo $[\alpha, \beta]$ con $a < \alpha< \beta < b$.\newline
\newline
Vale inoltre il \textbf{teorema di Abel}: se la serie di potenze $\sum a_n (x-x_0)^n$, converge negli estremi di un intervallo $[a,b]$, allora converge uniformemente in tutto l'intervallo $[a,b]$.\newline
\newline
Possibili casi:
\begin{itemize}
    \item converge in $x \in[a,b]$, converge uniformemente in $[a, b]$
    \item converge in $x \in[a,b)$, fissato un $\epsilon > 0$, converge uniformemente in $[a, b-\epsilon]$
    \item converge in $x \in(a,b]$, fissato un $\epsilon > 0$, converge uniformemente in $[a + \epsilon, b]$
    \item converge in $(a,b)$, fissato un $\epsilon > 0$ e $\delta > 0$, converge uniformemente in $[a + \delta, b-\epsilon]$
    \item converge per $x$ uguale a un solo punto, non ha senso parlare di convergenza uniforme
    \item converge per $x \in\mathbb{R}$, converge uniformemente in $[\alpha, \beta]$ con $-\infty < \alpha < \beta < + \infty$
\end{itemize}
\subsubsection{Sviuluppabilità in serie di Fourier}
Una funzione periodica $f: \mathbb{R} \rightarrow  \mathbb{R}$ è sviluppabile in serie di Fourier se vale una delle due condizioni seguenti:
\begin{itemize}
    \item $f$ è limitata e monotona a tratti su un periodo, oppure
    \item il quadrato di $f$ è integrabile su un periodo (vale meno di $\infty$).
\end{itemize}
Inoltre se la funzione $f$ è continua in $x_0$, allora la serie di Fourier converge al valore della funzione, viceversa se in $x_0$ è presente una discontinuità di prima specie (ma $f$ è limitata e monotona a tratti):
\[
    \lim_{x\rightarrow x_0^-} f(x) = f_-(x_0)
\]
\[
    \lim_{x\rightarrow x_0^+} f(x) = f_+(x_0)
\]
\[
    f_-(x_0) \neq f_+(x_0)
\]
allora (quale sia il valore che $f$ assume in $x_0$) il valore della serie in $x_0$ è la media aritmentica dei due valori:
\[
    \frac{f_-(x_0) + f_+(x_0)}{2}
\]
\newline
In parole povere, \textbf{per valutare se la funzione sia rappresentabile o meno con una serie di Fourier}, per prima cosa si controlla se è limitata e monotona tratti su un periodo (si capisce con una analisi grafica), se lo è allora la funzione è rappresentabile con Fourier. Altrimenti si può usare anche un metodo analitico: si prende la funzione, e si fa l'integrale $\int_{0}^{2\pi} f^2$, se l'integrale ha un valore finito, allora la funzione può essere sviluppata con una serie di Fourier. Da notare che l'integrale può essere svolto anche in senso generalizzato per risolvere puntii in cui la funzione va a $\infty$.\newline
\textbf{Per valutare per quali valori di $x$ la serie converge effettivamente alla funzione indicata}, si usa il seguente criterio: dove la funzione è continua, la serie converge al valore della funzione, dove ci sono discontinuità a salto la serie converge al valor medio degli estremi del salto, dove ci sono altri tipi di discontinuità non si può concludere nulla sul comportamento della seria.
\subsubsection{Calcolo delle serie di Fourier per funzioni $2\pi$-periodiche} 
Se una funzione $f$ è $2\pi$-periodica ed è sviluppabile in serie di Fourier, si ha che
\[
    f(x) \sim  \frac{a_0}{2} + \sum_{n=1}^{\infty}(a_n cos(nx) + b_n sin(nx))
\]
allora
\[
    a_n = \frac{1}{\pi} \int_{I} f(x) cos(nx) dx \;\;\;\; \;\forall\;n\geq0
\]
\[
    b_n = \frac{1}{\pi} \int_{I} f(x) sin(nx) dx \;\;\;\;\;\forall\;n\geq 1
\]
dove l'intervallo $I$ è un qualunque intervallo di ampiezza $2\pi$ (tipicamente si prende da $-\pi$ a $\pi$, per sfruttare eventuali simmetrie\dots) e il termine $a_0$ si calcola come $a_0 = \frac{1}{\pi} \int_{I} f(x) dx$.
\subsubsection{Semplificazioni nel calcolo delle serie di Fourier per funzioni pari e dispari}
I coefficienti di Fourier di funzioni $2\pi$-periodiche si possono calcolare integrando su un qualunque intervallo di ampiezza $2\pi$. Tipicamente si sceglie l'intervallo $[-\pi, \pi]$ perchè permette di sfruttare eventuali simmetrie della funzione $f$.\newline
Infatti, se la funzione $f$ è \textbf{pari} allora risulta pari anche $f(x)cos(nx)$ mentre risulta dispari $f(x)sin(nx)$, dunque:
\[
    f \; \text{pari}\; \Rightarrow a_n = \frac{2}{\pi} \int_{0}^{\pi}f(x) cos(nx) dx , \;\;\;\;\;b_n = 0
\]
Invece, se la funzione $f$ è \textbf{dispari} allora risulta dispari anche la funzioen $f(x)cos(nx)$ mentre risulta pari $f(x) sin(nx)$, dunque:
\[
    f \; \text{dispari}\; \Rightarrow a_n = 0, \;\;\;\;\;b_n = \frac{2}{\pi} \int_{0}^{\pi} f(x) sin(nx)dx
\]
\subsubsection{Riassunto dei criteri per la convergenza della serie di Fourier $\sum F$}
\textbf{Appunti prof:}\newline
Condizione di partenza:
\[
    \int_{0}^{2\pi}f^2 < \infty \Leftrightarrow \text{possiamo calcolare}\;\{a_n, b_n\}
\]
\begin{itemize}
    \item se $\sum_{n=1}^{\infty}(|a_n| + |b_n|) < \infty$ (converge) $\Rightarrow \sum F \rightarrow f$ totalmente $\Rightarrow \sum F \rightarrow  f$ uniformemente.\newline
    Proprietà: una successione/serie di funzioni continue che converge uniformemente ha limite continuo.
    \item se $f$ non è continua, la serie di Fourier $\sum F$ non può convergere uniformemente.
    \item $\int_{0}^{2\pi}f^2 < \infty $ e cioè $\sum_{n=0}^{\infty}(a_n^2 + b_n^2)< \infty \;\;\Longleftrightarrow \;\;\sum F \rightarrow f$ in media quadratica.
    \item Criterio di Dirichlet: se $a_n, b_n \downarrow 0 \Rightarrow  \sum F \rightarrow f$ puntualmente su $(0,2\pi)$.
    \item $f$ regolare a tratti $\Rightarrow \sum F(x) \rightarrow  \frac{f(x)^+ + f(x)^-}{2}$ puntualmente su $[0,2\pi]$
\end{itemize}
\textbf{Note dal libro di esercizi:}\newline
Data la generica funzione di Fourier:
\[
    f(x) \sim  \frac{a_0}{2} + \sum_{n=1}^{\infty} a_n cos(nx) + b_n sin(nx)
\]
per stabilirne la convergenza è possibile percorrere due strade:
\begin{itemize}
    \item Criterio di Dirichlet: se le successioni $a_n$ e $b_n$ sono monotone decrescenti e tendono a $0$, allora la serie di Fourier covnerge in tutti i punti, tranne al più in $x = 2k\pi$.
    \item Criterio di Weierstrass per le serie di funzioni: Data una serie di funzioni $\sum f_n(x)$, se per $x \in [a,b]$ si ha che
    \begin{itemize}
        \item $|f_n(x)| \leq c_n$
        \item $\sum c_n$ converge
    \end{itemize}
    allora la serie di partenza converge uniformemente in tutto l'intervallo $[a,b]$.\newline
    Per usarlo con le serie di Fourier: poichè
    \[
        |a_n cos(nx) + b_n sin(nx)| \leq |a_n| + |b_n|
    \]
    se la serie $\sum |a_n| + |b_n|$ converge, allora la serie di Fourier converge (uniformemente), e quindi la funzione a cui converge è continua.
\end{itemize}
Inoltre, poichè derivando la funzione termine a termine, otteniamo la serie
\[
    \sum_{n=1}^{\infty} (n b_n ) cos(nx) - (n a_n)sin(nx)
\]
è sufficiente riapplicare a qeust'ultima i criteri di convergenza, per controllare se la serie di partenza converge a una funzione derivabile.
\subsubsection{Armoniche}
Data una serie di Fourier:
\[
    f(x) \sim  \frac{a_0}{2} + \sum_{n=1}^{\infty}(a_n cos(nx) + b_n sin(nx))
\]
Allora può essere riscritta nella forma
\[
    f(x) \sim \text{valor medio}\; + \sum_{n=1}^{\infty}\alpha_n cos(nx + \theta_n)
\]
di cui dobbiamo ricavare $\alpha_n$ e $\theta_n$:
\[
    \alpha_n = \sqrt{a_n^2 + b_n^2}
\]
e
\[
    cos(\theta_n) = \frac{a_n}{\sqrt{a_n^2 + b_n^2}}
\]
\[
    sin(\theta_n) = -\frac{b_n}{\sqrt{a_n^2 + b_n^2}}
\]
da cui si ricava che $\theta_n$ è:
\[
    \theta_n = \begin{cases}
        arctan\left(\frac{-b_n}{a_n}\right) \;\;\;\; &se \; a_n > 0\\
        \pi + arctan\left(\frac{-b_n}{a_n}\right) \;\;\;\;& se \; a_n < 0
    \end{cases}
\]
se $a_n = 0$, non si può usare l'arcotangente, ma in tal caso la trasposizione è banale e diventa diventa:
\[
    b_n sin(nx) = b_n cos(nx - \frac{\pi}{2})
\]