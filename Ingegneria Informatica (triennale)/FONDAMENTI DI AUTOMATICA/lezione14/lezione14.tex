\section{Esercitazione 4}
\title{LEZIONE 14 1/04/2020}\newline
\textbf{link} \href{https://web.microsoftstream.com/video/7cfe3714-fd1e-453f-8394-f7abd4d747ad?list=user&userId=faa91214-a6f5-40d7-8875-253fd49b8ce1}{clicca qui}\newline
\newline
Appunti del prof con annotazioni \url{../pdf/FdA-L14-2020.04.01.pdf}\newline
Contenuto:
\begin{itemize}
    \item Diagrammi di Bode di fase e modulo;
    \item Regolo delle fasi.
\end{itemize}
\subsection{Regolo delle fasi}
Il tipico diagramma di Bode della fase che facciamo è una approssimazione asintotica che può variare molto rispetto al vero e proprio grafico. Il regolo delle fasi è utilizzato per calcolare la fase esatta (senza approssimazione asintotica) per una specifica frequenza $\bar{\omega}$ calcolando il contributo che ciascun polo e ciascun zero apporta a quella data frequenza $\bar{\omega}$.\newline
\newline
I passaggi da seguire sono i seguenti:
\begin{itemize}
    \item Si disegna il diagramma di Bode della fase approssimato;
    \item Si pone il regolo delle fasi con la freccia dei 45 gradi sulla $\bar{\omega}$ di cui si desidera calcolare la fase;
    \item Lungo il regolo si leggono i contributi (in modulo) di ogni polo e zero in corrispondenza delle loro frequenze d'angolo;
    \item Per calcolare la fase nel punto $\bar{\omega}$ è ora sufficiente combinare i valori trovati al punto precedente (ricordando di aggiungere il segno corretto e  di moltiplicarli per il numero di poli o zeri presenti), inoltre bisogna ricordarsi anche il valore di partenza della fase che ovviamente nel regolo non è espresso.
\end{itemize}