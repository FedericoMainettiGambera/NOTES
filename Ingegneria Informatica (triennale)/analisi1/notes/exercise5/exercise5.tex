\section*{5-ESERCITAZIONE}
22/10/19
\subsubsection*{Funzioni continue}
\textbf{es.} verificare che $f(x) = 5x -4$ è continua in $x=2$ con la definizione
\newline
Devo verificare che $f(2) = 6$ sia lo stesso risultato di:
\[
    \lim_{x\rightarrow 2} 5x-4 = 6
\]
\[
    \;\forall\; \epsilon > 0 \;\; \exists \;\; \delta \;\;:\;\; \;\forall\;x 0<|x-2|<\delta 
\]
\[
    |f(x) -6| < \epsilon
\]
\[
    |5x-4-6| < \epsilon
\]
\[
    -\epsilon < 5x-10< \epsilon
\]
\[
    10-\epsilon< 5x< 10 + \epsilon
\]
\[
    2- \frac{\epsilon}{5} < x < 2 + \frac{\epsilon}{5}
\]
Quindi se fisso $\delta = \frac{\epsilon}{5}$ il risultato è verificato.
\newline
\newline
\newline
\textbf{es.} verificare se $f(x)$ è continua nel suo dominio:
\[
    f(x) = \begin{cases}
        sen(x) \cdot e^{\frac{1}{x}} &\;\;\; per \;\;\; x\neq 0 \\
        0 & \;\;\;per \;\;\; x=0
    \end{cases}
\]
Il dominio è $\mathbb{R} - \{0\}$
\[
    \lim_{x\rightarrow 0 }f(x) = f(0)
\]
Verifichiamolo con la definizione:
\[
    \;\forall\; \epsilon> 0 \;\;\;\exists \;\; \delta \;\;:\;\; \;\forall\;x \;\; 0< |x|< \delta
\]
\[
    |f(x)-f(0)|<\epsilon
\]
\[
    |f(x)| < \epsilon
\]
Limite sinistro:
\[
    \lim_{x\rightarrow 0^-} f(x) = 0^-
\]
Quindi la funzione è continua da sinistra.
\newline
Limite destro:
\[
    \lim_{x\rightarrow 0^+} f(x) = \lim_{x\rightarrow 0^+}(x + o(x))\cdot  e ^{\frac{1}{x}} = \lim_{x\rightarrow 0^+} \frac{e^{\frac{1}{x}}}{\frac{1}{x}}
\]
La funzione non è continua da destra e quindi non è continua nel suo dominio.
\newline
\newline
\newline
\textbf{es.} per quali $ a \in \mathbb{R}$ la funzione
\[
    f(x) = \begin{cases}
        x^2 +2ax +a \;\;\;& x \in (0,+\infty) \\
        \sqrt{x+2}&x \in [-2,0]
    \end{cases}
\] 
è continua?
\newline
Dominio: $\mathbb{D} : [-2,+ +\infty)$
\newline
Il punto su cui investigare la continuità della funzione è lo $0$.
\[
    \lim_{x\rightarrow 0} f(x) = f(0)
\]
Vediamo $f(0)$: $\sqrt{0+2} = \sqrt{2}$.
\newline
Limite destro:
\[
    \lim_{x\rightarrow 0^+} x^2 + 2ax +a = a = \sqrt{2}
\]
Se impongo $a= \sqrt{2}$ la funzione è continua, ma se $a \neq \sqrt{2}$ abbiamo una discontinuità di prima specie in $0^+$.
\newline
\newline
\newline
\textbf{es.} Esercizio a crocette
\[
    x^3 +2x -2 = 0
\]
\begin{itemize}
    \item impossibile
    \item ha necessariamente tre radici reali
    \item verificata per almeno un $\alpha \in (0,1)$
    \item nessuna delle precedenti
\end{itemize}
Usiamo il teorema degli zeri (Bolzano) che dice:
\newline
$f$ continua in $[a,b]$, $f(a) \cdot f(b) < 0 \Rightarrow  \;\;\exists$ almeno un $c \in(a,b) \;\;:\;\; f(c) = 0$. Inoltre se f è strettamente monotona, lo zero è minimo.
\newline
\newline
$f(x)$ è continua su $\mathbb{R} \rightarrow  [a,b] \subset \mathbb{R}$ , $f(x)$ è continua in $[0,1]$.
\newline
$f(0) \cdot f(1) = -2 <0$, quindi per il teorema degli zeri $\;\;\exists\;\;$ almeno un $\alpha \in(0,1) \;\;:\;\; f(\alpha) = 0$.
\newline
Quindi la risposta è la c (terza, verificata per almeno un $\alpha \in (0,1)$).
\newline
\newline
\newline
\textbf{es.} Esercizio a crocette
\[
    f(x) = e^x +x
\]
\begin{itemize}
    \item è positiva in $\mathbb{R}$
    \item ha uno e un solo zero in $(-1,1)$
    \item ha uno zero in $(-1,1)$, ma non è unico
    \item nessuna delle precedenti
\end{itemize}
La prima è facilmente dimostrabile falsa: $f(-10) = e^{-10}-10 <0$.
\newline
Per la seconda: $f(x)$ è continua in $[-1,1]$, e per il teorema degli zeri: $f(-1) \cdot  f(1) = (e^{-1}-1)\cdot (e+1)<0$.
\newline
Per la terza:
\[
    e^x+x = 0
\]
\[
    e^x = -x
\]
Facendo un confronto grafico vediamo che esiste un solo punto di incontro fra le fuznioni $e^x$ e $-x$. Inoltre sono entrambi strettamente crescenti, quindi la somma di due funzioni strettamente crescenti può avere uno e un solo zero.
\newline
\newline
\newline
\textbf{es.} Scrivere lo sviluppo asintotico della seguente funzione:
\[
    f(x)  = 3x +1 + \sqrt{x^2+6x-3}
\]
Dominio:
\[
    x^2 +6x-3 \geq 0
\]
\[
    x_{1,2} = -3 \pm \sqrt{12}
\]
\[
    \mathbb{D}: (-\infty, -3-\sqrt{12}) \cup [-3 + \sqrt{12},+ \infty)
\]
[COMMENTED LINE]
%\[
    %f(x) \leq 3x +1 + |x| \cdot \sqrt{1 + \frac{6}{x} - \frac{3}{x^2}
%\]
\newline
Nell intorno di $+ \infty$:
\[
    f(x) = 3x +1+  x ( 1 + \frac{1}{2} ( \frac{6}{x} - \frac{3}{x^2}) + o(\frac{1}{x})) = 3x+1+x+3- \frac{3}{2x} +o(1) = 4x +4+o(1)
\]
per cui:
\[
    f(x) \sim 4x+4
\]
Nell intorno di $- \infty$:
\[
    f(x) = 3x+1+x-3+ \frac{3}{2x} + o(1) = 2x-2 + o(1)
\]
per cui:
\[
    f(x) \sim 2x-2
\]
\newline
\newline
\newline
\textbf{es.} Verificare che $\;\;\exists\;\;$ almeno un punto $c \in[-e^2,-1] \;\;:\;\; g(  alpha) = 2$
\[
    g(x) = ln(|x|) -x
\]
Si può usare il terema dei valori medi:
\newline
se $f$ è continua su $(a,b)$ e f assume i valore $y_1 < y_2$, allora f assume tutti i valori $y \;\;:\;\; y_1 < y< y_2$.
\newline
\newline
\[
    g(-2) = ln(e^2) +e^2 = 2+e^2
\]
\[
    g(-1)
\]
\subsubsection*{ASINTOTI}
\textbf{def.}  definita un funzione $f \;\;:\;\; I \rightarrow \mathbb{R}$ e $x_0$ un punto di accumulazione per $I$ se
\[
    \lim_{x\rightarrow x_0^-} f(x) = \pm \infty
\]
diciamo che $x = x_0$ è un \textbf{asintoto verticale sinistro}.
\newline
Altrimenti se
\[
    \lim_{x\rightarrow x_0^+} f(x)  = \pm \infty
\]
diciamo che $x= x_0$ è un \textbf{asintoto verticale destro}.
\newline
\newline
\newline
Preso $I$ intonro di $\infty$, se
\[
    \lim_{x\rightarrow -\infty} f(x) = l
\]
con $l$ finito, diciamo che $y= l$ è un \textbf{asintoto orizzontale per $- \infty$}
\newline
Altrimenti se
\[
    \lim_{x\rightarrow +\infty} f(x) = l
\]
con $l$ finito, diciamo che $y= l$ è un \textbf{asintoto orizzontale per $+ \infty$}
\newline
\newline
\newline
Se, invece, abbiamo
\[
    \lim_{x\rightarrow \infty} f(x) = \infty
\]
non c'è un asintoto orizzontale, ma se :
\[
    \;\;\exists\;\;m,q \;\; finiti \;\; \;\;:\;\; \lim_{x\rightarrow \infty}(f(x)-(mx+q))=0
\]
diciamo che $y= mx + q$  è un \textbf{asintoto obliquo} per $f(x)$.
\newline
\textbf{oss.} Per trovare l'asintoto obliquo si usano queste formule:
\[
    \lim_{x\rightarrow \infty} \frac{f(x)}{x}= m \;\;\; finito \;\; e \; \neq 0
\]
\[
    \lim_{x\rightarrow \infty}(f(x)-mx) = q \;\;\;finito
\]
\newline
\newline
\newline
\newline
\textbf{es.} Studiare la seguente funzione:
\[
    y= \frac{x^2 +1}{|x+1|}
\]
Dominio:
\[
    \mathbb{D}: \;\; |x+1| \neq 0 \rightarrow x \neq -1
\]
\[
    \mathbb{D}: \;\; (-\infty,-1)\cup(-1,+\infty)
\]
Segno della funzione:
\newline
la funzione ha $y>0 \;\forall\; x \in \mathbb{D}$
\newline
Intersezione con gli assi:
\[
    x=0 \rightarrow  y=1
\]
\[
    y=0 \rightarrow mai
\]
Asintoti:
\[
    \lim_{x\rightarrow -\infty} \frac{x^2 +1 }{|x+1|} = + \infty
\]
Non esiste asintoto orizontale, ma calcoliamo quello obliquo:
\[
    \lim_{x\rightarrow -\infty} \frac{x^2+1}{x \cdot |x+1|} = 
\]
per $x \rightarrow - \infty$ il modulo è negativo, quindi possiamo toglierlo e negare il contenuto
\[
    =\lim_{x\rightarrow - \infty} \frac{x^2+1}{-x^2 -x} = -1
\]
Quindi $m=-1$, calcoliamo ora la $q$:
\[
    \lim_{x\rightarrow -\infty} ( f(x) -mx) = \lim_{x\rightarrow -\infty} \frac{x^2 +1}{-x-1} +x = \lim_{x\rightarrow -\infty} \frac{x^2 +1 -x^2-x}{-x-1} = \frac{-1}{-1} = 1
\]
Quindi $q = -1$, e l'asintoto obliquo a $-\infty$ è $y= -x +1$.
\newline
Passiamo agli altri asintoti:
\[
    \lim_{x\rightarrow \pm 1^-} \frac{x^2+1}{|x+1|} = + \infty
\]
Quindi $ x= -1$ è asintoto verticale a sinistra e a destra per $-1$.
\newline
\[
    \lim_{x\rightarrow +\infty} \frac{x^2+1}{|x+1|} = 
\]
posso togliere il modulo perchè stiamo guardando valori positivi ($x \rightarrow  + \infty$)
\[
    = \lim_{x\rightarrow + \infty} \frac{x^2 +1}{x+1} = + \infty
\]
per cui non c'è asintoto orizzontale, allora cerchiamo quello obliquo se esite:
\[
    \lim_{x\rightarrow + \infty} \frac{f(x)}{x} = \lim_{x\rightarrow + \infty} \frac{x^2+1}{x^2+x} = 1 = m
\]
\[
    \lim_{x\rightarrow +\infty} (f(x) - mx) = \lim_{x\rightarrow + \infty} \frac{x^2+1}{x+1}+x = \lim_{x\rightarrow +\infty} \frac{x^2 +1-x^2 -x}{x+1} = -1
\]
Ora vogliamo capire come la funzione raggiunge gli asintoti obliqui, se da sopra o da sotto, quindi vediamo se si interseca con gli asintoti facendo un sistem:
\[
    \begin{cases}
        y = \frac{x^2+1}{|x+1|} & \\
        y= -x +1 &
    \end{cases}
\]
$x< -1$
\[
    \frac{x^2 +1}{-x-1} = \frac{(-x+1)}{-x-1}\cdot (-x-1)
\]
\[
    x^2 +1 = x^2 -1
\]
\[
    1 = -1 \Rightarrow \; impossibile
\]
Quindi andando verso $-\infty$ la nostra funzione rimane sopra il suo asintoto obliquo.
\newline
Analiziamo ora verso $+ \infty$:
\[
    \begin{cases}
        y = \frac{x^2 +1}{}
    \end{cases}
\]
$x> -1$
\[
    \frac{x^2 +1}{x+1} = x-1 
\]
\dots
\newline
[manca]
\newline
Comunquen on interseca l'asintoto neanche verso $+ \infty$
\newline
IMMAGINE: grafico dell'esercizio
\newline
\newline
\newline
\textbf{es.} Studiare la seguente funzione:
\[
    y = 2 \sqrt{2} x - ln(x^2-1)
\]
Dominio:
\[
    \mathbb{D}: x^2-1 > 0 \Rightarrow (- \infty, -1) \cup (1, + \infty)
\]
Non ci sono simmetrie.
\newline
Vediamo dove si interseca con gli assi:
\newline
L'asse delle $y$ non è nel dominio, quindi non lo controlliamo neanche, vediamo l'asse delle $x$
\[
    2 \sqrt{2}x - ln(x^2-1) \geq 0
\]
trasformiamo $2 \sqrt{2}x$ in $ln(e^{2 \sqrt{2}x})$
\[
    ln(e^{2 \sqrt{2}x}) - ln(x^2-1) \geq 0 
\]
\[
    ln (\frac{e^{2 \sqrt{2}x}}{x^2-1}) > 0
\]
\[
    \frac{e^{2 \sqrt{2}x}}{x^2-1} > 1
\]
\[
    e ^{2 \sqrt{2} x} > x^2-1
\]
Vedo graficamente che l'esponenziale $e ^{2 \sqrt{2} x}$ è maggiore della parabola $x^2-1$ per un certo $\alpha < -1$ (approssimazione per disegnare la fuznione).
\newline
Limiti:
\[
    \lim_{x\rightarrow - \infty} f(x) = - \infty
\]
Quindi non esiste asintoto orizzontale in $- \infty$, cerchiamo quello obliquo.
\newline
Cerchiamo la $m$ svolgendo questo limite e vedendo se il risultato è finito o meno:
\[
    \lim_{x\rightarrow -\infty} \frac{f(x)}{x} = \lim_{x\rightarrow -\infty} 2 \sqrt{2} - \frac{ln(x^2-1)}{x} = 2 \sqrt{2} = m
\]
Ora cerchiamo la $q$:
\[
    \lim_{x\rightarrow -\infty} f(x) - mx = \lim_{x\rightarrow -\infty} (2 \sqrt{2}x -ln (x^2-1) - 2 \sqrt{2}x) = -\infty
\]
Ma essendo $q = -\infty$, non esiste neanche quello obliquo.
\newline
Passiamo ora a studiare il punto $-1$
\[
    \lim_{x\rightarrow  + -1^-} f(x) = + \infty
\]
Per cui esiste un asintoto verticale a sinistra.
\newline
Passiamo a $1$:
\[
    \lim_{x\rightarrow 1^+}f(x) = + \infty
\]
Per cui esiste un asintoto verticale a destra.
\newline
Studiamo, infine, $+ \infty$:
\[
    \lim_{x\rightarrow +\infty} f(x) = [\infty- \infty] = \lim_{x\rightarrow + \infty} x (2 \sqrt{2} - \frac{ln(x^2-1)}{x})= + \infty
\]
Per la gerarchia degli infiniti $\frac{ln(x^2-1)}{x} \rightarrow 0$, quindi non esiste l'asintoto orizzontale.
\newline
Cerchiamo quindi quello obliquo:
\[
    \lim_{x\rightarrow + \infty} \frac{f(x)}{x} = 2 \sqrt{2} = m
\]
\[
    \lim_{x\rightarrow + \infty} 2 \sqrt{2}x - ln(x^2-1) - 2 \sqrt{2}x = - \infty = q
\]
Poichè $q$ non è finita, non esiste asintoto obliquo.
\newline
IMMAGINE: grafico della funzione.
\newline
\newline
\newline
\textbf{es.} Verificare che $e^x-sin(x) = 0$ ammette almeno una soluzione nell'intervallo $[-\frac{7}{2}\pi, 1]$ e discutere il numero di soluzioni in quell'intervallo.
\[
    f(x) = e^x -sin(x)
\]
la fuznione è definita su tutto $\mathbb{R}$ perchè è somma di funzioni continue su $\mathbb{R}$. Quindi in particolare sarà continua su un intervallo di $\mathbb{R}$: $[-\frac{7}{2}\pi, 1]$.
\newline
Usiamo il teorema di Bolzano
\[
    -\frac{7}{2}\pi \cdot f(1) = (e^{-\frac{7}{2}\pi} -1) \cdot (e-sin(1))
\]
$(e^{-\frac{7}{2}\pi} -1)$ è negativo, $(e-sin(1))$ è positivo, quindi il prodotto è negativo, e secondo il terema degli Zeri esiste almeno una soluzione dell'equazione $f(x) = 0$.
\[
    e^x = sin(x) \rightarrow \begin{cases}
        y=e^x &\\
        y= sin(x) &
    \end{cases}
\]
IMMAGINE: grafico dell'intersezione di $y=e ^x$ e $y=sin(x)$ nell'intervallo $[-\frac{7}{2}\pi, 1]$
\newline
Guardando questo grafico vediamo che ci sono tre intersezioni.
\newline
\newline
\newline
\textbf{es.} esercizio a crocette:
\[
    g(x) = \frac{x^2+x+|x|}{x}
\]
per $x \rightarrow  0$
\begin{itemize}
    \item $\rightarrow 1$
    \item $\rightarrow 0$
    \item non ammette limite
    \item nessuna
\end{itemize}
\[
    \lim_{x\rightarrow 0^+} \frac{x^2+x+|x|}{x} = \lim_{x\rightarrow 0^+} x+2 = 2
\]
\[
    \lim_{x\rightarrow  0^-} \frac{x^2 +x -x}{x} = 0
\]
Quindi la funzione non è continua e la risposta giusta è la c (terza, non ammette limite)
\newline
\newline
\newline
\textbf{es.} esercizio a crocette:
\[
    f(x) = \frac{1}{ln(|x|)}
\]
\begin{itemize}
    \item discontinua in $x = 0$ e definita su $\mathbb{R}$
    \item discontinuità a salto in $x=0$ 
    \item discontinuità eliminabile in $x=0$
    \item nessuna delle precedenti
\end{itemize}
La prima risposta è evidentemente sbagliata: il dominio mi fa dire che $x \neq 0 \land x \neq \pm 1$.
\newline
La funzioen si vede essere pari:
\[
    Y(-x) = \frac{1}{ln (|-x|)} = \frac{1}{ln (|x)} = y(x)
\]
Calcoliamo ora:
\[
    \lim_{x\rightarrow 0^-} \frac{1}{ln(|x|)} = 0^-
\]
\[
    \lim_{x\rightarrow 0^+} \frac{1}{ln(|x|)} = 0^-
\]
Quindi la risposta giusta è la c (terza, discontinuità eliminabile in $x=0$).
\[
    \lim_{n\rightarrow \infty} \frac
    {tg(\sqrt[3]{log(cos(\sqrt{n^2+2}-n))})}
    {(n-\sqrt{n-1})^{\alpha}[cos(log(sin(\frac{n\pi+2}{\pi+2n})))-1]}
\]
    lim (n -> ∞) (tg((log(cos((n^2+2)^(1/2)-n)))^(1/3)))/((n-(n-1)^(1/2))^(α)[cos(log(sin((nπ+2)/(π+2n))))-1])

    y=(tg((log(cos((x^2+2)^(1/2)-x)))^(1/3)))/((x-(x-1)^(1/2))[cos(log(sin((xπ+2)/(π+2x))))-1])
\newpage