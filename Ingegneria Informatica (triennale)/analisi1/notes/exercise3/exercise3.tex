\section*{3-ESERCITAZIONE}
15/10/19
\subsubsection*{Limiti}
Vediamo un piccolo riassunto dei concetti fondamentali sui limiti:
\[
    \lim_{x\rightarrow c} f(x) = l \;\;\;se \;\;\; \;\forall\;B_r(l) \;\exists \; B_r(c) \;\;:\;\; \;\forall\;x \in B_r(c) \cup D_f , f(x) \in B_r(l)
\]
\[
    f \sim c \;\;\; per \;\;\; x \rightarrow c \;\;\; se \;\;\; \lim_{x\rightarrow c}\frac{f(x)}{g(x)} = 1
\]
\newline
\[
    f(x) = o(g(x)) \;\;\; in \; B_r(c) \;\;\; se \;\;\; \lim_{x\rightarrow c} \frac{f(x)}{g(x)}= 0
\]
\newline
\[
    f \sim  g \Longleftrightarrow f(x) = g(x) + o(g(x)) \;\;\;\; per \; x \rightarrow c
\]
\newline
Vediamo alcuni degli asintotici notevoli più diffusi:
\newline
per tutti vale che $x \rightarrow 0$
\[
    sin(x) = x + o(x) \sim x
\]
\[
    cos(x) = 1- \frac{x^2}{2} + o(x^2) \sim  1- \frac{x^2}{2}
\]
\[
    ln(1+x) = x+o(x) \sim x
\]
\[
    e^x=1+x+o(x) \sim 1+x
\]
\[
    (1+x)^{\alpha} = 1 + \alpha \cdot x + o(x) \sim 1 + \alpha \cdot x 
\]
\[
    tan(x) = x + o(x) \sim x
\]
\[
    arcsin(x) = x + o(x) \sim x
\]
\[
    arctan(x) = x + o(x) \sim x
\]
\newline
Vediamo ora le proprietà di o-piccolo:
\[
    f(x) = o(g(x)) \;\; se \;\; \lim_{x\rightarrow c} \frac{f(x)}{g(x)} = 0
\]
\[
    \lim_{x\rightarrow c} \frac{o(g(x))}{f(x)} = 0
\]
\[
    c \cdot o(g(x)) = o (c \cdot g(x)) = o(g(x))
\]
\[
    o(f(x)) + o(f(x)) = o(f(x))
\]
\[
    g(x) \cdot o(f(x)) = o(g(x) \cdot  f(x))
\]
\[
    o(g(x)) \cdot  o(f(x)) = o(g(x) \cdot f(x))
\]
\[
    o(1) = \epsilon(x)
\]
\newline
Vediamo la gerarchia degli infiniti:
\newline
Per $x \rightarrow  \infty$ e $0 < \alpha < \beta$ e $1 < a < b$:
\[
    (log_a(x))^\alpha \;\; << \;\; x^\alpha \;\; << \;\; x^\beta \;\;<<\;\;a ^ x \;\;<<\;\;b^x \;\;<<\;\; b^n \;\;<<\;\; n! \;\;<<\;\;n^n
\]
\newline
\newline
\textbf{es.} Risolvere il seguente limite:
\[
    \lim_{x\rightarrow 0} ln(cos(x)) = ln(1- \frac{x^2}{2!} + o(x^2))= - \frac{x^2}{2} + o(x^2) + o(-\frac{x^2}{2}+o(x^2)) \sim  - \frac{x^2}{2}
\]
Da notare che $o(-\frac{x^2}{2}+o(x^2)) = o(x^2)$
\newline
\newline
\newline
\textbf{es.} Risolvere il seguente limite:
\[
    \lim_{x\rightarrow + \infty}\frac{2^x+ln(x)+sin(x^2)}{x+x \cdot ln(x) + 4^{-x} } = \frac{\infty}{\infty} = \lim_{x\rightarrow + \infty}\frac{2^x}{x \cdot ln(x)} = + \infty
\]
\newline
\newline
\textbf{es.} Risolvere il seguente limite:
\[
    \lim_{x\rightarrow +\infty} \frac{2x+\sqrt{x}}{x^2+\sqrt{x}} = \frac{\infty}{\infty} = 
\]
tramite la gerarchia degli infiniti otteniamo:
\[
    = \lim_{x\rightarrow +\infty}\frac{1}{x^2} = 0
\]
\newline
\newline
\textbf{es.}  Risolvere il seguente limite:
\[
    \lim_{x\rightarrow 0} \frac{2x+\sqrt{x}}{x^2+\sqrt{x}} = \frac{0}{0}= 
\]
tramite la gerarchia degli infiniti otteniamo:
\[
    = \lim_{x\rightarrow 0} \frac{\sqrt{x}}{\sqrt{x}} = 1
\]
Un altro modo per risolvere l'esercizio:
\[
    \lim_{x\rightarrow 0} \frac{2x+\sqrt{x}}{x^2 + \sqrt{x}} = \lim_{x\rightarrow 0} \frac{\sqrt{x} \cdot (2 \sqrt{x} +1)}{\sqrt{x} \cdot (x^{\frac{3}{2}}+1)} = 1
\]
\newline
\newline
\textbf{es.} Risolvere il seguente limite
\[
    \lim_{x\rightarrow + \infty} (1+ \frac{1}{x})^{3x} = 1 ^{\infty} = 
\]
Si può facilmente usare il limite che definisce il numero $e$ per svolgere l'esercizio, ma vediamo come risolverlo senza questo limite notevole:
\newline
Possiamo usare la seguente uguaglianza: $x= e^{ln(x)} = ln(e^x)$
\[
    = \lim_{x\rightarrow  + \infty} e^{3x \cdot ln(1+ \frac{1}{x})} =
\]
Ora usiamo $ln(1+x) = x +o(x)$ nell'intorno dell'origine e ricordiamoci le proprietà di o-piccolo: $f(x) \cdot  o(g(x)) = o(f(x) \cdot g(x)) \;\;\;$ e $\;\;\; c \cdot o(f(x)) = o(c \cdot f(x)) = o(f(x))$ con $c$ una costante
\[
    = \lim_{x\rightarrow + \infty} e^{3x(\frac{1}{x}+ o(\frac{1}{x}))} = \lim_{x\rightarrow +\infty} e^{3+o(1)} = 
\]
dove $o(1)$ è un infinitesimo, quindi:
\[
    = e^3
\]
\newline
\newline
\textbf{es.}  Risolvere il seguente limite:
\[
    \lim_{x\rightarrow 0} \frac{e^x - e ^-x}{sin(x)} = \frac{0}{0}=
\]
risolviamo questo limite con gli o-piccolo
\[
    =\lim_{x\rightarrow 0} \frac{1+x+o(x)-(1-x+o(x))}{x + o(x)} =
\]
Da notare come $o(-x) = o(x)$.
\newline
Perchè abbiamo deciso di usare gli o-piccoli? perchè se cerco di approssimare il limite senza gli o-piccoli perdo informazioni e non riesco a risolvere il limite e raggiungo la forma di indeterminazione $\frac{0}{0}$, con gli o-piccoli invece mantengo informazion ipiù dettagliate.
\[
    \lim_{x\rightarrow 0} \frac{2x+o(x)}{x+o(x)} = 2
\]
\newline
\newline
\textbf{es.} \textbf{es.}  Risolvere il seguente limite:
\[
    \lim_{x\rightarrow 0} \frac{sin(\sqrt[3]{x})}{2x} = \frac{0}{0}= \lim_{x\rightarrow 0} \frac{x^{\frac{1}{3}} + o (x ^{\frac{1}{3}})}{2x} = + \infty
\]
\newline
\newline
\textbf{es.}  Risolvere il seguente limite:
\[
    \lim_{x\rightarrow 1} \frac{ln(x)}{1-x^2} = \frac{0}{0} = 
\]
cerchiamo di trasformare l'argomento del logaritmo in forma $1 +$ infinitesimo, usiamo un cambio di variabile:
\[
    y=x-1
\]
\[
    =\lim_{y\rightarrow 0} \frac{ln(1+y)}{-y(y+2)} =
\]
Il denominatore viene da: $(1-x^2) = (1+x)\cdot(1-x)$
\[
    = \lim_{y\rightarrow 0} \frac{y +o(y)}{-y(y+2)} = -\frac{1}{2}
\]
Gli o-piccoli bisogna portarseli dietro fino alla fine, solo allora si decide se servono o se van ignorati, per esempio in questo esercizio solo nell'ultimo passaggio abbiamo deciso di ignorare $o(y)$ perchè era di un ordine di infinitesimo trascurabile.
\newline
\newline
\newline
\textbf{es.} Risolvere il seguente limite:
\[
    \lim_{n\rightarrow + \infty} \frac{e^{-n} \cdot cos(n)}{sin(\frac{1}{n})} = \frac{0}{0}
\]
Trascuriamo per il momento $cos(n)$ e valutiamo solo:
\[
    \lim_{n\rightarrow + \infty}\frac{e^{.n}}{sin(\frac{1}{n})} = \lim_{n\rightarrow  +\infty} \frac{e^{-n}}{\frac{1}{n} + o (\frac{1}{n})} \sim  \frac{e^{-n}}{\frac{1}{n}} = \frac{n}{e^n} \rightarrow 0
\]
Riaggiungiamo ora $cos(n)$:
\[
    \lim_{n\rightarrow + \infty} \frac{e^{-n} \cdot cos(n)}{sin(\frac{1}{n})} = \frac{n}{e^n}cos(n) = 0
\]
\newline
\newline
\textbf{es.} Scrivere lo sviluppo asintotico della seguente successione
\[
    a_n = (-)^n \cdot  (e^{\frac{1}{n}} - e^{-\frac{1}{n}}) \cdot  arctg(\frac{n^2}{1-n})
\]
Dobbiamo usare la seguente formula:
\[
    arcth(x) + arctg(\frac{1}{x}) = \begin{cases}
        \frac{\pi}{2} & x>0 \\
        -\frac{\pi}{2} & x>0
    \end{cases}
\]
Ll'argomento dell'arcotangente si comporta così
\[
    \frac{n^2}{1-n}<0 \;\;\;\;\;\;\;\forall\;n\geq 2
\]
Nell'intonro dell origine
\[
    arctg(x) = x +o(x) \sim x \;\;\;\;\;\;\;\;\;\;\;\;\;\;con \;\;f(x)\rightarrow 0 \;\;\;\;arctg(f(x)) = f(x) + o(f(x))
\]
\[
    arctg(\frac{n^2}{1-n}) = -\frac{\pi}{2}-arctg(\frac{1-n}{n^2})
\]
\[
    a_n = (-)^{n+1} \cdot 1+\frac{1}{n}+o(\frac{1}{n}) -(1-\frac{1}{n} +o(\frac{1}{n})) \cdot (-\frac{\pi}{2} \cdot  arctg(\frac{1}{n^2} - \frac{1}{n}))
\]
Poichè
\[
    \frac{1}{n^2} << \frac{1}{n}
\]
posso ignorare $\frac{1}{n^2}$:
\[
    a_n = (-)^{n+1} \cdot 1+\frac{1}{n}+o(\frac{1}{n}) -(1-\frac{1}{n} +o(\frac{1}{n})) \cdot (-\frac{\pi}{2} \cdot  arctg(\frac{1}{n})) = 
\]
ora usando $con \;\;f(x)\rightarrow 0 \;\;\;\;arctg(f(x)) = f(x) + o(f(x))$ ottengo:
\[
    = (-)^{n+1} (\frac{2}{n} + o (\frac{1}{n})) \cdot (\frac{\pi}{2} - \frac{1}{n} + o(\frac{1}{n}) + o( - \frac{1}{n} + o(\frac{1}{n})))=
\]
Svolgiamo ora tutti i conti
\[
    = (-)^{n+1} \cdot (\frac{\pi}{n} + o(\frac{1}{n})) =  (-)^{n+1}\frac{\pi}{n} + o(\frac{1}{n})
\]
\[
    a_n \sim (-)^{n+1}\frac{\pi}{n} 
\]
\newline
\newline
\textbf{es.} Risolvere il seguente limite:
\[
    \lim_{x\rightarrow 3} \frac{x^3 -5 x^2 +3x + 9}{x-3} = \frac{0}{0}=
\]
Per il teorema di Ruffini $P(3) = 0 \Longleftrightarrow P(x)/(x-3)$ ($P(x)$ significa polinomio di x...)
\[
    \lim_{x\rightarrow 3} \frac{(x-3)^2(x+1)}{(x-3)} = \lim_{x\rightarrow 3} (x-3) \cdot (x+1) = 0
\]
Abbiamo risolto il limite senza sviluppi particolari. Ma ora vogliamo sapere il comportamento del limite in un intorno di $3$:
\[
    4(x-3) \sim  f(x)
\]
quindi in un intorno di $3$ il comportamento asintotico è approssimabile a quello della retta $4(x-3)$
\newline
\newline
\newline
\textbf{es.} Risolvere il seguente limite:
\[
    \lim_{n\rightarrow + \infty} \sqrt[n]{n^2} = \lim_{n\rightarrow +\infty} n^{\frac{2}{n}} = \infty^0 =  
\]
Risolviamo passando all'esponenziale
\[
    = \lim_{n\rightarrow + \infty} e^{ln(n^{\frac{2}{n}})} = \lim_{n\rightarrow +\infty} e ^{\frac{2}{n} \cdot  ln(n)} = 
\]
a questo punto non dobbiamo approssimare con gli asintotici il logaritmo perchè $n \rightarrow  \infty$, invece usiamo la gerarchia degli infiniti:
\[
    = e^0 = 1
\]
\newline
\newline
\textbf{es.} Risolvere il seguente limite:
\[
    \lim_{x\rightarrow + \infty} (\sqrt{x^2-1}-x) = \lim_{x\rightarrow + \infty} (\sqrt{x^2(1- \frac{1}{x^2})}-x) = \lim_{x\rightarrow +\infty} x(\sqrt{1-\frac{1}{x^2}}-1) = 
\]
usiamo $(1+f(x))^{\alpha} = 1 +\alpha f(x) + o(f(x)) \sim 1 +\alpha f(x)$ con $f(x)$ funzione infinitesima
\[
    = \lim_{x\rightarrow + \infty} x (1 - \frac{1}{2x^2} + o(\frac{1}{x^2}))-1 = \lim_{x\rightarrow + \infty} x (- \frac{1}{2x^2} + o(\frac{1}{x^2})) =
\]
\[
    = \lim_{x\rightarrow + \infty} -\frac{1}{2x} + o(\frac{1}{x}) = 0
\]
\newline
\newline
\textbf{es.} Risolvere il seguente limite:
\[
    \lim_{x\rightarrow 1} \frac{sin (ln(x))}{ln(x)} = 1
\]
perchè sappiamo $ln(x) \rightarrow 0 \;\; per \; x \rightarrow 1 \;\;$ e che $\;\; sin(x) \sim  x \;\; per \;\; x \rightarrow 0$, quindi $\frac{sin(x)}{x} \rightarrow  1 \;\;\;\; per \;\; x \rightarrow 0$
\newline
Un altro modo di vedere la soluzione è il seguente:
\[
    \lim_{x\rightarrow 1} \frac{sin (ln(x))}{ln(x)} = \frac{ln(x) + o(ln(x))}{ln(x)} = 1
\]
che usa il principio che $sin(x) \sim x \;\; per \; x \rightarrow 0$
\newline
\newline
\newline
\textbf{es.} Risolvere il seguente limite:
\[
    \lim_{x\rightarrow 0} \frac{sin(x+3)- sin(x)}{x \cdot  cos(x)} = \frac{0}{0} =
\]
usiamo le formule di addizione del seno
\[
    = \lim_{x\rightarrow 0} \frac{sin(x)cos(3) + cos(x)sin(3) - sin(3)}{x \cdot cos(x)} = \lim_{x\rightarrow 0}  \frac{sin(x)cos(3)}{x \cdot cos(x)} +  \frac{cos(x)sin(3) - sin(3)}{x \cdot cos(x)} = 
\]
\[
    = \lim_{x\rightarrow 0} (\frac{sin(x)}{x} \cdot \frac{cos(3)}{cos(x)} + \frac{sin(3)}{cos(x)} \cdot \frac{cos(x)-1}{x}) = 
\]
Vediamo che $\frac{cos(x)-1}{x} \rightarrow 0$, poi $\frac{sin(3)}{cos(x)} \cdot \frac{cos(x)-1}{x} \rightarrow 0$, poi  $\frac{sin(x)}{x} \rightarrow  1$
\[
    = cos(3)
\]
\newline
\newline
\textbf{es.} Risolvere il seguente limite:
\[
    \lim_{x\rightarrow \pi} \frac{sin(x)}{x-\pi} = \frac{0}{0}
\]
Facciamo un cmabio di variabile
\[
    y = x-  \pi
\]
\[
    \lim_{y\rightarrow 0} \frac{sin(y+\pi)}{y} = \lim_{y\rightarrow 0} -\frac{sin(y)}{y} = -1
\]
\newline
\newline
\textbf{es.} Risolvere il seguente limite:
\[
    \lim_{x\rightarrow 0} \frac{1}{x} \cdot \sqrt{\frac{1-cos(x)}{2}} = \infty \cdot 0 = 
\]
in un intorno dell'origine $1-cos(x) \sim  \frac{x^2}{2}$
\[
    = \lim_{x\rightarrow 0} \frac{1}{x} \sqrt{\frac{x^2}{4}} = \lim_{x\rightarrow 0} \frac{|x|}{2x}=
\]
\[
    = \begin{cases}
        per \; x \rightarrow 0^+ & \frac{1}{2} \\
        per \; x \rightarrow 0^- & -\frac{1}{2}
    \end{cases}
\]
quindi il limite non esiste.
\newline
\newline
\newline
\textbf{es.} Risolvere i lseguente limite:
\[
    \lim_{x\rightarrow e} \frac{ln(x)-1}{x-e} =
\]
Facciamo un cambio variabile per avere un logaritmo più comodo
\[
    y = x-e
\]
\[
    \lim_{y\rightarrow 0} \frac{ln(y+e)-1}{y} = 
\]
Sappiamo che $ln(1+\epsilon(x)) \sim  \epsilon(x)$, per cui raccogliamo nel logaritmo per avere questa forma:
\[
    \lim_{y\rightarrow 0} \frac{ln[e(1+\frac{y}{e})]-1}{y} = \lim_{y\rightarrow 0}  \frac{1 + ln ( 1 + \frac{y}{e})-1}{y} = \lim_{y\rightarrow 0} \frac{ln ( 1 + \frac{y}{e})}{y} = \lim_{y\rightarrow 0} = \lim_{y\rightarrow 0} \frac{\frac{y}{e} + o(y)}{y} = \frac{1}{e}
\]

\newpage