\section*{4-ESERCITAZIONE}
21/10/19
\subsubsection*{esercitazione sostitutiva}
\textbf{es.} Dimostrare che:
\[
    n^{\frac{n}{2}} < n! \;\;\; per \;\;\; n \rightarrow + \infty
\]
\[
    n! = n \cdot (n-1) \dots (\frac{n}{2}) \dots 1 > (\frac{n}{2})^{\frac{n}{2}} \cdot (\frac{n}{2})!
\]
Quindi dopo questa trasformazione mi basta dimostrare:
\[
    (\frac{n}{2})^{\frac{n}{2}} \cdot (\frac{n}{2})! > n ^ {\frac{n}{2}}
\]
\[
    \frac{n^{\frac{n}{2}}}{2^{\frac{n}{2}}} \cdot  (\frac{n}{2})! > n^{\frac{n}{2}}
\]
\[
    (\frac{n}{2})! > 2 ^{\frac{n}{2}}
\]
se poniamo $x = \frac{n}{2}$
\[
    x! > 2^x
\]
che è vero e quindi dimostrato.
\newline
\newline
\newline
\textbf{es.} Svolgere il seguente limite
\[
    \lim_{x\rightarrow 3} \frac{x^3-5x^2+3x+9}{x-3} = \frac{0}{0}
\]
Ci sono diversi modi per fattorizzare il numeratore:
\begin{itemize}
    \item Ruffini
    \item con un cambio di variabile
\end{itemize}
Proviamo con il cambio di variabile $y= x-3$
\[
    \lim_{y\rightarrow 0} \frac{(y+3)^2-5(y+3)^2+3(y+3)+9}{y}= \lim_{y\rightarrow 0} \frac{y^3+4y^2}{y} = \lim_{y\rightarrow 0}\frac{y^2(y+4)}{y} = 0
\]
Proviamo con Ruffini:
\[
    \lim_{x\rightarrow 3} \frac{(x-3)^2(x+1)}{x-3}= \lim_{x\rightarrow 3}(x-3)(x+1) = 0
\]
\newline
\newline
\textbf{es.} Svolgere il seguente limite:
\[
    \lim_{x\rightarrow \pi} \frac{sin(x)}{x-\pi} = \frac{0}{0}
\]
Noi sappiamo che $\lim_{x\rightarrow 0} sin(x) \sim x$, ma se guardiamo il grafico della funzione seno, notiamo che nell'intorno di $\pi$ comunque ci avviciniamo a zero in modo estremamente simile.
\newline
Lavoriamo ora con un cambio variabile $y=x-\pi$
\[
    \lim_{y\rightarrow 0} \frac{sin(x+\pi)}{y} = \lim_{y\rightarrow 0} \frac{-sin(y)}{y}= \lim_{y\rightarrow 0}-\frac{y+o(y)}{y}= \lim_{y\rightarrow 0}\frac{y(-1+o(1))}{y} = -1
\]
\newline
\newline
\textbf{es.} Svolgere i seguenti due limiti:
\[
    a) \;\;\;\;\;\lim_{x\rightarrow 0} \frac{1}{x} \cdot \sqrt{\frac{1-cos(x)}{2}}
\]
\[
    b) \;\;\;\;\;\lim_{x\rightarrow 0} \frac{1}{|x|} \cdot \sqrt{\frac{1-cos(x)}{2}}
\]
vediamo a):
\[
    \lim_{x\rightarrow 0} \frac{1}{x}\sqrt{\frac{x^2}{4}} = \lim_{x\rightarrow 0} \frac{1}{x} \cdot \frac{|x|}{2}= \begin{cases}
        \lim_{x\rightarrow 0^+} &\frac{1}{2} \\
        \lim_{x\rightarrow 0^-} &-\frac{1}{2}
    \end{cases}
\]
Il limite non esiste.
\newline
vediamo b):
\[
    \lim_{x\rightarrow 0} \frac{1}{|x|} \cdot \sqrt{\frac{x^2}{4}} = \lim_{x\rightarrow 0} \frac{|x|}{2} \cdot \frac{1}{|x} = \frac{1}{2}
\]
Il limite esiste.
\newline
\newline
\newline
\subsubsection*{Ordini di infinitesimo}
\textbf{def.} $x \rightarrow x_0 \land f(x) \rightarrow 0 \;\;\;\;\;\;\; f$ è \textbf{infinitesimo} nell'intonro di $x_n$.
\newline
\newline
\textbf{def.} \textbf{infinitesimo campione di ordine k}:
\newline
con $x_0$ finito: $i_k(x) = |x-x_0|^k$ con $k>0$ si dice infinitesimo campione di ordine $k$ per $x \rightarrow x_0$
\newline
con $x_0$ infinito: $i_k(x) = \frac{1}{|x|^k}$ con $k>0$ si dice infinitesimo campione di ordine $k$ per $x \rightarrow x_0$
\newline
\newline
\textbf{def.} dico f infinitesimo di ordine k se:
\[
    per \;\;\; x \rightarrow x_0 \;\;\; \frac{f(x)}{i_k(x)} \longrightarrow c \;\;\;\;\;\;\; con \;\; c \neq 0\;\; e \;\;finito
\]
\newline
\newline
\newline
\textbf{es.} trovere l'ordine dei seguenti infinitesimi:
\[
    f(x) = \frac{x^3 - 5x^2 + 3x +9}{x-3}
\]
Questa funzione per $x \rightarrow  3$ tende a $0$, ma con che ordine?
\[
    \lim_{x\rightarrow 3} \frac{f(x)}{(x-3)^k} = \lim_{x\rightarrow 3} \frac{(x-3)^2(x+1)}{x-3} \cdot  \frac{1}{(x-3)^k} = \lim_{x\rightarrow 3} \frac{(x-3)^2(x+1)}{(x-3)^{k+1}} \rightarrow  4
\]
\[
    2 = k+1 \rightarrow  k= 1
\]
\[
    f(x) = 4(x-3) + o (x-3)
\]
\newline
\newline
\newline
\subsubsection*{Ordini di infinito}
\textbf{def.} $x \rightarrow  x_0$ e $f(x) \rightarrow \infty$, $f$ è \textbf{infinito} per $x \rightarrow x_0$
\newline
\newline
\textbf{def.} \textbf{infinito campione} 
\newline
per $x_0$ finito: $i_k(x) = \frac{1}{|x-x_0|^k}$ con $k > 0$ si dice infinito campione di k-esimo ordine
\newline
per $x_0$ infinito: $i_k(x) = |x|^k$ coon $k>0$ si dice infinito campione di k-esimo ordine.
\newline
\newline
\textbf{def.} dico f infinito campione di ordine k seguente
\[
    per \;\;\; x \rightarrow  x_0 \;\;\; \frac{f(x)}{i_k(x)} \rightarrow c \;\;\;\;\;\;\; c \neq 0 \;\;e \;\;finito
\]
\newline
\newline
\newline
\textbf{es.} Calcolare il seguente limite:
\[
    \lim_{x\rightarrow \pm\infty}\sqrt{x^2-1} -x
\]
\[
    \mathbb{D}(f) : (- \infty,1] \cup [1, + \infty)
\]
Vediamo questa funione per $+ \infty$
\[
    \lim_{x\rightarrow +\infty} f(x) = + \infty
\]
\[
    \lim_{x\rightarrow -\infty}\frac{f(x)}{x^k} = \lim_{x\rightarrow -\infty}\frac{\sqrt{x^2-1} -x}{x^k} =
\]
Facendo uscire l'$x^2$ dalla radice mi esce $|x|$, ma andando a $\rightarrow  -\infty$ posso sostituire il modulo con un "$-$":
\[
    =\lim_{x\rightarrow -\infty} \frac{-x((1-\frac{1}{x^2})^{\frac{1}{2}}+1)}{x^k} = c
\]
Che è vera per $k=1$
\newline
vediamo la funzione per $- \infty$
\[
    \lim_{x\rightarrow +\infty} x [(1 - \frac{1}{x^2})^{\frac{1}{2}}+1] = \lim_{x\rightarrow +\infty } x \cdot [-\frac{1}{2x^2} + o(\frac{1}{x^2})]= \lim_{x\rightarrow +\infty} - \frac{1}{2x} + o(\frac{1}{x}) = 0^-
\]
Ho quindi trovato che $lim \frac{f(x)}{\frac{1}{2}} = lim \frac{-\frac{1}{2} + o(\frac{1}{x})}{\frac{1}{x}} = - \frac{1}{2} $, che è un numero finito, quind $k = 1$, cioè infinito di ordine 1.
\newline
\newline
\newline
\textbf{es.} calcolare il seguente limite:
\[
    \lim_{x\rightarrow 1} \frac{(log(x))^2}{(2x-2)^2} = \frac{0}{0}
\]
Noi sappiamo che $log(1+z) = z + o(z)$ per $z \rightarrow 0$, quindi trasformiamo il limite così:
\[
    \lim_{x\rightarrow 1} \frac{[log(1+(x-1)) ]^2}{[2(x-1)]^2} = \lim_{x\rightarrow 1} \frac{[(x-1) + o(x-1)]^2}{4(x-1)^2} = \lim_{x\rightarrow 1} \frac{(x-1)^2 [1+o(1)]}{4(x-1)^2} = \lim_{x\rightarrow 1} \frac{ 1+o(1)}{4} = \frac{1}{4}
\]
Abbiamo trovato che:
\[
    per \;\;\;x \rightarrow  1 \;\;\; log^2(x) = (x-1)^2 + o((x-1)^2)
\]
quindi $k=2$, quindi abbiamo un ordine di infinitesimo di 2.
\newline
\newline
\newline
\textbf{es.}  calcolare il seguente limite
\[
    \lim_{x\rightarrow +\infty} x \cdot log(\frac{x+4}{x-1}) = [0 \cdot \infty]
\]
\[
    = \lim_{x\rightarrow +\infty} x \cdot  log(\frac{x-1 + 5+1}{x-1}) = \lim_{x\rightarrow +\infty} x \cdot log(1 + \frac{1}{x-1}) = \lim_{x\rightarrow +\infty} x \cdot [\frac{6}{x-1} + o(\frac{1}{x-1})] = 
\]
\[
    = \lim_{x\rightarrow +\infty} \frac{6x}{x}[\frac{1}{1-\frac{1}{x}}+ o(1)] = 6
\]
Si vede che $log(\frac{x+5}{x-1}) \rightarrow 0$ è infinitesima per $x \rightarrow  + \infty$,
\newline
ord. inf:
\[
    \frac{f(x)}{i_k(x)} = \frac{log(\frac{x+5}{x-1})}{\frac{1}{x^k}} = x^k \cdot log(\frac{x+5}{x-1}) \rightarrow 6 ( \neq 0) \;\;\; se \;\; k = 1
\]
quindi è di ordine 1.
\newline
\newline
\newline
\textbf{es.} calcolare il seguente limite:
\[
    \lim_{x\rightarrow +\infty} \frac{log(log(x))}{log(x)-4} = \frac{\infty}{\infty}
\]
Facciamo un cambio variabile $y = log(x)$
\[
    \lim_{y\rightarrow +\\infty} \frac{log(y)}{y-4} \rightarrow 0
\]
Cerchiamo ora l'ordine di infinitesimo:
\[
    \frac{f(x)}{i_k(x)} = \frac{log(log(x)) \cdot x^k }{log(x)-4} = \rightarrow y= log(x) \rightarrow =  \frac{log(y)}{y-4} \cdot e^{ky} \;\; , \;\;\;\;per \;\;x \rightarrow  + \infty : \;\;\;\; \begin{cases}
        +\infty & k>0 \\
        0 & k=0
    \end{cases}
\]
Poichè il valore o è infinito o zero, non esiste nessun ordine di infinitesimo.
\newline
\newline
\newline
\textbf{es.} studio locale di
\[
    f(x) = \frac{log(x)}{\sqrt[3]{x-1}}
\]
Tracciamo un grafico seguendo questo processo:
\begin{itemize}
    \item dominio, zeri e segno
    \item negli zeri e alla frontiera del dominio cerco sviluppi asintotici
\end{itemize}
Dominio:
\[
    \mathbb{D}(f) = \begin{cases}
        x >0 & \\
        x \neq 1 &
    \end{cases} = (0,1) \cup(1, + \infty)
\]
Segno:
\newline
è sempre positiva.
\newline
Zeri:
\newline
non ci sono zeri.
\newline
Frontiere del dominio:
\begin{itemize}
    \item se $x \rightarrow 1$: $ f= \frac{0}{0} = \frac{log(1+(x-1))}{(x-1)^{\frac{1}{3}}} = \frac{(x-1)+ o(x-1)}{(x-1)^{\frac{1}{3}}}= (x-1)^{\frac{2}{3}} \cdot [1+o(1)] \sim (x-1)^{\frac{2}{3}} \rightarrow 0$, quindi è una fuznione infinitesima di ordine $k = \frac{2}{3}$, cioè nell'intonro di 1 si comporta come $x^{\frac{2}{3}}$ (cuspide)
    \item se $x \rightarrow 0^+$: $f \sim -log(x) \rightarrow + \infty$, di cui, però, non si trova un ordine di infinito, è semplicemente un infinito logaritmico.
    \item se $x \rightarrow  +\infty$: $f=\frac{\infty}{\infty}= \frac{log(x)}{x^{\frac{1}{3}}\sqrt[3]{1-\frac{1}{x}}} \rightarrow 0^+$
\end{itemize}


\newpage