\title{Esercitazione 1 19/03/2020}\newline
\textbf{link} \href{https://web.microsoftstream.com/video/4f8d8d0e-8354-4151-bbc2-efa8e7399119?list=user&userId=c487c446-28dc-44c3-b1fb-2fe7e71e0737}{Clicca qui} (solo audio)
\section{Esercitazione I}
APPUNTI DEL PROF (Anno corrente): \url{../esercitazione1/pdf/ese_1_notes.pdf}. [consigliato! Ho preso note su questo pdf, in caso di dubbi confrontare le esercitazioni degli anni scorsi che sono più ordinate]\newline
\newline
ESERCITAZIONE -1 (Anno scorso): \url{../esercitazione1/pdf/01-Richiami e cinematica punto.pdf}.\newline
\newline
ESERCITAZIONE -2 (Anno scorso): \url{../esercitazione1/pdf/01-Ese aggiuntivo cinematica.pdf}.
\subsection{Vettori}
Il generico vettore $\vec{P}$ è definito come $\vec{P} = (P-O) = x_p \vec{i} + y_P \vec{j} + z_P \vec{k}$ con $\vec{i} = \left[\begin{matrix}
    1\\0\\0
\end{matrix}\right]$, $\vec{j} = \left[\begin{matrix}
    0\\1\\0
\end{matrix}\right]$ e $\vec{k}= \left[\begin{matrix}
    0\\0\\1
\end{matrix}\right]$.\newline
\newline
Un vettore è definito da 
\begin{itemize}
    \item modulo $|\vec{P}| = P = \sqrt{x_P^2 + y_P^2 + z_P^2}$;
    \item direzione;
    \item verso;
    \item punto di applicazione.
\end{itemize}
\subsubsection{Operazioni fra vettori}
\begin{itemize}
    \item \textbf{Prodotto scalare}: $c = \vec{a} \; x \; \vec{b} = |\vec{a}| | \vec{b}| cos(\alpha) = a b cos(\alpha)$, con $\alpha$ angolo compreso fra i due vettori, notiamo che $b cos(\alpha)$ è la proiezione di $b$ su $a$.
    \item \textbf{Somma di vettori}: $\vec{c} = \vec{a} + \vec{b} = (a_x \vec{i} + a_y \vec{j}) + (b_x \vec{i} + b_y \vec{j})$, graficamente si può usare la regola del parallelogramma.
    \item \textbf{Prodotto vettoriale}: $\vec{c} = \vec{a} \land \vec{b}$, direzione e verso di $\vec{c}$ sono decisi con la regola della mano destra (primo vettore che compare nel prodotto sul pollice, secondo vettore sull'indice, risultato nel medio), il modulo di $\vec{c}$ si trova come $|\vec{c}| = s b sin(\alpha)$.\newline
    Per fare il prodotto vettoriale si usa spesso questo metodo: 
    \[
        \vec{c} = \vec{a} \land \vec{b} = det \left[\begin{matrix}
            \vec{i} & \vec{j} & \vec{k}\\
            a_x & a_y & a_z \\
            b_x & b_y & b_z
        \end{matrix}\right]
    \]
\end{itemize}
\subsection{Cinematica del punto}
\subsubsection{Legge oraria e traiettoria}
Data una \textbf{legge oraria} $P(t) = x(t) \vec{i} + y(t) \vec{j}$ che descrive il moto di un punto nel tempo, si può ricavare la \textbf{traiettoria}:
\begin{itemize}
    \item Si esplicita la legge oraria in forma parametrica: $\begin{cases}
        x = x(t)\\ y= y(t)
    \end{cases}$
    \item si ricava $t$ dalla priama equazione e la si sostituisce nella seconda ricavando così un'equazione $y = f(x)$, che prende il nome di traiettoria
\end{itemize}
\subsubsection{Numeri complessi}
\textbf{Forma esponenziale}: $\vec{P} = r e^{j \theta}$.\newline
\textbf{Forma trigonometrica}: $\vec{P} = r cos(\theta) + j r sin(\theta)$.\newline
\newline
Per passare da una forma all'altra si usano le \textbf{relazioni di eulero}: $e^{j \theta} = cos(\theta) + j sin(\theta)$.\newline
\newline
Notiamo che $e^{j 0} = 1$, $e^{j \frac{\pi}{2}} = j$, e $e^{j \pi} = -1$.\newline
\newline
\textbf{oss.} Quando si lavora coi numeri complessi, per esempio nel cercare di definire la posizione di un certo punto, si cerca sempre di ottenere il risultato tramite somme di vettori, ciascuno dei quali varia nel tempo o il modulo o l'angolo, non entrambe assieme. E' preferibile usare più vettori "semplici", piuttosto che meno vettori "complessi". Utilizzare questo approccio facilita spesso i conti delle derivate per il calcolo della velocità e dell'accellerazione.\newline
\newline
\textbf{oss.} Ricordarsi sempre di definire gli angoli correttamene: si parte dall'asse reale e si ruota in senso antiorario.