\title{Esercitazione 2 02/04/2020}\newline
\textbf{link} \href{https://web.microsoftstream.com/video/490e9a91-e0f9-4769-a4ce-f9fc559ed5c1}{Clicca qui} [... praticamente solo audio]
\section{Esercitazione III}
\url{../esercitazione3/pdf/03-Cinematica cr-scala che scivola.pdf}\newline
\url{../esercitazione3/pdf/ese_3_notes.pdf}
\subsection{Tipico procedimento di risoluzione di un problema coi numeri complessi}
\begin{itemize}
    \item Disegnare il problema.
    \item Identificare i dati del problema. Porre attenzione alle variabili costanti e a quelle variabili nel tempo.
    \item Calcolare la posizione nei punti richiesti:
    \begin{itemize}
        \item Identificare i punti richiesti.
        \item Trovare una somma di vettori che abbiano ciascuno o modulo costante o angolo costante.
        \item Disegnare i vettori scelti.
        \item Scrivere l'equazione della somma di vettori in forma complessa (es: $a e^{i \alpha} = b e^{i \beta} + c e^{i \gamma}$).
        \item Ricordarsi che gli angoli ($\alpha, \beta, \gamma\dots$) vanno presi partendo dall'asse reale e girando in senso antiorario.
        \item Sostituire i dati del problema noti.
        \item Semplificare l'equazione con eventuali proprietà dei numeri complessi.
        \item Dividere parte reale e parte immaginaria (proiettare sull'asse delle $X$ e delle $Y$ l'equazione) in modo da ottenere un sistema.
        \item Ricavare dal sistema i valori richiesti.\newline 
        [Se il sistema è facilmente risolvibile si procede coi calcoli normalmente, cioè per sostituzione. Se il sistema non è facilmente risolvibile, si passa in forma matriciale che contenga una matrice, che chiameremo $A$ che moltiplica il vettore delle variabili incognite e il tutto che uguaglia il vettore dei termini noti (nullo se non ce ne sono). Per risolvere il sistema è sufficiente portare dall'altro lato dell'uguale la matrice $A$ come $A^{-1}$.]
        \item verificare se i risultati sono sensati.
    \end{itemize}
    \item Calcolare la velocità nei punti richiesti:
    \begin{itemize}
        \item Si deriva l'equazione della posizione.
        \item Si procede come per la posizione proiettando sull'asse delle $X$ e delle $Y$ l'equazione per ottenere un sistema.
        \item Dal sistema si ricavano le velocità cercate.
    \end{itemize}
    \item Calcolare l'accellerazione nei punti richiesti:
    \begin{itemize}
        \item Si deriva l'equazione della velocità.
        \item Si procede come per la velocità proiettando sull'asse delle $X$ e delle $Y$ l'equazione per ottenere un sistema.
        \item Dal sistema si ricavano le accellerazioni cercate.
    \end{itemize}
\end{itemize}