\title{Esercitazione 4 09/04/2020}\newline
\textbf{link} \href{https://web.microsoftstream.com/video/d71ffb36-a843-491f-a33d-da4a7d0b0af9}{Clicca qui}
\section{Esercitazione IV}
\url{../esercitazione4/pdf/04-Esercizi aggiuntivi.pdf}\newline
\url{../esercitazione4/pdf/04-Esercizi aggiuntivi.pdf}\newline
\url{../esercitazione4/pdf/ese_4_notes.pdf}
\subsection{Osservazioni sulla risoluzione di problemi col metodo dei moti relativi}
Nella risoluzione di esercizi con terne mobili occorre porre correttamente la terna e impostare l'equazione con il teorema dei moti relativi della velocità. Bisogna ora individuare le varie componenti dell'equazione (velocità di trascinamento, relative\dots) e in seguito per risolvere le incognite è sufficiente costruire un sistema uguagliando le componenti orizzontali con quelle verticali. Questo approccio è praticamente lo stesso usato per i numeri complessi quando si proiettano lungo gli assi le equazioni.\newline
Una volta ottenuta l'equazione, prima di disegnarne i vari termini, è utile fare una tabellina dei moduli e direzioni dei termini, specificando quali sono incognite e quali noti.\newline
Le stesse osservazioni si possono fare per le accellerazioni.