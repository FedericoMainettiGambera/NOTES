\section{LEZIONE 3 17/03/2020}
\textbf{link} \url{https://web.microsoftstream.com/video/b53b704d-3613-4728-9d95-917284509b0f}
\subsection{Analisi cinematica mediante osservatori in moto relativo}
\subsubsection{Moti relativi}
Fino ad ora abbiamo considerato sistemi di riferimento fissi, in questa lezione analiziamo sistemi di riferimento mobili rispetto a quello assoluto.\newline
\newline
\textbf{es.} [immagine dagli appunti del prof]\newline
Carrello su cui è incernierata una barra $AP$. Se andiamo a considerare un unico sistema di riferimento assoluto (in rosso), il moto del punto $P$ è un moto rototraslatorio, ma, se noi andassimo a inserire un sistema di riverimento mobile (in verde) che trasla insieme al punto $A$, il moto del punto $P$ diventa un moto di rotazione. Abbiamo dunque scomposto il moto rototraslatorio del punto $P$ nel moto traslatorio del nuovo sistema di riferimento mobile e nel moto rotatorio del punto $P$ in questo nuovo sistema di riferimento.\newline
\newline
\textbf{es.} [immagine dagli appunti del prof]\newline
Un asta $AB$ incernierata al pavimento che può ruotare, sull'asta può scivolare una seconda asta $CP$. Anche in questo caso il moto del punto $P$ rispetto a un unico sistema di riferimento assoluto (in rosso) sarebbe un moto rototraslatorio. Se però introduciamo un sistema di riferimento mobile (in verde) che abbia assi $Y_1$ e $X_1$ che ruotano insieme all'asse $AB$, il moto del punto $P$ diventa un moto traslatorio.\newline
\newline
[immagine dagli appunti del prof]\newline
Vogliamo descrivere il moto di un generico punto $P$ (velocità, posizione, accellerazione) andando a introdurre un nuovo sistema di riferimento mobile ($O_1, X_1, Y_1$) rispetto a un sistema di riferimento fisso ($O, X, Y$).\newline
Il sistema di riferimento mobile introdotto è in moto rototraslatorio noto rispetto al sistema di riferimento assoluto.\newline
Indichiamo con $\theta$ la rotazione del sistema di riferimento mobile rispetto al sistema di riferimento assoluto.
\subsubsection{Posizione}
Il punto generico $P$ ha coordinate $x_P, y_P$ rispetto al sistema assoluto, $x_{P,1}, y_{P,1}$ rispetto al sistema mobile. \newline
La posizione può essere vista come somma vettoriale (come abbiamo visto per i corpi rigidi): $(P-O) = (O_1 - O) + (P-O_1)$, dove $(P-O) = x_P \vec{i} + y_P \vec{j}$ e $(O_1 - O) = x_{O_1} \vec{i} + y_{O_1} \vec{j}$ e $(P-O_1) = x_{P,1} \vec{i_1} + y_{P,1} \vec{j_1}$. Se il moto del sistema di riferimento mobile è noto, conoscendo la posizione di $P$ rispetto al sistema assoluto, posso ricavare la posizione rispetto al sistema mobile, viceversa, conoscendo la posizione di $P$ rispetto al sistema mobile, posso ricavare la posizione rispetto al sistema assoluto.
\[
    (P-O) = x_{O_1}\vec{i} + y_{O_1}\vec{j} + x_{P,1} \vec{i_1} + y_{P,1} \vec{j_1}
\]
\subsubsection{Velocità}
$\vec{v_P} = \frac{d}{dt} (P-O) = \frac{d}{dt} (O_1 - O) + \frac{d}{dt} (P-O_1)$, sviluppando queste derivate, otteniamo: 
\[
    \vec{v_P} = 
        \dot{x}_{O_1}\vec{i} + 
        \dot{y}_{O_1} \vec{j} + 
        \dot{x}_{P,1} \vec{i_1} + 
        \dot{y}_{P,1} \vec{j_1} + 
        x_{P,1} \frac{d}{dt}\vec{i_1} + 
        y_{P,1} \frac{d}{dt} \vec{j_1}
\]
In teoria per i primi due termini se $\vec{i}$ e $\vec{j}$ variassero il loro orientamento rispetto al tempo dovrei derivare pure loro, ma siccome il sistema di riferimento assoluto fisso, la loro derivata è nulla. Per i secondi due termini invece i versori variano il proprio orientamento nel tempo perchè fanno riferimento a un sistema mobile, dunque non posso trascurare la loro derivata e per questo ci sono gli ultimi due addendi.\newline
La prima coppia di termini $\dot{y}_{O_1} \vec{j} + \dot{x}_{P,1} \vec{i_1}$ rappresenta la velocità $\vec{v}_{O_1}$ del punto $O_1$ rispetto al sistema di riferimento assoluto, anche detta velocità assoluta di $O_1$.\newline
La seconda coppia di termini $\dot{y}_{P,1} \vec{j_1} + x_{P,1} \frac{d}{dt}\vec{i_1}$ rappresenta la velocità $\vec{v}_{rel, P}$ relativa di $P$, cioè la velocità di $P$ rispetto al sistema di riferimento mobile.\newline
La terza e ultima coppia $x_{P,1} \frac{d}{dt}\vec{i_1} + y_{P,1} \frac{d}{dt} \vec{j_1}$ è più complicata da analizzare e dobbiamo prima capire cosa sono le derivate dei versori $\vec{i_1}$ e $\vec{j_1}$.\newline
[immagine dagli appunti del prof]\newline
Consideriamo i due sistemi di riferimento, mobile e assoluto, e poniamo due punti $A_1$ e $A_2$ sugli assi del sistema di riferimento mobbile a distanza unitaria dall'origine $O_1$. Questi due punti rappresentano i nostri versori $\vec{j_1}$ e $\vec{i_1}$.
\[
    (A_1 - O) = (O_1-O) + (A_1 - O_1) \;\text{, dove il termine $(A_1-O_1)$ è il versore $\vec{i_1}$}\;
\]
\[
    (A_2 - O) = (O_1-O) + (A_2 - O_1) \;\text{, dove il termine $(A_2-O_1)$ è il versore $\vec{j_1}$}\;
\]
Andiamo a derivare queste due equazioni per determinare la velocità dei punti $A_1$ e $A_2$. Facciamo i calcoli solo per $A_1$, per $A_2$ i procedimenti sono del tutto analoghi.
\[
    \frac{d}{dt}(A_1-O) = \frac{d}{dt}(O_1 - O) + \frac{d}{dt}(A_1 -O_1)
\]
Il termine $\frac{d}{dt}(A_1-O)$ rappresenta la velocità assoluta $\vec{v}_{A_1}$ di $A_1$, il secondo termine $\frac{d}{dt}(O_1 - O)$ è la velocità assoluta $\vec{v}_{O_1}$ di $O_1$, l'ultimo termine $\frac{d}{dt}(A_1 -O_1)$ è la derivata $\frac{d}{dt}\vec{i_1}$ rispetto al tempo del versore $\vec{i_1}$. Abbiamo quindi determinato la velocità assoluta del punto $A_1$.\newline
Potevamo ottenere il medesimo risultato usando il terorema di Rivals per un corpo rigido (il sistema di riferimento mobile), infatti la velocità di un punto qualsiasi di un corpo rigido è dato dalla somma di due componenti: una componente di traslazione di un punto generico del corpo (per esempio l'orgine $O_1$) e una componente di rotazione del corpo attorno al punto $O_1$. Se andiamo a definire il vettore velocità angolare $\vec{\omega} = \dot{\theta} \vec{k} = \omega \vec{k} = \omega \vec{k_1}$ del sistema mobile.
\[
    \vec{v}_{A_1} = \vec{v}_{O_1} + \vec{\omega} \land (A_1 - O_1) =\;\text{[dove $(A_1 - O_1) = \vec{i_1}$]}\; = \vec{v}_{O_1} + \vec{\omega}\land \vec{i_1}
\]
\newline
Riprendendo entrambi i metodi visti otteniamo:
\[
    \vec{v}_{A_1} = \cancel{\vec{v}_{O_1}} + \frac{d}{dt}\vec{i_1} = \cancel{\vec{v}_{O_1}} + \vec{\omega} \land \vec{i_1} \Longrightarrow \frac{d}{dt}\vec{i_1} = \vec{\omega}\land \vec{i_1}
\]
Procedendo in maniera analoga anche per il punto $A_2$, otteniamo le \textbf{formule di Poisson}
\[
    \begin{cases}
        \frac{d}{dt}\vec{i_1} = \vec{\omega}\land \vec{i_1}\\
        \\
        \frac{d}{dt}\vec{j_1} = \vec{\omega}\land \vec{j_1}
    \end{cases}
\]
Possiammo ora andare a sostituire all'interno della formula della velocità scritta in precedenza ($
    \vec{v_P} = 
        \dot{x}_{O_1}\vec{i} + 
        \dot{y}_{O_1} \vec{j} + 
        \dot{x}_{P,1} \vec{i_1} + 
        \dot{y}_{P,1} \vec{j_1} + 
        x_{P,1} \frac{d}{dt}\vec{i_1} + 
        y_{P,1} \frac{d}{dt} \vec{j_1}
$):
\[
    \vec{v_P} = 
        \vec{v}_{O_1} + \vec{v}_{rel, P} +
        x_{P,1} \vec{\omega}\land \vec{i_1} + 
        y_{P,1} \vec{\omega}\land \vec{j_1}
\]
dove i termini $x_{P,1} \vec{\omega}\land \vec{i_1} + y_{P,1} \vec{\omega}\land \vec{j_1}$ possono essere riscritti come $\vec{\omega}(x_{P,1} \vec{i_1} + y_{P,1} \vec{j_1})$, dove ciò fra parentesi è il vettore $(P-O_1)$.\newline
\newline
\textbf{teor.} Teorema dei moti relativi per le velocità
\[
    \vec{v_P} = 
        \vec{v}_{O_1} + \vec{v}_{rel, P} + \vec{\omega} \land (P-O_1) = 
        \vec{v}_{tr,P} + \vec{v}_{rel, P}
\]
dove la somma di $\vec{v}_{O_1} + \vec{\omega} \land (P-O_1)$ prende il nome di velocità di trascinamento $\vec{v}_{tr,P}$ del punto $P$, che è la velocità che il punto $P$ avrebbe se fosse rigidamente collegato al sistema di riferimento mobile.\newline
Questo teorema esprime la relazione fra la velocità assoluta di un punto $P$ e la velocità relativa a un sistema di riferimento in moto relativo rispetto a quello assoluto. La velocità assoluta è quindi la somma di due componenti, la velocità di trascinamento e la velocità relativa rispetto al sistema mobile.\newline
Quindi una volta noto il moto di trascinamento del sistema mobile è possibile passare dalla velocità assoluta a quella relativa e viceversa.
\subsubsection{Accellerazione}
Anche per l'accellerazione vogliamo cercare una relazione fra l'accellerazione del punto $P$ rispetto al sistema di riferimento assoluto e l'accellerazione del punto $P$ rispetto al sistema di riferimento mobile.\newline
