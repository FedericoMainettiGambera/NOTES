\section{LEZIONE 2 12/03/2020}
\textbf{link} \url{https://web.microsoftstream.com/video/7f38ca42-5bcf-4a08-b9cf-d42ca1289062}
\subsection{Cinematica di un corpo}
\subsubsection{Definizioni}
\begin{itemize}
    \item Corpo: Un corpo è un insieme continuo di infiniti punti che assume dimensioni finite.
    \item Posizione del corpo: è l'insieme di tutti i vettori posizione relativi a ciascun punto appartenente al corpo.
    \item Spostamento, velocità, accellerazione: definiziamo spostamento, velocità, accellerazione, l'insieme di tutti i vettori spostamento, velocità, accellerazione relativi a ciascun punto appartenente al corpo.
    \item Moto piano: in questo corso faremo sempre riferimento a un moto piano, che rappresenta il caso in cui tutti i vettori posizione, velocità e accellerazione di tutti i punti appartenenti al corpo sono paralleli a un piano, detto piano direttore.
    \item Spostamento infinitesimo: lo spostamento infinitesimo è una condizione di moto per cui ogni punto che appartiene al corpo subirà uno spostamento di dimensione infinitesima.
    \item Atto di moto: l'atto di moto è l'insieme delle velocità di tutti i punti che appartengono al corpo nell'istante di tempo geneico considerato. L'atto di moto rappresenta una "fotografia istantanea" del suo campo di velocità. Possiamo definire un'analogia fra lo spostamento infinitesimo e l'atto di moto: siccome la velocità di un generico punto $P$ è $\vec{v_P} = \frac{d \vec{P}}{dt}$, l'atto di moto può essere visto come lo spostamento infinitesimo di $P$ fratto l'intervallo di tempo infinitesimo $dt$ in cui esso avviene. Perciò tutte le regole cinematiche che definiremo per l'atto di moto varranno anche per lo spostamento infinitesimo.
\end{itemize}
Tutte le definizione appena viste valgono per un qualsiasi corpo, ma noi nel corso vedremo solo corpi rigidi.\newline
Per un corpo deformabile ci servono $\infty^2$ gradi di libertà (caso piano) per descrivere ciascuno degli infiniti punti che lo rappresentano.\newline
Nel caso di un corpo rigido saranno sufficienti $3$ gradi di libertà per definire la posizione del corpo nel piano.
\subsubsection{Corpo rigido}
Un corpo si definisce rigido se esso può definire solamente spostamenti rigidi. Uno spostamento si può definire rigido se a fronte di esso il corpo non subisce alcuna variazione nè di forma nè di dimensioni.\newline
[immagine dagli appunti del prof]\newline
Più analiticamente diciamo che uno spostamento è rigido se a seguito dello spostamento esiste un nuovo sistema di riferimento per cui la posizione del corpo rigido risulta la stessa di partenza.\newline
Se per esempio il corpo subisce un rimpicciolimento o una deformazione a seguito dello spostamento, non siamo in presenza di uno spostamento rigido.\newline
Ne conseguono due proprietà:
\begin{itemize}
    \item La distanza fra due punti qualsiasi di un corpo rigido si mantiene immutata.
    \item L'angolo formato dalle rette passanti fra due coppie di punti appartenenti al corpo rimane immutato.
\end{itemize}
Il principale vantaggio di studiare corpi rigidi è che dobbiamo usare solo 3 coordinate (3 gradi di libertà) per descrivere pienamente degli spostamenti.\newline
[immagine del professore]\newline
Per capire quali tre coordinate scegliere si seleziona un punto qualsiasi $A$ all'interno del corpo:
\begin{itemize}
    \item la prima è l'ascissa del punto $A$ (come per il punto), $x_A(t)$;
    \item la seconda è l'ordinata del punto $A$ (come per il punto), $y_A(t)$.
    \item La terza è la coordinata angolare $\phi$ di un segmento qualsiasi che collega il punto $A$ con un altro generico punto $B$ interno al corpo. Ogni punto $B$ mantiene invariata la sua distanza dal punto $A$ a seguito di un qualsiasi spostamento e, studiando come varia l'orientamento di questo segmento $AB$, sono in grado di ricostruire la posizione di ciascuno dei punti all'interno del corpo rigido. La rotazione $\phi$ avviene attorno ad un asse $z$ che esce dal piano del corpo rigido. Qualsiasi segmeno all'interno del corpo subirà la stessa variazione angolare (stessa rotazione): la rotazione $\phi$ è una proprietà dell'intero corpo rigido.
\end{itemize}
\subsubsection{Moto in grande}
\begin{itemize}
    \item \textbf{Traslazione}: é un moto nel quale un corpo non varia il proprio orientamento, ovvero in cui la coordinata angolare rimane costante. Tutti i punti del corpo subiranno lo stesso esatto spostamento, dunque $\vec{v_A} = \vec{v_B} = \dots$, $\vec{a_A} = \vec{a_B} = \dots$ e le traiettorie di ciascun punto saranno le stesse.
    \newline[immagine dagli appunti del prof]
    \item \textbf{Rotazione}: è un moto nel quale un punto (anche esterno al corpo), detto centro di rotazione, che mantiene la sua posizione fissa durante lo spostamento. Tutti gli altri punti invece subiranno una rotazione $\phi$. La traiettoria di ogni punto seguirà un moto circolare. Possiamo definire un vettore detto $\vec{\phi} = \phi \vec{k}$ ovvero con direzione uscente dal piano.
    \newline[immagine dagli appunti del prof]
    \item \textbf{Rototraslazione}: il corpo rigido andrà a modificare la propria posizione senza però che sia possibile individuare un punto che rimane fermo. Per studiare il moto rototraslatorio si può lavorare considerando due spostamenti successivi, prima una traslazione e poi una rotazione.
    \newline[immagine dagli appunti del prof]
\end{itemize}
\subsubsection{Atto di moto (moto in piccolo)}
Per atto di moto si intende un moto in cui gl ispostamenti e le rotazioni sono di dimensione infinitesima. L'atto di moto rappresenta una "fotografia istantanea" del campo di velocità del corpo.\newline
Andando ad osservare movimenti in piccolo quindi ci ritroviamo difronte a moti o rotatori o traslatori, non rototraslatori. Se la velocità di tutti i punti è uguale in modulo direzione e verso, l'atto di moto è di tipo traslatorio. Viceversa, se esiste un punto, detto centro di istantanea rotazione, in cui la velocità è nulla, siamo in presenza di un moto di tipo rotatorio.\newline
\textbf{es.} Esempio di rotazione:\newline
[immagine dagli appunti del prof]\newline
Presi i punti $A$ e $B$ interni al corpo e le rispettive velocità $\vec{v_A}$ e $\vec{v_B}$, essendo il corpo rigido, le proiezioni delle velocità sulla retta $r_{AB}$ devono essere di medesima lunghezza (nell'immagine evidenziate in giallo).\newline
Consideriamo ora le rette $r_A$ passante per $A$ e perpendicolare a $\vec{v_A}$ e la retta $r_B$ passante per $B$ e perpendicolare a $\vec{v_B}$; tutti i punti che si trovano sulla retta $r_A$ devono avere velocità perpendicolare alla retta stessa (analogo per la retta $r_B$), perchè altrimenti il corpo subirebbe una deformazione, quindi tutti i punti su $r_A$ (o $r_B$) devono avere velocità perpendicolare a $\vec{v_A}$ (o $\vec{v_B}$). Il punto $C$ di intersezione di queste due rette dovrebbe avere velocità perpendicolare sia a $\vec{v_A}$ sia a $\vec{v_B}$, ma questo è possibile solo se $\vec{v_C} = 0$, dunque il punto $C$ è il centro di istantanea rotazione del corpo rigido e siamo dunque in presenza di una rotazione.\newline
Il centro di istantanea rotazione differisce dal centro di rotazione (dei moti in grande), il primo ha velocità nulla solo nell'istante che stiamo considerando, mentre il secondo è fermo per tutto l'arco della rotazione. In poche parole il centro di istantanea rotazione ha velocità nulla, ma la sua accellarazione può non esserlo.\newline
\textbf{es.} Esempio di traslazione:\newline
[immagine dagli appunti del prof]\newline
In questo caso le rette $r_A$ e $r_B$ sono parallele e non si incontrano mai, il centro di istantanea rotazione non è definibile e dunque il moto e traslatorio.\newline
\textbf{es.} Esempio di moto di un corpo non rigido:\newline
[immagine dagli appunti del prof]\newline
Se i due punti $A$ e $B$ hanno $\vec{v_A}$ e $\vec{v_B}$ di modulo diverso, le proiezioni di queste due velocità sulla retta $r_{AB}$ non sono identiche e dunque il corpo si sta deformando.\newline
\textbf{es.} Un altro esempio di rotazione:\newline
[immagine dagli appunti del prof]\newline
Se i due punti $A$ e $B$ hanno $\vec{v_A}$ e $\vec{v_B}$ di direzione opposta, la congiungete fra le due velocità (disegnata in rosso) ci mostra che la velocità di tutti i punti lungo il segmeno $AB$ deve diminuire man mano che ci avviciniamo al punto $C$, che quindi ha velocità nulla e rappresenta il centro di istantanea rotazione.