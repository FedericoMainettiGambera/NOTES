\title{Esercitazione 7 07/05/2020}\newline
\textbf{link} \href{https://web.microsoftstream.com/video/0b9f9350-a8e9-4502-8400-a731263975b5}{Clicca qui}
\section{Esercitazione VII}
\url{../esercitazione7/pdf/07-Dinamica01.pdf}\newline
\url{../esercitazione7/pdf/ese_7_notes.pdf}
\subsection{Ripasso dinamica}
La dinamica studia il legame fra le forze e le coppie agenti su un sistema meccanico e il moto che ne deriva.
\subsubsection{Equazioni di D'alambert}
Le equazioni di D'Alambert sono equazioni di equilibrio dinamico, cioè avendo un corpo e delle forze e coppie applicate ad esso, ne consegue un moto descrivibile con un accellerazione del baricentro $\vec{a}_G$ e una certa velocità angolare $\dot{\omega}$.\newline
\newline
Forze e coppie di inerzia:
\[
    \begin{cases}
        F_{in} = -m \vec{a}_G\\
        C_{in} = - J_G \dot{\vec{\omega}}
    \end{cases}
\]
\ \newline
Le equazioni di D'Alambert sono:
\[
    \sum_i \vec{F}_i + \vec{F}_{in} = 0
\]
che proiettate sugli assi ci porta ad avere un sistema di due equazioni nel piano per ogni corpo.
\[
    \sum_i (P_i - O) \land \vec{F}_i + \sum_j \vec{C}_j + (G-O) \land \vec{F}_{in} + C_{in} = 0
\]
che è un equazione nel piano per ogni corpo.\newline
Quindi in totale si hanno $3 \cdot n_c$ equazioni, con $n_c$ il numero di corpi, che ci permettono di calcolare i legami dinamici tra le forze e coppie applicate del corpo e l'accellerazione e velocità angolare del copro, in più ci permettono pure di calcolare le reazioni vincolari.
\subsubsection{Bilancio di Potenze o teorema dell'energia cinetica}
Questo approccio alla dinamica è un approccio energetico.\newline
\newline
Abbiamo visto che la potenza $W$ delle forze e coppie attive applicate sul corpo sommati alla potenza $W_{in}$ delle forze e coppie di inerzia è nullo:
\[
    W + W_{in} = 0
\]
\ \newline
In alternativa possiamo scrivere
\[
    W_{in} = - \frac{d E_c}{dt}
\]
e quindi il teorema dell'energia cinetica:
\[
    W = \frac{d E_c}{dt}
\]
\ \newline
Questo approccio è più diretto perchè non ci richiede di calcolare le reazioni vincolari.