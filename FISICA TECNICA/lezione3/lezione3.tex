\section{LEZIONE 3 11/03/2020}
\textbf{link} https://web.microsoftstream.com/video/93bcc66f-72e2-4da5-8463-2b5a45e95ad1
\subsection{Slide: L02}
In questa lezione andremo a veder il primo e il seconod principio della termodinamica, e a definire le variabili di stati quali energia interna e entropia.\newline
Come principi di conservazione ne abbiamo tre, il primo è quello della massa. Un sistema chiuso non scambia massa e quindi la massa totale è sempre costante, per i sistemi aperti il discorso è differente, ma ne parleremo più avanti. Il secondo principio è quello della conservazione dell'energia (primo principio della termodinamica) e il terzo riguarrda la ocnservazione dell'entropia (secondo principio della termodinamica).\newline
Il primo principio della termodinamica per sistemi chiusi, che esprime la conservazione dell'energia, dice che (formula assiomatica): "per un sistema semplice all'equilibrio è definita una proprietà intrinseca (funzione di stato) detta energia interna U la cui variazione è il risultato di interazioni del sistema con l'ambiente esterno".
\[
    \Delta U = Q^{\leftarrow} - L^{\rightarrow}
\]
dove $\leftarrow$ indica uno scambio dall'esterno verso il sistema (entrante) e $\rightarrow$ indica uno scambio dal sistema verso l'esterno (uscente).\newline
Notiamo anche che $Q^{\leftarrow} = -Q^{\rightarrow}$.
Lavoro L = energia fornita ad un sistema termodinamico semplice che sia riconducibile alla variazione di quota di un grave.\newline
Calore Q = energia fornita ad un sistema termodinamico semplice che non è riconducibile alla variazione di quota di un grave.\newline
Una proprietà dell'energia interna totale di un sistema (energia interna riferita alla intera massa del sistema $M$) è che, oltre ad essere una funzione di stato, è una grandezza estensiva e perciò additiva: 
\[
    U = M \cdot u
\]
Vediamo ora il primo principio in forma differenziale:
\[
    du =  \delta q^{\leftarrow } - \delta l^{\rightarrow }
\]
dove $d$ è un differenziale esatto e $\delta$ indica il differenziale di una grandezza che non è una funzione di stato (dal punto di vista matematico hanno lo stesso significato, è solo una notazione usata per indicare se si sta parlando di funzioni di stato o meno).\newline
Essendo $U$ una grandezza estensiva e additiva, se il sistema $Z$ e fatto da due sottosistemi $A$ e $B$, l'energia interna totale è: 
\[
    U_Z = U_A + U_B
\]
In un sistema isolato (semplice o composto) i bilancio energetico diviene:
\[
    \Delta U_{\text{isolato}} = 0
\]
[sto prendendo appunti troppo precisi, aggiugerò solo le informazioni non scritte nelle slide...]
\newline[8]
\newline[9] Riguardo il primo principio della termodinamica abbiamo visto la definizione assiomatica che si focalizza sul bilancio energetico, e finchè questo bilancio vale, il principio rimane valido. La definizione classica del primo principio definisce l'energia interna dicendo che è tutta quell'energia che non è riconducibile allo spostamento di un oggetto o ad altre forme di energia. Questa definizione viene dall'esperimento di Joule: un recipiente riempito con acqua ferma, un elica viene azionato da un peso che fa muovere l'acqua, dopo un determinato tempo l'acqua torna a fermarsi.
\newline[10] le pareti (in rosso) sono dette adiabatiche, nel senso che sono isolate dallo scambio con l'esterno. Però il sistema ha la capacità di scambiare lavoro con l'esterno tramite il meccanismo dell'elica. L'energia cinetica nel nostro sistema è zero, così come l'energia potenziale (l'acqua non si è mossa verticalmente). Se guardiamo il principio di conservazione meccanica (una variazione di lavoro è data da una variazione di energia cinetica e/o una variazione di energia potenziale) otteniamo: $\delta L = d E_c + dE_p$. Nell'esperimento noi abbiamo fornito lavoro al sistema, quindi un $\delta L$ esiste, ma l'energia cinetica e potenziale sono nulle, di conseguenza per forza deve esistere una forma di energia che appunto chiameremo energia interna. Inoltre Joule si accorse che l'acqua aveva incrementato la sua temperatura, e quindi per forza c'era dell'energia di cui non era a conoscenza.
\newline[11] Joule ha eseguito molti esperimenti come quello visto precedentemente e ha cercto di determinare l'equivalenza tra il lavoro scambiato e il calore scambiato. Nell'immagine l'1 e il 2 rappresentano due stati, per esempio due temperature dell'acqua. Joule ha eseguito due cicli, in entrambi partendo dallo stato di temperatura 1 e arrivando allo stato 2. Nell'immagine il primo ciclo è "a" e "c", il secondo e "b" e "c". "c" rappresenta un raffreddamento dell'acqua (per esempio facendola raffreddare con l'esterno). "a" e "b" rappresentano due esperimenti (come quello precedente) in cui aumenta la temperatura dell'acqua con diverse quantità di lavoro. Ha visto che in entrambi i casi c'è un equivalenza di calore scambiati fra "a" e "c" o fra "b" e "c", così con vari calcoli è giunto alla conclusione che $\Delta U_{12a} = \Delta_{12b}$. Si ha una dimostrazione che la variazione di energia interna è una funzione di stato e cioè non dipende dalla trasformazione eseguita (non dipende dal percorso).
\newline[12]
\newline[13]
\newline[14]
\newline[15]
\newline[16]
\newline[17]
\newline[18]
\newline[19]
\newline[20]
\newline[21]
\newline[22]
\newline[23]
\newline[24]
\newline[25]
\newline[26]
\newline[27]
\newline[28]
\newline[29]
\newline[30]
\newline[31]
\newline[32]
\newline[33]
