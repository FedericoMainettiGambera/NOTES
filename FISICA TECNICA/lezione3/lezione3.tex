\section*{LEZIONE 3 11/03/2020}
\textbf{link} \url{https://web.microsoftstream.com/video/93bcc66f-72e2-4da5-8463-2b5a45e95ad1}
\subsection*{Slide: L02}
In questa lezione andremo a veder il primo e il seconod principio della termodinamica, e a definire le variabili di stati quali energia interna e entropia.\newline
Come principi di conservazione ne abbiamo tre, il primo è quello della massa. Un sistema chiuso non scambia massa e quindi la massa totale è sempre costante, per i sistemi aperti il discorso è differente, ma ne parleremo più avanti. Il secondo principio è quello della conservazione dell'energia (primo principio della termodinamica) e il terzo riguarrda la ocnservazione dell'entropia (secondo principio della termodinamica).\newline
Il primo principio della termodinamica per sistemi chiusi, che esprime la conservazione dell'energia, dice che (formula assiomatica): "per un sistema semplice all'equilibrio è definita una proprietà intrinseca (funzione di stato) detta energia interna U la cui variazione è il risultato di interazioni del sistema con l'ambiente esterno".
\[
    \Delta U = Q^{\leftarrow} - L^{\rightarrow}
\]
dove $\leftarrow$ indica uno scambio dall'esterno verso il sistema (entrante) e $\rightarrow$ indica uno scambio dal sistema verso l'esterno (uscente).\newline
Notiamo anche che $Q^{\leftarrow} = -Q^{\rightarrow}$.
Lavoro L = energia fornita ad un sistema termodinamico semplice che sia riconducibile alla variazione di quota di un grave.\newline
Calore Q = energia fornita ad un sistema termodinamico semplice che non è riconducibile alla variazione di quota di un grave.\newline
Una proprietà dell'energia interna totale di un sistema (energia interna riferita alla intera massa del sistema $M$) è che, oltre ad essere una funzione di stato, è una grandezza estensiva e perciò additiva: 
\[
    U = M \cdot u
\]
Vediamo ora il primo principio in forma differenziale:
\[
    du =  \delta q^{\leftarrow } - \delta l^{\rightarrow }
\]
dove $d$ è un differenziale esatto e $\delta$ indica il differenziale di una grandezza che non è una funzione di stato (dal punto di vista matematico hanno lo stesso significato, è solo una notazione usata per indicare se si sta parlando di funzioni di stato o meno).\newline
Essendo $U$ una grandezza estensiva e additiva, se il sistema $Z$ e fatto da due sottosistemi $A$ e $B$, l'energia interna totale è: 
\[
    U_Z = U_A + U_B
\]
In un sistema isolato (semplice o composto) i bilancio energetico diviene:
\[
    \Delta U_{\text{isolato}} = 0
\]
[sto prendendo appunti troppo precisi, aggiugerò solo le informazioni non scritte nelle slide...]
\newline\textbf{[8]}\;
\newline\textbf{[9]}\; Riguardo il primo principio della termodinamica abbiamo visto la definizione assiomatica che si focalizza sul bilancio energetico, e finchè questo bilancio vale, il principio rimane valido. La definizione classica del primo principio definisce l'energia interna dicendo che è tutta quell'energia che non è riconducibile allo spostamento di un oggetto o ad altre forme di energia. Questa definizione viene dall'esperimento di Joule: un recipiente riempito con acqua ferma, un elica viene azionato da un peso che fa muovere l'acqua, dopo un determinato tempo l'acqua torna a fermarsi.
\newline\textbf{[10]}\; le pareti (in rosso) sono dette adiabatiche, nel senso che sono isolate dallo scambio con l'esterno. Però il sistema ha la capacità di scambiare lavoro con l'esterno tramite il meccanismo dell'elica. L'energia cinetica nel nostro sistema è zero, così come l'energia potenziale (l'acqua non si è mossa verticalmente). Se guardiamo il principio di conservazione meccanica (una variazione di lavoro è data da una variazione di energia cinetica e/o una variazione di energia potenziale) otteniamo: $\delta L = d E_c + dE_p$. Nell'esperimento noi abbiamo fornito lavoro al sistema, quindi un $\delta L$ esiste, ma l'energia cinetica e potenziale sono nulle, di conseguenza per forza deve esistere una forma di energia che appunto chiameremo energia interna. Inoltre Joule si accorse che l'acqua aveva incrementato la sua temperatura, e quindi per forza c'era dell'energia di cui non era a conoscenza.
\newline\textbf{[11]}\; Joule ha eseguito molti esperimenti come quello visto precedentemente e ha cercto di determinare l'equivalenza tra il lavoro scambiato e il calore scambiato. Nell'immagine l'1 e il 2 rappresentano due stati, per esempio due temperature dell'acqua. Joule ha eseguito due cicli, in entrambi partendo dallo stato di temperatura 1 e arrivando allo stato 2. Nell'immagine il primo ciclo è "a" e "c", il secondo e "b" e "c". "c" rappresenta un raffreddamento dell'acqua (per esempio facendola raffreddare con l'esterno). "a" e "b" rappresentano due esperimenti (come quello precedente) in cui aumenta la temperatura dell'acqua con diverse quantità di lavoro. Ha visto che in entrambi i casi c'è un equivalenza di calore scambiati fra "a" e "c" o fra "b" e "c", così con vari calcoli è giunto alla conclusione che $\Delta U_{12a} = \Delta U_{12b}$. Si ha una dimostrazione che la variazione di energia interna è una funzione di stato e cioè non dipende dalla trasformazione eseguita (non dipende dal percorso).
\newline\textbf{[12]}\; Il secondo principio della termodinamica ha diverse formulazioni, qui presentiamo quella assiomatica. L'entropia è identificata con la lettera $S$. Dal punto di vista ingegneristico l'entropia è quella grandezza, quel numero che ci permette di dire se quello che pensiamo di fare a livello di trasformazioni è fattibile o no e se lo stiamo facendo in maniera ottimale o meno. L'entropia è uno strumento fondamentale qualora vogliamo migliorare un sistema e trovare una soluzione ideale.
\newline\textbf{[13]}\; Anche l'entropia è una grandezza estensiva. Con la $s$ minuscola identifichiamo l'entropia specifica rispetto alla massa. C'è un forte legame fra l'entropia e la trasformazione reversibile (che è quella ideale).
\newline\textbf{[14]}\; E' una grandezza adittiva.
\newline\textbf{[15]}\; $S_Q^{\leftarrow  }$ è l'entropia entrante, cioè la variazione di entropia legata allo scambio di calore, di conseguenza il segno di $S_Q^{\leftarrow  }$ è uguale al segno di $Q^{\leftarrow }$. $S_{irr}$ è l'entropia generata per via delle irreversibilità.
\newline\textbf{[16]}\; Il primo principio ci permette di quantificare gli scambi di calore e di lavoro nel sistema. cioè stabilisce le equivalenze fra calore e lavoro, però non pone nessun limite sul poter trasformare il calore in lavoro. Il secondo principio è stato introdotto per tradurre ciò che estato osservato durante gli esperimenti in maniera matematica con dei vincoli.
\subsection*{Slide: L03}
\textbf{[1]}\;
\newline\textbf{[2]}\; Funzioni di stato e variabili di stato sono la stessa cosa. Per variabile di stato si intende una grandezza che dipende dallo stato del sistema, viceversa variabili come il lavoro e il calore (che non sono di stato) non dipendono dallo stato del sistema, ma bensì dal percorso che hanno seguito per raggiungere quel determinato stato. Queste slide sono fondamentali per riuscire a trattare gli esercizi dell'esame.
\newline\textbf{[3]}\; Una quantità di energia meccanica (Lavoro) può essere calcolata da una forza e uno spostamento, se traduciamo questo concetto in termodinamica otteniamo la formula scritta sulle slide. Notiamo che $P$ è la pressione e $A$ è un area e $P \cdot A$ è una forza. $ds$ è uno spostamento infiniteimo. Quindi forza per spostamento ci dà il lavoro. Possiamo però riscrivere $A \cdot ds$ come una variazione infinitesima di volume, e quindi $dV$. 
\newline\textbf{[4]}\; si può anche scrivere il in maniera specifica (le lettere minuscole significa che sono grandezze specifiche, in questo caso specifiche rispetto alla massa). Se il sistema evolve da uno stato iniziale e uno stato finale attraverso una succesione di stati di equilibrio (quindi descrivibili con pressione e temperatura, e quindi evolve lentamente) allora c'è l'equazione della trasformazione (la formula scritta nella slide) che può essere integrata.
\newline\textbf{[5]}\; appunto si può integrare
\newline\textbf{[6]}\; questa trasformazione isoterma reversibile è un espansione (volume iniziale minore di volume finale e pressione iniziale maggiore di pressione finale). Durante tutta la trasformazione la temperatura rimane la stessa, per cui le pareti non sono adiabatiche, se fossero adiabatiche, durante un espansione la temperatura dovrebbe scendere. Guardiamo l'equazione di stato di un gas ideale $PV = MR^{*}T$ in cui la massa è costante perchè il sistema è chiuso, $R^*$ è una costante del gas ideale, inoltre anche la temperatura di questa trasformazione è costatnte, dunque il termine a destra dell'uguale è costante e quindi possiamo ricavare la funzione matematica della pressione in funzione del volume: $P(V) = \frac{MR^*T}{V}$ (funzione iperbolica). Quest'ultima funzione  è integrabile, da cui otteniamo la formula scritta nella slide.
\newline\textbf{[7]}\; Vediamo la differenza fra trasformazione reversibile e irreversibile. Con trasformazione reversibile spesso si intende una trasformazione lenta (con trasformazioni infinitesimali, cioè in cui la variazione di spostamento del pistone azzurro è minore della dimensione delle molecole che compongono il sistema, si lascia tempo al sistema di tornare a uno stato omogeneo, cioè in cui lo stato del sistema è ben definibili). Una trasformazione irreversibile invece è spesso una trasformazione veloce, in cui, per esempio, fra un istante e l'altro la pressione non è omogenea nel sistema. Si capisce bene guardando la seconda immagine: il pistone blu si sta spostando verso sinistra, la zona rossa rappresenta la parte in cui la pressione è rimasta quella iniziale, invece la zona azzurra e tratteggiata è la zona in cui la pressione è maggiore della pressione iniziale. Quindi lo stato del sistema non è ben definito (non sappiamo dire quanto sia la pressione totale, perchè non è uguale ovunque) e calcolare il lavoro non è facile, ma fortunatamente la pressione dell'ambiente che consideriamo come pressione finale $P_f$ rimane ben definita e possiamo usarla per calcolare il lavoro come $P_f \cdot A$ (rimane un'approssimazione).
\newline\textbf{[8]}\; 
\newline\textbf{[9]}\; Confrontiamo i due casi. Nel caso reversibile il lavoro è l'area sotto alla curva della pressione (di colore verde, che è un integrale), nel caso irreversibile il lavoro compiuto sarà la massa moltiplicata per l'area al di sotto dello stato finale (di colore rosso). Nella maggior parte dei casi il lavoro irreversibile è minore del lavoro reversibile (per un espansione, il contrario per una compressione). 
\newline\textbf{[10]}\; La variazione di altitudine (funzione di stato) dallo stato iniziale allo stato finale utilizzando un cammino o l'altro non cambia, ma invece il tempo (non funzione di stato) per esempio dipende dal cammino intrappreso. 
\newline\textbf{[11]}\; 
\newline\textbf{[12]}\; Un ciclo è un trasformazione che termina con lo stato iniziale. In funzione di come avviene il ciclo, orario o antiorario per esempio, avremo che il lavoro uscente dal sistema è positivo o negativo. Nel piano $PV$ chiamiamo macchine a ciclo diretto (motrici) quelle che eseguono trasformazioni in senso orario, mentre chiameremo macchine a ciclo inverso (operatrici) quelle che eseguono trasformazioni in senso antiorario.
\newline\textbf{[13]}\; Il calore non è temperatura (per esempio il corpo umano ha vari nervi e non percepisce la temperatura dell'ambiente, ma piuttosto il flusso termico che abbiamo sulla pelle, per esempio l'aria di una sauna a 70 gradi non ci fa stare male, ma l'acqua calda a 70 gradi ci ustiona). Il calore dipende dalle sostanze che stiamo considerando e anche dalla differenza di temperatura. Per calcolare il calore abbiamo diversi strumenti, uno dei più fondamentali è la capacità termica (ritornando agli esempi di prima possiamo dire che diverse sostanze hanno capacità termica diversa). Spesso non useremo la capacità termica, ma il calore specifico ,che è la medesima grandezza ma divisa per la massa del sistema. 
\newline\textbf{[14]}\; diventeremo molto familiari con queste formule. Se siamo in grado di misurare queste due (prima riga) grandezze allora possiamo calcolare i due coefficienti (seconda riga) che sono quindi proprietà intrinseche di una sostanza.
\subsection*{Slide: P03}
\textbf{[1]}\;
\newline\textbf{[2]}\; ci sono scambi di energia elettrica e termica verso l'esterno (prelievo di lavoro dall'ambiente, cioè energia elettrica,  e rilascio di calore verso l'ambiente).
\newline\textbf{[3]}\; Gli scambi sono quindi due, scambio di calore $Q_{DC}$ e scambio di lavoro $L_{DC}$. Nel tempo la temperatura media del sistema di calcolo non varia, quindi il $\Delta U_{DC} = 0$ e di conseguenza $Q_{DC}^\leftarrow - L_{DC}^\rightarrow  = 0$.
\newline\textbf{[4]}\;
\newline\textbf{[5]}\; $L_DC$ è la quantità di lavoro prelevato dall'ambiente (energia elettrica presa dalla rete elettrica).
\newline\textbf{[6]}\; 
\newline\textbf{[7]}\; PUE è il rapporto della potenza utilizzata da tutto il centro di calcolo e l'effettiva potenza usata dalle funzionalità IT.
\newline\textbf{[8]}\;
\newline\textbf{[9]}\; E' possibile far rientrare i costi sfruttando l'energia che rilasciamo all'ambiente (calore a bassa temperatura)?