\section{LEZIONE 2 05/03/2020}
\textbf{link} \url{https://web.microsoftstream.com/video/2f03bda0-02f4-4ce6-ac58-77bcaaa3cc52}
\subsection{Slide: L01}
\subsubsection{Esempi}
[14] A sinistra centrale a carbone, in centro in basso una centrale termoelettrica, uno degli impianti più efficienti esistenti, a destra una centrale nucleare.\newline
[15] Vediamo ora sistemi rinnovabili. A sinistra energia eolica che trasforma energia meccanica in energia elettrica (l'energia eolica non è affidabile, perchè non è costante). In centro in basso non sono pannelli fotovoltaici, ma tanti specchi che concentrano il calore sulla punta della torre, se ne facciamo un confronto con la centrale termoelettrica della slide precendente, ci accorgiamo che è poco efficiente. Sulla destra c'è la centrale di energia rinnovabile più tipica, è una centrale idroelettrica. Le centrali idroelettriche rappresentano il 90\% dell'energia elettrica rinnovabile prodotta. Quella nell'immagine è la centrale elettrica (cinese) capace di erogare la maggiore potenza di tutte quelle nel mondo, anche più di quelle non rinnovabili.\newline
[16] a sinistra un motore di un automobile (circa 500 cavalli), in centro in basso un motore di aereo, a destra un motore diesel delle cosiddette applicazioni pesanti, in questo caso di una nave.\newline
[17] a sinistra un impianto di ventilazione industriale, a destra un centro di calcolo con un sistema di raffreddamento, in centro in basso un impianto per la pastorizzazione di cibi.\newline
[18] a sinistra un condizionatore e a destra un frigorifero (da notare che i 200W non sono usati costantemente, ma rappresenta la potenza istantanea).\newline
\subsubsection{Definizioni}
Queste slide andavano viste autonomamente, le trattiamo velocemente.\newline
[19-20] definizioni varie \newline
[21] isotropo: che presenta le stesse proprietà in tutte le direzioni (es. il calore si diffonde omogeneamente all'interno dello spazio di osservazione). Non sempre considereremo sistemi semplici, alcune volte certe caratteristiche non sono ignorabili. L'ultimo punto significa che non andremo a considerare effetti locali a livello molecolari.\newline
[22-23-24] \newline
[25] per determinare lo stato di equilibrio di un sistema si usa la Legge di Duhem.\newline
[26] La regola di Duhem ci dice che C=2 all'interno della formula di Gibbs \newline
[27] equazione di stato ci dice che se conosciamo due grandezze, la terza è nota, ma in molti casi l'equazione di stato è ignota.\newline
[28]\newline
[29] nelle prime lezioni vedremo principalmene sistemi chiusi. Nell'immagine contorno fisico in grigio e contorno immaginario in rosso.\newline
[30] una trasformazione termodinamica è l'insieme degli stati d'equilibrio intermedi quando il nostro sistema termodinamico è soggetto a variazioni, cioè tutti gli stati del sistema a partire da quello iniziale fino a quello finale a seguito di una variazione.\newline
[31]
\subsubsection{Equzione di stato}
[32]\newline
[33-34] L'equazione di stato è nota solo per la regione in blu dei gas ideali e vale PV = NRT.\newline
[35] Se il sistema non è nella regione dei gas ideali allora la formula vista prima non vale più. Nella regione dei gas ideali per l'equazione di stato si usa il modello di van del Waals (ma ne esistono altre, va usato quello più adatto alla sostanza in oggetto).\newline
[36] per liquidi e solidi non esiste un modello di equazione di stato, ma avendo molte misure sperimentali abbiamo calcolato due coefficienti $\beta$ e $K_T$. Quindi anche se non conosciamo f(P,v,T)= 0, possiamo determinare la variazione di una grandezza rispetto alle altre v = v(P,T)\newline
[37] spesso si usa un modello semplificato, come si vede nell'immagine la parte solida e liquida è ripida e si usa un modello con solidi e liquidi incomprimibili.
\subsection{Slide: P02}
[3] In alto: articolo da leggere. Immagine: schema di un centro di calcolo ideale. Idea di cercare di conservare il calore rilasciato dal datacenter per riscaldare le abitazioni.\newline
[4] Un Rack del nostro datacenter può essere considerato come un sistema aperto.\newline
[5] Ogni rack ha 16 server nel nostro progetto. \newline
[6] Avremo 40 rack nella stanza detta IT room. Nella IT room c'è anche un'unità CRAC che possiamo vedere, pure lui, come sistema aperto. La stanza intera, invece, possiamo considerarlo come un sistema chiuso (ovviamente le pareti non saranno del tutto isolate dall'esterno, ma queste perdite sono minime, quindi ignorabili).\newline
[7] La IT support area è la zona con tutte le strumentazioni per consentire lo svolgimento delle operazioni in modo idoneo, quindi avremo un sisema di raffreddamento che sarà poi collegato al CRAC, abremo anche UPS (batterie che in caso di interruzione dell elettricità possono mantenere il nostro sitema attivo), sistema di distribuzione che avrà perdite etc, ma per ora possiamo trascurare quest'ultimo.\newline
[8] La terza zona la chiameremo spazio ausiliare-\newline
[9] Tutte le zone assieme.\newline
[10]