\section{Stati bifase: grandezze di stato e trasformazioni}
Lo \textbf{stato bifase} di una sostanza pura corrisponde alla coesistenza di due diverse fasi (solidoliquido, solido-vapore o liquido-vapore) in equilibrio tra loro (\textbf{stato eterogeneo}). Un sistema
in uno stato eterogeneo può essere pensato come un insieme di più sottosistemi in stati
omogenei. Si definisce \textbf{fase} di un sistema eterogeneo, il sottosistema omogeneo caratterizzato
dagli stessi valori di tutte le proprietà specifiche. Un sistema eterogeneo è in uno stato di
equilibrio se tutte le sue proprietà intensive hanno valori ovunque uguali (cioè sono uguali
per tutte le fasi). \newline
\newline
Si definisce \textbf{transizione di fase} la trasformazione che porta un sistema che si trova in uno
stato omogeneo a separarsi in due o più fasi. Le transizioni di fase avvengono a \textbf{temperatura} e \textbf{pressione costanti}.\newline
\newline
Un sistema monocomponente in uno stato bifase
ha un solo grado di libertà (nella regola di Gibbs: $V=1$). \newline
\newline
La pressione P e la temperatura T di un sistema monocomponente bifase sono infatti legate,
in qualsivoglia passaggio di stato, dalla relazione di Clausius-Clapeyron: 
\[
    \left(\frac{dP}{dT}\right)_{a \rightarrow b} = \frac{(s_a-s_b)}{(v_a-v_)b} \;\;\;\;\;\;\;\;\;\;\;\;\;\;\; \left( \frac{dP}{dT} \right)_{a \rightarrow  b} = \frac{(h_a-h_b)}{T(v_a-v_b)}
\]
Lo stato termodinamico di un sistema monocomponente bifase è quindi compiutamente
descritto da una variabile intensiva, essendo l'altra univocamente definita, e da una variabile
estensiva specifica o anche dalla coppia estensiva-estensiva. \newline
\newline
Di qui le varie rappresentazioni grafiche per il bifase: (P, v), (T,s), (P,h), (h,s) che contengono
sempre almeno una quantità estensiva specifica. 
\subsection{Proprietà termodinamiche estensive}
Faremo ora sempre assunzione allo stato bifase \textbf{liquido-vapore}.\newline
\newline
Indicando con $Z$ una generica grandezza estensiva, per una miscela bifase liquido-vapore si ha
\[
    Z = Z_l + Z_v
\]
cioè il valore della proprietà $Z$ per la miscela è pari alla somma del valore della proprietà estensiva della fase liquida $Z_l$ e della fase vapore $Z_v$. Adottando le grandezze estensive specifiche corrispondenti di $Z$, la relazione precedente può essere riscritta come 
\[
    Z = M_l z_l + M_v z_v
\]
dove $M_l$ e $M_v$ sono rispettivamente la massa di liquido e quella di vapore.\newline
\newline
Definendo \textbf{titolo} $x$ (espresso in massa) il rapporto tra la massa di vapore e la massa totale conenuta nel sistema:
\[
    x = \frac{M_v}{M_l+M_v}
\]
si ottiene la seguente relazione valida per ottenere il valore di qualsivoglia grandezza
estensiva specifica di un sistema bifase avendo noto il titolo: 
\[
    z = z_l(1-x) + xz_v \;\;\;\;\;\text{oppure}\;\;\;\;\; z = z_l + x z_{lv}
\]
dove $z_{lv}$ rappresenta la variazione della proprietà estensiva specifica nella transizione di fase. \newline
Con questa formula si possono esprimere \textbf{l'energia interna, il volume, l'entalpia e l'entropia specifiche} di un sistema bifase.\newline
\newline
Come si può notare per valutare il valore assunto da una grandezza estensiva specifica in un
sistema bifase è necessario conoscere il valore assunto dalla grandezza estensiva in
condizione di saturazione (liquido e vapore).
\subsection{Tabelle}
Questi valori sono riportati su tabelle. Le tabelle termodinamiche sono di due tipi: 
\begin{itemize}
    \item  tabelle dello \textbf{stato eterogeneo} dove sono riportate le proprietà termodinamiche del liquido
    saturo e del vapore saturo; il dato di ingresso (argomento) è o la temperatura o la pressione. \newline
    I dati in corrispondenza di questa sono : \newline
    \begin{itemize}
        \item la pressione o la temperatura di saturazione (univocamente definita); 
        \item il volume specifico del liquido e del vapore saturi e la loro differenza ($m^3/kg$); 
        \item la entalpia specifica del liquido e del vapore saturi e la loro differenza ($kJ/kg$);
        \item la entropia specifica del liquido e del vapore saturi e la loro differenza ($kJ/kgK$);
        \item l'energia interna specifica del liquido e del vapore saturi e la loro differenza ($kJ/kg$);
    \end{itemize}
    L' ultima grandezza non è sempre presente, ma è comunque valutabile dal legame
    esistente tra l’entalpia e l’energia interna ($h= u+ Pv$). 
    \item tabelle dello stato omogeneo che portano le proprietà del vapore surriscaldato; i dati di
    ingresso sono la temperatura e pressione (le due intensive che definiscono lo stato).\newline
    I dati in
    corrispondenza della particolare pressione sono:
    \begin{itemize}
        \item il volume specifico del vapore surriscaldato ($m^3/kg$);
        \item la entalpia specifica del vapore surriscaldato ($kJ/kg$);
        \item la entropia specifica del vapore surriscaldato ($kJ/kgK$);
        \item la energia interna specifica del vapore surriscaldato ($kJ/kg$);
    \end{itemize}
    L' ultima grandezza non è sempre presente, ma è comunque valutabile dal legame
    esistente tra l’entalpia e l’energia interna ($h= u+ Pv$). 
\end{itemize}
\ \newline
Qualora lo stato termodinamico di cui si stanno valutando le proprietà termodinamiche non
coincide con gli stati discreti riportati sulle tabelle è necessario eseguire una \textbf{interpolazione}
dei dati.\newline
\newline
\textbf{Interpolazione lineare}:
\[
    Y = Y_A + \frac{Y_B - Y_A}{X_B - X_A} (X - X_A)
\]
con $Y$ grandezza che si vuole ricavare, $X$ grandezza conosciuta e $A,B$ stati di riferimento (presenti in tabella, $X_A<X<X_B$).\newline
Oppure si può usare anche:
\[
    \frac{X_X - X_A}{X_B - X_A} = \frac{Y_X - Y_A}{Y_B - Y_A}
\]
\textbf{Interpolazione bilineare}:
Nel caso in cui più di una grandezza non corrisponda a nessun valore preciso della tabella usiamo la formula di \textbf{interpolazione bilineare}
\[
    Y = Y_A + \frac{Y_B - Y_A}{X_B - X_A}(X - X_A)
\]
\[
    Y_A = Y_{A1} + \frac{Y_{A2}- Y_{A1}}{X_{A2}- X_{A1}} (X_A - X_{A1})
\]
\[
    Y_B = Y_{B1} + \frac{Y_{B2}- Y_{B1}}{X_{B2}- X_{B1}} (X_B - X_{B1})
\]
\ \newline
L’utilizzo delle tabelle termodinamiche consente di valutare le proprietà di una sostanza,
normalmente, in condizioni di bifase liquido-vapore o nello stato di vapore surriscaldato.
L’analisi di alcuni processi può richiedere informazioni sulle proprietà termodinamiche in
stati diversi quali liquido sottoraffreddato o solido. \newline
In questi casi, se non sono disponibili tabelle specifiche, occorre ricorrere a valutazioni
approssimate che si ottengono con l’ausilio di equazioni di stato. \newline
Occorre porre attenzione che l’utilizzo contemporaneo di dati estratti da una tabella o valutati
con l’ausilio di relazioni approssimate richiede l’adozione dello stesso stato di riferimento
rispetto a cui valutare le grandezze di interesse (entalpia, entropia, energia interna, etc.). \newline
In prima approssimazione, ipotizzando che lo stato di riferimento sia lo stato triplo e fase
liquida, si possono seguire queste indicazioni: 
\begin{itemize}
    \item \textbf{Stato di liquido sottoraffreddato}:
    \begin{itemize}
        \item entalpia: \[
            h(T,P) = h(T,P_s) + v (P-P_s)
        \]
        \[
            h(T,P) \sim h(T,P_s)
        \]
        \item entropia: \[
            s(T,P) = s(T,P_s)
        \]
    \end{itemize}
    dove $h(T,P_s)$ e $s(T,P_s)$ sono l'entalpia e l'entropia dello stato di liquido saturo alla temperatura $T$ (e alla conseguente pressione di saturazione $P_s$).
    \item \textbf{Stato di solido (ghiaccio)}:
    \begin{itemize}
        \item entalpia:\[
            h(T,P) = h_{ls,t} + c_s(T-T_t) + v(P-P_t)
        \]
        \item entropia:\[
            s(T,P) = s_{lv,t} +c_s ln \frac{T}{T_t}
        \]
    \end{itemize}
    dove $h_{ls,t}$ e $s_{ls,t}$ sono l'entalpia e l'entropia associate alla transizione di fase liquido-solido allo stato triplo, $T_t$ e $P_t$ sono la temperatura e la pressione dello stato triplo mentre $c_s$ è il calore specifico della fase solida.\newline
    Per la transizione di fase liquido-solido dell'acqua ($T_t = 273.16 K, P_t = 0.6122kPa$) si ha:
    \[
        h_{ls,t} = -333kJ/kg
    \]
    \[
        s_{ls,t} = \frac{h_{ls,t}}{T_t} \;\;\;\;\;\;\;\;\;\;s_{ls,t} = \frac{-333}{273.16}kJ/kgK = -1.219 kJ/kgK
    \]
    \[
        c_s = 2.093 kJ/kgK
    \]
\end{itemize}