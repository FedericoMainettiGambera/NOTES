\section{Equazioni di bilancio per i sistemi chiusi}
Un \textbf{sistema chiuso} è un sistema per il quale non sono consentiti scambi di massa attraverso il suo contorno. Per tale sistema non è in uso applicare \textbf{equazioni di bilancio di massa} essendo implicita la sua conservazione nella definizione di sistema chiuso.\newline
\newline
Nelle soluzioni di problemi con sistemi chiusi le equazioni fondamentali sono \textbf{l'equazione di bilancio energetico} e \textbf{l'equazione di bilancio entropico} che assumono la forma:
\[
    \Delta U = Q^\leftarrow  -L^\rightarrow  
\]
\[
    \Delta S = S_Q + S_{irr}
\]
cioè la variazione di energia interna è pari alla differenza tra il calore entrante nel sistema e il lavoro ceduto dal sistema; la variazione di entropia è pari alla somma della entropia $S_Q$ entrante attraverso il contorno del sistema con il calore Q e della quantità $S_{irr}$ generata all'interno del sistema per irreversibilità. Questa ultima quantità è sempre positiva e tende a zero col tendere dei processi alla reversibilità. Inoltre il segno del termine $S_Q$ è sempre uguale al segno di $Q$.\newline
\newline
L'\textbf{energia interna} e l'\textbf{entropia} sono proprietà \textbf{estensive e additive}. Quindi dato un sistema $Z$ composto da due sottosistemi $A$ e $B$, posso scrivere:
\[
    \Delta U_Z = \Delta U_A + \Delta U_B = Q_Z^\leftarrow  - L_Z^\rightarrow  \;\;\;\;\;\;\;\;\;\;\Delta S_Z = \Delta S_A + \Delta S_B = S_{Q,Z} + S_{irr,Z}
\]
\ \newline
\newline
Se il \textbf{lavoro scambiato} è lavoro meccanico quasi-statico (internamente reversibile), è determinabile con l'espressione:
\[
    L = \int_{i}^{f} P dV
\]
Questa espressione è integrabile se è nota l'equazione della trasformazione ovvero la funzione:
\[
    P = P(V)
\]
\ \newline
Il \textbf{calore scambiato}, nell'ipotesi di trasformazione quasi-statica, può essere determinato avendo noto il calore specifico della trasformazione:
\[
    Q = M c_x (T_2-T_1)
\]
\ \newline
L'\textbf{energia interna} di un sistema in uno stato di equilibrio può essere espressa in funzione di altre proprietà intensive ed estensive specifiche del sistema come la temperatura, il volume specifico, la pressione. Particolarmente utili sono i due seguenti casi in cui è possibile esprimere in una forma analitica elementare il legame funzionale tra la variazione di energia interna tra due stati di equilibrio e la variazione di temperatura. \newline
\textbf{Gas perfetto}: $\Delta u = c_V (T_2 - T_1)$\newline
\textbf{Liquido incomprimibile perfetto}: $\Delta u = c(T_2-T_1)$\newline
Inoltre:
\begin{itemize}
    \item Per un sistema isolato $\Delta U_{isolato} = 0$.
    \item Per un sistema che subisce una trasformazione ciclica: $\Delta U_{ciclo} = 0$
\end{itemize}
\ \newline
\newline
Il caso di trasformazioni quasi-statiche a pressione costante o trasformazioni irreversibili in un sistema con stato iniziale e finale alla stessa pressione consente di scrivere l’equazione di bilancio energetico per il sistema chiuso come:
\[
    \Delta H = Q \;\;\;\;\;\;\;\;\;\;\;\;\;\;\;\text{dove $h = u + Pv$}\;
\]
Per esempio la miscelazione adiabatica e isobara di due sottosistemi produce uno stato finale caratterizzato da un valore di entalpia totale pari alla somma delle entalpia iniziale dei due sottosistemi. \newline
L'\textbf{entalpia specifica} di un sistema in uno stato di equilibrio può essere espressa
in funzione di altre proprietà intensive ed estensive specifiche del sistema come la
temperatura, il volume specifico, la pressione. Particolarmente utili sono i due seguenti casi
in cui è possibile esprimere in una forma analitica elementare il legame funzionale tra la
variazione di entalpia specifica tra due stati di equilibrio e la variazione di temperatura. \newline
\textbf{Gas perfetto}: $\Delta h = c_P (T_2 - T_1)$\newline
\textbf{Liquido incomprimibile perfetto}: $\Delta h = c(T_2-T_1) + v(P_2-P_1)$\newline
\newline
L'\textbf{entropia} è una proprietà del sistema e quindi dipende solo dallo stato del sistema: ne consegue che noti gli stati iniziale e finale di un processo, la determinazione della variazione di entropia prescinde dalla conoscenza di qualsiasi dettaglio del processo (ivi compresa la sua reversibilità). Particolarmente utili sono i due seguenti casi in cui è possibile esprimere in una forma analitica elementare il legame funzionale tra l'entropia e altre proprietà del sistema. Per il gas perfetto sono possibili 3 diverse espressioni che fanno uso di 3 diverse coppie di coordinate termodinamiche indipendenti. \newline
\textbf{Gas perfetto}:
\[
    \begin{matrix}
        \Delta s = c_P ln \frac{T_2}{T_1} - R^* ln \frac{P_2}{P_1} \;\;\;\;\;\;\;\;\;\;\;\;\;\;\;\Delta s = c_V ln \frac{T_2}{T_1} + R^* ln \frac{V_2}{V_1}\\
        \Delta s = c_V ln \frac{P_2}{P_1} + c_P ln \frac{V_2}{V_1}
    \end{matrix}
\]
\textbf{Liquido incomprimibile perfetto}: $\Delta s = c ln \frac{T_2}{T_1}$\newline
Inoltre:
\begin{itemize}
    \item Per definizione la variazione di entropia di un serbatoio di lavoro ($\Delta S_{SL}$) è nulla.
    \item Per definizione la variazione di entropia di un serbatoio di calore ($\Delta S_{SC}$) ha solo la componente reversibile, gli scambi avvengono con trasformazioni quasi-statiche (internamente reversibili).
    \item Un processo è impossibile da realizzare nel momento in cui $S_{irr} < 0$.
    \item La variazione di entrpia per una trasformazione reversibile è data da $\Delta S = \int_{i}^f \frac{1}{T} \delta Q_{rev}^\leftarrow$ (per i casi in cui $T$ è costante, allora $\Delta S = \frac{Q}{T}$).
    \item La variazione di entropia totale di un sistema isolato sede di trasformazioni termodinamiche è
    sempre maggiore di zero e tende a zero con il tendere dei processi alla reversibilità $\Delta S_{isolato} \geq 0$.
    \item $S_{irr} \geq 0$ è sempre maggiore di zero e il segno di $S_{Q}^\leftarrow $ è uguale al segno di $Q^\leftarrow$
\end{itemize}
\ \newline
\newline
\textbf{L’energia interna, l’entalpia e l’entropia} sono proprietà del sistema e quindi dipendono solo
dallo stato del sistema: ne consegue che noti gli stati iniziale e finale di un processo, la
determinazione della loro variazione prescinde dalla conoscenza di qualsiasi dettaglio del
processo (ivi compresa la sua reversibilità). Non è invece possibile determinare \textbf{il lavoro ed il
calore} scambiato dal sistema nel caso in cui si conoscano solo stato iniziale e finale e in tal caso non si ricorre pertanto alla scrittura del bilancio di energia (primo principio) e di entropia (secondo principio) che risulterebbero
indeterminati, ma si ricorre alle equazioni di stato (per i gas perfetti o per i liquidi incomprimibili perfetti\dots) che utilizzano solamente dati dello stato iniziale e finale.
