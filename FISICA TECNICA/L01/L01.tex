\section{L01-Introduzione}
\subsection{Introduzione}
La \textbf{termodinamica} è la scienza che studia \textbf{l’energia}, la \textbf{materia} e le \textbf{leggi} che governano le loro interazioni (scambi).
\subsection{Sistema termodinamico}
Il sistema termodinamico è inteso come porzione di spazio limitata da un \textbf{contorno} che lo racchiude completamente (il contorno è costituito da una superficie reale o immaginaria, rigida o deformabile).\newline
\newline
Tutto ciò che è esterno al sistema termodinamico è il \textbf{mondo esterno} e quando il mondo esterno è di massa infinita viene chiamato \textbf{ambiente}.\newline
\newline
I termini \textbf{serbatoio}, \textbf{sorgente} o \textbf{pozzo} fanno riferimento ad ambienti che interagiscono con il sistema termodinamico.\newline
\newline
Un \textbf{sistema composto} è un insieme di sistemi e sottosistemi a massa finita e/o infinita.\newline
\newline
Il sistema può essere \textbf{monocomponente} (sostanza pura o miscela di sostanze pure in rapporto fisso, quale ad esempio l'aria) o \textbf{policomponente} cioè composto da più componenti.\newline
\newline
Ogni sistema monocomponente può essere in diversi \textbf{stati di aggregazione} (solido, liquido, aeriforme). I sistemi saranno \textbf{monofase} o \textbf{polifase}.