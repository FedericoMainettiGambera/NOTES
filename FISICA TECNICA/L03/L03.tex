\section{L03-Trasformazioni}
Per variabile di stato si intende una grandezza che dipende dallo stato del sistema, viceversa variabili come il lavoro e il calore (che non sono di stato) non dipendono dallo stato del sistema, ma bensì dal percorso che hanno seguito per raggiungere quel determinato stato.
\subsection{Il lavoro termodinamico}
In un dispositivo cilindro-pistone, uno squilibrio di forze infinitesimo tra forze esterne e forza interna ($P \cdot A$) provoca uno spostamento infinitesimo del pistone a cui corrisponde un \textbf{lavoro}
\[
    \delta L^\rightarrow  = P A ds = P \cdot dV
\]
\begin{center}
    \includegraphics[height=3cm]{../L03/img1.PNG}
\end{center}
In termini di grandezze specifiche, la relazione diventa
\[
    \delta l^\rightarrow = P \cdot dv
\]
Quando il sistema evolve da uno stato \textbf{iniziale} (i) ad uno stato \textbf{finale} (f) attraverso una successione di \textbf{stati di equilibrio}, allora sarà possibile esprimere una legge, detta \textbf{equazione della trasformazione}, tra le variabili di stato $P$ e $v$ e la integrazione di $P dv$ rappresenterà il lavoro scambiato durante la trasformazione.\newline
Il lavoro termodinamico è dunque calcolabile come
\[
    l^\rightarrow = \int_{i}^{f}Pdv
\]
che è un integrale calcolabile solo se si conosce la funzione $P = P (v)$ detta equazione della trasformazione, che si può ricavare dall'equazione di stato.
\subsubsection{Il lavoro termodinamico nelle trasformazioni reversibili e irreversibile}
Nel caso di trasformazione \textbf{reversibile} in un cilindro-pistone la pressione interna è sempre omogenea all'interno del cilindro.
\begin{center}
    \includegraphics[height=2cm]{../L03/img2.PNG}
\end{center}
Nel caso di trasformazione \textbf{irreversibile} in un cilindro-pistone la pressione interna non è omogenea all'interno del cilindro. Quindi per ricavare la forza applicata al pistone si usa la pressione dell'ambiente esterno che agisce sul pistone.
\begin{center}
    \includegraphics[height=2cm]{../L03/img3.PNG}
\end{center}
\ \newline
\newline
Solitamente il lavoro reversibile è maggiore del lavoro irreversibile, come si vede bene dal seguente grafico:
\begin{center}
    \includegraphics[height=5cm]{../L03/img4.PNG}
\end{center}
L'area verde insime all'area a linee rosse (cioè tutta quella sottesa alla curva) rappresenta il lavoro nel caso di trasformazione reversibile, la sola area a linee rosse rappresenta, invece, il lavoro nel caso di trasformazione irreversibile.
\subsubsection{Il lavoro termodinamico di un ciclo}
Un ciclo è un trasformazione che termina con lo stato iniziale. In funzione di come avviene il ciclo, orario o antiorario per esempio, avremo che il lavoro uscente dal sistema è positivo o negativo. Nel
piano $PV$ chiamiamo macchine a \textbf{ciclo diretto} (motrici) quelle che eseguono trasformazioni in senso orario, mentre chiameremo macchine a \textbf{ciclo inverso} (operatrici) quelle che eseguono trasformazioni in senso antiorario.
\begin{center}
    \includegraphics[height=3cm]{../L03/img5.PNG}
\end{center}
\subsection{Il calore}
\textbf{Capacità termica}: è il rapporto fra il calore fornito al sistema e la variazione di temperature del sistema stesso
\[
    C_x = \left(\frac{\delta Q^\leftarrow }{d T}\right)_x
\]
\textbf{Calore specifico}: è il rapporto tra la capacità termica del sistema e la sua massa
\[
    c_x = \frac{1}{M}\left(\frac{\delta Q^\leftarrow }{dT}\right)_x
\]
I calori specifici possono essere interpretati come \textbf{derivate parziali di funzioni termodinamiche}.\newline
\newline
Il pedice $x$ precisa la \textbf{trasformazione} lungo la quale viene scambiato il calore $\delta Q$. Vediamo i casi in cui $x$ è la pressione e $x$ è il volume.
\subsubsection{Calori specifici a volume costante $c_V$}
\[
    c_V =\frac{1}{M}\left(\frac{\delta Q^\leftarrow }{dT}\right)_V = \left(\frac{\delta q^\leftarrow }{dT}\right)_V
\]
Partendo dal primo principio della termodinamica e dalla definizione di lavoro data precedentemente si può scrivere che $\delta q^\leftarrow  = d u + Pdv$ e che quindi $\delta q^\leftarrow  = \left(\frac{\delta u}{\delta T}\right)_vdT + \left(\frac{\delta u}{\delta T}\right)_T dv + Pdv$, e proseguendo considerando il fatto che il volume è costante ricaviamo che $\delta q^\leftarrow  = \left(\frac{\delta u}{\delta T}\right)_VdT$, da cui ricaviamo che 
\[
    c_V = \left( \frac{\delta u}{\delta T} \right)_V
\]
Siccome è una \textbf{derivata di una funzione di stato}, può in generale essere espresso come funzione di una coppia di variabili termodinamiche (in particolare della coppia $T$,$P$):
\[
    c_V = c_V(T,P)
\]
\subsubsection{Calori specifici a pressione costante $c_P$}
\[
    c_P =\frac{1}{M}\left(\frac{\delta Q^\leftarrow }{dT}\right)_P = \left(\frac{\delta q^\leftarrow }{dT}\right)_P
\]
Per lavorare sul calore specifico a pressione costante dobbiamo introdurre la funzione di stato \textbf{entalpia}, che esprime la quantità di energia che un sistema può scambiare con l'ambiente ed è definita come
\[
    h = u + Pv
\]
Per le trasformazioni che avvengono a pressione costante, la variazione di entalpia è uguale al calore scambiato dal sistema con l'ambiente esterno. Col suo differenziale possiamo riscrivere il primo principio come $dh = du + vdP + Pdv$, da cui, ricordando che siamo a pressione costante, ricaviamo che $\delta q^\leftarrow = dh - vdP = \left(\frac{\delta h}{\delta T}\right)_P dT + \left(\frac{\delta h}{\delta T}\right)_T dP - vdP$, e quindi $\delta q^\leftarrow  =  \left(\frac{\delta h}{\delta T}\right)_P dT$, da cui ricaviamo che
\[
    c_P =  \left(\frac{\delta h}{\delta T}\right)_P
\]
Siccome è una \textbf{derivata di una funzione di stato}, può in generale essere espresso come funzione di una coppia di variabili termodinamiche (in particolare della coppia $T$,$P$): 
\[
    c_P = c_P(T,P)
\]
\subsubsection{$c_V$ e $c_P$ per i gas ideali}
Per un gas ideale la variazione di energia interna e l'entalpia sono funzioni della sola temperatura $u = u(T)$ e $h = h(T)$, per cui i calori specifici da derivate parziali diventano derivate esatte:
\[
    c_V = c_V(T) = \left(\frac{\delta u}{\delta T}\right)
\]
\[
    c_P = c_P(T) = \left( \frac{\delta h}{ \delta T}\right)
\]
Vale inoltre la \textbf{realzione di Mayer}
\[
    c_P = c_V + R^*
\]
\subsubsection{$c_V$ e $c_P$ per i gas perfetti}
Per i gas ideali i calori specifici dipendono dalla temperatura, ma questa relazione di dipendenza è molto debole, per cui in intervalli ristretti di temperatura i calori specifici si ritengono spesso costanti: in questo caso il gas viene definito \textbf{perfetto}.\newline
\newline
Per calcolare i calori specifici si usano le seguenti relazioni:
\begin{itemize}
    \item \textbf{Gas monoatomico} ($He$, $Ar$)
    \[
        c_v = \frac{3}{2}R^*; \;\;\;\;\;\;\;\;\;\;\;\;\;\;\;c_P = \frac{5}{2}R^*
    \]
    \item \textbf{Gas biatomico o poliatomico lineare}: ($N_2$, $O_2$, $CO_2$)
    \[
        c_v = \frac{5}{2}R^*; \;\;\;\;\;\;\;\;\;\;\;\;\;\;\;c_P = \frac{7}{2}R^*
    \]
    \item \textbf{Gas poliatomico non lineare}: ($CH_4$)
    \[
        c_v = \frac{6}{2}R^*; \;\;\;\;\;\;\;\;\;\;\;\;\;\;\;c_P = \frac{8}{2}R^*
    \]
\end{itemize}
\subsubsection{$c_V$ e $c_P$ per i liqudi (e solidi) incomprimibili ideali}
\[
    c_V = c_P = c(T)
\]
\subsubsection{$c_V$ e $c_P$ per i liqudi (e solidi) incomprimibili perfetti}
\[
    c_V = c_P = c = \text{costante}
\]
\subsection{Le trasformazioni politropiche}
\subsubsection{Trasformazione politropica per un gas ideale}
\subsubsection{Trasformazione politropica per un gas perfetto}
\subsubsection{Espressioni della politropica}
\subsubsection{Politropiche per trasformazioni elementari}
\subsubsection{Lavoro lunga una generica politropica}
\subsubsection{Politropiche nel diagramma T-S}
\subsection{Calcolo delle grandezze termodinamiche}
\dots