\newpage
\title{LEZIONE 7 19/03/2020}\newline
\textbf{link} \href{https://web.microsoftstream.com/video/2feefb06-11cb-4de4-ac19-dcc5fe175a98?list=user&userId=faa91214-a6f5-40d7-8875-253fd49b8ce1}{clicca qui}
\section{Esercizi di ricapitolazione degli argomenti fino ad ora trattati}
\subsection{Es. Equilibri e stabilità in un sistema non lineare} Dato il SD NL a TC:
\[
    \begin{cases}
        \dot{x_1} = x_1^3 + x_2 + u\\
        \dot{x_2} = x_1 e^{x_2}\\
        y = x_1 (x_2 - u) + u^2
    \end{cases}
\]
Domande:
\begin{itemize}
    \item Trovare gli stati e le uscite di equilibrio per $u(t) = \bar{u} = -1$;
    \item Trovare le stabilità degli equilibri eventualmente trovati al punto precedente.
\end{itemize}
\ \newline
\textbf{Equilibri}:\newline
essendo a TC, la caratteristica degli equilibri è che le derivate dello stato sono nulle (devono essere ferme)
\[
    \bar{x}_1^3 + \bar{x}_2 + \bar{u} = 0
\]
\[
    \bar{x}_1 e^{\bar{x}_2} = 0
\]
Con la seconda equazione ricaviamo immediatamente che $\bar{x}_1 = 0$ e con la prima equazione invece ricaviamo che $\bar{x}_2 = - \bar{u}$.\newline
Quindi per $\bar{u} = -1$ esiste il solo equilibrio 
$\bar{x} = \left[\begin{matrix}
    0 \\ 
    1
\end{matrix}\right]$.\newline
\newline
\textbf{Sistema linearizzato e stabilità degli equilibri}:
\begin{itemize}
    \item Sistema linearizzato è asintoticamente stabile, ($\Rightarrow$) allora abbiamo un equilibrio asintoticamente stabile.
    \item Matrice $A$ del sistema linearizzato con almento un autovalore con parte reale positiva, ($\Rightarrow $) allora equilibrio instabile.\newline
    \textbf{oss.} un sistema lineare (o linearizzato) può essere instabile anche senza avere autovalore con parte positiva (ricontrolla la teoria per saperne di più).
\end{itemize}