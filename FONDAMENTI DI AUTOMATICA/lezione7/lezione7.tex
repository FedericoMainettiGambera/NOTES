\newpage
\title{LEZIONE 7 19/03/2020}\newline
\textbf{link} \href{https://web.microsoftstream.com/video/2feefb06-11cb-4de4-ac19-dcc5fe175a98?list=user&userId=faa91214-a6f5-40d7-8875-253fd49b8ce1}{clicca qui}
\section{Esercizi di ricapitolazione degli argomenti fino ad ora trattati}
\subsection{Es. Equilibri e stabilità in un sistema non lineare, TDE 06/05/2014 E1} Dato il SD NL a TC:
\[
    \begin{cases}
        \dot{x_1} = x_1^3 + x_2 + u\\
        \dot{x_2} = x_1 e^{x_2}\\
        y = x_1 (x_2 - u) + u^2
    \end{cases}
\]
Domande:
\begin{itemize}
    \item Trovare gli stati e le uscite di equilibrio per $u(t) = \bar{u} = -1$;
    \item Trovare le stabilità degli equilibri eventualmente trovati al punto precedente.
\end{itemize}
\ \newline
\textbf{Equilibri}:\newline
essendo a TC, la caratteristica degli equilibri è che le derivate dello stato sono nulle (devono essere ferme)
\[
    \bar{x}_1^3 + \bar{x}_2 + \bar{u} = 0
\]
\[
    \bar{x}_1 e^{\bar{x}_2} = 0
\]
Con la seconda equazione ricaviamo immediatamente che $\bar{x}_1 = 0$ e con la prima equazione invece ricaviamo che $\bar{x}_2 = - \bar{u}$.\newline
Quindi per $\bar{u} = -1$ esiste il solo equilibrio 
$\bar{x} = \left[\begin{matrix}
    0 \\ 
    1
\end{matrix}\right]$.\newline
Stato dell'uscita d'equilibrio:
\[
    \bar{y} = \bar{x} (\bar{x}_2 - \bar{u}) + \bar{u}^2 = 1
\]
\newline
\textbf{Sistema linearizzato e stabilità degli equilibri}:\newline
Ripasso dei criteri:
\begin{itemize}
    \item Sistema linearizzato è asintoticamente stabile, ($\Rightarrow$) allora abbiamo un equilibrio asintoticamente stabile.
    \item Matrice $A$ del sistema linearizzato con almento un autovalore con parte reale positiva, ($\Rightarrow $) allora equilibrio instabile.\newline
    \textbf{oss.} un sistema lineare (o linearizzato) può essere instabile anche senza avere autovalore con parte positiva (ricontrolla la teoria per saperne di più).
    \item Altrimenti non si può dire nulla, può essere qualunque cosa.
\end{itemize}
Applichiamo questi criteri:\newline
Calcoliamo $f_x = \left[\begin{matrix}
    3x_1^2 & 1 \\
    e^{x_2} & x_1 e^{x_2}
\end{matrix}\right]$. \newline
Nell'equilibrio $f_x |_{\bar{x}, \bar{u}}=  [\bar{x}_1 = 0, \bar{x}_2 = 1, \bar{u} = -1] = \left[\begin{matrix}
    0 & 1 \\ e & 0
\end{matrix}\right]$.\newline
Autovalori: $\;\;\;\;\;det \left[\begin{matrix}
    s & -1 \\ -e & s
\end{matrix}\right] = 0 \;\;\;\;\; \rightarrow  \;\;\;\;\; s^2 - e = 0 \;\;\;\;\;$, quindi abbiamo un autovalore con parte reale positiva e quindi siamo in presenza di un equilibrio instabile.\newline
\rule{\textwidth}{0,4pt}
\subsection{Es. , TDE 04/05/2015 E1}
Dato il SD LTI SISO a TC 
\[
    \begin{cases}
        \dot{x} = \left[\begin{matrix}
            -11 & 9 \\ -12 & 10 
        \end{matrix}\right]x + \left[\begin{matrix}
            -3\\-3
        \end{matrix}\right] u \\
        \\
        y = \left[\begin{matrix}
            2 &1
        \end{matrix}\right] x
    \end{cases}
\]
determinare
\begin{itemize}
    \item Se è asintoticamente stabile, stabile o instabile;
    \item la funzione di trasfermineto $G(s)$;
    \item $y(t)$ prodotto da $x(0) = \left[\begin{matrix}
        0\\0
    \end{matrix}\right]$ (cioè solo movimento forzato) e $u(t) = 2 sca(t)$. 
\end{itemize}
\textbf{Stabilità}:\newline
Autovalori di $A$:
\[
    det \left[\begin{matrix}
        s +11 & -9 \\ 12 & s-10
    \end{matrix}\right] = 0 \;\;\;\;\;\rightarrow \;\;\;\;\; s^2 + s -110 +108 = 0 \;\;\;\;\; \rightarrow  \;\;\;\;\; s^2 + s -2 = 0
\]
Abbiamo una variazione di segno e un autovalore con parte reale positiva, quindi il sistema è instabile.\newline
La risposta a questa domanda termina qua, senza dover calcolare gli autovalori, ma noi proseguiamo coi conti, perchè tanto ci servono per i punti successivi.\newline
\[
    s_{1,2} = - \frac{1}{2} \mp \sqrt{\frac{1}{4} + 2} = - \frac{1}{2} \mp \sqrt{\frac{9}{4}} = \begin{cases}
        -2\\ 1
    \end{cases}
\]
\textbf{Funzione di trasferimento}:
\[
    G(s) = c(sI-A)^{-1} b +d = \left[\begin{matrix}
        2 &1
    \end{matrix}\right] \left[\begin{matrix}
        s+11 & -9 \\ 12 & s-10
    \end{matrix}\right]^{-1} \left[\begin{matrix}
        -3\\-3
    \end{matrix}\right] =
\]
[per il calcolo della trasposta di una matrice 2x2 basta cambiare di posizione i termini della diagonale principale e invertire i segni dei termini della diagonale opposta]
\[
    = \frac{1}{s(+2)(s-1)} \left[\begin{matrix}
        2 &1
    \end{matrix}\right] \left[\begin{matrix}
        s-10 & 9 \\ -12 & s+11
    \end{matrix}\right] \left[\begin{matrix}
        -3\\-3
    \end{matrix}\right] = \frac{1}{s(+2)(s-1)} \left[\begin{matrix}
        2 & 1
    \end{matrix}\right] \left[\begin{matrix}
        -3s+3\\-3s + 3
    \end{matrix}\right] =
\]
\[
    = \frac{1}{s(+2)(s-1)} (-6s + 6 -3s +3) = \frac{-9s +9}{(s+1)(s-1)} = \frac{-9 \cancel{(s-1)}}{(s+2)\cancel{(s-1)}} = \frac{-9}{s+2}
\]