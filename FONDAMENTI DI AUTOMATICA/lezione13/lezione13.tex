\title{LEZIONE 12 30/03/2020}\newline
\textbf{link} \href{https://web.microsoftstream.com/video/22a546ac-3e6b-43e0-b0f9-45a630661700?list=user&userId=faa91214-a6f5-40d7-8875-253fd49b8ce1}{clicca qui}
\begin{itemize}
    \item $G_b(s)= \frac{1}{s^g} \rightarrow G(j \omega) = \frac{1}{(j \omega)^g} \rightarrow \begin{cases}
        |G_b(j \omega)| = \frac{1}{\omega^g} \rightarrow |G_b(j \omega)|_{dB}= - 20 g log(\omega)\\
        arg(G_b(j \omega)) = - g \cdot 90^o
    \end{cases} $\newline
    [immagine dagli appunti del prof]\newline \newline
    diagramma di bode del modulo: Il diagramma di bode del modulo corrispondente interseca sempre l'asse delle ascisse nel punto $\omega = 1$ e la pendenza è $-20g \frac{dB}{decade}$ (spesso abbreviato come "pendenza $-g$"), dove la \textbf{decade} è la distanza corrispondente a un rapporto che vale $10$.\newline \newline
    diagramma di bode della fase: Il diagramma di bode delle fasi è orizzontale al valore $-g \cdot 90^o$.\newline \newline
    Da notare è che fino ad ora non abbiamo fatto nessuna approssimazione.
    \item $G_c(s)= 1 + st \rightarrow  G_c(j \omega) = 1 + j \omega t \rightarrow \begin{cases}
        |G_c(j \omega)| = \sqrt{1 + (\omega t)^2}\\
        arg(G_c(j \omega))= arctan(\omega t)
    \end{cases}$\newline
    Per facilitare i conti applichiamo un approssimazione, che è il motivo del perchè stiamo facendo diagrammi di bode asintotici:
    \begin{itemize}
        \item se $| \omega t| >> 1$ (molto maggiore di $1$), allora $G_c(j \omega) \sim  j \omega t$, per cui otteniamo \newline che $\begin{cases}
            |G_c(j \omega)| \sim  | \omega t| \\
            arg(G_c(j \omega)) \sim  \begin{cases}
                90^o  \;\;& t>0\\
                -90^o & t <0
            \end{cases}
        \end{cases}$
        \item se $| \omega t| << 1$ (molto minore di $1$), allora $G_c(j \omega) \sim 1$, per cui otteniamo \newline ce $\begin{cases}
            |G_c(j \omega)| \sim  1 \\
            arg(G_c(j \omega)) \sim 0^o
        \end{cases}$
    \end{itemize}
    [immagine dagli appunti del prof]\newline \newline
    diagramma di bode del modulo: Definiamo la \textbf{frequenza d'angolo} $\frac{1}{|t|}$ nel diagramma di Bode del modulo. Grazie alle approssimazioni che abbiamo fatto, andando a sinistra nell'asse delle $\omega$, cioè verso il valore di $0 _{dB}$, il modulo vale circa $1$. Facciamo valere questa approssiamazione fino al valore di frequenza d'angolo. Superata la frequenza d'angolo il modulo cresce con pendenza $+1$, cioè di $20 \frac{dB}{decade}$. Questa rappresentazione prende il nome di diagramma di bode del modulo asintotico (il diagramma di bode del modulo esatto è mostrato in figura, e la differenza è che non ha una curva "netta").\newline \newline
    diagramma di bode della fase: approssimiamo tutto ciò che precede la frequenza d'angolo con $0^o$, alla frequenza d'angolo c'è un salto in cui se $t$ è positivo prota a $90^o$ (rossa nel disegno), se è negativo a $-90^o$ (blu nel disegno). La rappresentazione non approssimata dovrebbe seguire la linea tratteggiata in rosso nel disegno.\newline
    \newline
    Notiamo che l'approssimazione del modulo è molto buona, mentre quella della fase non molto.
    \item $G_d(s) = 1 + 2 \frac{\xi}{\omega_n}s + \frac{1}{\omega_n^2}s^2 \rightarrow  G_d(j \omega) = 1 + 2 \frac{\xi}{\omega_n} j \omega + \frac{1}{\omega_n^2}(j \omega)^2 = 1- \frac{\omega^2}{\omega_n^2} + j 2 \xi \frac{\omega}{\omega_n} $ con $1- \frac{\omega^2}{\omega_n^2}$ parte reale e $j 2 \xi \frac{\omega}{\omega_n}$ parte immaginaria
    \begin{itemize}
        \item per $\omega \rightarrow 0$: $\begin{cases}
            \text{parte reale}\; \rightarrow  1\\
            \text{parte immaginaria}\; \rightarrow 0
        \end{cases} \Rightarrow \begin{cases}
            |G_d(j \omega)| \rightarrow 1 \rightarrow |G_d(j \omega)|_{dB} \rightarrow 0\\
            arg(G_d(j \omega)) \rightarrow  0^o
        \end{cases}$ 
        \item per $\omega \rightarrow + \infty$: \newline
        [immagine dagli appunti del prof]\newline
        Chiamiamo le generiche radici coniugate complesse  la coppia $s_1$ e $s_2$ di $G_d(s) = \frac{1}{\omega_n^2}(s-s_1)(s-s_2)$ e rappresentiamole nel grafico.\newline \newline
        Facciamo un attimo un excursus dal caso $\omega \rightarrow  \infty$ e dimostriamo i risultati ottenuti precedentemente per $\omega \rightarrow 0$: [colore blu nel disegno] prendiamo il punto $j \omega$ con $\omega=0$, cioè $j0$, i vettori che connettono le radici $s_1$ e $s_2$ al punto $j0$ hanno modulo $\omega_n$, quindi il modulo di $|G_d(j0)|$ vale $\frac{\omega_n \cdot \omega_n}{\omega_n^2} = 1$. Possiamo anche dimostrare che la fase di $G_d$ per $\omega \rightarrow 0$, cioè in $j0$, che vale $0^o$, infatti gli angoli di $s_1$ e $s_2$ rispetto a un asse orizzontale sono opposti e si annullano a vicenda.\newline \newline
        Vediamo ora il caso in cui, invece di considerare il punto $j0$, consideriamo il generico punto $j \omega$. Analiziamo i vettori che connettono il generico punto $j \omega$ e $s_1$ e $s_2$ [in rosso nel disegno], questi vettori $j \omega - s_i$ per $\omega \rightarrow \infty$ (cioè per facendo salire lungo l'asse immaginario il generico punto $j \omega$) hanno entrami modulo che tende a $\infty$ e fase che tende a $90^o$ (quindi in totale $180^o$).\newline
        Quindi per $\omega \rightarrow  \infty \Rightarrow \begin{cases}
            |G_d(j \omega)| \rightarrow \infty \;\;\text{allo stesso modo in cui tende}\;\omega^2\\
            arg(G_d(j \omega)) \rightarrow  180^o
        \end{cases}$ \newline
        \newline
        [immagine dagli appunti del prof]\newline
        \newline
        \textbf{oss.} Il modulo del vettore $|j \omega - s_2|$ è monotono crescente, mentre il modulo del vettore $|j \omega - s_1|$ no, infatti ha un minimo per $\omega = Im(s_1)$, il perchè si vede graficamente.\newline
        \newline
        \textbf{oss.} più $s_1$ e $s_2$ sono vicini all'asse immaginario, più il minimo di $s_1$ è pronunciato e la variazione di fase avviene bruscamente.\newline
        \newline
        [immagine dagli appunti del prof]\newline
        \newline
        Diagramma di Bode del modulo: Segnamo la frequenza $\omega_n$ che prende il nome di \textbf{frequenza naturale}. Approssimiamo tutto ciò che precede $\omega_n$ con modulo uguale a $1$ ($0dB$), invece dalla frequenza naturale in poi il modulo sale con pendenza $+2$ (cioè $40 \frac{dB}{decade}$). Questo è il diagramma asintotico. Il diagramma esatto è mostrato in figura ed è diverso in base al termine $\xi$ ($|\xi| = 1$ abbiamo due radici reali coincidenti, $|\xi| = 0$ abbiamo $2$ radici immaginarie, in mezzo a questi due casi ci sono tutti gli altri casi possibili)\newline
        \newline
        [immagine dagli appunti del prof]\newline
        \newline
        Diagramma di Bode della fase: Il diagramma asintotico (approssimato) è fatto a scalino e va da $0^o$ a $+ 180^o$ se $\xi > 0$ o a $-180^o$ se $\xi < 0$. Il diagramma esatto è mostrato in figura (tratteggiato in rosso) e può avere una pendenza più o meno ripida per $|\xi| \rightarrow 0$.
    \end{itemize} 
\end{itemize}