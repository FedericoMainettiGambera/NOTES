\newpage
\title{LEZIONE 10 25/03/2020}\newline
\textbf{link} \href{https://web.microsoftstream.com/video/562a82e0-19cc-4f81-9183-eee77c9c45a4?list=user&userId=faa91214-a6f5-40d7-8875-253fd49b8ce1}{clicca qui} \;\;\textbf{[0:26:39 - fine lezione]}\newline
\newline
Vogliamo capire sotto quali ipotesi la rappresentazione di stato ($A, b, c, d$) e la funzione di trasferimento ($G(s)$) sono rappresentazioni equivalenti di un sistema e se è possibile studiare la stabilità di un sistema dalla fuznione di trasferimento tenendo in mente che la funzione di trasferimento può avere delle "parti nascoste".
\section{Raggiungibilità (SD LTI a TC SISO)}