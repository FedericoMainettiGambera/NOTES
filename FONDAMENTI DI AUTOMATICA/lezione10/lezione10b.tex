\newpage
\title{LEZIONE 10 25/03/2020}\newline
\textbf{link} \href{https://web.microsoftstream.com/video/562a82e0-19cc-4f81-9183-eee77c9c45a4?list=user&userId=faa91214-a6f5-40d7-8875-253fd49b8ce1}{clicca qui} \;\;\textbf{[0:26:39 - fine lezione]}\newline
\newline
Vogliamo capire sotto quali ipotesi la rappresentazione di stato ($A, b, c, d$) e la funzione di trasferimento ($G(s)$) sono rappresentazioni equivalenti di un sistema. Inoltre affronteremo il problema di studiare la stabilità di un sistema dalla fuznione di trasferimento, tenendo in mente che la funzione di trasferimento può avere delle "parti nascoste".
\section{Raggiungibilità (SD LTI a TC SISO)}
Uno stato $\tilde{x}$ si dice \textbf{raggiungibile} (da zero) se esiste un ingresso $\tilde{u}(t)$ tale che 
\[
    \begin{rcases}
        x(0) &= 0 \\
        u(t) &= \tilde{u}(t) \;\; t \geq 0
    \end{rcases} \longrightarrow x(\tilde{t}) = \tilde{x}  \;\;\;\; \tilde{t}< \infty
\]
Cioè uno stato ($\tilde{x}$) è raggiungibile se esiste un ingresso ($\tilde{u}(t)$) che partendo da zero porta il sistema in quel determinato stato in un tempo finito.\newline
\newline
Un sistema si dice (completamente) \textbf{raggiungibile} se ogni stato è raggiungibile.\newline
\newline
Come si determina se un sistema dinamico è o meno raggiungibile?\newline
\newline
Teorema di Cayley-Hamilton: Ogni matrice annulla il polinomio caratteristico.\newline
Capiamo meglio questo terema guardando il polinomio caratteristico di una matrice $A$ che è il determinante della matrice $sI-A$:
\[
    \Pi(s) = det(sI-A) = s^n + \beta_1 s^{n-1} + \dots + \beta_n
\]
quindi, se calcoliamo il polinomio caratteristico in $A$, vediamo questo si che si annulla:
\[
    \Pi(A) = det(AI-A) = det(0) = 0
\]
Ne segue che
\[
    A^n + \beta_1 A^{n-1} + \dots + \beta_n I = 0
\]
\[
    A^n = -\beta_1 A^{n-1} - \beta_2 A^{n-1} - \dots - \beta_n I
\]
cioè la potenza ennesima della matrice $A$ può essere scritta come combinazione lineare di tutte le potenze inferiori, fino alla potenza $0$, cioè la matrice identità.\newline
\newline
Applichiamo questo concetto al calcolo del movimento.\newline
Scriviamo la formula di Lagrange nel caso $x(0) = 0$ (cioè solamente movimento forzato):
\[
    x(t) = \int_{0}^{t}e^{A(t-\tau)} b u(\tau) d \tau
\]
dove la matrice $e^{A(t - \tau)}$ è
\[
    e^{A(t - \tau)} = I +A(t-\tau) + \frac{A^2(t-\tau)^2)}{2!} + \dots + \frac{A^{n-1} (t - \tau)^{n-1}}{(n-1)!} +\; \text{[combinazione lineare dei termini precedenti]}\;\dots 
\]
di conseguenza posso fare dei raccoglimenti di tutti itermini "combinazione lineare del termini precedenti" e scrivere questo termine grazie a una sommatoria:
\[
    e^{A(t - \tau)} = \sum_{l=0}^{n-1} \gamma_l(t-\tau) \cdot  A^l
\]
Ora quindi sostituiamo questo risultato nell'espressione del movimento forzato:
\[
    x(t) = \int_{0}^{t} \sum_{l=0}^{n-1} \gamma_l(t-\tau) \cdot  A^l b u(\tau)d \tau=
\]
dove $\gamma_l(t-\tau)$ sono termini che non mi interessa calcolare, ma mi basta sapere che ci sono e che rappresentano i termini "combinazione lineare dei termini precedenti",
\[
    = \sum_{l=0}^{n-1} A^l b \int_{0}^{t}\gamma_l (t-\tau) u(\tau)d \tau=
\]
chiamiamo ora l'integrale $\zeta_l (t) = \int_{0}^{t}\gamma_l (t-\tau) u(\tau)d \tau$, che dipende solo da $t$ perchè $\tau$ muore durante l'integrazione, inoltre, siccome $\gamma_l$ sono termini di cui non ci interessa la forma, non ci interessa calcolare $\zeta_l (t)$, ci basta sapere che è presente.\newline
\newline
Siamo quindi giunti a scrivere che
\[
    x(t) = \sum_{l=0}^{n-1} A^l b \zeta_l(t)
\]
Il termine $\zeta_l(t)$ contiene
\begin{itemize}
    \item i coefficienti del polinomio caratteristico di $A$;
    \item l'ingresso.
\end{itemize}