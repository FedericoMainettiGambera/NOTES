\title{LEZIONE 5 17/03/2020}\newline
\textbf{link} \href{https://web.microsoftstream.com/video/72b5f396-c34b-4ff6-8d6e-44bf6832dd2e?list=user&userId=faa91214-a6f5-40d7-8875-253fd49b8ce1}{clicca qui}\newline
\newline
(continuo della lezione scorsa)
\subsection{Esempi}
\textbf{es.} Prendiamo un polinomio caratteristico $\Pi(s) = 5 s^2 +s$: questo è chiaramente non asintoticamente stabile (c'è una radice nulla).\newline
\newline
\textbf{es.} $\Pi(s) = s^3 -s^2 +s +4$: anche questo non è asintoticamente stabile (c'è un coefficiente discorde)\newline
\newline
\textbf{es.} $\Pi(s) = s^5 +4 s^3 +3s^2 +s + 5$: anche questo non è asintoticamene stabile (manca il termine $s^4$ che quindi ha coefficiente nullo).\newline
\newline
\textbf{es.} $\Pi(s) = s^4 +2s^3+4s^2+s+5$: in questo esempio le condizioni necessarie sono soddisfatte, ma non sappiamo dire se è o meno asintoticamente stabile. Ci serve ora un criterio per stabilire se è asintoticamente stabile.
\subsection{Routh}
il criterio di Routh è una condizione necessaria e succificiente per la stabilità asintotica di un SD LTI a TC (l'analogo a TD è il criterio di Jury, ma noi non lo tratteremo).\newline
\newline
Il criterio di Routh si basa sulla tabella di Routh che si costruisce a partire dal polinomio caratteristico $\Pi(s)$.
\subsubsection{Tabella di Routh}
Definiamo il polinomio caratteristico come
\[
    \Pi(s) = a_0s^n + a_1 s^{n-1} + \dots + a_{n-1}s + a_n
\]
\textbf{oss.} il polinomio deve avere tutti i termini, altrimenti violiamo una delle condizioni necessarie che abbiamo visto alla lezione scorsa.\newline
\newline
La tabella di Routh si costruisce nel seguente modo:
\begin{itemize}
    \item Si compilano le prime due righe a "zig-zag" (come mostrato con dalle frecce) con i coefficienti del polinomio.
    \[
        \begin{matrix}
            a_0 & \;\;\;\;\;\; a_2 & \;\;\;\;\;\;\dots\\
            \;\;\downarrow & \nearrow \;\; \downarrow & \nearrow \;\; \downarrow\\
            a_1 & \;\;\;\;\;\; a_3 & \;\;\;\;\;\;\dots 
        \end{matrix}
    \]
    \item l'ultima colonna può terminare in due modi:
    \[
        \begin{matrix}
            \dots & a_{n-1}\\
            \;\\
            \dots & a_n
        \end{matrix}\;\;\;\;\;\; \text{oppure}\;\;\;\;\;\; \begin{matrix}
            \dots &a_n\\
            \;\\
            \dots & 0 
        \end{matrix}
    \]
    \item Le righe successive si costruiscono a partire dalle prime due.\newline
    In totale, considerando anche le prime due righe, ci sono $n+1$ righe.\newline
    Ogni riga dalla terza in poi dipende dalle due precedenti seguendo una regola:
    \[
        \begin{matrix}
            h_1 & h_2 & h_3 &\dots\\
            \;\\
            q_1 & q_2 & q_3 & \dots\\
            \;\\
            w_1 & w_2 & w_3 & \dots
        \end{matrix}
    \]
    prese due generiche righe ($h_i$ e $q_i$), la riga successiva ($w_i$) si genera come $w_i = - \frac{1}{q_1} det\left[\begin{matrix}
        h_1 & h_{i+1} \\
        q_1 & q_{i+1}
    \end{matrix}\right]$.\newline
    Gli elementi mancanti al termine delle righe soprastanti si assumono nulli.
    \item Se troviamo un elemento nullo in prima colonna, ci si ferma, sicuramente il sistema non è asintoticamente stabile, e siamo in presenza di un caso particolare che non ci permette di calcolare la tabella di Routh.
\end{itemize}
\subsubsection{Criterio di Routh}
Un SD con polinomio caratteristico $\Pi(s)$ è asintoticamente stabile se e solo tutti gli elementi della prima colonna della tabella di Routh sono concordi (e non nulli).\newline
\newline
\textbf{Corollario}\newline
Se non vi sono elementi nulli in prima colonna, allora il numero di inversioni di segno sulla prima colonna è uguale al numero di radici di $\Pi(s)0$ con $Re>0$. [non lo useremo mai questo corollario].
\subsubsection{Esempi}
\textbf{es.} $\Pi(s) = s^4 + 2s^3 + 4s^2 + s + 5$: soddisfa le condizioni necessarie, quindi faciamo la tabella di Routh. Siccome $n=4$ la tabella avrà $n+1 = 5$ righe:
\[
    \begin{matrix}
        1 & 4 & 5 \\
        2 & 1 & 0 \\
        \alpha & \beta\\
        \gamma\\
        \delta
    \end{matrix} \;\;\;\;\;\;\;\;\;\;\;\;\;\;\;\;\;\;\;\;\alpha = - \frac{1}{2}det\left[\begin{matrix}
        1&4\\2&1
    \end{matrix}\right] = \frac{7}{2}
\]
\[
     \beta = - \frac{1}{2}det\left[\begin{matrix}
        1&5\\2&0
    \end{matrix}\right]=5 \;\;\;\;\; \gamma= -\frac{1}{\alpha} det \left[\begin{matrix}
        2&1\\\alpha&\beta
    \end{matrix}\right] = - \frac{13}{7} \;\;\;\;\; \delta = - \frac{1}{\gamma} det \left[\begin{matrix}
        \alpha & \beta \\ \gamma &0
    \end{matrix}\right]
\]
Siccome $\gamma$ è discorde, sappiamo che non è asintoticamente stabile.\newline
Inoltre fra $\alpha$ e $\gamma$ c'è un'inversione di segno e fra $\gamma$ e $\delta$ c'è un'altra inversione di segno. Avendo due inversioni di segno, so che ci sono due radici con $Re > 0$. Da notare che, anche se abbiamo un solo elemento discorde, ci sono due inversioni di segno.\newline
\newline
\textbf{es.} Dato il SD LTI a TC con polinomio caratteristico $\Pi(s) = s^3 + 2s^2 + hs +k$, dire per quali valori di $(h,k)$ esso è asintoticamente stabile.\newline
Deduciamo che dovremo avere $h>0$ e $k>0$ (altrimenti violo una delle condizioni necessarie viste la lezione scorsa). Usiamo ora Rout, l'unico caso in cui si può evitare di usare Routh è se il polinomio caratteristico è di secondo grado.
\[
    \begin{matrix}
        1&h\\
        2&k\\
        \alpha\\
        \beta
    \end{matrix} \;\;\;\;\;\;\;\;\;\;\;\;\;\;\;\alpha = - \frac{1}{2}det \left[\begin{matrix}
        1&h\\2&k
    \end{matrix}\right] = \frac{2h-k}{2} \;\;\;\;\;\;\;\;\;\;\;\;\;\;\; \beta = - \frac{1}{\alpha} det \left[\begin{matrix}
        2 & k \\ \alpha & 0
    \end{matrix}\right] = k
\]
Disequazioni per imporre che i termini della prima colonna siano concordi:
\[
    \begin{cases}
        h - \frac{k}{2}>0\\
        k>0
    \end{cases} \Rightarrow \begin{cases}
        k>0\\
        k>2h
    \end{cases} \text{ricordando che} \; k>0 \;\text{e}\;h>0
\]
\newpage
\section{Segnali e trasformate}
Consideriamo un sistema $S$ dinamico LTI a TC (SISO), con un ingresso $u$ e un'uscita $y$.\newline
\newline
[immagine dagli appunti del prof]\newline
\newline
\textbf{Dominio del tempo}: fra il segnale $u(t)$ (la causa) e il segnale $y(t)$ (effetto) per noi c'è un legame differenziale ($u(t) \rightarrow y(t)$). Ciò che attribuisce a un sistema il carattere dinamico è la presenza di equazioni differenziali. Nel dominio del tempo abbiamo $t,u,y \in \mathbb{R}$\newline
\newline
\textbf{Dominio delle trasformate}: Supponiamo di poter associare a $u(t)$ del dominio del tempo, con un operazione che chiamiamo "trasformazione", un'altra funzione $U(s)$ del dominio delle trasformate ($u(t) \rightarrow U(s)$), dove $U$ è una funzione e $s$ è una variabile complessa. Facciamo la medesima cosa, ma con verso opposto, con $Y(s)\rightarrow y(t)$, dove questa operazione prende il nome di "antitrasformazione". Nel dominio delle trasformate abbiamo $s, U ,Y \in \mathbb{C}$.\newline
Date queste premesse, il legame fra $U(s)$ e $Y(s)$, il cui corrispondente nel dominio del tempo è differenziale , nel dominio delle trasformate è di tipo algebrico ($U(s) \rightarrow  Y(s)$).
\subsection{Serie di Fourier}
Dato un segnale $v(t)$ periodico di periodo $T$, posso esprimerlo come 
\[
    v(t)= v_0 + \sum_{k=1}^{\infty} v_k sin(k \omega_0 t + \phi_k)
\]
dove $\omega_0 = \frac{2\pi}{T}$. Questo significa che posso esprimere un segnale periodico come somma di infinite (infinito numerabile) sinusoidi di frequenze multiple di una fondamentale ($\omega_0$, di periodo $T$).
\subsection{Trasformata di Fourier}
Dato un segnale $v(t)$ definito su tutto $\mathbb{R}$ (non necessariamente periodico), chiamiamo la sua trasformata di Fourier come 
\[
    V(j \omega) = \mathcal{F}[v(t)] = \int_{-\infty}^{+\infty} v(t) e^{-j \omega t} dt \;\;\;\;\;\text{(se esiste)}\;
\]
L'antitrasformata di Fourier, invece, è
\[
    v(t) = \mathcal{F}^{-1}[V(j \omega)] = \frac{1}{2\pi j} \int_{-\infty}^{+\infty} V(j \omega) e^{j \omega t} d \omega
\]
\textbf{oss.} una trasformata è definita dal suo nucleo, che è $e^{-j \omega t}$.\newline
\textbf{oss.} l'antitrasformata è un integrale sulla variabile $\omega$. L'integrale è stato introdotto ad analisi con le somme integrali (per calcolare l'area sottesa a una funzione, si fanno i rettangolini e si sommano, poi si fa tendere la dimensione dei rettangolini a zero, etc). Immaginiamo, con la metafora dei rettangolini, di fare tanti rettangolini dell'asse della nostra variabile $\omega$, prendiamo uno di questi rettangolini e il suo contributo a $v(t)$ nell'antitrasformata è $e^{j \omega t}$ moltiplicata per un numero complesso $V(j \omega )$. Il termine $e^{j \omega t}$, il nostro nucleo, può avere due aspetti, se $\omega= 0$ è una costante, altrimenti è una sinusoide (ricordando che è un numero complesso e può essere espresso come somma di seni e coseni). Ora prendiamo il caso in cui il nucleo è una sinusoide, moltiplicarla per un numero complesso $V(j \omega)$ significa attenuarla o amplificarla (modulo) e sfasarla (argomento). Quindi noi per ricostruire $v(t)$ con l'antitrasformata devo sommare infinite sinusoidi, ognuna delle quali caratterizzate da una propria ampiezza e una propria fase indicata dal valore di $V(j \omega)$, \textbf{cioè $V(j \omega)$ ci dice quanto pesa ogni frequenza in $v(t)$}. In questo caso $v(t)$ è somma di infinte (infinito del continuo) sinusoidi.
\subsection{Trasformata di Laplace}
Dato un segnale $v(t)$ definito per $t\geq 0$ (o equivalentemente nullo per $t < 0$), definiamo trasformata di Laplace come
\[
    V(s) = \mathcal{L}[v(t)] = \int_{0}^{\infty} v(t) e^{-st}dt
\]
con $s,V \in \mathbb{C}$.\newline
L'antitrasformata di Laplace, invece, è
\[
    v(t) = \mathcal{L}[V(s)] = \frac{1}{2\pi j} \int_{\alpha + j \infty}^{\alpha - j \infty} V(s) e^{st}ds
\]
dove gli estremi dell'integrale rappresentano il fatto che quanto integriamo rispetto a una variabile complessa ($s$) dobbiamo specificare su quale linea si muove la variabile (spiegato molto "alla buona", non dare peso a questo fatto): la variabile di integrazione $s$ si muove su una retta parallela all'asse immaginario di parte reale (ascissa) $\alpha$, andando con la sua parte immaginaria (ordinata) da $- \infty$ a $+ \infty$.\newline
\textbf{oss.} Abbiamo che il nucleo è $e^{st} = e^{()\alpha + j \omega) t} = e^{\alpha t} (cos(\omega t) + j sin(\omega t))$. I segnali con questa forma sono, non solo le costanti e le sinusoidi (come nel caso del nucleo di Fourier), ma anche le sinusoidi che si smorzano, le sinusoidi che divergono, i segnali che esponenzialmente divergono, i segnali che esponenzialmente convergono. I segnali che si lasciano trasformare secondo Fourier, sono i segnali che si lasciano ricostruire per mezzo di una somma infinita di segnali che partono dal nucleo $e^{j \omega t}$, cioè costanti e sinusoidi, moltiplicato per un numero complesso $V(j \omega)$, cioè amplificate o attenuate e sfasate; i segnali che si lasciano trasformare secondo Laplace, sono molti di più, perchè il nucleo di partenza $e^{st}$ rappresenta un insieme di segnali molto più grande. Tutti i segnali trasformabili secondo Fourier sono trasformabili anche secondo Laplace, ma non vale il viceversa.
\subsubsection{Esempi}
Vediamo tre trasformate di Laplace notevoli (da imparare a memoria, perchè saranno molto frequenti).\newline
\newline
\textbf{es.} $v(t) = sca(t) := \begin{cases}
    1 \;\;\;\; &t\geq\\ 0 &t<0 
\end{cases}$, la trasformata di Laplace è
\[
    \mathcal{L}[sca(t)] = \int_{0}^{\infty} sca(t)e^{-st} dt = \int_{0}^{\infty} 1 \cdot e^{-st}dt = \left[\frac{e^{-st}}{-s}\right]_0^\infty = 0 - \frac{1}{-s} = \frac{1}{s}
\]
\textbf{es.} $v(t) = imp(t) := \begin{cases}
    imp(t)=0 \;\;\;&\forall\;t \neq 0\\
    \int_{-\infty}^{\infty}imp(t) dt =1
\end{cases}$\newline
[immagine dagli appunti del prof]\newline
La trasformata di Laplace è
\[
    \mathcal{L}[imp(t)] = \lim_{\epsilon\rightarrow 0} \mathcal{L}[f_\epsilon(t)] = \lim_{\epsilon\rightarrow 0}\int_{0}^{\infty}f_\epsilon (t) e^{-st} dt = \lim_{\epsilon\rightarrow 0}\int_{0}^{\epsilon}\frac{1}{\epsilon}e^{-st}dt =
\]
\[
    = \lim_{\epsilon\rightarrow 0}\left[\frac{e^{-st}}{-s\epsilon}\right]_0^\epsilon = \lim_{\epsilon\rightarrow 0}\left( \frac{e^{-s\epsilon}}{-s\epsilon}- \frac{1}{-s\epsilon} \right) = \lim_{\epsilon\rightarrow 0} \frac{1- e^{-s \epsilon}}{s\epsilon} = [F.I., Hopital] = \lim_{\epsilon\rightarrow 0}\frac{\cancel{s}e^{-s\epsilon}}{\cancel{s}} = 1
\]
\textbf{es.} $v(t) = e^{at}$ per $t\geq 0$ o equivalentemente $v(t) = e^{at} sca(t)$, la trasformata di laplace è
\[
    \mathcal{L}[e^{at} sca(t)] = \int_{0}^{\infty}e^{at}e^{-st}dt = \int_{0}^{\infty}e^{(a-s)t}dt = \left[\frac{e^{(a-s)t}}{a-s}\right]_0^\infty = \frac{1}{s-a} \;\;\;\text{se $Re(a-s)<0$, cioè $Re(s)>a$}\;
\]
\rule{\textwidth}{0,4pt}\newline
\textbf{Riassunto dei concetti chiave introdotti}:
\begin{itemize}
    \item Le trasformate sono strumenti che legano biunivocamente sengali nel tempo a funzioni complesse di variabile complessa.
    \item Le trasformate di Fourier interpretano un segnale come somma di infinite sinusoidi, invece le trasformate di Laplace non considera sinusoidi, ma delle esponenziali complesse (categoria nella quale rientrano anche le sinusoidi).
\end{itemize}
\rule{\textwidth}{0,4pt}\newline
\subsubsection{Proprietà della trasformata di Laplace}
\textbf{Time stamp: 2:14:16} sono stanco sorry fede del futuro, mancano solo 15 minutiiiii\dots