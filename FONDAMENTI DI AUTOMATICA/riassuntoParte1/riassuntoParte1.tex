Un \textbf{sistema dinamico} (SD) è un sistema in cui la conoscenza degli ingressi su un intervallo di tempo non è sufficiente per determinare l'andamento delle uscite sullo stesso intervallo di tempo.\newline
\newline
Le quantità di cui occorre il valore iniziale per conoscere l'uscita, noto l'ingresso, si dicono \textbf{variabili di stato} e si indicano tipicamente con $x$.\newline
\newline
Sistemi dinamici con solo un ingresso e solo un'uscita si dicono \textbf{SISO} (Single Input, Single Output).\newline
\newline
Il numero di variabili di stato prende il nome di \textbf{ordine} del sistema.\newline
\rule{\textwidth}{0,4pt}\newline
\newline
Nei sistemi dinamici a \textbf{tempo continuo}:
\[
    \begin{rcases}
        \dot{x}_1(t) &= f_1(x_1(t),x_2(t), \dots, x_n(t), u(t), t)\\
        \dots \;\;\;&= \;\;\;\dots\\
        \dot{x}_n(t) &= f_n(x_1(t),x_2(t), \dots, x_n(t), u(t), t)
    \end{rcases} \rightarrow \text{equazione (differenziale) di stato}\;
\]
\[
    \begin{rcases}
        \\
        \;\;\;\; y(t) = g(x_1(t),x_2(t), \dots, x_n(t), u(t), t)\\
        \\
    \end{rcases} \rightarrow \text{equazione o trasformazione d'uscita}\;
\].\newline
In espressione vettoriale:
\[
    x(t) = \left[\begin{matrix}
        x_1(t)\\
        \dots\\
        x_n(t)
    \end{matrix}\right] \Rightarrow  \begin{cases}
        \dot{x}(t) = f(x(t), u(t),t)\\
        y(t) = g(x(t),u(t),t)
    \end{cases} \;\;\;\;u,y \in \mathbb{R}, \;\;\;x \in \mathbb{R}^n
\]
Per tempo discreto si intende che l'evoluzione temporate è "a passi", esiste infatti un indice temporale $k$ intero che tiene traccia dei numeri di passi.\newline
Nei sistemi dinamici a \textbf{tempo discreto}: 
\[
    \begin{rcases}
        \dot{x}_1(k) &= f_1(x_1(k-1),x_2(k-1), \dots, x_n(k-1), u(k-1), k)\\
        \dots \;\;\;&= \;\;\;\dots\\
        \dot{x}_n(k) &= f_n(x_1(k-1),x_2(k-1), \dots, x_n(k-1), u(k-1), k)
    \end{rcases} \rightarrow \text{equazioni (di stato) alle differrenze}\;
\]
\[
    \begin{rcases}
        \\
        \;\;\;\; y(k) = g(x_1(k),x_2(k), \dots, x_n(k), u(k), k)\\
        \\
    \end{rcases} \rightarrow \text{equazione o trasformazione d'uscita}\;
\]
Nel caso a tempo continuo usavamo l'integrazione per esprimere la dipendenza dagli stati passati, qui abbiamo un equivalente a tempo discreto, in cui lo stato attuale ($k$) dipende dallo stato di prima ($k-1$).\newline
\rule{\textwidth}{0,4pt}\newline
\newline
Se $f$ e $g$ sono lineari in $x$ e in $u$, allorà dirò che il sistema dinamico è \textbf{lineare}.\newline
\newline
Se $f = f(x,u)$ (non compare esplicitamente $t$) e $g = g(x,u)$ (non compare esplicitamente $t$), allora dirò che il sistema dinamico è \textbf{tempo invariante o stazionario}.\newline
\newline
Se $g = g(x,t)$ (non compare $u$), allora dirò che il sistema dinamico è \textbf{strettamente proprio}.\newline
\rule{\textwidth}{0,4pt}\newline
\newline
I sistemi dinamici \textbf{lineari tempo invarianti} sono sistemi in cui $f$ e $g$ sono lineari in $x$ e in $u$, e per cui non c'è dipendenza esplicita da $t$ (TC) o da $k$ (TD).\newline
Caso a \textbf{tempo continuo} (TC):
\[
    \begin{cases}
        \dot{x}_1(t) &= a_{11}x_1(t) + \dots+ a_{1n}x_{n}(t) + b_1 u(t)\\
        \dots &= \;\;\; \dots\\
        \dot{x}_n(t) &= a_{n1}x_1(t) +\dots+ a_{nn}x_n(t) + b_nu(t)\\
        y(t) &= c_1x_1(t) + \dots + c_n x_n(t) + d u(t)
    \end{cases}
\]
In forma vettoriale:
\[
    x = \left[\begin{matrix}
        x_1\\
        \dots\\
        x_n
    \end{matrix}\right], \;\;\; A = \left[\begin{matrix}
        a_{11} & \dots & a_{1n}\\
        \dots & \dots & \dots\\
        a_{n1} & \dots & a_{nn}
    \end{matrix}\right], \;\;\; b = \left[\begin{matrix}
        b_1\\
        \dots\\
        b_n
    \end{matrix}\right], \;\;\; c = [c_1 \;\; \dots \;\; c_n]
\]
\[
    \begin{cases}
        \dot{x} = Ax+bu\\
        y = cx +du
    \end{cases}
\]
I termini $A, b, c, d$ prendono il nome di \textbf{descrizione di stato}.\newline
Caso a \textbf{tempo discreto} (TD) è analogo:
\[
    \begin{cases}
        x(k) = Ax(k-1) + bu(k-1)\\
        y(k) = cx(k) +d u(k)
    \end{cases}
\] 
\textbf{oss.} Nei libri di testo si può trovare scritto così: $\begin{cases}
    x(k+1) = Ax(k) + bu(k)\\
    y(k+1) = cx(k+1) +d u(k+1)
\end{cases}$, comunque la cosa importatne è che nella prima equazione del sistema ci siano due valori successivi di $k$ (quello corrente e quello successivo oppure quello corrente e quello precedente) e nella seconda ci sia un solo valore di $k$ (stesso valore consistente, che sia quello corrente, quello successivo o quello precedente, etc.).\newline
\rule{\textwidth}{0,4pt}\newline
\newline
Equilibrio:
\[
    \begin{rcases}
        x(0) &= \bar{x}\\
        u(t) \;\text{(o $u(k)$)}\; &= \bar{u} \;\;\text{per $t$ (o $k$) $\geq 0$}\;
    \end{rcases} \rightarrow x(t) \;\text{(o $x(k)$)}\; = \bar{x} \;\;\;\;\text{per $t$ (o $k$) maggiore di 0}\;
\]
cioè: con un ingresso costante deciso, esiste un qualche stato costante tale che, prendendo proprio questo stato come stato iniziale e applicando l'ingresso costante deciso, lo stato non cabia?\newline
\newline
Vediamo il caso a \textbf{tempo continuo} (TC). Se $x$ deve rimanere costante, significa che la derivata deve essere nulla $\dot{x} = 0$. Quindi in generale con $\dot{x} = f(x,u)$, noto il valore $\bar{u}$ segnato che applico, gli eventuali $\bar{x}$ sono le soluzioni di $f(\bar{x}, \bar{u}) = 0$.\newline
\newline
Vediamo il caso a \textbf{tempo discreto} (TD). Se $x$ deve rimanere costante, significa che $x(k+1) = x(k)$ per ogni $k$, cioè che $x(k+1)$ deve rimanere costante. Quindi in tal caso dovrò risolvere per $\bar{x}$ l'equazione $f(\bar{x}, \bar{u}) = \bar{x}$.\newline
\newline
Vediamo ora il caso \textbf{tempo continuo} (TC) \textbf{lineare tempo invariante} (LTI) (per sistemi non lineari guardare l'osservazione più avanti). In questo caso l'equazione da risolvere diventa $0 = A \bar{x} + b \bar{u}$ (risolvendo questo sistema matriciale si trovano i vari stati di equilibrio). Se $A$ non è singolare, allora esiste uno e uno solo $\bar{x} = -A^{-1} b u$, altrimenti o non esiste $\bar{x}$ o esistono infiniti $\bar{x}$.\newline
\newline
Vediamo ora il caso \textbf{tempo discreto} (TD) \textbf{lineare tempo invariante} (LTI). In questo caso l'equazione da risolvere diventa $\bar{x} = A \bar{x} + b \bar{u}$. Questa equazione si scrive anche $(I-A)\bar{x} = b \bar{u}$ e se $I-A$ non è singolare, cioè se $A$ non ha autovalori in 1, allora esiste uno e uno solo $\bar{x} = (I-A)^{-1} b \bar{u}$, altrimenti o non esiste $\bar{x}$ o esistono infiniti valori $\bar{x}$.\newline
\newline
\textbf{oss.} Nel caso non lineare $g(\bar{x}, \bar{u})$ potrebbe anche non avere significato, quindi non esiste un'uscita d'equilibrio. Invece nel caso lineare (sia TC sia TD) se esiste $\bar{x}$, allora esiste sempre $\bar{y} = c \bar{x} + d \bar{u}$.\newline
\newline
