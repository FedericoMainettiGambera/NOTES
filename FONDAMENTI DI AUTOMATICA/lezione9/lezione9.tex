\section{Esercitazione 2}
\title{LEZIONE 9 24/03/2020}\newline
\textbf{link} \href{https://web.microsoftstream.com/video/c0596b77-fd44-467c-8085-683611779644?list=user&userId=faa91214-a6f5-40d7-8875-253fd49b8ce1}{clicca qui}\newline
\newline
Appunti del prof con annotazioni \url{../pdf/FdA-L09-2020.03.24.pdf}\newline
Contenuto:
\begin{itemize}
    \item Equilibrio;
    \item Stabilità per sistemi non lineari (linearizzazione);
    \item Linearizzazione di sistemi non lineari;
    \item Calcolo del movimento con la funzione di trasferimento;
    \item Heaviside con ritardi;
    \item Realizzazione;
    \item Relazione stabilità e Forma di Jordan).
\end{itemize}
\subsection{Stabilità per sistemi non lineari}
Vediamo il caso generico per come calcolare la matrice $A$ nell'intorno di un equilibrio per un sistema non lineare con due variabili di stato:\newline
\newline
Dato un sistema non lineare $\begin{cases}
    \dot{x}_1 = a_{11}x_1 + a_{12}x_2 + b_{1}u\\ 
    \dot{x}_2 = a_{21}x_1 + a_{22}x_2 +b_2u\\
    y = c_1x_1 + c_2x_2 + d
\end{cases}$, e determinati i suoi equilibri $\bar{x} = \left[\begin{matrix}
    \bar{x}_1\\\bar{x}_2
\end{matrix}\right]$ per l'uscita $\bar{u}$, la matrice $A$ linearizzata all'equilibrio si ottiene nel seguente modo:
\begin{itemize}
    \item Si calcola la matrice delle derivate parziali
    \[
        f_x = \left[\begin{matrix}
            \frac{\delta f}{\delta x_1} (a_{11}x_1) & \frac{\delta f}{\delta x_2} (a_{12}x_2)\\
            \frac{\delta f}{\delta x_1} (a_{21}x_1) & \frac{\delta f}{\delta x_2} (a_{22}x_2)
        \end{matrix}\right]
    \]
    \item Per trovare la matrice $A$ linearizzata nell'intorno di uno dei suoi equilibri è sufficiente sostituire i valori dell'equilibrio e dell'uscita all'interno di questa matrice.
\end{itemize}
\textbf{oss.} Una volta stabilita la matrice $A$ del sistema linearizzato, è possibile procedere con l'analisi della stabilità. Per i sistemi non lineari i procedimenti da seguire sono i seguenti:
\begin{itemize}
    \item Se il sistema linearizzato nell'intorno di un equilibrio è asintoticamente stabile, allora l'equilibrio è asintoticamente stabile e per capirlo si usano i soliti criteri per la matrice $A$;
    \item Se il sistema linearizzato nell'intorno di un quilibrio ha almeno un autovalore con parte reale positiva, allora l'equilibrio è instabile.
    \item In tutti gli altri casi non si può dire nulla (la soluzione dipende da termini di sviluppo in serie maggiori del primo), non ci verrà richiesto in esame!
\end{itemize}
\textbf{oss.} Per i sistemi non lineari la stabilità non è una proprietà del sistema, quindi gli equilibri non condividono per forza la stessa "tipologia" di stabilità (AS/S/I). Di conseguenza è necessario analizzare la stabilità per ciascuno degli equilibri.
\subsection{Linearizzazione di un sistema non lineare nell'intorno di un equilibrio}
Dato un sistema non lineare 
\[
    S: \begin{cases}
        \dot{x} = f(x, u)\\ y = g(x,u)
    \end{cases}
\]
La sua linearizzazione è della forma:
\[
    \mathcal{L}^\delta : \begin{cases}
        \delta \dot{x} = f_x |_{\bar{x},\bar{u}} \delta x + f_u|_{\bar{x},\bar{u}} \delta u\\
        \delta y = g_x|_{\bar{x},\bar{u}} \delta x + g_u |_{\bar{x},\bar{u}} \delta u
    \end{cases}
\]
con $ \delta u = u - \bar{u}$, $\delta x = x -\bar{x}$, $ \delta y = y - \bar{y}$.\newline
\newline
Bisogna ora calcolare $f_u$, $f_x$, $g_u$ e $g_x$:\newline
\newline
Nel caso di sistema scalare i termini $f_u$, $f_x$, $g_u$ e $g_x$ sono le varie dervate parziali.\newline
\newline
Vediamo ora il caso in cui il sistema non è scalare, ma matriciale, per esempio prendiamo il caso a due dimensioni in cui il sistema è della forma $\begin{cases}
    \dot{x}_1 = f_1(x_1, x_2, u)\\ 
    \dot{x}_2 = f_2(x_1,x_2, u)\\
    y = g(x_1,x_2)
\end{cases}$.
\[
    f_x=\left[\begin{matrix}
        \frac{\delta f}{\delta x_1} (f_1) & \frac{\delta f}{\delta x_2} (f_1)\\
        \frac{\delta f}{\delta x_1} (f_2) & \frac{\delta f}{\delta x_2} (f_2)
    \end{matrix}\right]
\]
\[
    f_u = \left[\begin{matrix}
        \frac{\delta f}{\delta u} (f_1)\\
        \frac{\delta f}{\delta u} (f_2)
    \end{matrix}\right]
\]
\[
    g_x = \left[\begin{matrix}
        \frac{\delta f}{\delta x_1} (g) & \frac{\delta f}{\delta x_2} (g)
    \end{matrix}\right]
\]
\[
    g_u = \frac{\delta f}{\delta u} (g)
\]
Una volta determinati tutti questi valori, per scrivere il sistema linearizzato nell'intorno di un equilibrio $\bar{x} = \left[\begin{matrix}
    \bar{x}_1 \\ \bar{x}_2
\end{matrix}\right]$ per l'ingresso $\bar{u}$ e uscita $\bar{y}$ è sufficiente sostituire i valori:
\[
    \begin{cases}
        \delta \dot{x} = f_x|_{x_1 = \bar{x}_1, x_2 = \bar{x}_2} \delta x + f_u|_{u = \bar{u}}\delta u\\
        \delta y = g_x|_{x_1 = \bar{x}_1, x_2 = \bar{x}_2} \delta x + g_u|_{u = \bar{u}} \delta u
    \end{cases}
\]
con $ \delta u = u - \bar{u}$, $\delta x = x -\left[\begin{matrix}
    \bar{x}_1 \\ \bar{x}_2
\end{matrix}\right]$, $ \delta y = y - \bar{y}$.
\subsection{Ritardi nelle trasformate}
Data una funzione $f(t)$ nel dominio del tempo, un ritardo $\tau$ è solitamente espresso come $f(t-\tau)$. Sia $f$ una dei segnali di cui sappiamo le trasformate notevoli, allora la corrispondente trasformata di Laplace è la trasformata di Laplace della funzione $f$ senza il ritardo, moltiplicata per il termine $e^{- \tau s}$ che rappresenta il ritardo (es. $ram(t-\tau) \rightarrow \frac{1}{s^2}e^{- \tau s}$).\newline
\newline
Questo concetto è da tenere a mente anche durante l'operazione inversa di antitrasformazione secondo Heaviside: Heaviside è utilizzato per antitrasformare trasformate di Laplace che sono razionali fratte ($\frac{N(s)}{D(s)}$), ma nel caso in cui la trasformata con cui sto lavorando presenta un termine esponenziale dovrò aggiungere un ritardo. L'importante è tenere la parte razionale fratta separata dalla parte esponenziale, inizialmente si ignora l'esponenziale e si eseguono tutti i passaggi di Heaviside sulla razionale fratta, e, solo alla fine, si aggiunge il ritardo espresso dall'esponenziale:
\[
    \mathcal{L}^{-1}[\frac{N(s)}{D(s)}e^{-s \tau}] \rightarrow \;\text{Heaviside di $\frac{N(s)}{D(s)}$ ottengo $y(t)$, poi con $e^{-s \tau}$ aggiungo ritardo $\tau$}\; \rightarrow  y(t- \tau)
\]