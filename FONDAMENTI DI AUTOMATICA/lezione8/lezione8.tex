\title{LEZIONE 8 23/03/2020}\newline
\textbf{link} \href{https://onedrive.live.com/?authkey=%21AATVJK3srNwxGzs&id=EE092FF4FF7B5B0E%212158&cid=EE092FF4FF7B5B0E}{link a una registrazione di back up}\newline
\newline
Appunti del prof con annotazioni \url{../pdf/FdA-L08-2020.03.23.pdf}
\begin{comment}
\section{Esercitazione 1}
\subsection{Es. 1}
Dato il sistema
\[
    \begin{cases}
        \dot{x}_1 = -2 x_1 + 6 x_2 + u\\
        \dot{x}_2 = -2x_1 + 5 x_2
    \end{cases}
\]
\begin{itemize}
    \item Stabilità del sistema (AS/S/I)?
    \item ponendo $x(0) = \left[\begin{matrix}
        1\\0
    \end{matrix}\right]$ e $u(t) = sca(t)$, quanto vale $x(t)$ per $t\geq 0$?
\end{itemize}
\textbf{Stabilità}:
\[
    A= \left[\begin{matrix}
        -2 & 6 \\ -2 &5
    \end{matrix}\right]
\]
Senza fare conti aggiuntivi ci accorgiamo che la traccia $tr(A)$ è positiva e quindi il sistema è instabile.\newline
\newline
\textbf{Movimento dello stato}:\newline
Lo facciamo in più di un modo.\newline
\newline
\textbf{Primo metodo:} iniziamo calcolando il movimento libero $x_L(t)$ tramite $e^{At}$ e poi il movimento forzato $x_F(t)$ con la trasformata di Laplace.\newline
Calcoliamo gli autovalori di $A$: 
\[
    det(sI-A) = 0
\]
\[
    det \left[\begin{matrix}
        s+2 & -6 \\ 2 & s-5
    \end{matrix}\right] = 0
\]
\[
    (s+2)(s-5) + 12 = 0
\]
\[
    s^2 - 3s +2 = 0 \;\; \Longrightarrow \;\; s= \frac{2\mp \sqrt{9-8}}{2} = \begin{cases}
        s_1 = 1\\ s_2 = 2
    \end{cases}
\]
Notiamo, come osservazioni aggiuntive, che in $s^2 - 3s +2 = 0$ ci sono due variazioni, quindi due autovalori con parte reale positiva e che $s_1 + s_2$ sono la traccia di $A$.\newline
Calcoliamo gli autovettori:\newline
$s_1=1$:
\[
    Az = s_1 z
\]
\[
    \left[\begin{matrix}
        -2 & 6 \\-2 & 5
    \end{matrix}\right] \left[\begin{matrix}
        z_1 \\z_2
    \end{matrix}\right] = 1 \cdot \left[\begin{matrix}
        z_1\\z_2
    \end{matrix}\right]
\]
\end{comment}