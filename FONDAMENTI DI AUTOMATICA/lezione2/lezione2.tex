\title{LEZIONE 2 11/03/2020}\newline
\textbf{link} \href{https://web.microsoftstream.com/video/c3fbadab-4a18-4fbd-bd5c-3a50914235b4?list=user&userId=faa91214-a6f5-40d7-8875-253fd49b8ce1}{clicca qui}
\section{Sistemi dinamici} 
[immagine dagli appunti del prof]\newline
Premettendo che per questa trattazione la presenza del disturbo non è influente, ci poniamo la seguente domanda: se conosco $U(t)$ sull'intervallo $[t_0, t]$, questo mi basta per conoscere $y[t_0,t]$, cioè l'andamento del segnale di $y$ nell'intervallo $[t_0,t]$?\newline
Se la risposta a questa domanda è sì, significa che siamo in presenza di un sistema dinamico, se la risposta è no, il sistema non è dinamico.\newline
\newline
Un sistema dinamico è un sistema in cui la conoscenza degli ingressi su un intervallo di tempo non è sufficiente per determinare l'andamento delle uscite sullo stesso intervallo di tempo.\newline
\newline
\textbf{es.} \newline
[immagine dagli appunti del prof]\newline
La tensione $U(t)$ è l'ingresso, la corrente sulla resistenza $R$ è l'uscita $y(t)$. La legge che governa questo circuito è $y(t) = \frac{1}{R}u(t)$, quindi noto $U(t)$ conosco $y(t)$. Siamo in presenza di un sistema non dinamico.\newline
\newline
\textbf{es.} \newline
[immagine dagli appunti del prof]\newline
La tensione $U(t)$ è l'ingresso, la corrente sulla capacità $C$ è l'uscita $y(t)$. Per conoscere $y[t_0,t]$ mi occorrono $u[t_0,t]$ e $y(t_0)$. Siamo in presenza di un sistema dinamico.\newline
\newline
