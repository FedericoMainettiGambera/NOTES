\title{LEZIONE 6 18/03/2020}\newline
\textbf{link} \href{https://web.microsoftstream.com/video/0c2093a0-44b9-4c83-a0b1-7e61fbd111b5?list=user&userId=faa91214-a6f5-40d7-8875-253fd49b8ce1}{clicca qui}
\subsection{Teorema del valore iniziale}
\[
    V(s) = \mathcal{L}[v(t)] \Longrightarrow v(0) = \lim_{s\rightarrow \infty}s V(s)
\] 
dove possiamo usare $v(0^+)$ se $v$ è discontinua in $0$.\newline
\newline
\textbf{es.} dato $v(t) = sca(t)$, la sua trsformata è $V(s) = \frac{1}{s}$ e $v(0^+) = 1$ perchè $\lim_{s\rightarrow \infty}s \cdot \frac{1}{s} = 1$.
\subsection{Teorema del valore finale}
\[
    V(s) = \mathcal{L}[v(t)] \Longrightarrow \text{se esiste}\; \lim_{t\rightarrow \infty} v(t) \text{, allora}\;\lim_{t\rightarrow \infty} v(t) = \lim_{s\rightarrow 0}sV(s)
\]
\subsection{Trasformate di Laplace notevoli}
\renewcommand{\arraystretch}{2}
\begin{center}
    \begin{tabular}{ |c|c| } 
     \hline
     \;\;\;\;\;\;\;\;\;\;\;\;\;\;\;$v(t)$ \;\;\;\;\;\;\;\;\;\;\;\;\;\;\;& \;\;\;\;\;\;\;\;\;\;\;\;\;\;\;$V(s)$ \;\;\;\;\;\;\;\;\;\;\;\;\;\;\;\\ 
     \hline
     $imp(t)$ & $1$ \\ 
     $sca(t)$ & $\frac{1}{s}$  \\ 
     $ram(t) = t \cdot  sca(t)$ & $\frac{1}{s^2}$ \\
     $e^{at}sca(t)$ & $\frac{1}{s-a}$ \\ 
     $t^{n}\cdot e^{at}sca(t)$ & $\frac{n!}{(s-a)^{n+1}}$\\ 
     \hline
    \end{tabular}
\end{center}
Da notare è che la rampa è l'integrale dello scalino, che a sua volta è l'integrale dell'impulso.\newline
Queste prime tre trasformate prendono il nome di segnali canonici, e sono tutte quelle trasformate di Laplace che valgono $\frac{1}{s^n}$.\newline
Notiamo che segnali come trigonometrici sono rappresentabili come esponenziali complessi, come $sin(\omega t) = \frac{e^{j \omega t}- e^{-j \omega t}}{2j} \dots$ e così possiamo ricavare le trasformate di Laplace di tutte le funzioni trigonometriche.
\subsection{Antitrasformazione $\mathcal{L^{-1}}$ secondo Heaviside}
Questo metodo vale per trasformate di Laplace razionali fratte, cioè trasformate che possono essere scritte come
\[
    V(s) = \frac{N(s)}{D(s)}
\] dove $N, D$ sono polinomi in $s$, inoltre il grado di $N$ e sempre minore o uguale del grado di $D$.\newline
\newline
Il meccanismo di antitrasformazione di Heaviside consiste nella scomposizione di $V(s)$.