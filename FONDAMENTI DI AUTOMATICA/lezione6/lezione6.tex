\title{LEZIONE 6 18/03/2020}\newline
\textbf{link} \href{https://web.microsoftstream.com/video/0c2093a0-44b9-4c83-a0b1-7e61fbd111b5?list=user&userId=faa91214-a6f5-40d7-8875-253fd49b8ce1}{clicca qui}
\subsection{Teorema del valore iniziale}
\[
    V(s) = \mathcal{L}[v(t)] \Longrightarrow v(0) = \lim_{s\rightarrow \infty}s V(s)
\] dove possiamo usare $v(0^+)$ se $v$ è discontinua in $0$.\newline
\newline
\textbf{es.} dato $v(t) = sca(t)$, la sua trsformata è $V(s) = \frac{1}{s}$ e $v(0^+) = 1$ perchè $\lim_{s\rightarrow \infty}s \cdot \frac{1}{s} = 1$.\newline
\newline
