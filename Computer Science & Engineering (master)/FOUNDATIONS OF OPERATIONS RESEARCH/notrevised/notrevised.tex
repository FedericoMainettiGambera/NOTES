\title{LEZIONE 1 16/09/2020}\newline
\newline
\url{../other/info-FOR-20-21-presentation.pdf}\newline
\url{../other/Introduction-FOR-20-21-presentation.pdf}
\newpage
\title{LEZIONE 2 22/09/2020}\newline
\newline
Esercitazione..\newline
\newline
\url{../other/exercises-1.1-1.2-text.pdf}\newline
\newline
Solutions will be available next week on beep.\newline
\newline
We see two simple and common example of decision making problem with linear optimization.\newline
\newline
\textbf{1.1 Portfolio optimization}\newline
\newline
The "risk factor" is the fraction of the investement that you can be lost. A risk factor of $1$ means that all the investment can be lost.\newline
\newline
The term "Programming" in Linear Programming is just a storical term, in real its meaning is "Optimization".\newline
\newline
Lets defined the Decision variables:
\begin{itemize}
    \item $x_i =$ amount of money invested in stock of types $i$, with $i = 1,2$;
\end{itemize}
\ \newline
What do we want to maximize? $max \{ 0,15 \cdot x_1 + 0,25 \cdot x_2 \}$.
\newline
\newline
Lets now define all the costraints:
\begin{itemize}
    \item (a) $x_1 + x_2 \leq C$;
    \item (b) $\frac{1}{3} \cdot  x_1 + 1 \cdot x_2 \leq \frac{C}{2}$;
    \item (c) $x_2 \leq 2 \cdot x_1$;
    \item (d) $x_1 \geq \frac{1}{6} C$;
    \item $x_1 \geq 0$ and $x_2 \geq 0$ (obvious).
\end{itemize}
\ \newline
[immagine: grafico]\newline
\newline
bla bla bla non c'ho cazzi.
\newline
\newline
\rule{\textwidth}{0,4pt}
\newline
\newline
\textbf{1.2 Gasoline mixture}\newline
\newline
bla bla bla non c'ho cazzi.