\section{What is AI?}
\textbf{Professor's slides}:\newline
\url{../other/professor's slides/Part II - What is AI.pdf}\newline
\newline
Artificial Intelligence (AI) is an area of Computer Science and Engineering whose goal is to
develop computational systems that can be considered intelligent in some meaningful way.\newline
\newline
Since the beginning, the goal of AI has been to program digital computers in order to
replicate aspects of human intelligence.\newline
Of course, this presupposes that we know what we are talking about when we use the
terms “intelligence”, “intelligent”, and so on.\newline
\newline
So, what do we actually mean by “intelligence”, “intelligent”, etc.?\newline
We have an intuitive understanding, but no rigorous definition that is universally accepted.\newline
\newline
A possible interpretation of the term “intelligent” tends to identify it with the term
“rational”, which in turn is viewed under the rather restricted perspective that is typical of
the economical sciences: being rational means being able to act in such a way that some
utility function is maximised (or some cost function is minimised).\newline
\newline
Under this view, the predicate “intelligent” may be applied:
\begin{itemize}
    \item to an \textbf{agent}, that is, to a systems that we regard as capable of performing actions
    (persons, dogs, …, maybe certain types of artificial systems);
    \item to a specific \textbf{action} perfromed by an agent (e.g., an intelligent answer);
    \item to the \textbf{result} or product of an action (e.g., an intelligent coffee machine).
\end{itemize}
If, as it seems reasonable, we opt for the first one, this implies that we are ready to regard a computer
program as an \textbf{agent}.